\documentclass[10pt]{report}
\usepackage[utf8]{inputenc}
\usepackage{amsfonts}
\usepackage{amsmath}
\usepackage{amssymb}
\usepackage{amsopn}
\usepackage{commath}
\usepackage[ngerman]{babel}
\usepackage{enumitem}
\usepackage{booktabs}
\usepackage{longtable}
\usepackage{relsize}

\DeclareMathOperator{\grad}{grad}

\newcommand{\K}{\mathbb{K}}
\newcommand{\N}{\mathbb{N}}
\newcommand{\Z}{\mathbb{Z}}
\newcommand{\Q}{\mathbb{Q}}
\newcommand{\R}{\mathbb{R}}
\newcommand{\C}{\mathbb{C}}
\newcommand{\an}{{(a_n)}_{n=1}^\infty}
\newcommand{\sn}{{(s_n)}_{n=1}^\infty}
\newcommand{\ReP}{\operatorname{Re}}
\newcommand{\ImP}{\operatorname{Im}}
\newcommand*{\nthSqrt}[2]{\sqrt[\leftroot{-2}\uproot{2}#1]{#2}}
\newcommand{\setbigcup}{\mathop{\bigcup}\displaylimits}
\newcommand{\setbigcap}{\mathop{\bigcap}\displaylimits}
\newcommand{\setbigtimes}{\mathop{\times}\displaylimits}
\newcommand{\ddx}{\dfrac{\text{d}}{\text{d}x}}
\newcommand{\dx}{\text{ d}x}
\newcommand{\dy}{\text{ d}y}
\newcommand{\dt}{\text{ d}t}
\newcommand{\ds}{\text{ d}s}
\newcommand{\ido}{\text{ d}o}
\newcommand{\intd}[1]{\text{ d}#1}
\newcommand{\rot}{\operatorname{rot}}
\newcommand{\vdiv}{\operatorname{div}}
\newcommand{\vspan}{\operatorname{span}}
\newcommand{\Kern}{\operatorname{Kern}}
\newcommand{\diag}{\operatorname{diag}}
\newcommand{\Image}{\operatorname{Im}}
\newcommand{\rg}{\operatorname{rg}}
\newcommand{\sgn}{\operatorname{sgn}}

\setlength\parindent{0pt}

\title{Zusammenfassung Höhere Mathematik}
\author{Paul Nykiel}

\begin{document}
    \maketitle
    \pagebreak
    Schlagzahl erhöhen.
    \pagebreak
    \tableofcontents
    \pagebreak

    \part{HM 1 --- Zusammenfassung}
    \chapter{Vorkurs}
    \input{HM1/Vorkurs}

    \chapter{Grenzwerte}
    \input{HM1/Grenzwerte}

    \chapter{Differentialrechnung}
    \section{Ableitung}

\subsection{Definition Differenzen-Quotient}
Sei $f: D \subseteq \R \rightarrow \R $. Dann heißt $f$ in
$x_0 \in D$ differentierbar, falls
\begin{equation*}
    f'(x_0) := \lim_{x \rightarrow x_0}
        \frac{f(x) - f(x_0)}{x - x_0}
\end{equation*}
für alle $x_0 \in D$ existiert.

\subsection{Rechtsseitige und linksseitige Ableitung}
Im Fall der Existenz heißen
\begin{eqnarray*}
    f'(x_0^+) &:=& \lim_{x \rightarrow x_0^+}
        \frac{f(x) - f(x_0)}{x-x_0} \text{ bzw.}\\
    f'(x_0^-) &:=& \lim_{x \rightarrow x_0^-}
        \frac{f(x) - f(x_0)}{x-x_0}
\end{eqnarray*}
die rechts- bzw.\ linksseitige Ableitung in $x_0$

\subsubsection{Bemerkung}
\begin{equation*}
    f'(x_0) \text{ ex.} \Leftrightarrow
    f'(x_0^+) \text{ und } f(x_0^-) \text{ ex.\ und }
    f'(x_0^+) = f'(x_0^-)
\end{equation*}

\subsection{Ableitungsrechenregeln}
Seien $f, g: D \rightarrow \R$ differentierbar in $x_0 \in D$, dann gilt:
\begin{enumerate}[label= (\alph*)]
    \item $(f+g)'(x_0) = f'(x_0) + g'(x_0)$
    \item $(f \cdot g)'(x_0) = f'(x_0) \cdot g(x_0) +
                                f(x_0) \cdot g'(x_0)$
    \item Falls $g(x_0) \neq 0$: $\left( \dfrac{f}{g} \right) '(x_0) =
            \dfrac{f'(x_0) g(x_0) - f(x_0) g'(x_0)}{{g(x_0)}^2}$
\end{enumerate}

\subsection{Alternative Definition der Ableitung}
Sei $f: D \subseteq \R \rightarrow \R$ und $x_0 \in D$. Dann gilt:
f differenzierbar in $x_0 \Leftrightarrow$
\begin{equation*}
    \exists A \in \R \text{ und } r: D \rightarrow \R \text{ mit }
    \lim_{x \rightarrow x_0} r(x) = 0 \text{ so dass gilt: }
    f(x) = f(x_0) + A \cdot (x - x_0)  + r(x) \cdot (x-x_0)
\end{equation*}

\subsection{Zusammenhang Differentierbarkeit --- Stetigkeit}
Ist $f: D \rightarrow \R$ differentierbar in $x_0 \in D \Rightarrow$
$f$ stetig in $x_0$

\subsection{Differentiation von Potenzreihen}
Sei $f(x) = \sum_{k=0}^\infty a_k {(x-x_0)}^k$ eine Potenzreihe mit $R>0$, dann
ist $f$ für $x$ mit $\abs{x-x_0}<R$ differentierbar, und es gilt:
\begin{equation*}
    f'(x) = \sum_{k=1}^\infty a_k \cdot k \cdot {(x - x_0)}^{k-1}
\end{equation*}

\subsubsection{Bemerkung}
Der Konvergenzradius von $f'(x)$ ist ebenfalls $R$.

\subsection{Ableitung der Umkehrfunktion}
Seien $I, J \subseteq \R$ Intervalle und $f: I \rightarrow J$ sei differentiarbar
und bijektiv, dann ist auch $f^{-1}: J \rightarrow I$ differentierbar und es
gilt:
\begin{equation*}
    {(f^{-1})}'(y_0) = \frac{d}{dx} f^{-1}(y_0) =
    \frac{1}{f'(f^{-1}(y_0))} \forall y_0 \in J \text{ für ein } y_0=f(x_0)
    \text{ und } f'(y_0) \neq 0
\end{equation*}

\subsection{Ketternregel}
Seien $f: A \rightarrow B$, $g: B \rightarrow \R$ mit $A,B \subseteq \R$
differentierbar auf $A$ bzw. $B$, dann ist auch $g \circ f$ auf $A$
differentierbar und es gilt:
\begin{equation*}
    (g \circ f)'(x_0) = g'(f(x_0)) \cdot f'(x_0)\ \forall x_0 \in A
\end{equation*}

\section{Mittelwertsätze}

\subsection{Satz von Rolle}
Sei $f: [a, b] \rightarrow \R$ stetig und auf $(a,b)$ differentierbar. Falls
$f(a) = f(b)$ gilt, existiert ein $x_0 \in (a,b)$ mit $f'(x_0)=0$

\subsection{Definition lokaler Extrempunkt}
Sei $f: D \rightarrow \R$ und $x_0 \in D$. Dann besitzt$f$ in $x_0$ ein
lokales Maximum (bzw. Minimum)$:\Leftrightarrow$
\begin{equation*}
    \exists \delta > 0: f(x) \leq f(x_0)
    \text{ (bzw. } f(x) \geq f(x_0) \text{) } \forall x \in D \cap U_\delta(x_0)
\end{equation*}

\subsection{Notwendige Bedingung für lokale Extrema}
Sei $f: D \rightarrow \R$ differentierbar in $x_0 \in D$ und $x_0$ sei kein
Randpunkt, dann gilt:

Liegt bei $x_0$ ein lokales Maximum/Minimum $\Rightarrow f'(x_0)=0$.

\subsection{2. Mittelwertsatz}
Seien $f,g: [a,b] \rightarrow \R$ stetig und auf $(a,b)$ differentierbar
dann existiert ein $x_0 \in (a,b)$ mit
\begin{equation*}
    f'(x_0) \cdot (g(b) - g(a)) = g'(x_0) \cdot (f(b)-f(a))
\end{equation*}
Bzw.\ falls nicht durch Null geteilt wird:
\begin{equation*}
 \frac{f'(x_0)}{g'(x_0)} = \frac{f(b)-f(a)}{g(b)-g(a)}
\end{equation*}

\subsection{1. Mittelwertsatz (Folgerung aus 2. Mittelwertsatz)}
Sei $f: [a,b] \rightarrow \R$ stetig und auf $(a,b)$ differentierbar
\begin{equation*}
    \Rightarrow \exists x_0 \in (a,b) \text{ mit } f'(x_0)=\frac{f(b)-f(a)}{b-a}
\end{equation*}

\subsection{L'Hospital}
Seien $f,g: [a,b) \rightarrow \R (a < b, b \in (\R \cup {\infty}))$
differentierbar auf $(a,b)$ mit $g'(x) \neq 0\ \forall x \in (a,b)$. Falls
der Grenzwert $\alpha = \lim\limits_{n \rightarrow b^-} \dfrac{f'(x)}{g'(x)}$
ex.\ und:
\begin{enumerate}[label= (\alph*)]
    \item $\lim\limits_{n \rightarrow b^-} f(x) =
        \lim\limits_{n \rightarrow b^-} g(x) = 0$ oder
    \item $\lim\limits_{x \rightarrow b^-} g(x) = \infty$
\end{enumerate}
dann gilt:
\begin{equation*}
    \lim_{x \rightarrow b^-} \frac{f(x)}{g(x)} =
    \lim_{x \rightarrow b^-} \frac{f'(x)}{g'(x)}
\end{equation*}

\subsection{Satz von Taylor}
Sei $f: [a,b] \rightarrow \R$ $n+1$ mal differentierbar auf $(a,b)$ und
$x_0 \in (a,b)$. Dann gilt für ein $\xi \in (x_0, x)$:
\begin{equation*}
    f(x) = \sum_{k=0}^n \frac{f^{(k)}(x_0)}{k!} \cdot {(x-x_0)}^k +
    \frac{f^{(n+1)}(\xi)}{(n+1)!} \cdot {(x-x_0)}^{n+1}
\end{equation*}


    \part{HM 2 --- Zusammenfassung}
    \chapter{Integration}
    \section{Integration}

\subsection{Definition Zerlegung, Zwischenwerte}
Eine Teilmenge $T$ von $[a,b]$ mit $a, b \in T$ nennt man eine
Unterteilung, Zerlegung oder Partitionierung von $[a, b]$ wenn
gilt:
\begin{eqnarray*}
    T = \{ x_0, x_1, \ldots , x_n\} \text{ mit}\\
    a = x_0 < x_1 < \ldots < x_n = b
\end{eqnarray*}

Schreibweise für diese Menge $T$ sei:
\begin{equation*}
    T: a = x_0 < x_1 < \ldots < x_n = b
\end{equation*}

Ist T eine Zerlegung, dann heißt:
\begin{enumerate}[label= (\alph*)]
    \item Die Zahl $\mu(T) :=
        \max{ \{\ \abs{x_k - x_{k+1}}, k = 1, \ldots, n \} }$
        das Feinheitsmaß von $T$.
    \item Ein Vektor $\xi = (\xi_1, \ldots, \xi_n) \in \R^n$ heißt
        ein Zwischenwertvektor zu $T$, wenn gilt
        \begin{equation*}
            x_{k-1} \leq \xi_k \leq x_k \text{ für } k = 1, \ldots, n
        \end{equation*}
        Dann heißt die Komponente $\xi_k$ ein Zwischenwert von
        $x_{k-1}$ und $x_k$.
\end{enumerate}

\subsection{Definition Riemannsumme}
Ist $f: [a,b] \rightarrow \R$ eine Funktion, $T: a=x_0<\ldots<x_n=b$
eine Zerlegung von $[a, b]$ und $\xi = (\xi_1, \ldots, \xi_n)$ ein
Zwischenwertevektor zu $T$, dann nennen wir die Summe
\begin{equation*}
    S(f; T, \xi) = S_f(T, \xi) = \sum_{k=1}^n f(\xi_k)(x_k - x_{k-1})
\end{equation*}
die Riemansumme von $f$ bezüglich $T$ und $\xi$.

\subsection{Definition Riemann-Integral}
Eine Funktion $f: [a, b] \rightarrow \R$ heißt Riemann-Integrierbar
unter $[a, b]$ wenn für jede Folge ${(T_N)}_{N=1}^\infty$ von Zerlegungen
von $[a,b]$ mit $\mu(T_N) \rightarrow 0$ für $N \rightarrow \infty$
und jede Folge ${(\xi_N)}_{N=1}^\infty$ von Zwischenpunktvektoren
der Grenzwert
\begin{equation*}
    \lim_{N \rightarrow \infty} S(f; T_N, \xi_N) \text{ existiert.}
\end{equation*}

\subsubsection{Behauptung}
Der Grenzwert ist im Fall der Existenz für jede Folge identisch.

\subsubsection{Bemerkung}
\begin{enumerate}[label= (\alph*)]
    \item Im Fall der Existenz bezeichnet man den Grenzwert durch:
        \begin{equation*}
            \int_{a}^b f(x) \dx = \lim_{N \rightarrow \infty} S(f; T_N, \xi_N)
        \end{equation*}
    \item Zu ${(T_N)}_{N=1}^\infty$, also $T_1, T_2, T_3, \ldots$:
        \begin{eqnarray*}
            &T_1:& a = x_0^{(1)} < \ldots < x_n^{(1)} = b\\
            &T_2:& a = x_0^{(2)} < \ldots < x_n^{(2)} = b\\
            &T_3:& a = x_0^{(3)} < \ldots < x_n^{(3)} = b\\
            &\vdots&\\
            &T_l:& a = x_0^{(l)} < \ldots < x_n^{(l)} = b
        \end{eqnarray*}
    \item Zu ${(\xi_N)}_{N=1}^\infty$, also $\xi_1, \xi_2, \xi_3, \ldots$:
        \begin{eqnarray*}
            &\xi_1& = (\xi_1^{(1)}, \ldots, \xi_n^{(1)})
            \text{ mit } x_{k-1}^{(1)} \leq \xi_k^{(2)} \leq x_k^{(1)}
            \text{ mit } 1 \leq k \leq n_1\\
            &\xi_2& = (\xi_1^{(2)}, \ldots, \xi_n^{(2)})
            \text{ mit } x_{k-1}^{(2)} \leq \xi_k^{(2)} \leq x_k^{(2)}
            \text{ mit } 1 \leq k \leq n_2\\
            &\xi_3& = (\xi_1^{(3)}, \ldots, \xi_n^{(3)})
            \text{ mit } x_{k-1}^{(3)} \leq \xi_k^{3} \leq x_k^{(3)}
            \text{ mit } 1 \leq k \leq n_3\\
            &\vdots&\\
            &\xi_l& = (\xi_1^{(l)}, \ldots, \xi_n^{(l)})
            \text{ mit } x_{k-1}^{(l)} \leq \xi_k^{l} \leq x_k^{(l)}
            \text{ mit } 1 \leq k \leq n_l\\
        \end{eqnarray*}
    \item Sei $f$ integrierbar und ${(T_N)}_{N=1}^\infty$ und
        ${(\xi_N)}_{N=1}^\infty$ sowie ${(\tilde{T}_N)}_{N=1}^\infty$ und
            ${(\tilde{\xi}_N)}_{N=1}^\infty$ entsprechende Folgen,
            d.h. $\mu(T_N) \rightarrow 0, \mu(\tilde{T}_N) \rightarrow 0$
            für $N \rightarrow \infty$. Dann gilt gilt für ${(\hat{T}_N)}_{N=1}^\infty$
            und ${(\hat{\xi}_N)}_{N=1}^\infty$ mit
            \begin{equation*}
                \hat{T}_N :=
                \begin{cases}
                    T_N &\text{ für } N \text{ gerade}\\
                    \tilde{T}_N &\text{ für } N \text{ ungerade}
                \end{cases}
            \end{equation*}
            und
            \begin{equation*}
                \hat{\xi}_N :=
                \begin{cases}
                    \xi_N &\text{ für } N \text{ gerade}\\
                    \tilde{\xi}_N &\text{ für } N \text{ ungerade}
                \end{cases}
            \end{equation*}
            dass
            \begin{equation*}
                \lim_{N \rightarrow \infty} S(f; \hat{T}_N, \hat{S}_N)
            \end{equation*}
            existiert, da $f$ integrierbar ist.

            Dann stimmt der Grenzwert von
            $\lim_{N \rightarrow \infty} S(f; \tilde{T}_N, \tilde{S}_N)$ und
            $\lim_{N \rightarrow \infty} S(f; T_N, S_N)$ überein.
\end{enumerate}

\subsection{Menge der Riemann-Integrierbaren Funktionen}
Mit $R [a, b]$ oder $R ([a, b])$ bezeichnen wir die Menge von Funktionen
$f: [a, b] \rightarrow \R$ die auf $[a, b]$ Riemann integrierbar sind.

\subsection{Kriterien für Riemann-Integrierbarkeit}
\begin{enumerate}[label= (\alph*)]
    \item
        \begin{equation*}
            f \in R[a, b] \Rightarrow f \text{ ist auf } [a,b] \text{beschränkt}
        \end{equation*}
    \item Ist $f, g \in R [a, b]$ und $c \in \R$ dann sind auch die Funktionen
        \begin{eqnarray*}
            f&+&g\\
            f&-&g\\
            c &\cdot& f
        \end{eqnarray*}
        Riemann integrierbar auf $[a, b]$.
    \item Ist $f, g \in R[a, b]$, dann ist auch
        \begin{equation*}
            f \cdot g \in R[a, b]
        \end{equation*}
    \item Ist $f, g \in R[a, b]$ und falls
        $\abs{g(x)} > \delta > 0\ \forall x \in [a, b]$ dann ist auch
        \begin{equation*}
            \frac{f}{g} \in R[a, b]
        \end{equation*}
    \item Für beliebiges $c \in [a, b]$  gilt:
        \begin{equation*}
            f \in R[a, b] \Leftrightarrow f \in R[a, c] \land f \in R[c, b]
        \end{equation*}
        und weiter gilt:
        \begin{equation*}
            \int_a^b f(x) \dx = \int_a^c f(x) \dx + \int_c^b f(x) \dx
        \end{equation*}
    \item
        \begin{equation*}
        f \in R[a, b] \Rightarrow \abs{f} \in R[a, bv]
        \end{equation*}
        und
        \begin{equation*}
            \abs{\int_a^b f(x)\dx} \leq \int_a^b \abs{f(x)} \dx
        \end{equation*}
\end{enumerate}

\subsection{Änderung von Funktionen}
Wenn $f \in R[a, b]$ ist und durch endlich viele Änderungen daraus
$g: [a,b] \rightarrow \R$ konstruiert werden kann, d.h.
\begin{equation*}
    g(x) =
    \begin{cases}
        f(x) &\text{ falls } x \notin \{x_1, \ldots, x_n \} \\
        y_1 &\text{ falls} x=x_1\\
        \vdots
    \end{cases}
\end{equation*}
dann gilt $g \in R[a, b]$ und
\begin{equation*}
    \int_a^b f(x) \dx = \int_a^b g(x) \dx
\end{equation*}

\subsection{Zusammenhang Stetigkeit und Integrierbarkeit}
Es gilt:
\begin{equation*}
    f \in C[a, b] \Rightarrow f \in R[a, b]
\end{equation*}

\subsection{Stückweise Integration}
Falls $f: [a, b] \rightarrow \R$ stückweise stetig ist, d.h.\ es existieren
endlich viele Intervall-Stücke auf denen $f$ stetig ist, dann ist
$f \in R[a,b]$ und es gilt:
\begin{equation*}
    \int_a^b f(x) \dx = \int_{x_0}^{x_1} f(x) \dx + \int_{x_1}^{x_2} f(x)
    + \ldots + \int_{x_{n-1}}^{x_n} f(x)
\end{equation*}


    \chapter{Gleichmäßige Konvergenz}
    %!TEX root = ../main.tex
\section{Gleichmäßige Konvergenz}
\subsection{Definition Funktionenfolge und Funktionenreihe}
Sei $M$ eine Menge und $m \in \Z$. Ist jedem $n \in \{m, m+1, \ldots\}$ eine
Funktion $f_n: M \rightarrow \R$ zugeordnet, so nennt man:
\begin{enumerate}[label= (\alph*)]
    \item Die Folge ${(f_n)}_{n=m}^\infty$ eine Funktionenfolge auf $M$
    \item Die Reihe $\sum_{n=m}^\infty f_n(x)$ eine Funktionenreihe auf $M$
\end{enumerate}

konvergiert ${(f_n)}_{n \geq m}$ (bzw. $\sum_{n=m}^\infty f_n(x)$) für alle
$x \in \tilde{M} \subseteq M$ so heißt die durch $f(x) = \lim\limits_{n \rightarrow \infty}
f_n(x)$ (bzw. $f(x) = \sum_{n=m}^\infty f_n(x)$) definierte Funktion $f:
\tilde{M} \rightarrow \R$ die Grenzfunktion von ${(f_n)}_{n=m}^\infty$
(bzw. $\sum_{n=m}^\infty f_n$).

\subsection{Gleichmäßige Konvergenz}
Sei $M$ eine Menge und sei $f: M \rightarrow \R$ eine Funktion.
\begin{enumerate}[label= (\alph*)]
    \item Eine Funktionenfolge ${(f_n)}_{n=1}^\infty$ heißt auf $M$ gleichmäßig
        konvergent gegen $f$ wenn gilt:
        \begin{equation*}
            \forall \varepsilon > 0\  \exists\ n_0(\varepsilon):
                \abs{f_n(x) - f(x)} < \varepsilon\
                \forall x \in M \text{ und } n \geq n_0(\varepsilon)
        \end{equation*}
    \item Eine Funktionenfolge $\sum_{n=m}^\infty f_n$ konvergiert gleichmäßig
        auf $M$ wenn gilt:
            \begin{equation*}
                \forall \varepsilon > 0\ \exists\ n_0(\varepsilon): \abs{\sum_{k=m}^n f_k(x) - f(x)}
                    < \varepsilon\ \forall x \in M \text{ und } n \geq n_0(\varepsilon)
            \end{equation*}
\end{enumerate}

\subsubsection{Bemerkung}
Offensichtlich gilt:
\begin{equation*}
    \text{Gleichmäßig konvergent} \Rightarrow \text{Punktweise Konvergent}
\end{equation*}


\subsection{Stetigkeit der Grenzfunktion}
Sei ${(f_n)}_{n=1}^\infty$ (bzw. $\sum_{n=1}^\infty f_n(x)$) gleichmäßig konvergent gegen
$f$ auf einem Intervall $I$ und alle $f_n$ stetig auf $I$. Dann ist auch die
Grenzfunktion $f$ stetig.

\subsection{Integration der Grenzfunktion}
Sei ${(f_n)}_{n=1}^\infty$ eine Folge von integrierbaren Funktionen auf $[a,b]$
\begin{enumerate}[label= (\alph*)]
    \item Falls ${(f_n)}_{n=1}^\infty$ gleichmäßig gegn $f$ konvergiert, dann ist
    auch $f$ auf $[a,b]$ integrierbar und es gilt:
        \begin{equation*}
            \lim_{n \rightarrow \infty} \int_a^b f_n(x) \dx =
            \int_a^b \lim_{n \rightarrow \infty} f_n(x) \dx
        \end{equation*}
    \item Analog für Funktionenreihen
\end{enumerate}

\subsection{Cauchy Kriterium für gleichmäßige Konvergenz}
\begin{enumerate}[label= (\alph*)]
    \item Eine Funktionenfolge ${(f_n)}_{n=1}^\infty$ konvergiert genau dann
        gleichmäßig auf einer Menge $M$ ($\subseteq$ Definitionsbereich), wenn
        gilt:
        \begin{equation*}
            \forall \varepsilon > 0\ \exists n(\varepsilon):
            \abs{f_n(x) - f_m(x)} < \varepsilon\ \forall n \geq n(\varepsilon) \forall x \in M
        \end{equation*}
    \item Analog für Funktionenreihen
\end{enumerate}

\subsection{Differentiation der Grenzfunktion}
Sei ${(f_n)}_{n=1}^\infty$ eine auf dem Intervall $I$ differentierbare Folge von
Funktionen.
\begin{enumerate}[label= (\alph*)]
    \item Konvergiert die Folge ${(f_n')}_{n=1}^\infty$ gleichmäßig auf $I$ und konvergiert
    für ein beliebiges, festes $x_0 \in I$ die reele Folge ${(f_n(x_0))}_{n=1}^\infty$
    dann ist auch die Grenzfunktion $f$ von ${(f_n)}_{n=1}^\infty$ differentierbar
    und es gilt:
    \begin{equation*}
        \lim_{n \rightarrow \infty} \ddx f_n(x) = \ddx \lim_{n \rightarrow \infty} f_n(x)
    \end{equation*}
    \item Analog für Funktionenreihen
\end{enumerate}

\subsubsection{Bemerkung}
Außerdem gilt dass ${(f_n)}_{n=1}^\infty$ (bzw. $\sum_{n=1}^\infty f_n$) auf jedem
beschränkten Teilintervall von $I$ gleichmäßig konvergiert.

\subsection{Majorantenkriterium auf Potenzreihen anwenden}
Für eine reele Potenzreihe $f(x) = \sum_{k=0}^\infty a_k {(x-x_0)}^k$ mit
Konvergenzradius $R > 0$ gilt:
\begin{enumerate}[label= (\alph*)]
    \item $f$ ist stetig auf $(x_0 - R, x_0 + R) =: I$
    \item $f$ ist differentierbar auf $I$ und
        \begin{equation*}
            f'(x) = \sum_{k=0}^\infty a_k \cdot k \cdot {(x-x_0)}^{k-1}
        \end{equation*}
    \item $f$ ist integrierbar auf $I$ und hat die Stammfunktion
        \begin{equation*}
            F(x) = \sum_{k=0}^\infty \frac{a_k}{k+1} {(x-x_0)}^{k+1}
        \end{equation*}
\end{enumerate}

\subsubsection{Bemerkung}
Wurde alles schon in HM1 gezeigt aber mühsam.

\subsection{Majorantenkriterium für Funktionenreihen}
Falls $\abs{f_n(x)} \leq a_n$ und $\sum_{n=1}^\infty a_n$ konvergiert
$\Rightarrow \sum_{n=1}^\infty f_n(x)$ ist gleichmäßig konvergent.


    \chapter{Differentialrechung mit mehreren Variablen}
    \section{Der n-dimensionale Euklidische Raum}

\subsection{Definitionen}
Sind $n, m \in \N$, so gelten folgende Bezeichungen:
\begin{eqnarray*}
    \R^n &:=& \Bigg\{
        \begin{pmatrix}
            x_1\\
            \vdots \\
            x_n
        \end{pmatrix}
        \text{ für }
        x_1, \ldots, x_n \in \R^n
    \Bigg\} \\
    \R^{m \times n} &:=& \Bigg\{
        \begin{pmatrix}
            a_{11} & \cdots & a_{1n}\\
            \vdots & \ddots\\
            a_{m1} & \ldots & a_{mn}
        \end{pmatrix}
        \text{ für }
        a_{ij} \in \R, 1 \leq i \leq m, 1 \leq j \leq n
    \Bigg\}\\
    <x, y> &:=& x \cdot y := x^T y := \sum_{k=1}^n x_k y_k \text{ (Skalarprodukt)}\\
    \norm{x} &:=& {\Vert x \Vert}_2 := \abs{x} := \sqrt{\sum_{k=1}^n x_k^2}
    \text{ euklidische Norm des }\R^n\text{ /Betrag in }\R^n
\end{eqnarray*}

\subsection{Folgerungen}
\begin{enumerate}
    \item
        \begin{equation*}
            \norm{x}_\infty = \max_{k=1\ldots n} \abs{x_k}
            \leq \norm{x}_2 \leq \sqrt{n} \max_{k=1 \ldots n} \abs{x_k}
            \ \forall x \in \R^n
        \end{equation*}
    \item
        \begin{equation*}
            \norm{x}_1 = \sum_{k=1}^n \abs{x_k}
        \end{equation*}
        und
        \begin{equation*}
            \norm{x}_2 \leq \norm{x}_1
        \end{equation*}
    \item $\norm{x}_1$, $\norm{x}_2$, $\norm{x}_\infty$ sind drei mögliche Festlegungen
        für Vektornormen. Allgemein hat eine Norm $\norm{\cdot}_2$
        ($\norm{\cdot}_2: \R^2 \rightarrow \R$) folgende Eigenschaften:
        \begin{eqnarray*}
            &\norm{x} \geq 0&\ \forall x \in \R^n \land \norm{x}=0 \Leftrightarrow
            x = \begin{pmatrix}
            0 \\ 0
            \end{pmatrix}
            = \vec{0} \\
            &\norm{\alpha \cdot x} = \abs{\alpha} \cdot \norm{x}&\ \forall \alpha
            \in \R\land \forall x \in \R^n \\
            &\norm{x + y} \leq \norm{x} + \norm{y}&\ \forall x, y \in \R^n
        \end{eqnarray*}
    \item Der Einheitskreis ist bezüglich verschiedener Normen nicht immer ein Kreis
    \item p-Norm:
        \begin{equation*}
            \norm{x}_p = \nthSqrt{p}{\sum_{k=1}^n \abs{x_k}^p}
        \end{equation*}
     \item $x \cdot y$ im $\R^2$ hat die anschauliche Bedeutung
        \begin{equation*}
            <x,y> = x \cdot y = \norm{x}_2 \cdot \norm{y}_2 \cdot \cos(\alpha)
        \end{equation*}
        Daraus folgt die Cauchy-Schwarzsche-Ungleichung (CSU)
        \begin{equation*}
            <x,y> \leq \norm{x}_2 \cdot \norm{y}_2
        \end{equation*}
\end{enumerate}

\subsection{Konventionen}
\begin{enumerate}[label= (\alph*)]
    \item In $\R^n$ sei stets $A^c := \R^n \ A$ für eine Menge $A \subseteq \R^n$
    \item Mit $\norm{\cdot}$ bezeichnen wir die euklidische Norm $\norm{\cdot}_2$.
    Außer es wird explizit gesagt, dass $\norm{\cdot}$ eine allgemeine Norm ist
    (z.B. \glqq{} Sei $\norm{\cdot}$ eine Norm auf $\R^n$)
\end{enumerate}


    \chapter{Integration in mehreren Veränderlichen}
    \section{Parameterintegrale}
\subsection{Eigentliche Parameterintegrale}
Sei $f(x,t)$ reel und stetig in $[\alpha, \beta] \times [a,b]$ (also $x\in[\alpha, \beta],
t \in [a,b]$). Dann gilt für
\begin{equation*}
    F(x) := \int_a^b f(x,t) \dt
\end{equation*}
\begin{enumerate}[label= (\alph*)]
    \item $F$ ist stetig auf $[\alpha, \beta]$
    \item Ist $f_x$ stetig in $[\alpha, \beta] \times [a,b]$, so ist
        $F \in C^1([\alpha, \beta])$ und $F'(x) = \int_a^b f_x(x,t) \dt$
    \item Satz von Fubini:
        \begin{equation*}
            \int_\alpha^\beta F(x) \dx = \int_\alpha^\beta \int_a^b f(x,t) \dt \dx
            = \int_a^b \int_\alpha^\beta f(x,t) \dx \dt
        \end{equation*}
\end{enumerate}

\subsection{Leibniz Regel}
Seien $f(x,t), f_x(x,t)$ stetig in $[\alpha, \beta] \times [a,b]$ und
$u, v \in C^1 ( [a,b])$. Dann ist
\begin{equation*}
    F(x) = \int_{u(x)}^{v(x)} f(x,t) \dt \in C^1([a,b])
\end{equation*}
und
\begin{equation*}
    F'(x) = \int_{u(x)}^{v(x)} f_x(x,t) \dt +
    f(x, v(x)) v'(x) - f(x, u(x)) u'(x)
\end{equation*}

\subsection{Uneigentliche Parameterintegrale}
Ist für jedes $x \in M \subseteq \R$ ein uneigentliches Integral
\begin{equation*}
    \int_a^b f(x,t) \dt
\end{equation*}
mit kritischem Punkt $a$ oder $b$ gegeben, so heißt dieses gleichmäßig konvergent
in $M$, wenn gilt:
\begin{equation*}
    \forall \varepsilon > 0\ \exists L \in (a,b): \abs{\int_{T_1}^{T_2} f(x,t) \dt}
    <\varepsilon \ \forall x \in M \forall T_1, T_2 \in (a,L) (\text{bzw.}
    \forall T_1, T_2 \in (L,b))
\end{equation*}

\subsection{Majorantenkriterium}
Ein uneigentliches Integral $\int_a^b f(x,t) \dt$ konvergiert gleichmäßig in $M$ wenn ein konvergentes
Integral
\begin{equation*}
    \int_a^b g(t) \dt \text{ ex.\ mit} \abs{f(x,t)} \leq g(t)
\end{equation*}

\subsection{Fubini für uneigentliche Parameterintegrale}
Ist $f(x,t)$ stetig in $[\alpha, \beta] \times [a,b]$ und konvergiert
\begin{equation*}
    F(x) = \int_a^b f(x,t) \dt
\end{equation*}
gleichmäßig auf $[\alpha, \beta]$ dann ist $F$ stetig und
\begin{equation*}
    \int_\alpha^\beta \int_a^b f(x,t) \dt \dx =
    \int_a^b \int_\alpha^\beta f(x,t) \dx \dt
\end{equation*}

\subsection{Konvergenzkriterien}
Sind $f(x,t), f_x(x,t)$ stetig auf $[\alpha, \beta] \times [a,b]$ und ist
\begin{equation*}
    \int_a^b f(x, t) \dt
\end{equation*}
für ein $x_0 \in [\alpha, \beta]$ konvergent und ist
\begin{equation*}
    \int_a^b f_x(x,t) \dt
\end{equation*}
gleichmäßig konvergent. Dann gilt:
\begin{equation*}
    F(x) = \int_a^b f(x,t) \dt\ \forall x \in [\alpha, \beta]
\end{equation*}
und
\begin{equation*}
    F'(x) = \int_a^b f_x(x,t) \dt\ \forall x \in (\alpha, \beta)
\end{equation*}
existiert und ist stetig.

\section{Kurvenintegrale}
\subsection{Äquivalenz für Kurven}
Zwei stetige Funktionen $x: [a, b] \subseteq \R \rightarrow \R^n, y:[\alpha, \beta]
\subseteq \R \rightarrow \R^n$ heißen Äquivalent (schreibweise $x \sim y$), wenn eine
streng monoton wachsende Funktion
\begin{equation*}
    \phi: [a,b] \rightarrow [\alpha, \beta]
\end{equation*}
gibt mit
\begin{equation*}
    x(t) = y(\phi(t))\ \forall t \in [a,b]
\end{equation*}

\subsubsection{Bemerkung}
Es gilt:
\begin{enumerate}[label= (\alph*)]
    \item $x \sim x$ (Reflexivität)
    \item $x \sim y \Rightarrow y \sim x$ (Symmetrie)
    \item $x \sim y \land y \sim z \Rightarrow x \sim z$ (Transitivität)
\end{enumerate}

\subsection{Kurven im $\R^n$}
Ist $x: [a,b] \subseteq \R \rightarrow \R^n$ stetig, so nennt man die Menge
\begin{equation*}
    \K := \{y: [\alpha, \beta] \subseteq \R \rightarrow \R^n \text{ mit } x \sim y\}
\end{equation*}
die Kurve $\K$ mit Parameterdarstellung $x$ und den Punkt $x(a)$ Anfangspunkt und
$x(b)$ Endpunkt.

\subsubsection{Schreibweise}
\begin{equation*}
    \K: x(t), a \leq t \leq b
\end{equation*}
Die Menge
\begin{equation*}
    T(\K) := \{x(t): t \in [a,b]\} = x([a,b])
\end{equation*}
nennt man den Träger der Kurve $\K$.

\subsubsection{Bemerkung}
Verschieden Kurven können also den gleichen Träger haben.

Man nennt $K$:
\begin{enumerate}[label= (\alph*)]
    \item Geschlossen, wenn $x(a) = x(b)$
    \item Einfach oder Jordankurve, wenn $x(t) \neq x(s)\ \forall t,s: a \leq t < s <b$
\end{enumerate}

\subsection{Eigenschaften von Parameterdarstellungen}
\begin{enumerate}[label= (\alph*)]
    \item Eine Parameterdarstellung $x:[a,b] \rightarrow \R^n$ einer Kurve heißt
        stückweise stetig differentierbar, wenn eine Zerlegung
        \begin{equation}
            T: a=t_0 < \ldots < t_k = b
        \end{equation}
        existiert und $x$ auf $(t_l, t_{l+1})\ l\in\{0,\ldots,k-1\}$
        differentierbar ist.
    \item Besitzt eine Kurve $\K$ eine (stückweise) stetig differentierbare
        Parameterdarstellung $x(t), t\in [a,b]$ mit $\dot{x}(t) \neq \vec{0}$
        für $t \in [a,b]$ so heißt $\K$ stückweise glatt oder stückweise regulär.
    \item Ist eine Parameterdarstellung $x$ von $\K$ differentierbar und glatt,
        so heißt
        \begin{equation*}
            T(t) := \frac{\dot{x}(t)}{\norm{\dot{x}(t)}}
        \end{equation*}
        der Tangential (einheits) vektor von $x$ und $\K$
    \item Ist auch $T$ differentierbar und glatt (also $\dot{T}(k) \neq \vec{0}$)
        so heißt
        \begin{equation*}
            N(t) := \frac{\dot{T}(t)}{\norm{\dot{T}(t)}}
        \end{equation*}
        der (Haupt-) Normalen (einheits) vektor von $\K$ und $x$ bei $t$
    \item Und falls $n=3$
        \begin{equation*}
            B(t) = T(t) \times N(t)
        \end{equation*}
        der Binormalen (einheits) vektor von $\K$ und $x$ bei $t$
        (Man nennt dann $T(t), N(t), B(t)$ ein begleitendes Dreibein von $\K$)
    \item Existiert $T(t)$, so nennt man die Gerade
        \begin{equation*}
            \{x(t) + \lambda \dot{x}(t): \lambda \in \R\}
        \end{equation*}
        die Tangente von $\K$ bei $\lambda$
    \item Existiert auch $N(t)$ so nennt man die Ebene
        \begin{equation*}
            \{ x(t) + \lambda \dot{x}(t) + \mu \ddot{x}(t): \lambda, \mu \in \R\}
        \end{equation*}
        die Schmiegeebene von $\K$ bei $t$.
\end{enumerate}

\subsubsection{Bemerkung}
Sei $x(t) = y(\phi(t))$ mit $a\leq t \leq b$ zwei Parameterdarstellungen von $x$.
Dann gilt:
\begin{equation*}
    T(t) = \frac{\dot{x}(t)}{\norm{\dot{x}(t)}} = \frac{\dot{y}(\varphi(t)) \cdot \dot{\varphi}(t)}
    {\norm{\dot{y}(\phi(t)) \cdot \dot{\varphi}(t)}} = \frac{\dot{y}(\varphi(t))}
    {\norm{\dot{y}(\varphi(t))}}
\end{equation*}
Das heißt die Berechnung von $T$ ist unabhängig von der konkreten Parameterdarstellung

Existiert $N(t)$ dann gilt:
\begin{equation*}
    N(t) \bot T(t)
\end{equation*}
Existiert auch $B(t)$ (im $\R^3$), dann gilt:
$N(t), T(t), B(t)$ sind paarweise Orthogonal.

\subsection{Weitere Definitionen zu Kurven}
\begin{enumerate}[label= (\alph*)]
    \item Ist $\K: x(t), a \leq t \leq b$ eine Kurve, so heißt:
        \begin{equation*}
            - \K: y(t), a \leq t \leq b \text{ mit } y(t)=x(a+b-1)
        \end{equation*}
        die zu $\K$ entgegengesetzte Kurve
    \item Sind $\K: x(t), a \leq t \leq b$ und $\mathbb{L}: y(t), \alpha \leq
        t \leq \beta$ zwei Kurven und gilt $x(b) = y(\alpha)$ dann ist
        \begin{equation*}
            \K + \mathbb{L}: z(t), a \leq t \leq (\beta-\alpha) + b
        \end{equation*}
        und
        \begin{equation*}
            z(t) =
            \begin{cases}
                x(t) &, a \leq t \leq b\\
                y(t-b+\alpha) &, b \leq t \leq (\beta-\alpha) + b
            \end{cases}
        \end{equation*}
        die Aus $\K$ und $\mathbb{L}$ zusammengesetzte Kurve.
\end{enumerate}

\subsection{Kurventintegrale 2. Art}
Sei $\K$ eine Kurve im $\R^n$ und
\begin{equation*}
    f: T(\K) \rightarrow \R^n
\end{equation*}
\begin{enumerate}[label= (\alph*)]
    \item Sei $x: [a,b]  \rightarrow \R^n$ eine Parameterdarstellung von $\K$
        \begin{enumerate}[label = (\roman*)]
            \item Für eine Zerlegung $T: a = t_0 < \cdots < t_n = b$,
                Zwischenpunte $Z: (\xi_1, \ldots, \xi_n)$ mit
                $t_{k-1} \leq \xi_k \leq t_k$
                heißt
                \begin{equation*}
                    S(f,x,T,Z) := \sum_{k=1}^n f(x(\xi_k)) \cdot (x(t_k) - x(t_{k-1}))
                \end{equation*}
                die Riemann-Summe von $f, T, Z$ bezüglich $x$.
            \item Exitiert eine Zahl $I \in \R$ derart, dass für jede Folge von
                Zerlegungen $T_n$ mit
                \begin{equation*}
                    \lim_{n \rightarrow \infty} \mu(T_n) = 0
                \end{equation*}
                stets
                \begin{equation*}
                    \lim_{n \rightarrow \infty} S(f,x,T_n, Z_n) = I
                \end{equation*}
                folgt, so heißt $I$ das Kurvenintegral (2. Art) von
                $f$ längs $\K$ bzgl. $x$.
        \end{enumerate}
    \item Gibt es stets ein $I$ wie in (a) so heißt $f$ längs $\K$ (Riemann-)
        integrierbar und man nennt $I$ das (unbestimmte) Kurvenintegral von $f$
        längs $\K$ und schreibt:
        \begin{equation*}
            I = \int_\K f = \int_\K f(x) \cdot \dx = \int_\K f_1(x) \dx_1 + \cdots
            + f_n(x) \dx_n
        \end{equation*}
\end{enumerate}

\subsection{Substitutionsregel}
Ist $\K: x(t), a \leq t \leq b$ eine Kurve im $\R^n$ und $x(t)$ stückweise
differentierbar, sowie $f: T(\K) \rightarrow \R^n$ stetig, so ist $f$ längs
$\K$ integrierbar und es gilt:
\begin{equation*}
    \int_K f(x) \dx = \int_a^b f(x(t)) \text{d}x(t) = \int_a^b f(x(t)) \cdot
    \dot{x}(t) \dt
\end{equation*}

\subsection{Definition Wegunabhängigkeit}
Sei $f\in C(G, \R^n)$ mit $G \subseteq \R^n$ ein Gebiet:
\begin{enumerate}[label= (\alph*)]
    \item Gilt für zwei Wege $\K$ und $\mathbb{L}$ mit gleichem Anfangs- und
        Endpunkt stets
        \begin{equation*}
            \int_\K f = \int_{\mathbb{L}} f
        \end{equation*}
        dann heißen die Kurvenintegrale Wegunabängig in $G$.
    \item Eine Funktion $F \in C^1(G,\R)$ heißt Stammfunktion von $f$ in $G$, wenn
        \begin{equation*}
            \nabla F(x) = f(x)\ \forall x \in G
        \end{equation*}
        gilt.
    \item Man nennt
        \begin{equation*}
            P := -F
        \end{equation*}
        das Potential von $f$.
    \item Man nennt $f$ konservativ in $G$ oder ein Potentialfeld oder Gradienentenfeld
        in $G$, wenn $f$ eine Stammfunktion hat.
\end{enumerate}

\subsection{1. Hauptsatz für Kurvenintegral}
Sei $f$ konservativ in $G$ mit Stammfunktion $F$ und Potential $P$ dann gilt
für jeden Weg $\K$ in $G$ mit Anfangspunkt $A \in G$ und Endpunkt $B \in G$:
\begin{equation*}
    \int_\K f = F(B) - F(A) = P(A) - P(B)
\end{equation*}
insbesondere ist also das Integral wegunabhängig.

\subsection{Äquivalente Aussagen zu Stammfunktionen}
\begin{enumerate}[label= (\alph*)]
    \item
        \begin{equation*}
            \int_\K f \text{ ist wegunabhängig in } G
        \end{equation*}
    \item
        $f$ besitzt eine Stammfunktion
    \item
        \begin{equation*}
            \int_\K f = 0 \text{ für jede geschlossene Kurve } \K
        \end{equation*}
\end{enumerate}

\subsubsection{Bemerkung}
Rechenregeln für zwei Kurven $\K$ und $\mathbb{L}$:
\begin{enumerate}[label= (\alph*)]
    \item
        \begin{equation*}
            \int_{\K + \mathbb{L}} f = \int_\K f + \int_\mathbb{L} f
        \end{equation*}
    \item
        \begin{equation*}
            \int_{-\K} f = - \int_\K f
        \end{equation*}
\end{enumerate}

\subsection{Definition einfach zusammenhängende Gebiete}
Ein Gebiet $G \subseteq \R^n$ heißt einfach zusammenhängend, wenn sich jede
geschlossene Kurve in $G$ innerhalb von $G$ \glqq{}auf einen beliebigen
Punkt zusammenziehen lässt\grqq{}.

\subsection{Sternförmige Gebiete}
Eine Menge $G \subseteq \R^n$ heißt Sternförmig bezüglich $x_0 \in G$, wenn für
alle $x \in G$ gilt, dass $\overline{x_0x} \subseteq G$ (d.h.\ jedes $x$ ist von
$x_0$ durch einen Streckenzug erreichbar). $G$ ist ein sternförmiges Gebiet,
wenn $G$ offen und sternförmig ist.

\subsubsection{Bemerkung}
\begin{equation*}
    G \text{ sternförmig} \Rightarrow G \text{ einfach zusammenhängend}
\end{equation*}

\subsection{2. Hauptsatz für Kurvenintegrale}
Sei $f \in C^1(G, \R^n), G \subseteq \R^n$ ein Gebiet, dann gilt:
\begin{enumerate}[label= (\alph*)]
    \item Besitzt $f$ eine Stammmfunktion in $G$, so erfüllt $f$ in $G$ die
        Integrabilitätsbedingung:
        \begin{equation*}
            \frac{\partial f_l}{\partial x_k} = \frac{\partial f_k}{\partial x_l}\
            k,l \in \{ 1, \ldots, n\}
        \end{equation*}
        D.h.\ die Jacobi-Matrix von $f$ ist symetrisch.

        Kurz:
        \begin{equation*}
            f \text{ hat Stammfunktion} \Rightarrow f' = (f')^T
        \end{equation*}
    \item Ist $G$ einfach zusammenhängend und erfüllt $f$ die Integrabilitätsbedingung
        dann besitzt $f$ eine Stammfunktion.

        Kurz:
        \begin{equation*}
            G \text{ einfach zusammenhängend} \land f' = (f')^T \Rightarrow
                \exists F: \nabla F = f
        \end{equation*}
\end{enumerate}

\subsection{Definition Rotation}
Sei $G \subseteq \R^3$ offen und $f: G \rightarrow R^3$ partiell differentierbar,
dann heißt die Funktion $\rot f: G \rightarrow \R$ mit
\begin{equation*}
    \rot f(x) :=
    \begin{pmatrix}
        \dfrac{\partial f_3}{\partial x_2} - \dfrac{\partial f_2}{\partial x_3} \\
        \dfrac{\partial f_1}{\partial x_3} - \dfrac{\partial f_3}{\partial x_1} \\
        \dfrac{\partial f_2}{\partial x_1} - \dfrac{\partial f_1}{\partial x_2} \\
    \end{pmatrix}
\end{equation*}
die Rotation von $f$ in $G$.

\subsubsection{Bemerkung}
Im Fall $f: G \subseteq \R^2 \rightarrow \R^2$ definiert man
\begin{equation*}
    \rot f(x_1, x_2) = \frac{\partial f_2}{\partial x_1} - \frac{\partial f_1}{\partial x_2}
\end{equation*}
Formal betrachtet man die Hilfsfunktion
\begin{equation*}
    \tilde{f}(x,y,z) :=
    \begin{pmatrix}
        f_1(x,y)\\
        f_2(x,y)\\
        0
    \end{pmatrix}
\end{equation*}

\subsection{Zusammenhang Rotation und Integrabilitätsbedingung}
Ist $f \in C^1(G, \R^3), G$ ein Gebiet, dann gilt
\begin{enumerate}[label= (\alph*)]
    \item $f$ besitzt eine Stammfunktion $\Rightarrow \rot f = \vec{0}$
    \item $G$ einfach zusammenhängend und $\rot f = \vec{0} \Rightarrow f$
        hat Stammfunktion.
\end{enumerate}

\subsection{Definition Linienintegral/Kurvenintegral 1. Art}
Sei $\K: x(t), a \leq t \leq b$ ein Weg, und $x$ stückweise differentierbar.
Für ein $\phi \in  C(T(\R), \R)$ heißt
\begin{equation*}
    \int_\K \phi \ds := \int_a^b \phi(x(t)) \norm{ \dot{x}(t)} \dt
\end{equation*}
ein Linienintegral oder Kurvenintegral 1. Art von $\phi$ längs $\K$.

\subsubsection{Bemerkung}
\begin{enumerate}[label= (\alph*)]
    \item
        Mit $\phi \equiv 1$:
        \begin{equation*}
            \int_\K 1 \ds = \int_a^b \phi(x,t) \norm{\dot{x}(t)} \dt
            \int_a^b  \norm{\dot{x}(t)} \dt = l(\K)
        \end{equation*}
        d.h.\ mit Linienintegralen können auch Weglängen berechnet werden,
        bzw. Weglängen berechnet man mit $\phi = 1$.
    \item
        $\phi: [a,b] \rightarrow \R$ wähle $\K: x(t) = a + t \cdot (b-a)\
        t \in [0,1]$:
        \begin{equation*}
            \int_\K \phi \ds = \int_0^1 \phi (a + t \cdot (b-a)) \norm{b-a} \dt
            = \int_a^b \phi(t) \dt
        \end{equation*}
    \item Linienintegrale hängen nicht von der Parameterdarstellung ab.
    \item Man schreibt (falls Parameter-Darstellung bekannt ist) oft
        \begin{equation*}
            \ds = \norm{\dot{x}(t)} \dt
        \end{equation*}
        und nennt $\ds$ Bogensegment oder Liniensegment.
    \item Ist $f \in C(T(\K), \R^n)$ und $\dot{x}(t) \neq \vec{0}\ \forall
        t \in [a,b]$, dann ist:
        \begin{eqnarray*}
            \int_\K f &=& \int_\K f(x) \dx = \int_a^b f(x(t)) \dot{x}(t) \dt \\
            &=& \int_a^b \frac{f(x(t)) \dot{x}(t)}{\norm{\dot{x}(t)}} \norm{\dot{x}(t)}
            \dt \\ &=& \int_a^b f(x(t)) \cdot T(t) \norm{\dot{x}(t)} \dt  \\ &=&
            \int_\K \phi \ds \text{ mit } \phi(t) = f(x(t)) \cdot T(t)
        \end{eqnarray*}
\end{enumerate}

\section{Bereichsintegrale}
Hier: $f: G \subseteq \R^n \rightarrow \R$ und
\begin{equation*}
    \int_G f = \int_G f(x_1, \ldots, x_n) \intd{(x_1, \ldots x_n)}
\end{equation*}
sollen anschaulich bedeuten:

Welches Volumen schließt der Graph von $f$ mit der Grundfläche $G$ ein.

\subsection{Intervalle im $\R^n$}
Für $a,b \in \R^n$ bezeichnet die Menge
\begin{equation*}
    [a,b] := [a_1, b_1] \times \cdots \times [a_n, b_n]
\end{equation*}
einen (kompakten) Quader oder (kompaktes) Intervall im $\R^n$. Die Zahl
\begin{equation*}
    V([a,b]) =
    \begin{cases}
        \prod_{k=1}^n (b_k - a_k) &, \text{falls }  b_k > a_k \text{ für } k=1,\ldots\\
        0 &,\text{sonst}
    \end{cases}
\end{equation*}
bezeichnet das Volumen, und die Zahlen $b_1 - a_1, \ldots b_n - a_n$ als
Kantenlängen.

\subsection{Definition Zerlegung}
Ist $[a,b] = [a_1, b_1] \times \cdots \times [a_n, b_n]$ und ist für jedes
$k \in \{1, \ldots, n\}$ mit
\begin{equation*}
    T^{(k)} : a_k = x_0 < \cdots < x_{l_k} = b_k
\end{equation*}
eine Zerlegung von $[a_k, b_k]$ dann heißt die Menge
\begin{equation*}
    I_{l_1, \ldots, l_n} = [x_{l_1 - 1}^{(1)} - x_{l_1}^{(1)}] \times \cdots \times
    [x_{l_1 - 1}^{(n)} - x_{l_1}^{(n)}]
\end{equation*}
mit $l_k \in \{1, \ldots, l_k\}$ für $k \in \{1, \ldots, n\}$ eine Zerlegung $T$
von $[a,b]$.

Das Feinheitsmaß von $T$ ist
\begin{equation*}
    \mu(T) = \max_{l_1, \ldots, l_n} V(I_{l_1, \ldots, l_n})
\end{equation*}

Allgemein ist ein Intervall von der Form
\begin{equation*}
    [x_i^{(1)}, x_{i+1}^{(1)}] \times [x_j^{(2)}, x_{j+1}^{(2)}]
\end{equation*}
mit $i \in \{0, \ldots, l_1-1\}$ und $j \in \{0, \ldots l_2-1\}$.

\subsection{Definition Riemann-Summe}
Sei $T$ eine Zerlegung eines kompakten Quaders $I \subseteq \R^n$ mit Teilquadern
$I_1, \ldots, I_l$ mit $l=l_1 \cdot \cdots \cdot l_n$ (entstehen indem man die
Zerlegungsintervalle fortlaufend durchnummeriert) und Zwischenpunkte
$\xi = (\xi_1, \ldots, \xi_l)$ mit $\xi_i \in I_i (i \in \{l, \ldots, n\})$
und $f: I \to \R$ (d.h. Skalarwertige Funktion). Dann heißt
\begin{equation*}
    S(f, T, \xi) = \sum_{i=1}^l f(\xi_i) f(\xi_i) \mu(I_1)
\end{equation*}
die Riemann-Summe von $f$ bezüglich $T$ und $\xi$.

\subsection{Riemann integrierbare Bereichsintegrale}
Sei $f: I \to \R$ eine Funktion, $I \subseteq \R^n$ ein Quader. Gibt es eine Zahl
$\alpha \in \R$, so dass für jede Folge von Zerlegungen ${(T_k)}_{k=1}^\infty$ mit
Zwischenpunkten ${(\xi_k)}_{k=1}^\infty$ mit $\lim\limits_{k \to \infty} \mu(T_k)=0$
die Riemann-Summe $S(f, T_k, \xi_k)$ gegen $\alpha$ konvergiert für $k \to \infty$
dann heißt $f$ Riemann integrierbar über $I$ und $\alpha$ nennen wir das
Bereichsintegral von $f$ über $I$.

\subsubsection{Schreibweise}
\begin{equation*}
    \alpha = \int_I f(x) \dx
\end{equation*}
Zur Schreibweise: z.B. $n=2$ auch:
\begin{equation*}
    \alpha = \iint_I f(x,y) \intd{(x,y)} := \int_I f(x,y) \intd{(x,y)}
\end{equation*}
oder Angabe von $I$ an dem Integral:
\begin{equation*}
    \alpha = \int_{[a_1, b_1]\times[a_2,b_2]} f(x,y) \intd{(x,y)}
\end{equation*}

\subsection{Bereichsintegrale über beschränkte Mengen}
Sei $M \subseteq \R^n$ beschränkt und $I = [a_1, b_1] \times \cdots \times [a_n,
b_n]$ ein Quader mit $M \subseteq I$. Dann heißt $f: M \to \R$ über $M$ integrierbar
wenn die Funktion
\begin{equation*}
    \tilde{f}: I \to \R \text{ mit } \tilde{f}(x) =
    \begin{cases}
        f(x) &, x\in M\\
        0 &, \text{ sonst}
    \end{cases}
\end{equation*}
über $I$ Bereichs-Riemann integrierbar ist. Wir definieren:
\begin{equation*}
    \int_M f(x) \dx = \int_I \tilde{f}(x) \dx
\end{equation*}

\subsection{Cavalieri}
Sei $M \subseteq \R^n (n >1)$ und bezeichne
\begin{equation*}
    M' = \{ x \in \R: {(x,y)}^T \in M \text{ für ein } y \in \R^{n-1} \}
\end{equation*}
und für $x \in M'$
\begin{equation*}
    M(x) = \{ y \in \R^{n-1}: {(x,y)}^ \in M \}
\end{equation*}
dann gilt für $f \in C(\bar{M})$ (falls $M, M', M(x)$ sogenannte messbare Mengen
sind, d.h $\mu(M), \mu{M'}, \mu{M(x)}$ sind definiert)
\begin{equation*}
    \int_M f(x,y) \intd{(x,y)} = \int_{M'} \left[ \int_{M(x)} f(x,y) \dy \right]
        \dx
\end{equation*}
mit $x \in \R$ und $y \in \R^{n-1}$.

\subsection{Fubini}
Im Fall $n=2$ steht nach Cavalieri ein Parameterintegral und mit Fubini gilt:
\begin{equation*}
    \int_{M'} \int_{M(x)} f(x,y) \dy \dx =
    \int_{\tilde{M}'} \int{\tilde{M}(y)} f(x,y) \dx \dy
\end{equation*}
wobei $\tilde{M}', \tilde{M}(y)$ analog zu $M', M(x)$ bezüglich $y$ definiert
sind.

\subsection{Definition Meßbare-Mengen}
Eine beschränkte Menge $M \subseteq \R^n$ heißt (Jordan-) meßbar, wenn
\begin{equation*}
    \int_M 1 \dx
\end{equation*}
existiert, in diesem Fall nennt man
\begin{equation*}
    \mu(M) := \int_M 1 \dx
\end{equation*}
das Volumen von $M$. Ist $\mu{M} =0$, so nenntn man $M$ eine Nullmenge.

\subsection{Definition $2 \times 2$ Determinante}
Für
\begin{equation*}
    A =
    \begin{pmatrix}
        a & c \\
        b & d \\
    \end{pmatrix}
    \in \R^{2 \times 2}
\end{equation*}
definieren wir die Funktion
\begin{equation*}
    \det: \R^{2 \times 2} \to \R
\end{equation*}
durch $A \mapsto \det(A) = a \cdot d -  c \cdot b$ und nennen die
Funktionsauswertung die Determinante von $A$.

\subsection{Mehrdimensionale Substitutonsregel}
Sei $M \subseteq \R^n$ meßbar und $G \supseteq M$ ein Gebiet. Ist $T\in C^1(G, \R^m)$
und gilt $\det(T'(x)) \neq 0\ \forall x \in M \setminus N$ für eine Nullmenge $N$,
dann gilt:
\begin{equation*}
    \int_{T(M)} f(x_1, \ldots, x_n) \intd{(x_1, \ldots, x_n)} =
    \int_M f(T(u_1, \ldots, u_n)) \cdot \abs{\det(T'(u_1, \ldots, u_n))} \intd{(u_1, \ldots
    u_n)}
\end{equation*}

\section{Integralsätze in der Ebene}
\subsection{Positiv berandete Menge}
Eine beschränkte Menge $B \subseteq \R^2$ heißt positiv berandet durch einen Weg,
(Randkurve) $\K$, wenn $T(\K) = \partial B$ ist und wenn $\K$ eine stückweise
stetig differentierbare Parameterdarstellung $x: [a,b] \to \R^2$ hat mit:
\begin{enumerate}[label = (\roman*)]
    \item $\dot{x}(t) \neq 0$ für fast alle $t \in [a,b]$
    \item der Normalenvektor von $x(t)$ zeigt nach außen
\end{enumerate}

\subsection{Satz von Green}
Ist $B \subseteq \R^2$ positiv berandet, dann gilt für alle $f \in C^1(B, \R^2)$
\begin{equation*}
    \iint_B \frac{\partial f_2}{\partial x} - \frac{\partial f_1}{\partial y}
    \intd{(x,y)} =
    \int_{\partial B} f(x,y) \intd{(x,y)}
\end{equation*}

\subsection{Definition Normalbereiche}
Eine Menge $B \subseteq \R^2$ heißt Normalbereich bezüglich der $x$-Achse
(bzw. $y$-Achse), wenn es ein Intervall $[a,b]$ gibt und die Funktion
$\varphi, \psi$ mit
\begin{equation*}
    B = \{ {(x,y)}^T: a \leq x \leq b, \phi(x) \leq y \leq \psi(x) \}
\end{equation*}

\subsection{Gauß'sche Integralsätze in der Ebene}
Sei $B \subseteq \R^2$ ein positiv berandeter Bereich und $f \in C^1(B, \R^2)$
bzw. $f \in C^2(B, \R^2)$ und bezeichne $\nu$ die nach außen gerichtete Normale
auf $\partial B$. Dann gelten die Integralsätze:
\begin{enumerate}[label = (\roman*)]
    \item
        \begin{equation*}
            \iint_B (\vdiv f)(x,y) \intd{(x,y)} = \int_{\partial B} f \cdot \nu \ds
        \end{equation*}
    \item
        \begin{equation*}
            \iint_B f_1(x,y) \Delta f_2(x,y) - f_2(x,y) \Delta f_1(x,y) \intd{(x,y)}
            = \int_{\partial B} f_1 \frac{\partial f_2}{\partial \nu} - f_2
            \frac{\partial f_1}{\partial \nu}
        \end{equation*}
\end{enumerate}

\section{Oberflächenintegrale und Integralsätze im $\R^3$}
\subsection{Definition Reguläre Flächen}
Sei $B \subseteq \R^2$ und $x: B \to \R^3$,
\begin{equation*}
    x(u,v) =
    \begin{pmatrix}
        x_1(u,v)\\
        x_2(u,v)\\
        x_3(u,v)
    \end{pmatrix}
\end{equation*}
eine stetig diffbare Funktion, für die $x_u = \frac{\partial x}{\partial u}$ und
$x_v = \frac{\partial x}{\partial v}$ linear unabhängig sind (d.h.\ die Vektoren
$x_u$ und $x_v$ zeigen nicht in die gleiche oder entgegengesetzte Richtung) (für
fast alle ${(u,v)}^T \in B$) die Menge der Ausnahmen muss $\tilde{B} \subseteq B$
muss $\mu{\tilde{B}} = 0$ erfüllen.

Das Bild einer solchen Funktion, d.h.\ die Menge
\begin{equation*}
    A = x(B) := \{ x(u,v) \vert {(u,v)}^T \in B \}
\end{equation*}
heißt dann eine reguläre Fläche im $\R^3$ und die Funktion $x$ heißt die
Parametrisierung von $A$.

Man nennt
\begin{enumerate}[label = (\roman*)]
    \item $x_u(u,v), x_v(u,v)$ die Tangentialvektoren in ${(u,v)}^T$
    \item $n(u,v) := \frac{(x_u \times x_v)(u,v)}{\norm{(x_u \times x_v)}(u,v)}$
        \begin{itemize}
            \item Vektor mit Länge $1$ der Senkrecht auf den Tangentialvektoren
                steht
            \item Rechnerisch zu enthalten durch das Kreuzprodukt der
                Tangentialvektoren
        \end{itemize}
\end{enumerate}
der (Flächen-) Normalenvektor in ${(u,v)}^T$ falls $x_u(u,v)$ und $x_v(u,v)$
linear unabhängig sind.

Ist $B$ positiv berandet durch $\K: y(t), a \leq t \leq b$ so nennt man $A$ positiv
berandet durch Kurve mit Parameterdarstellung $x(y(t)), a \leq t \leq b$.

\subsection{Defintion Oberflächenintegral}
Sei $A$ eine reguläre Fläche im $\R^3$ mit Parameterdarstellung $x: B \to \R^3,
B \subseteq \R^2$ meßbar und $x$ injektiv auf $B \setminus N$ für eine Nullmenge
$N$.
\begin{enumerate}[label = (\alph*)]
    \item Für jedes $f \in C(A, \R)$ heißt
        \begin{equation*}
            \iint_A f \cdot \ido= \iint_B f(x(u,v)) \cdot \norm{(x_u \times x_v)
            (u,v)} \intd{(u,v)}
        \end{equation*}
        das Oberflächenintegral von $f$ über $A$ und man nennt
        \begin{equation*}
            \ido= \norm{(x_u \times x_v)
            (u,v)} \intd{(u,v)}
        \end{equation*}
        das Oberflächenelement.
    \item $O(A) := \iint_A 1 \do$ heißt Oberflächeninhalt von $A$.
\end{enumerate}

\subsubsection{Bemerkung}
\begin{enumerate}
    \item Das Oberflächenintegral hängt nicht von der Parameterdarstellung ab.
    \item Ein Summand des Obefĺöchenintegrals sieht so aus:
        \begin{equation*}
            f(x(u,v)) \cdot \norm{(x_u \times x_v)(u,v)} \cdot \Delta u \Delta v
        \end{equation*}
\end{enumerate}

\subsection{Satz von Stokes}
Sei $A$ eine reguläre Fläche im $\R^3$ und $\partial A$ positiv berandet. Dann
gilt für $f \in C^1(A, \R^3)$
\begin{equation*}
    \iint_A \rot f \cdot n \ido= \int_{\partial A} f
\end{equation*}
Mit $n$:
\begin{enumerate}[label = (\roman*)]
    \item Normalenvektor
    \item Länge 1
    \item Senkrecht auf Fläche
    \item Immer auf der gleichen Seite von $A$
\end{enumerate}
also
\begin{equation*}
    \iint_B \rot(f(x(u,v))) \cdot n(x(u,v)) \cdot \norm{(x_u \times x_v)(u,v)}
    \intd{(u,v)} = \int_{\partial A} f
\end{equation*}
mit
\begin{equation*}
    n(x(u,v)) = \pm \frac{(x_u \times x_v)(u,v)}{\norm{(x_u \times x_v)(u,v)}}
\end{equation*}

\subsection{vdivgenzsatz von Gauß}
Sei $M \subseteq \R^3$ kompakt und $\partial M$ ergebe sich als endliche
Vereinigung von regulären Flächen, deren Normale $n$ (normiert) nach Außen
zeigt. Dann gilt fpr jedes $f \in C^1(M, \R^3)$
\begin{equation*}
    \iiint_M \div f = \iint_{\partial M} f \cdot n \ido
\end{equation*}


    \chapter{Lineare Algebra}
    \section{Der Begriff Vektorraum}
\subsection{Definition Vektorraum}
Gegeben sei eine abelsche Gruppe $V$ und ein Körper $\K$ (bei uns wird $\K=\R$
oder $\K=\C$ gelten) und eine Abbildung:
\begin{equation*}
    \cdot: \K \times V \to V, \cdot(\alpha, x) \mapsto \alpha \cdot x =:
    \alpha x \text{ (Skalierung)}
\end{equation*}
Dann nennt man $V$ einen Vektorraum über $\K$, wenn die folgenden Vektorraumaxiome
erfüllt sind:
\begin{eqnarray*}
    &\text{(V1)}& \alpha \cdot (\beta \cdot v) = (\alpha \cdot \beta) \cdot v
    \text{ (Assoziativgesetz)}\\
    &\text{(V2)}& \alpha \cdot (x+y) = (\alpha \cdot x) + (\alpha \cdot y) =
    \alpha x + \alpha y \\
    && (\alpha + \beta) \cdot x = \alpha x + \beta x
    \text{ (Distributivgesetzte)}\\
    &\text{(V3)}& 1 \cdot x = x \text{ für die } 1 \in \K \text{ (Gesetz der Eins)}
\end{eqnarray*}

In einem Vektorraum $V$ über $\K$ nennt man Elemente aus $V$ Vektoren, die Elemente
aus $\K$ Skalare, $\K$ den Skalarkörper und \glqq{}$\cdot$\grqq{} die
Multiplikation mit Skalaren. Die \glqq{}$+$\grqq{} Verknüpfung in $V$ die $V$ die
Vektoraddition und das neutrale Element $\vec{0} \in V$ den Nullvektor.

\subsection{Rechenregeln}
Ist $V$ ein Vektorraum über $\K$, so gilt für $\alpha, \beta \in \K$ und
$x, y \in V$:
\begin{enumerate}
    \item
        \begin{enumerate}[label = (\alph*)]
            \item $0 \cdot x = \vec{0} = \alpha \vec{0}$
            \item $\alpha \cdot x = \vec{0} \Rightarrow \alpha = 0 \lor x=\vec{0}$
        \end{enumerate}
     \item
        \begin{equation*}
            \alpha (-x) = (-\alpha) x = - (\alpha x)
        \end{equation*}
\end{enumerate}

\section{Unterräume}
\subsection{Definition Unterraum}
Eine Teilmenge $U$ eines Vektorraums $V$ über $\K$ heißt Unterraum von $V$, wenn
$U$ bezüglich der in $V$ definierten Vektoraddition und Skalierung ein Vektorraum
ist.

\subsection{Unterraumkriterien}
Für $U \subseteq V$ und $U \neq \emptyset$ sind folgende Aussagen äquivalent
\begin{enumerate}[label = (\alph*)]
    \item $U$ ist ein Unterraum von $V$
    \item
        \begin{equation*}
            x,y \in U, \alpha, \beta \in \K \Rightarrow
            \alpha x + \beta y \in U
        \end{equation*}
    \item
        \begin{equation*}
            (x,y \in U \Rightarrow x+y \in U) \land
            (\alpha \in K, x \in U \Rightarrow \alpha x \in U)
        \end{equation*}
\end{enumerate}

\subsection{Durchschnitt von Unterräumen}
Der Durchschnitt von Unterräumen ist wieder ein Unterraum, d.h.:
\begin{equation*}
    U_i\ i \in J\ (J\text{ eine Indexmenge}) \text{ sind Unterräume}
    \Rightarrow \bigcap_{i \in J} U_i \text{ ist Unterraum}
\end{equation*}

\subsection{Definition lineare Hülle}
\begin{itemize}
    \item Ist $M$ eine beliebige Teilemenge eines Vektorraums. Dann heißt
        \begin{equation*}
            \vspan (M) := \bigcap_{U \in S} U \text{ mit }
            S := \{ U \subseteq V: U \text{ ist Unterraum}, U \supseteq M \}
        \end{equation*}
        der von $M$ aufgespannte Unterraum oder die lineare Hülle von M.
    \item Ist $U$ ein Unterraum und $M \subseteq V$ mit $\vspan(M) = U$, dann heißt
        $M$ ein erzeugendes System von $U$.
\end{itemize}

\subsubsection{Bemerkung}
\begin{enumerate}
    \item $\vspan(M)$ ist der kleinste Unterraum, der $M$ enthält
    \item $\vspan(\emptyset) = \vec{0}$
    \item $M \subseteq N \Rightarrow \vspan(M) \subseteq \vspan(N)$
    \item Ist $U$ ein Unterraum, dann gilt $U=\vspan(U)=\vspan(U \setminus \{\vec{0}\})$
\end{enumerate}

\subsection{Definition Linearkombination}
Ist $V$ ein Vektorraum über $\K$ und $x_1, \ldots x_n \in V, \alpha_1, \ldots,
\alpha_n \in \K$ dann heißt
\begin{equation*}
    \sum_{k=1}^n a_k x_k \in V
\end{equation*}
eine Linearkombination von $x_1, \ldots, x_n$ (mit Koeffizienten $\alpha_1, \ldots
\alpha_n$).

\subsection{Zusammenhang lineare Hülle --- Linearkombination}
Sei $V$ ein Vektorraum über $\K$ und $M \subseteq V$, dann gilt $\vspan{M}$ ist
die Menge aller Linearkombinationen, d.h.
\begin{equation*}
    \vspan(M) = \{\alpha_1 x_1 + \cdots + \alpha_n x_n \vert x \in \N,
    x_1, \ldots, x_n \in M, \alpha_1, \ldots, \alpha_n \in \K \}
\end{equation*}
im Fall $M=\{x_1, \ldots, x_n\}$ gilt:
\begin{equation*}
    \vspan(M) =  \{\alpha_1 x_1 + \ldots \alpha_n x_n \vert \alpha_1, \ldots
    \alpha_n \in \K \}
\end{equation*}

\section{Lineare Unabhängigkeit}
\subsection{Definition Lineare Unabhängigkeit}
Sei $V$ ein Vektorraum über $\K$
\begin{enumerate}[label = (\alph*)]
    \item Eine endliche Liste $a_1, \ldots, a_n \in V$ heißt linear unabhängig
        (l.u.), wenn gilt
        \begin{equation*}
            \alpha_1 a_1 + \cdots + \alpha_n a_n = \vec{0}
            \Rightarrow \alpha_1 = \cdots = \alpha_n  = 0
        \end{equation*}
        Andernfalls heißen $a_1, \ldots, a_n$ linear abhängig (l.a.).
    \item Eine beliebige Teilmenge $M \subseteq V$ heißt linear unabhängig, wenn
        für eine beliebige endliche Liste $a_1, \ldots, a_n \in M$ gilt, dass
        diese linear unabhängig sind. Andernfalls ist $M$ linear abhängig.
\end{enumerate}

\subsection{Rechenregeln für lineare Unabhängigkeit}
Für Vektoren $a, a_1, \ldots, a_n, b_1, \ldots, b_n$ eines Vektorraumes $V$ gilt:
\begin{enumerate}[label = (\alph*)]
        \item
            \begin{equation*}
                a \text{l.u.} \Leftrightarrow \{a\} \text{ l.u.} \Leftrightarrow
                a \neq \vec{0}
            \end{equation*}
            Bemerkung:

            $a_1, a_2$ mit $a_1 = a_2$ ist linear unabhängig, aber $M=\{a, a\} =
            \{a\}$ ist nur dann linear abhängig wenn $a = \vec{0}$.
        \item
            $a_1, \ldots, a_n$ linear abhängig $\Rightarrow$ $a_1, \ldots a_n,
            b_1, \ldots, b_k$ sind linear abhängig für $k \geq 0$.
        \item $a_1, \ldots, a_n$ linear unabhängig $\Rightarrow a_1, \ldots, a_k$
            linear unabhängig für $k \leq n$
        \item $a_1, \ldots, a_n$ linear unabhängig $\Rightarrow a_1, \ldots a_n$
            sind paarweise verschieden
        \item Für $n \geq 2$ sind $a_1, \ldots, a_n$ genau dann linear abhängig,
            wenn ein Vektor als Linearkombination darstellbar ist. D.h.:
            \begin{equation*}
                \exists i \in \{1, \ldots, n\}: a_i = \sum_{k=1, k \neq i}^n
                \alpha_k a_k \text{ für } \alpha_1, \ldots, \alpha_{i-1},
                \alpha_{i+1}, \ldots, \alpha_n \in \K
            \end{equation*}
        \item Sind $a_1, \ldots, a_n$ linear unabhängig und $a_1, \ldots, a_n, a$
            linear abhängig, so ist $a$ die linear Kombination von $a_1, \ldots,
            a_n$ und die Koeffizienten sind eindeutig.
        \item Ist $a$ eine Linearkombination von $a_1, \ldots, a_n$ und jeder
            Vektor $a_k$ eine Linearkombination von $b_1, \ldots, b_m$ so ist
            $a$ eine Linearkombination von $b_1, \ldots, b_m$
\end{enumerate}

\subsubsection{Bemerkung}
Für Teilmengen $M, N$ eines Vektorraums $V$ gilt:
\begin{enumerate}[label= (\alph*)]
    \item $M$ l.a. $M \subseteq N \Rightarrow N$ l.a.
    \item $M=\emptyset \Rightarrow M$ l.u.
    \item $\vec{0} \in M \Rightarrow M$ l.a.
\end{enumerate}

\paragraph{Für $V = \R^3$}
\begin{enumerate}[label= (\alph*)]
    \item $a_1, a_2, a_3$ seien linear abhängig und $a_1, a_2$ linear unabhängig

        $\Leftrightarrow$ also $a_3$ ist in der von $a_1$ und $a_2$ aufgespannten
        Ebene

        $\Rightarrow$ Spat mit Kanten $a_1, a_2, a_3$ hat Volumen $0$

        $\Leftrightarrow \det(a_1, a_2, a_3) = 0$
    \item $a_1, a_2$ linear abhängig $\Rightarrow a_2$ ist auf der von $a_1$
        aufgespannten Gerade

        $\Rightarrow \det(a_1, a_2) = 0$
\end{enumerate}

\section{Basis und Dimension}
\subsection{Definition Hamel-Basis}
\begin{enumerate}[label= (\alph*)]
    \item Eine Teilmenge $B$ eines Vektorraums $V$ heißt (Hamel-) Basis von $V$,
        wenn gilt
        \begin{enumerate}[label= (\roman*)]
            \item $B$ ist linear unabhängig
            \item $V = \vspan(B)$
        \end{enumerate}
        Kurz:
        \begin{equation*}
            B \text{ ist ein linear unabhängiges Erzeuger-System
            von }V
        \end{equation*}
    \item Man sagt Vektoren $b_1, \ldots, b_n$ bilden eine Basis von $V$, wenn
        gilt $B=\{b_1, \ldots, b_n\}$ ist eine Basis von $V$.
\end{enumerate}


    \part{HM 3 --- Zusammenfassung}
	\chapter{Exkurs Funktionanlanalysis}
    %!TEX root = ../main.tex
\section{Normen und innere Produkte}
\subsection{Definition Vektornorm}
Sei $V$ ein Vektorraum über $\K$ eine Abbildung
\begin{equation*}
	\norm{\cdot}: V \rightarrow \R
\end{equation*}
nennt man eine Norm, wenn für $\lambda \in \K, x, y \in V$  stets gilt:
\begin{enumerate}[label= (\alph*)]
	\item 
		\begin{equation*}
			\norm{x} \geq 0\ \forall x \in V\ \land\ \norm{x} = 0 \Leftrightarrow x = \vec{0}
		\end{equation*}
	\item 
		\begin{equation*}
			\norm{\lambda \cdot x} = \abs{\lambda} \cdot \norm{x}
		\end{equation*}
	\item
		\begin{equation*}
			\norm{x + y } \leq \norm{x} + \norm{y}
		\end{equation*}
\end{enumerate}
In dem Fall heißt heißt $V$ normierter Vektorraum. Einen Vektor $a \in V$ nennen wir normiert, wenn $\norm{a} = 1$ ist.
\subsubsection{Bemerkung}
\begin{enumerate}
	\item In einem normierten Vektorraum, kann man jedes $a \in V \backslash \{\vec{0}\}$ durch
		\begin{equation*}
			\tilde{a} = \frac{1}{\norm{a}} \cdot a
		\end{equation*}
		normieren, dann
		\begin{equation*}
            \norm{\tilde{a}} = \norm{\frac{1}{\norm{a}} \cdot a} = \abs{\frac{1}{\norm{a}}} \norm{a} = \frac{1}{\norm{a}} \norm{a} = 1
		\end{equation*}
	\item Es gilt in einem normierten Vektorraum auch, dass
		\begin{equation*}
			\abs{\norm{x} - \norm{y}} \leq \norm{x - y}\ \forall x,y \in V
		\end{equation*}
		Bew.: folgt aus (N3) (vgl. HM1)
\end{enumerate}

\subsection{Skalarprodukt / inneres Produkt} Sei $V$ ein Vektorraum über $K =
\C$ oder $K=\R$. Eine Abbildung: \begin{equation*}     < \cdot, \cdot >: V
\times V \rightarrow \K, (x,y) \mapsto <x,y> \end{equation*}  heißt inneres
Produkt oder Skalarprodukt in $V$, wenn für $\lambda \in \K, x,y \in V$ stets gilt:
\begin{enumerate}
	\item
		\begin{equation*}
			<x,x> \geq 0\ \land\ <x,x> = 0 \Leftrightarrow x = \vec{0}
		\end{equation*}
	\item Homogenität:
		\begin{equation*}
			<x, \lambda y> = \lambda <x, y>
		\end{equation*}
	\item Additivität:
		\begin{equation*}
			<x, y+z> = <x,y> + <x,z>
		\end{equation*}
	\item
		\begin{equation*}
			<x,y> = \overline{<y,x>}
		\end{equation*}
\end{enumerate}
Ist $<\cdot,\cdot>$ ein inneres Produkt von $V$, dann heißt $V$ innerer Produktraum
und genauer:
\begin{enumerate}[label= (\alph*)]
	\item euklidischer Raum für $V = \R^n$
	\item unitärer Raum für $V = \C^n$
\end{enumerate}
\subsubsection{Bemerkung}
In (S1) wird implizit verlangt, dass $<x,x> \in \R$ gilt.

\subsection{Definition induzierte Norm}
Ist $V$ ein innerer Produktraum bezüglich $<\cdot,\cdot>$ so heißt die durch
$\norm{x}:=\sqrt{<x,y>}$ definierte Abbildung
\begin{equation*}
	\norm{\cdot}: V \rightarrow \R
\end{equation*}
die induzierte Norm.
\subsubsection{Bemerkung}
Dass die induzierte Norm eine Norm ist, ist noch zu zeigen.

\subsection{Äquivalente Aussagen zu induzierten Normen}
Sei $V$ ein Vektorraum über $\K$ und $\norm{\cdot}$ eine Norm, dann sind folgende
Aussagen äquivalent:
\begin{enumerate}
	\item Die Norm wird durch ein Skalarprodukt induziert
	\item Es gilt die sogennante Parallelogram Identität:
		\begin{equation*}
			{\norm{x+y}}^2 + {\norm{x-y}}^2 = 2(\norm{x}^2 + \norm{y}^2)
		\end{equation*}
\end{enumerate}

\subsection{Rechenregeln für Skalarprodukte}
Seien $x,y \in V, \lambda \in \K$:
\begin{enumerate}
	\item
		\begin{equation*}
			<\lambda x, y>  = \overline{\lambda} <x,y>
		\end{equation*} 
	\item 
		\begin{equation*}
			<x+y, z> = <x,z> + <y,z>
		\end{equation*}
	\item
		\begin{equation*}
			<\vec{0}, y> = <x, \vec{0}> = 0
		\end{equation*}
\end{enumerate}

\subsection{Cauchy-Schwarz'sche-Ungleichung}
Sei $V$ ein Vektorraum mit Skalarprodukt $<\cdot, \cdot>$ und induzierter Norm
$\norm{\cdot}$ dann gilt $<x,y>\leq\norm{x}\norm{y}$

\section{Orthogonalität}
\subsection{Orthogonaltität}
Sei $V$ ein innerer Produktraum, dann nennt man
\begin{enumerate}[label= (\alph*)]
	\item Zwei Vektoren $x,y \in V$ orthogonal (schreibweise $x \perp y$)
		\begin{equation*}
			:\Leftrightarrow <x,y> = 0
		\end{equation*}
	\item $M \subseteq V$ ein Orthogonalsystem (OGS) in $V$, wenn gilt:
		\begin{equation*}
			x,y \in M \land x \neq y \Rightarrow x \perp y
		\end{equation*}
	\item $M \subseteq V$ ein Orthonormalsystem (ONS) in $V$, wenn $M$ ein OGS ist
		und gilt
		\begin{equation*}
			x \in M \Rightarrow \norm{x} = 1
		\end{equation*}
	\item $B \subseteq V$ eine Orthogonalbasis von $V$ wenn $B$ eine Basis von 
		$V$ ist und $B$ ein OGS ist.
	\item $B \subseteq V$ eine Orthonormalbasis von $V$ wenn $B$ eine Basis von
		$V$ ist und $B$ ein ONS ist.
\end{enumerate}

\subsection{Orthogonalität und lineare Abhängigkeit}
Sei $V$ ein innerer Produktraum. Dann gilt:
\begin{enumerate}[label= (\alph*)]
	\item Seien $b_1, \ldots, b_n \in V$ orthogonal und alle $\neq \vec{0} \Rightarrow$
		$b_1, \ldots, b_n$ sind linear unabhängig.
	\item Ist $B = \{b_1, \ldots, b_n\}$ ein OGS und $x \in \vspan{B}$, dann gilt
		\begin{equation*}
			x = \alpha_1 b_1 + \ldots + \alpha_n b_n
		\end{equation*}
		mit
		\begin{equation*}
			\alpha_k = \frac{<b_k, x>}{<b_k, b_k>}
		\end{equation*}
\end{enumerate}

\subsection{Gram-Schmidt (Orthogonalisierung von Vektoren)}
Seien $x_1, \ldots, x_n$ linear unabhängige Vektoren eines inneren Produktraums.
Definiert man:
\begin{eqnarray*}
	y_1 &:=& x_1 \\
	y_k &:=& x_k - \sum_{l=1}^{k-1} \frac{<y_l, x_k>}{<y_l, y_l>} y_l\  \text{für}\ 
	k\in\{2, \ldots, n\}
\end{eqnarray*}
dann gilt: $\vspan{(x_1, \ldots, x_n)} = \vspan{(y_1, \ldots, y_n)}$ und $y_l \perp y_k$
für $l \neq k$.

\subsection{Orthogonale Projektion}
Sei $V$ ein innerer Produktraum und $x_1, \ldots, x_n$ linear unabhängig. Dann gilt
für $x \in V$
\begin{enumerate}[label= (\alph*)]
	\item
		\begin{equation*}
			\exists! \hat{x} \in V: \norm{x - \hat{x}} = 
			\min \{\norm{x-u}: u \in \vspan(x_1, \ldots, x_n) \}
		\end{equation*}
	\item
		Für $\hat{x}$ aus (a) gilt $x-\hat{x} \perp U$ d.h.
		\begin{equation*}
			x - \hat{x} \perp u\ \forall u \in U
		\end{equation*}
		Sind $y_1, \ldots, y_n$ ein OGS mit $U = \vspan(y_1, \ldots, y_n)$ dann gilt:
		\begin{equation*}
			\hat{x} = \sum_{k=1}^{n} \frac{<y_k,x>}{<y_k, y_k>} y_k
		\end{equation*}
\end{enumerate}


    \chapter{Gewöhnliche Differentialgleichungen}
    %!TEX root = ../main.tex
\section{Einführung}
\subsection{Explizite DGL 1. Ordnung}
Gegeben sei ein Gebiet $G \subseteq \R^2$ und eine stetige Funktion $f: G \rightarrow \R$, dann heißt
eine Funktion $y: I \rightarrow \R$ (mit $I \subseteq \R$ ein Intervall) eine Lösung der (expliziten) Differentialgleichung (DGL)
\begin{equation*}
	y' = f(x, y)
\end{equation*}
wenn gilt:
\begin{enumerate}[label= (\alph*)]
	\item 
		\begin{equation*}
			\begin{pmatrix}
				x \\ y(x)
			\end{pmatrix}
			\in G\ \forall x \in I
		\end{equation*}
	\item 
		\begin{equation*}
			y'(x) = f(x,y(x))\ \forall x \in I		
		\end{equation*}
\end{enumerate}
Ferner heißt $y$ eine Lösung des Anfangswertproblems
\begin{equation*}
	(\text{AWP})  =
	\begin{cases}
		y' &= f(x,y) \\
		y(x_0) &= y_0
	\end{cases}
\end{equation*}
wenn $y$ eine Lösung von $y'=f(x,y)$ ist und $y(x_0) = y_0$ gilt.

\subsection{Implizite DGL 1. Ordnung}
\begin{enumerate}[label= (\alph*)]
	\item Eine DGL der Form
		\begin{equation*}
			g(x,y,y') = 0
		\end{equation*}
		mit $g: M \subseteq \R^3 \rightarrow \R$ heißt eine implizite DGL 1. Ordnung und
		\begin{equation*}
			(\text{AWP}) =
			\begin{cases}
				g(x,y,y') &= 0 \\
				y(x_0) &= y_0
			\end{cases}
		\end{equation*}
		ist das dazugehörige AWP.
	\item Das lässt sich (manchmal) auf eine explizite DGL zurückführen:
		\begin{equation*}
			g(x,y,z) = 0
		\end{equation*}
		hat bei $(x_0, y_0)$ eine Auflösung nach $z=f(x,y)$ falls die Determinate der Jakobi-Matrix
		$J = \frac{\partial}{\partial z} g(x,y,z)$ nicht null ist.
\end{enumerate}

\subsection{DGL-System 1. Ordnung}
Sei $G \subseteq \R^{n+1}$ und für $k\in\{1, \ldots, n\}$ sei $f_k: G \rightarrow \R$ stetig.
$y: I \rightarrow \R^n$ mit $y(x) = {(y_1(x), \ldots, y_n(x))}^T$ heißt Lösung des Systems
\begin{equation*}
	\begin{cases}
		y_1' &= f_1(x, y_1, \ldots, y_n) \\
		\vdots \\
		y_n' &= f_n(x, y_1, \ldots, y_n)
	\end{cases}
\end{equation*}
kurz:
\begin{equation*}
	y' = f(x,y)
\end{equation*}
von Differentialgleichungen 1. Ordnung, wenn gilt:
\begin{enumerate}
	\item 
		\begin{equation*}
			\begin{pmatrix}
				x \\ y(x)
			\end{pmatrix}
			=
			\begin{pmatrix}
				x \\ y_1() \\ \vdots \\ y_n(x)
			\end{pmatrix}
			\in G\ \forall x \in I
		\end{equation*}
	\item $y_k$ ist stetig differentierbar auf $I$
	\item $y_k'(x) = f_k(x, y_1(x), \ldots, y_n(x)) = f_k(x, y(x))$
\end{enumerate}
Außerdem heißt $y$ eine Lösung des zugehörigen AWP
\begin{equation*}
	(\text{AWP}) = 
	\begin{cases}
		y' = f(x,y) \\
		y(x_0) = 
		\begin{pmatrix}
			y_1(x_0) \\ \vdots \\ y_n(x_0)
		\end{pmatrix} = 
		\begin{pmatrix}
			\tilde{y_1} \\ \vdots \\ \tilde{y_n}
		\end{pmatrix} = 
		\tilde{y} = y_0
	\end{cases}
\end{equation*}
wenn gilt: $y$ ist eine Lösung von $y' = f(x,y)$ und $y(x_0) = y_0$.

\subsection{Umschreiben von DGLen}
Eine sogenannte DGL $n$-ter Ordnung
\begin{equation*}
	y^{(n+1)} = f(x, y, y', \ldots, y^{(n+1)})
\end{equation*}
kann man stets in ein System 1. Ordnung umschreiben.

\begin{enumerate}
	\item $y_1, \ldots, y_n$ definieren als:
		\begin{eqnarray*}
			y_1 &:=& y\\
			y_2 &:=& y_1'\\
			y_n &:=& y_{n-1}'
		\end{eqnarray*}
	\item Das ursprüngliche Problem einsetzten:
		\begin{equation*}
			y_n' = y^{(n)} = f(x,y, y', \ldots, y^{(n-1)}) =
			f(x, y_1, \ldots, y_n)
		\end{equation*}
	\item Als System formulieren:
		\begin{equation*}
			{\begin{pmatrix}
				y_1 \\ \vdots \\ y_n
			\end{pmatrix}}'
			=
			\begin{pmatrix}
				f_1(x, y_1, \ldots, y_n) \\
				\vdots \\
				f_n(x, y_1, \ldots, y_n) \\
			\end{pmatrix}
			=
			\begin{pmatrix}
				y_2 \\ y_3 \\ \vdots \\ f(x, y_1, \ldots, y_n)
			\end{pmatrix}
		\end{equation*}
	\item Die Anfangsbedinung formulieren:
		\begin{eqnarray*}
			y(x_0) &=& \tilde{y_1} \\
			y'(x_0) &=& \tilde{y_2} \\
			& \vdots & \\
			y^{(n-1)}(x_0) &=& \tilde{y_n}
		\end{eqnarray*}
		ergibt den Vektor
		\begin{equation*}
			\tilde{y} := 
			\begin{pmatrix}
				\tilde{y_1} \\ \vdots \\ \tilde{y_n}
			\end{pmatrix}
		\end{equation*}
\end{enumerate}

    \chapter{DGL-Systeme und DGLen n-ter Ordnung}
    %!TEX root = ../main.tex
\section{Systeme von DGLen 1. Ordnung}
\subsection{Definition DGL-System}
Sei $G \subseteq \R^{n+1}$ und für $k \in \{1, \ldots, n\}$ sei $f_k: G \to \R$ stetig:
\begin{enumerate}[label= (\alph*)]
	\item $y(x) = (y_1(x), \ldots, y_n(x))^T$ heißt Lösung des Systems:
		\begin{eqnarray*}
			y_1' &=& f_1(x, y_1, \ldots, y_n) \\
			&\vdots& \\
			y_n' &=& f_n(x, y_1, \ldots, y_n) \\
		\end{eqnarray*}
		von DLGen $n$-ter Ordnung, wenn gilt:
		\begin{enumerate}
			\item 
				\begin{equation*}
					\begin{pmatrix}
						x \\ y(x)
					\end{pmatrix} \in G\ \forall x \in I
				\end{equation*} 
			\item $y$ ist differentierbar auf $I$ (Komponentenweise, d.h. $y_k$ ist diff'bar $\forall k$)
			\item $y_k'(x) = f_k(x, y_1(x), \ldots, y_n(x))\ \forall x \in I$ und bel. $k\in\{1, \ldots, n\}$ 
		\end{enumerate}
	\item Ist $(x,y)^T = (x_0, y_1^{(0)}, \ldots, y_n^{(0)})^T \in G$ dann löst $y: I \to \R^n$ das AWP.
		\begin{equation*}
			\text{(AWP)} = 
			\begin{cases}
				y_1'(x) &= f_1(x, y_1(x), \ldots, y_n(x)) \\
				&\vdots \\
				y_n'(x) &= f_n(x, y_1(x), \ldots, y_n(x))\\
				y_k(x_0) &= y_k^{(0)}\ k \in \{1, \ldots, n\}
			\end{cases}
		\end{equation*} 
		wenn $y$ die DGL löst und $y(x_0)=(y_1^{(0)}, \ldots,y_n^{(0)})^T$ gilt.
\end{enumerate}

\subsection{Schreibweise}
\begin{enumerate}[label= (\alph*)]
	\item DGL:
		\begin{enumerate}
			\item 
				\begin{equation*}
					y: I \to \R^n, y(x) := \begin{pmatrix}
						y_1(x) \\ \vdots \\ y_n(x)
					\end{pmatrix} \text{ oder }
					y = \begin{pmatrix}
						y_1 \\ \vdots \\ y_n
					\end{pmatrix}
				\end{equation*}
			\item
				\begin{equation*}
					f_k(x,y) = f_k(x, y_1, \ldots, y_n) \text{ bzw. }
					f_k(x,y(x)) = f_k(x, y_1(x), \ldots, y_n(x))\ \forall x \in I
				\end{equation*}			
			\item $f: G \subseteq \R^{n+1} \to \R^n$ mit
				\begin{equation*}
					f(x,y) = f(x, y_1, \ldots, y_n) = \begin{pmatrix}
						f_1(x, y_1, \ldots, y_n) \\
						\vdots \\
						f_n/(x, y_1, \ldots, y_n)
					\end{pmatrix}
				\end{equation*}
		\end{enumerate}
	\item AWP:
		\begin{equation*}
			\text{(AWP)} = \begin{cases}
				y' &= f(x,y) \\
				y(x_0) &= y_0
			\end{cases} \text{ mit }
			y_0 = \begin{pmatrix}
				y_1^{(0)} \\ \vdots \\ y_n^{(0)}
			\end{pmatrix}
		\end{equation*}
	\item Integration von Vektoren:
		Für $g:[a,b] \subseteq \R \rightarrow \R^n, g(x) = (g_1(x), \ldots, g_n(x))^T$ definieren wir (falls $g_1, \ldots, g_n$ int'bar sind):
		\begin{equation*}
		 \int_a^b g(x) \dx = \begin{pmatrix}
		 		\int_a^b g_1(x) \dx \\
		 		\vdots \\
		 		\int_a^b g_n(x) \dx \\
		 	\end{pmatrix}
		\end{equation*}
\end{enumerate}

\subsection{Voltera'sche Integralgleichung}
Sei $G \subseteq \R^{n+1}$ ein Gebiet, $f: G \to \R^n$ stetig und
\begin{equation*}
	\begin{pmatrix}
		x_0 \\ y_1^{(0)} \\ \vdots \\ y_n^{(0)}
	\end{pmatrix} = 
	\begin{pmatrix}
		x_0 \\ y_0
	\end{pmatrix}
	\in G
\end{equation*} 
dann ist $y: I \subseteq \R \to \R^n$ genau dann eine Lösung des AWP $y'=f(x,y), y(x_0)=y_0$, wenn gilt
\begin{equation*}
	y(x) = y_0 + \int_{x_0}^x f(t, y(t)) \dt
\end{equation*}

\subsection{Definition Lipschitz-Bedingung}
Sei $f: M \subseteq \R^{n+1} \to \R^n$. Dann erfüllt $f$ eine L-Bed. Mit $L>0$, wenn gilt:
\begin{equation*}
	\norm{f(x,y_1) - f(x, y_2)} < L \cdot \norm{y_1-y_2}\ \forall 
	\begin{pmatrix}
		x, y_1
	\end{pmatrix},
	\begin{pmatrix}
		x, y_2
	\end{pmatrix} \in M
\end{equation*}

\subsection{Bemerkung}
Welche Norm hier verwendet wird ist egal, denn es existiert ein $c_1, c_2$ mit
\begin{equation*}
	c_1 \norm{\cdot}_\infty \leq \norm{\cdot} \leq c_2 \norm{\cdot}
\end{equation*}

\subsection{Satz von Picard-Lindelöf}
Seien $r,s > 0; x_0 \in \R, y_0 = (y_1^{(0)}, \ldots, y_n^{(n)})^T \in \R^n$ und $M = \{
(x, y_1, \ldots, y_n)^T\in \R^n: x \in [x_0, x_0 + r], y_k \in [y_k^{(0)}-s, y_k^{(0)}+s]\}$
und $f: G \to \R^n$ stetig.

Falls $f$ auf $M$ eine L-Bed. erfüllt so exitiert auf $I = [x_0, x_0 + \alpha]$ mit $\alpha = \min \{r, \frac{s}{c} \}$
für
\begin{equation*}
	c := \max_{(x,y)^T \in M} \norm{f(x,y)}_\infty
\end{equation*}

\subsection{Satz von Peano}
Falls die L-Bed. im Satz von Picard-Lindelöf nicht erfüllt ist, alle anderen Vorraussetzungen aber gelten kann man auf
$I = [x_0, x_0 + \alpha]$ (mit $\alpha$ wie oben) trotzdem die Existenz einer Lösung zeigen, nicht aber deren Eindeutigkeit.

\subsection{Stabilität}
Analog gilt, falls $f$ eine L-Bed. erfüllt und eine Lösung existiert:
\begin{equation*}
	\norm{y_1(x) - y_2(x)}_\infty \leq (\norm{y_0 - \tilde{y_0}}_\infty c (x-x_0)) \exp{\left(L(x-x_0)\right)}
\end{equation*}
dabei ist $y_1$ Lösung von $y'=f_1(x,y), y(x_0)=y_0$ und $y'=f_2(x,y), y(x_0) = \overline{y_0}$

\section{Lineare DGL-Systeme 1. Ordnung}
\subsection{Definition}
Ein DGL System der Form
\begin{equation*}
	\begin{pmatrix}
		y_1'(x) \\ \vdots \\ y_n'(x)
	\end{pmatrix} = 
	\begin{pmatrix}
		a_{11}(x) & \cdots & a_{1n}(x) \\
		\vdots & \ddots & \vdots \\
		a_{n1}(x) & \cdots & a_{nn}(x)
	\end{pmatrix}
	\begin{pmatrix}
		y_1(x) \\ \vdots \\ y_n(x)
	\end{pmatrix} +
	\begin{pmatrix}
		b_1(x) \\ \vdots \\ b_n(x)
	\end{pmatrix}
\end{equation*}
oder in kurz:
\begin{equation*}
	y'(x) = A(x) y(x) + b(x)
\end{equation*}
mit stetigen Funktionen $a_{ij}: I \to \R, b_i:I \to \R\ \forall i,j \in \{1, \ldots, n\},\ I \subseteq \R$ ein Intervall heißt ein lineares DGL-System 1. Ordnung.

\subsection{Definition Frobenius-Norm}
Für $A \in \R^{m \times n}$ ist durch
\begin{equation*}
	\norm{A}_F = \sqrt{\sum_{i=1}^n \sum_{j=1}^n \abs{a_{ij}}^2}
\end{equation*}
die sogenannte Frobenius-Norm $\norm{\cdot}_F: \R^{m \times n} \rightarrow \R$ definiert.

\subsection{Matrixnorm und Vektornorm}
\begin{enumerate}[label= (\alph*)]
	\item $\norm{\cdot}_F$ ist eine Vektornorm auf dem Vektorraum $\R^{m \times n}$ (oder $\C^{m \times n}$)
	\item Die Matrix-Norm $\norm{\cdot}_F$ ist bezüglich der Vektornorm $\norm{\cdot}_2$ submultiplikativ, d.h. es gilt
		\begin{equation*}
			\norm{A \cdot x}_2 \leq \norm{A}_F \cdot \norm{x}_2
		\end{equation*}
	\item Bezüglich einer beliebigen Vektornorm auf $\R^n$ gilt:
		\begin{equation*}
			\norm{A x} \leq c_1 \norm{A x}_2 \leq \norm{A}_F \norm{x}_2 \leq c_2 \norm{A}_F \norm{x}
		\end{equation*}
\end{enumerate}

\subsection{Grenznorm}
Die sogenannte Grenznorm für eine Matrix-Norm wird bezüglich einer Vektornorm $\norm{\cdot}$ so definiert:
\begin{equation*}
	\norm{A} := \sup_{x \in \R^n \land x\neq \vec{0}} \frac{\norm{A x}}{\norm{x}} = \max_{\norm{x}=1} \norm{A x}
\end{equation*}
Daraus folgt:
\begin{equation*}
	\norm{A x} \leq \norm{A} \norm{x}
\end{equation*}
und für mindestens ein $x$ gilt $=$ (d.h die Abschätzung ist scharf).

Die Bezeichnung der Grenznorm wird von der Vektornorm übernommen. Es gilt also:
\begin{equation*}
	\norm{A}_p = \sup_{x \in \R^n \land x \neq \vec{0}} \frac{\norm{A x}_p}{\norm{x}_p}
\end{equation*}
die Grenznorm bezüglich $\norm{\cdot}_p$.

\subsubsection{Bemerkung}
Es gilt:
\begin{equation*}
	\norm{A}_2 \leq \norm{A}_F
\end{equation*}

\subsection{Lösungsmengen von linearen Gleichungssystemen}
\begin{enumerate}[label= (\alph*)]
	\item Die Lösungsmenge von $y' = A(x) y$ ist ein Unterraum des Vektorraums der stetigen Funktionen
	\item Die Lösungsmenge von $y' = A(x) y + b(x)$ ist ein affiner Unterraum des Vektorraums der stetigen Funktionen
\end{enumerate}

\subsection{Definition Fundamentalsystem}
Sei $I \subseteq \R$ ein nicht entartetes Intervall und $A: I \to \R^{n \times n}$ stetig
\begin{enumerate}[label= (\alph*)]
	\item Eine Basis des Lösungsraums von $y' = A(x) y$ heißt Fundamentalsystem (FS). Ist $y_1, \ldots, y_n$ so ein FS,
		dann heißt die Matrix 
		\begin{equation*}
			Y(x) = (y_1(x), \ldots, y_n())
		\end{equation*}
		die Fundamentalmatrix
	\item Für $y_1, \ldots, y_n: I \to \R^n$ heißt $W: I \to \R$ mit $W(x) = \det (Y(x))$ die Wronski-Determinante
		des Fundamentalsystems.
\end{enumerate}

\subsection{Definition Determinante}
Eine Funktion $\det: \R^{n \times n} \to \R$ heißt Determinante, wenn gilt
\begin{enumerate}[label= (\alph*)]
	\item Mit $I$ der Einheitsmatrix:
		\begin{equation*}
			\det(I) = 1
		\end{equation*}
	\item $\det$ ist linear in jeder Spalte, d.h. für $1\leq l \leq n$ gilt:
		\begin{equation*}
			\det(a_1 \ldots, a_{l-1}, \alpha x + \beta y, a_{l+1}, \ldots, a_n) =
			\alpha \det(a_1 \ldots, a_{l-1}, x, a_{l+1}, \ldots, a_n) + 
			\beta \det(a_1 \ldots, a_{l-1}, y, a_{l+1}, \ldots, a_n)
		\end{equation*}
	\item Spalten tauschen ändert Vorzeichen:
		\begin{equation*}
			\det(a_1, \ldots, a_i, \ldots, a_j, \ldots, a_n) =
			\det(a_1, \ldots, a_j, \ldots, a_i, \ldots, a_n)
		\end{equation*}
\end{enumerate}

\subsection{Entwicklungssatz von Laplace}
Sei $A \in \R^{n \times n}$ und bezeichne $A_{ij}$ die $(n-1)\times(n-1)$ Matrix, die aus $A$ durch Streichen
der $i$-ten Zeile und $j$-ten Spalte entsteht. Die Einträge von $A$ bezeichnen wir mit $A = (a_{ij})_{1 \leq i,j \leq n}$. Dann gilt:
\begin{enumerate}[label= (\alph*)]
	\item Für $j \in \{1, \ldots, n \}$ beliebig aber fest ist
		\begin{equation*}
			\det(A) = \sum_{i=1}^n a_{ij} \det{A_{ij}} {(-1)}^{i+j}
		\end{equation*}
	\item Für $i \in \{1, \ldots, n \}$ beliebig aber fest ist
		\begin{equation*}
			\det(A) = \sum_{j=1}^n a_{ij} \det{A_{ij}} {(-1)}^{i+j}
		\end{equation*}
\end{enumerate}

\subsection{Leibniz-Formel für Determinanten}
Sei $A \in \R^{n \times n}$ und bezeichne $S_n$ die Gruppe der Permutationen von
$\{1, \ldots, n\}$ und sei $\sgn(\sigma)$ definiert durch $\sgn: S_n \to \{-1, 1\}$:
\begin{equation*}
	\sgn{\sigma} = \begin{cases}
		1 & \text{ falls $\sigma$ durch eine gerade Anzahl Permutationen entsteht} \\
		-1 & \text{ sonst}
	\end{cases}
\end{equation*}
wobei eine Transposition einer paarweisen Vertauschung entspricht.

Dann gilt:
\begin{equation*}
	\det(A) = \det((a_{ij})_{1\leq  i,j \leq n}) = 
	\sum_{\sigma \in S_n} \sgn(\sigma) \prod_{k=1}^n a_{k\sigma(k)}
\end{equation*}

\subsection{Berechnung der Wronski-Determinante ohne bekanntes FS}
Für die Wronski-Determinante gilt (d.h. FS existiert, sonst ist Wronski-Determinante nicht
definiert):
\begin{enumerate}[label= (\alph*)]
	\item $W(x)$ kann durch eine lineare DGL 1. Ordnung bestimmt werden:
		\begin{equation*}
			W'(x) = \operatorname{spur}(A(x)) \cdot W(x)
		\end{equation*}
		mit
		\begin{equation*}
			\operatorname{spur}(A) = \sum_{i=1}^n a_{ii}
		\end{equation*}
	\item
		Durch Lösen der DGL folgt:
		\begin{equation*}
			W(x) = W(x_0) \cdot \exp\left(\int_{x_0}^x \operatorname{spur}(A(t)) \dt \right)
		\end{equation*}
\end{enumerate}

\subsubsection{Bemerkung}
Falls $y' = A(x) y$ eine Anfangsbedingung hat ist $W(x_0)$ durch die Anfangsbedinung bekannt, dadurch ist
$W(x)$ bekannt.


\subsection{Ableitung der Determinante}
Sei $I \subseteq \R$ ein nicht-entartetes Interval und $A(x)$ (komponentenweise) differentierbar auf
$I$, dann ist $\det(A(x))$ differentierbar und es gilt:
\begin{equation*}
	\ddx \det(A(x)) = \sum_{i=1}^n \det \begin{pmatrix}
		a_{11}(x) & \cdots & a_{1n}(x)\\
		& \vdots \\
		a_{i-1,1}(x) & \cdots & a_{i-1,n}(x)\\
		a'_{i1}(x) & \cdots & a'_{in}(x)\\
		a_{i+1,1}(x) & \cdots & a_{i+1,n}(x)\\
		& \vdots \\
		a_{n1}(x) & \cdots & a_{nn}(x)\\
	\end{pmatrix}
\end{equation*}


\subsection{Äquivalente Aussagen zu FSen und Wronski-Determinaten}
Sei $I \subseteq \R$ ein nicht entartetes Interval, $A: I \to \R^{n \times n}$ stetig.
Sind $y_1, \ldots, y_n$ Lösungen von $y' = A(x)  y$ und bezeichne $W(x) = \det(y_1(x), \ldots, y_n(x))$
dann sind folgende Aussagen äquivalent:
\begin{enumerate}[label= (\alph*)]
	\item $y_1, \ldots, y_n$ bilden ein FS
	\item $W(x) \neq 0\ \forall x \in I$
	\item $\exists x_0 \in I: W(x_0) \neq 0$
\end{enumerate}		

\subsection{Partikuläre Lösung aus Wronksi-Determinante}
Sei $I \subseteq \R$ ein nicht entartetes Intervall, $A: I \to \R^{n \times n}$ stetig und $b: I \to \R^n$
stetig, weiter sei $y_1, \ldots, y_n$ ein FS des homogenen Systems $y' = A(x) y$.

Dann ist:
\begin{equation*}
	y_p: I \to \R^n \text{ mit } y_p(x) = \sum_{k=1}^n y_k(x) \int_{x_0}^x \frac{W_k(t)}{W(t)} \dt
\end{equation*}
mit $x_0 \in I$ beliebig und 
\begin{equation*}
	W_k(x) = \det(y_1(x), \ldots, y_{k-1}(x), b(x), y_{k+1}(x), \ldots, y_n(x))
\end{equation*}
eine Lösung des inhomogenen Systems
\begin{equation*}
	y'(x) = A(x) y(x) + b(x)
\end{equation*}

    \chapter{Ergänzung zur Analysis}
    %!TEX root = ../main.tex
\section{Äquivalenzrelation und Äquivalenzklassen}
\subsection{Definition}
Sei $X$ eine beliebige Menge, mit $\tilde{}$ wird eine Eigenschaft zwischen zwei
Elementen definiert (Formal: $\tilde{}: X \times X \to \{\text{Wahr}, \text{Falsch}\}$).
Diese Relation heißt Äquivalenzrelation wenn gilt:
\begin{enumerate}
    \item $a \tilde{} a \ \forall a \in X$ 
    \item $a \tilde{} b \Rightarrow b \tilde{} a\ \forall a,b \in X$
    \item $a \tilde{} b \land b \tilde{} c \Rightarrow a \tilde{} c\ \forall a,b,c \in X$
\end{enumerate}
Mit so einer Relation kann man $X$ in Äquivalenzklassen $\hat{x}$ zerlegen:
\begin{equation*}
    \hat{x} := \{A \subseteq X: \text{Für} a,b \in X \text{ gilt } a \tilde{} b \land \text{für kein} y \in A^C  \text{gilt} y \tilde{} a \text{ für } a \in A\}
\end{equation*}
$A$ ist die größte Teilmenge von $X$ in der alle Elemente in Relation stehen.

\section{Distributionen}
\subsection{Testfunktionen}
\begin{enumerate}
    \item Eine Funktion $\Phi: \R \to \R$ heißt Testfunktion, wenn $\Phi \in C_0^\infty$ gilt, d.h $\Phi \in C^\infty$
        und
        \begin{equation*}
            \exists c > 0:\ \Phi(x)=0\ \forall x \notin [-c, c]
        \end{equation*}
        die Menge aller Testfunktionen bezeichnen wir mit $D$ d.h. $D = C_0^\infty$.
    \item Eine Folge ${(\Phi_n)}_{n=1}^\infty \subseteq D$ heißt konvergent, wenn ein $\Phi \in D$ existiert mit
        $\Phi_n$ konvergiert gleichmäßig gegen
        \begin{equation*}
            \Phi \Leftrightarrow \forall \varepsilon > 0: \exists n_0:\ \abs{\Phi(x) - \Phi_n(x)} < \varepsilon\ 
            \forall x \in \R, n \geq n_0
        \end{equation*}
\end{enumerate}

\subsection{Distributionen}
Eine Funktion $f: D \to \R$ heißt Distribution, wenn
\begin{enumerate}
    \item $f$ ein lineares Funktional ist, das heißt es gilt:
        \begin{enumerate}
            \item Der Wertebereich von $f$ ist $\R$ (oder $\C$)
            \item $\forall \Phi, \Psi \in D, \alpha, \beta \in \R$ gilt
                \begin{equation*}
                    f(\alpha \Phi + \beta \Psi) = \alpha f(\Phi) + \beta f(\Psi)
                \end{equation*}
        \end{enumerate}
    \item $f$ ist stetig, d.h.
        \begin{equation}
            \Phi_n \to \Phi (n \to \infty) \Rightarrow f(\Phi_n) \to f(\Phi) (n \to \infty)
        \end{equation}
\end{enumerate}

\subsubsection{Bemerkung:}
Die Menge aller Distribution nennen wir $D'$

\subsection{Duale Paarung und Repräsentanten}
\begin{enumerate}
    \item Ist $f \in D'$ und $\Phi \in D$ dann schreibt man
        \begin{equation*}
            <f, \Phi> = f(\Phi)
        \end{equation*}
    \item Eine Funktion $\tilde{f}: \R \to \R$ heißt Repräsentant von $f \in D'$, wenn gilt
        \begin{equation*}
            <f, \Phi> = \int_{-\infty}^\infty \tilde{f}(x) \Phi(x) \dx = <\tilde{f}, \Phi>
        \end{equation*}
    das heißt $f$ kann man sich vorstellen.
\end{enumerate}
\subsubsection{Bemerkung:}
Wenn eine Distribution einen Repräsentanten besitzt ist dieser nicht eindeutig.

\subsection{Ableitung einer Distribution}
Sei $f\in D'$, dann heißt $f'$ die schwache Ableitung von
$f$ wenn gilt:
\begin{equation*}
    <f', \Phi> = -<f, \Phi'>\ \forall \Phi \in D
\end{equation*}

\section{Fouriertransformation}
\subsection{Definition Fourier-Trafo}
Sei $f: \R \to \C$ stückweise stetig und gelte
\begin{equation*}
    \int_{-\infty}^\infty \abs{f(t)} \dt < \infty
\end{equation*}
dann heißt
\begin{equation*}
    \hat{f}(x) = F_f(x) = \infty_{-\infty}^\infty
    f(t) \exp(-itx) \dt
\end{equation*}
die Fourier-Transformierte von $f$.

\subsection{Stetigkeit der Fourier-Transformierten}
Sei $f: \R \to \C$ stückweise stetig und $\int_{-\infty}^\infty \abs{f(t)} \dt < \infty$
dann ist $F_f(x)$ beschränkt und stetig.

\subsection{Zeitliche Verschiebung und Skalierung der Fouriertransformierten}
Seien $f_1, f_2, f: \R \to \C$ stückweise stetig und $\int_{-\infty}^\infty \abs{f(t)} \dt < \infty$ sowie
$\int_{-\infty}^\infty \abs{f_k(t)} \dt\ k \in \{1,2\}$.

Definiere $g: \R \to \C$ mit
\begin{equation*}
    g(t) = f(a*t + b)
\end{equation*}
dann gilt
\begin{enumerate}
    \item 
        \begin{equation*}
            F_g(x) = \frac{1}{\abs{x}} \exp(i x \frac{b}{a}) \cdot F_f(\frac{x}{a})\ \text{für } a \neq 0
        \end{equation*}
    \item
        \begin{equation*}
            F_{\alpha f_1 + \beta f_2} (x) = \alpha F_{f_1}(x) + \beta F_{f_2}(x)
        \end{equation*}
\end{enumerate}

\subsection{Ableitung der Fouriertransformierten}
Ist $f: \R \to \C$ stückweise stetig, $\int_{-\infty}^\infty \abs{f(t)} \dt < infty$ sowie
$\int_{-\infty}^\infty \abs{t f(t)}\dt < \infty$. Definiere $g(t) = t f(t)$. Ist $f$ differentierbar, dann ist 
$F_f$ ebenfalls differentierbar und es gilt:
\begin{equation*}
    F_f'(x) = \ddx F_f(x) = -i F_g(x)
\end{equation*}


    \part{Beweisansätze}
    \chapter{HM 1}

\subsection{Eindeutigkeit des Grenzwert einer Folge }
 Zeige, dass Grenzwert a = Grenzwert b, nahrhafte 0
\subsection{Konvergente Folgen sind beschränkt }
 Nahrhafte 0, Dreiecks-ugl.
\subsection{Grenzwertrechenregeln }
 Nahrhafte 0, Dreiecks-ugl.
$a_n \leq \gamma\ \forall n \Rightarrow a \leq \gamma$
 Ausgehend von a über nahrh. 0 zu Def Konvergenz
$a_n \leq b_n\ \forall n \Rightarrow a\leq b$
 Definiere Hilfsfolge, argumentiere nach s.o
Sandwich-Theorem
 Zeige, dass $-\varepsilon < c_n < \varepsilon$  (Quasi Epsilon-Schlauch)
\subsection{Monotoniekriterium }
 Da $\abs{a_n} < c\ \forall n$, argumentiere über das Supremum der Menge, die aus $a_n$ besteht
\subsection{Grenzwert einer konv. Folge = Grenzwert jeder Teilfolge }
 Def. Konvergenz + Def Teilfolge
\subsection{Charakterisierung $\varlimsup$ und $\varliminf$ }
 Argumentiere über Eigneschaften sup und inf
\subsection{Folge konv.\ $\varlimsup = \varliminf$ }
 Hin: Eindeutigkeit des Grenzwert;\@Rück: Charakterisierung limSup und limInf
\subsection{Bolzano-Weierstraß }
 Zunächst für reelle Folge (trivial), dann für komplex: Realteil ist klar, Imaginärteil: Teilfolge konstruieren
\subsection{Cauchykriterium }
 Hin: nahrhafte 0; Rück: zeige Beschränktheit, dann folge daraus, dass ein Häufungswert existiert und benutze diesen als Grenzwert-Kandidat
\subsection{Reihe konv.  Folge ist Nullfolge }
 Cauchy für Reihen
\subsection{GrenzwertRR für Reihen }
 GrenzwertRR für Folgen
\subsection{Reihe konv g. 0 }
 Restreihe als Differenz darstellen
\subsection{Leibniz }
 Cauchy für Reihen
\subsection{Absolut konv.\ $\Rightarrow$  konv. }
 Cauchy und Dreiecks-ugl.
\subsection{Majorantenkriterium }
 Cauchy
\subsection{Minorantenkriterium }
 Kontradiktion von Majorantenkriterium
\subsection{Wurzelkriterium }
 Majorantenkrit: geom. Summe über $Q:=q+\varepsilon<1$, in $q$ das Wurzelkriteriumeinsetzen, Charakterisierung $\varlimsup$
\subsection{Quotientenkriterium }
 Majorantenkrit: setze in $q$ das Quotientenkriteriumein und Argumentation über $\varlimsup$
\subsection{Hadamard }
 Wurzelkriterium+ Fallunterscheidung für Sonderfälle
\subsection{Differenzieren / Integrieren von Potenzreihen}
 Wurzelkriterium
\subsection{Lemma zu sin, cos und exp }
 Cauchy-Produkt + Definitionen
\subsection{$e^z \neq 0$ und $e^{-z} = \frac{1}{e^{z}}$ }
 Inverses Element der Multiplikation
\subsection{Pythagoras }
 3.\ binomische Formel
\subsection{$e^x > 0\ \forall x \in \R$ }
 Betrachte $x \geq 0$, angeordneter Körper
\subsection{$1+x \leq e^x\ \forall x \in \R$ }
 Bernoulli
\subsection{$x<y \Rightarrow e^x < e^y$ }
 nahrhafte 0
\subsection{Folgenkriterium }
 Hin: Def. Folgenkonv.\ und dann Def Funktionsgrenzwert einsetzen; \@Rück: Wähle versch. $\delta$ und zeige Widerspruch
\subsection{Cauchy für Funktionen }
 Hin: Def. FunktionsGrenzwert + nahrhafte 0; \@Rück: Cauchy für Folgen
\subsection{Grenzwerte an Intervallgrenzen }
 Argumentiere über Supremum / Infimum
\subsection{Verknüpfungen stetiger Funktionen stetig }
 Folgenkriterium
\subsection{Potenzreihen sind innerhalb des Konvergenzradius stetig }
 Abschätzung: $\exists r>0 : \abs{x-x_0 \text{ bzw. }
 x_1} \leq r$, dann einfach $\abs{f(x)-f(x_1)}$ nach oben abschätzen
\subsection{Umgebung pos. Funktionswerte }
 Wähle $\varepsilon = \frac{f(x_0)}{2}$, Def. Stetigkeit
\subsection{Zwischenwertsatz }
 Definiere $x_0 := \sup \{x \in [a,b] : f(x) \leq y \}$ und zwei Hilfsfolgen, die gegen $x_0$ konvergieren
\subsection{Existenz $\log$ }
 Zeigen $\exp$ ist bijektiv (Zwischenwertsatz)
\subsection{Beschränktheit stetiger Funktionen}
 Annahme $f$ nicht beschränkt Folgenkriterium
\subsection{Weierstraß existenz min bzw.\ max }
 Zeigen das $\sup=\max$

 \chapter{HM 2}


    \part{Klausurvorbereitung}
    %!TEX root = main.tex
Hier findest du eine kurze Übersicht über alle Themen, die du für die jeweilige
Klausur beherrschen solltest:
Wichtige Definitionen und Beweise, die man gut in der Klausur abfragen kann,
besonders trickreiche Aufgaben, die mehrmals in der Vorlesung oder in der Übung
besprochen wurden und generelle Kompetenzen, die höchstwahrscheinlich von dir
verlangt werden.
\chapter{HM1}
%@TODO
\chapter{HM2}
\section{Integration}

\subsection{Wichtige Beweise}
\begin{itemize}
  \item 1.\ und 2. Mittelwertsatz der Integralrechnung
  \item Hauptsatz der Differential- und Integralrechnung
\end{itemize}

\subsection{Typische Aufgaben}
Berechne den GW von z.B. folgender Reihe (hast du also das Prinzip der Riemann-Summen verstanden?)
\begin{equation*}
    \lim\limits_{n \rightarrow \infty}{\frac{1}{n}\sum_{k=0}^n \sin{\frac{k\pi}{n}}}
\end{equation*}

Untersuche Reihen auf Konvergenz (wende das Integralkriterium an)
\begin{equation*}
    \sum_{n=-m}^\infty \frac{1}{1+n^2} \quad (m \in \N)
\end{equation*}
Oder diese hier (Tipp: Eulersche Gammafunktion)
\begin{equation*}
    \sum_{n=0}^\infty n^3 e^{-n^2}
\end{equation*}

Untersuche uneigentliche Integrale auf Konvergenz

\subsection{Trickreiche Aufgaben}
Schwierige uneigentliche Integrale. Konvergiert beispielsweise dieses Integral? (Ja, tut es)
\begin{equation*}
    \int_1^\infty \frac{\sin{x}}{x}\dx
\end{equation*}

\subsection{Weitere hilfreiche Dinge}
Schau dir uneigentliche Integrale an, die man gut als Majorante oder Minorante
verwenden kann, z.B.:
\begin{equation*}
    \int_0^1 \frac{1}{x^\alpha}\dx
\end{equation*}

\section{Gleichmäßige Konvergenz}
\subsection{Wichtige Beweise}
\begin{itemize}
  \item Stetigkeit der Grenzfunktion
  \item Satz von Dini (ziemlich tricky, aber die Idee sollte man im Kopf haben)
\end{itemize}

\subsection{Typische Aufgaben}
Untersuche Reihen auf gleichmäßige Konvergenz, z.B.:
\begin{equation*}
    \sum_{k=0}^\infty x{(1-x)}^k,\quad \forall x \in [0,1] \quad \text{bzw} \quad \forall x \in [a,1] \quad \text{mit} \quad 0 < a \leq 1
\end{equation*}

\subsection{Trickreiche Aufgaben}
Auf welchem Intervall konvergiert die Riemannsche Zeta-Funktion gleichmäßig?
\begin{equation*}
    \zeta(s):=\sum_{n=1}^\infty \frac{1}{n^s}
\end{equation*}

\section{Differentialrechnung mit mehreren Veränderlichen}
\subsection{Wichtige Beweise}
\begin{itemize}
  \item Beweise über z.B. die Vereinigung von beliebig vielen offenen Mengen
  \item Mehrdimensionaler Mittelwertsatz
  \item Notwendige Bedingung für Extrema
  \item Satz über konstante Funktionen
  \item Satz von Taylor
  \item Hinreichende Bedingung für Extrema
  \item Herleitung für die Ableitung der Auflösung (kann sehr hilfreich sein,
        wenn man die Formel vergessen hat)
\end{itemize}

\subsection{Typische Aufgaben}
Kannst du:
\begin{itemize}
  \item dir Mengen vorstellen und zeichnen?
  \item Funktionsgrenzwerte berechnen?
  \item Funktionen auf Stetigkeit prüfen?
  \item partielle Ableitungen und Richtungsableitungen berechnen?
  \item Funktionen auf totale Diff'barkeit prüfen?
  \item Extremwerte von Funktionen finden und klassifizieren?
  \item mit Matrizen rechnen und Inverse bestimmen?
  \item prüfen, ob eine Funktion umkehrbar ist und die Umkehrung bestimmen?
  \item die Ableitung einer unbekannten Umkehrfunktion bestimmen?
  \item prüfen, ob eine Funktion nach einer / mehreren Variablen auflösbar ist und die Auflösung bestimmen?
  \item die Ableitung einer unbekannten Auflösung berechen?
  \item Extrema unter Nebenbedingung bestimmen?
\end{itemize}

\section{Integration in mehreren Veränderlichen}
\subsection{Wichtige Beweise}
\begin{itemize}
	\item Fubini
	\item Ableitung eines Parameterintegrals
	\item Leibniz-Formel herleiten können
	\item 1. Hauptsatz für Kurvenintegrale
\end{itemize}

\subsection{Typische Aufgaben}
Kannst du:
\begin{itemize}
	\item die Länge von Kurven bestimmen?
	\item Funktionen auf Wegunabhängigkeit prüfen?
	\item Potentiale und Stammfunktionen berechnen?
	\item Flächen und Volumina berechnen mit:
	\begin{itemize}
		\item Fubini und Cavaleri?
		\item der Substitutionsregel?
	\end{itemize}
	\item Integralsätze:
	\begin{itemize}
		\item verifizieren?
		\item geschickt anwenden?
	\end{itemize}
\end{itemize}

\subsection{Trickreiche Aufgaben}
Schau dir schwierige uneigentliche Integrale an, wie z.B.:
\begin{equation*}
	\int_0^1{\frac{t^b-t^a}{\log{t}}\dt}
\end{equation*}
oder
\begin{equation*}
	\int_{-\infty}^{\infty}{e^{-x^2}\dx}
\end{equation*}

\section{Lineare Algebra}

\subsection{Typische Aufgaben}
Kannst du:
\begin{itemize}
	\item prüfen, ob eine Menge ein Vektorraum ist?
	\item prüfen, ob eine Menge ein Unterraum ist?
	\item lineare Unabhängigkeit nachprüfen?
	\item die Basis eines VR bestimmen?
	\item die Dimension einers VR bestimmen?
	\item LGS lösen?
\end{itemize}




    \part{Appendix}
    \chapter{Grenzwerte}
\section{Konvergenzkriterien}
Zusammenfassung verschiedener Konvergenzkriterien nach Wikipedia (Seite: Konvergenzkriterium):
\begin{center}
    \begin{tabular}{lcccccccp{2cm}}
         \toprule
         Kriterium & {nur f.\ mon. F.} & Konv. & Div. & abs. Konv. & Absch. & Fehlerabsch.\\
         \midrule
         Nullfolgenkriterium &  &  & X &  &  & \\
         Monotoniekriterium &  & X &  & X &  & \\
         Leibniz-Kriterium & X & X &  &  & X & X\\
         Cauchy-Kriterium &  & X & X &  &  & \\
         Abel-Kriterium & X & X &  &  &  & \\
         Dirichlet-Kriterium & X & X &  &  &  & \\
         Majorantenkriterium &  & X &  & X &  & \\
         Minorantenkriterium &  &  & X &  &  & \\
         Wurzelkriterium &  & X & X & X &  & X\\
         Integralkriterium & X & X & X & X & X & \\
         Cauchy-Kriterium & X & X & X & X &  & \\
         Grenzwertkriterium &  & X & X &  &  & \\
         Quotientenkriterium &  & X & X & X &  & X\\
         Gauß-Kriterium &  & X & X & X &  & \\
         Raabe-Kriterium &  & X & X & X &  & \\
         Kummer-Kriterium &  & X & X & X &  & \\
         Bertrand-Kriterium &  & X & X & X &  & \\
         Ermakoff-Kriterium & X & X & X & X &  & \\
         \bottomrule
    \end{tabular}
\end{center}

\chapter{Integration}
\section{Riemann-Integrierbarkeit}
\begin{center}
    \begin{tabular}{lcc}
        \toprule
        Kriterium & Integrierbar & Nicht Integrierbar \\
        \midrule
        Funktion nicht beschränkt & & X \\
        Verknüpfung Riemann-Integrierbarer Funktionen & X \\
        Stetige Funktion & X\\
        Endliche vielen Änderungen zu Riemann-Int.barer Funktion & X\\
        Monotone Funktion & X\\
        \bottomrule
    \end{tabular}
\end{center}

\chapter{Integration in mehreren Veränderlichen}
\section{Häufige Additionstheoreme}
\begin{eqnarray*}
    \sin^2(t) &=& \frac{1}{2} (1 - \cos(2t))\\
    \cos^2(t) &=& \frac{1}{2} (1 + \cos(2t))\\
    \sin(t) \cos(t) &=& \frac{1}{2} \sin(2t)
\end{eqnarray*}

\end{document}
