\documentclass[10pt]{report}
\usepackage[utf8]{inputenc}
\usepackage{amsfonts}
\usepackage{amsmath}
\usepackage{amssymb}
\usepackage{amsopn}
\usepackage{commath}
\usepackage[ngerman]{babel}
\usepackage{enumitem}
\usepackage{booktabs}
\usepackage{longtable}
\usepackage{relsize}

\DeclareMathOperator{\grad}{grad}

\newcommand{\K}{\mathbb{K}}
\newcommand{\N}{\mathbb{N}}
\newcommand{\Z}{\mathbb{Z}}
\newcommand{\Q}{\mathbb{Q}}
\newcommand{\R}{\mathbb{R}}
\newcommand{\C}{\mathbb{C}}
\newcommand{\an}{{(a_n)}_{n=1}^\infty}
\newcommand{\sn}{{(s_n)}_{n=1}^\infty}
\newcommand{\ReP}{\operatorname{Re}}
\newcommand{\ImP}{\operatorname{Im}}
\newcommand*{\nthSqrt}[2]{\sqrt[\leftroot{-2}\uproot{2}#1]{#2}}
\newcommand{\setbigcup}{\mathop{\bigcup}\displaylimits}
\newcommand{\setbigcap}{\mathop{\bigcap}\displaylimits}
\newcommand{\setbigtimes}{\mathop{\times}\displaylimits}
\newcommand{\ddx}{\dfrac{\text{d}}{\text{d}x}}
\newcommand{\dx}{\text{ d}x}
\newcommand{\dy}{\text{ d}y}
\newcommand{\dt}{\text{ d}t}
\newcommand{\ds}{\text{ d}s}
\newcommand{\ido}{\text{ d}o}
\newcommand{\intd}[1]{\text{ d}#1}
\newcommand{\rot}{\operatorname{rot}}
\newcommand{\vdiv}{\operatorname{div}}
\newcommand{\vspan}{\operatorname{span}}
\newcommand{\Kern}{\operatorname{Kern}}
\newcommand{\diag}{\operatorname{diag}}
\newcommand{\Image}{\operatorname{Im}}
\newcommand{\rg}{\operatorname{rg}}


\setlength\parindent{0pt}

\title{Zusammenfassung Höhere Mathematik}
\author{Paul Nykiel}

\begin{document}
    \maketitle
    \pagebreak
    Schlagzahl erhöhen.
    \pagebreak
    \tableofcontents
    \pagebreak

    \part{HM 1 --- Zusammenfassung}
    \chapter{Vorkurs}
    %!TEX root = ../main.tex
\section{Aussagenlogik}
\subsection{Definition Aussage}
Eine Aussage ist ein Satz, der entweder wahr oder falsch ist.

\subsubsection{Bemerkung}Wir beschäftigen uns mit der klassischen zweiwertigen
Logik. Es gibt auch Logiken mit 3 bzw.\ 4 Werten.

\subsection{Verknüpfungen}
Formal kann eine Oder-Verknüfung mit dem $\lor$-Zeichen durch eine
Wahrheitstabelle definiert werden:
\begin{center}
    \begin{tabular}{ccc}
        \toprule
        $A$ & $B$ & $A \lor B$ \\
        \midrule
        $1$ & $1$ & $1$ \\
        $1$ & $0$ & $1$ \\
        $0$ & $1$ & $1$ \\
        $0$ & $0$ & $0$ \\
        \bottomrule
    \end{tabular}
\end{center}

Analog kann eine Und-Verknüpfung mit dem $\land$-Zeichen durch eine
Wahrheitstabelle definiert werden:
\begin{center}
    \begin{tabular}{ccc}
        \toprule
        $A$ & $B$ & $A \land B$ \\
        \midrule
        $1$ & $1$ & $1$ \\
        $1$ & $0$ & $0$ \\
        $0$ & $1$ & $0$ \\
        $0$ & $0$ & $0$ \\
        \bottomrule
    \end{tabular}
\end{center}

Und eine Negation wird definiert durch:
\begin{center}
    \begin{tabular}{cc}
        \toprule
        $A$ & $\lnot A$ \\
        \midrule
        $1$ & $0$ \\
        $0$ & $1$ \\
        \bottomrule
    \end{tabular}
\end{center}

Eine sog.\ Implikation wird durch das $\Rightarrow$-Zeichen dargestellt und
ist definiert durch:
\begin{center}
    \begin{tabular}{ccc}
        \toprule
        $A$ & $B$ & $A \Rightarrow B$ \\
        \midrule
        $1$ & $1$ & $1$ \\
        $1$ & $0$ & $0$ \\
        $0$ & $1$ & $1$ \\
        $0$ & $0$ & $1$ \\
        \bottomrule
    \end{tabular}
\end{center}

\subsubsection{Bemerkung}
Bei mehr als einer Verknüpfung muss klar sein welche Verknüpfung
als erstes ausgewerted werden muss, hierfür werden Klammern verwendet.

\subsection{Mehr zu Implikationen}
Bei der Aussage $A \Rightarrow B$ bezeichnet man $A$ als hinreichende
Bedingung und $B$ als notwendige Bedingung.

Die Aussage $A \Rightarrow B$ ist äquivalent zu $\lnot B \Rightarrow \lnot A$.

\subsection{Bezeichnung von Aussagen}
Eine Aussageform heißt:
\begin{enumerate}[label= (\alph*)]
    \item Allgemeingültig (oder Tautologie), wenn sie als Wahrheitswert stets
        den Wert wahr annimmt.
    \item Erfüllbar, wenn die Wahrheitstabelle mindestens einmal den Wert
        wahr enthält.
    \item Unerfüllbar (oder Kontradiction), wenn die Wahrheitstabelle nur
        falsch-Einträge enthält.
\end{enumerate}

\subsection{Satz der Identität}
Mit $A \Leftrightarrow B$ kürzen wir die Aussage:
\begin{equation*}
    (A \Rightarrow B) \land (B \Rightarrow A)
\end{equation*}
ab.

\subsubsection{Bemerkung}
Für den allg. Fall sagt man zu $A \Leftrightarrow B$: A ist äquivalent
zu B. Das heißt aber nicht, dass $A=B$ ist.


\section{Mengen}
\subsection{Defintion: Mengen nach Cantor}
Unter einer Menge versteht man eine Zusammenfassung
bestimmter wohlunterscheidbarer Objekte unsere Anschauung oder unseres Denkens
zu einem Ganzen.

\subsection{Begrifflichkeiten und Schreibweise}
Objekte einer Menge bezeichnet man als Elemente einer Menge.

Schreibweise:
\begin{enumerate}[label= (\alph*)]
    \item $x \in M$ oder $x \notin M$
    \item Mengen können durch Aufzählen der Elemente beschrieben werden:
        $M= \{ a, b, c \}$
    \item Mengen können durch Eigenschaften der Elemente beschrieben werden:
        $M= \{x: x \text{ hat Eigenschaft\ldots} \}$
\end{enumerate}

\subsection{Leere Menge, Teilmengen}
\begin{enumerate}[label= (\alph*)]
    \item Die Menge, die kein Element enthält, heißt leere Menge.
        Wir bezeichnen diese mit $\emptyset$.
    \item Eine Menge $M_1$ heißt Teilmenge einer Menge $M_2$
        (Schreibweise $M_1 \subseteq M_2$) falls jedes Element von $M_1$ auch
        Element von $M_2$ ist. D.h.\ es gilt:
        \begin{equation*}
            x \in M_1 \Rightarrow x \in M_2
        \end{equation*}
    \item Zwei Mengen sind gleich wenn gilt:
        \begin{equation*}
            M_1 = M_2 \Leftrightarrow M_1 \subseteq M_2 \land M_2 \subseteq M_1
        \end{equation*}
    \item $M_1$ heißt echte Teilmenge von $M_2$ wenn gilt:
        \begin{equation*}
            M_1 \subseteq M_2 \land M_1 \neq M_2
        \end{equation*}
        Schreibweise: $M_1 \subset M_2$ oder $M_1 \subsetneq M_2$.
\end{enumerate}

\subsection{Transitivität u.a.}
Für Mengen $M, M_1, M_2, M_3$ gilt stets:
\begin{enumerate}[label= (\alph*)]
    \item Aus $M_1 \subseteq M_2$ und $M_2 \subseteq M_3$ folgt stets:
        $M_1 \subseteq M_3$
    \item
        $M_1 = M_2 \Leftrightarrow M_1 \subseteq M_2 \land M_2 \subseteq M_1$
    \item $M \subseteq M$ und $\emptyset \subseteq M$
\end{enumerate}

\subsection{Verknüpfung von Mengen}
Für Mengen $M_1$ und $M_2$ definiert man:
\begin{enumerate}[label= (\alph*)]
    \item Die Vereinigung von $M_1$ und $M_2$ durch:
        \begin{equation*}
            M_1 \cup M_2 := \{ x: x \in M_1 \lor x \in M_2 \}
        \end{equation*}
    \item Den Schnitt von $M_1$ und $M_2$ durch:
        \begin{equation*}
            M_1 \cap M_2 := \{ x: x \in M_1 \land x \in M_2 \}
        \end{equation*}
    \item Die Differenz von  $M_1$ und $M_2$ durch:
        \begin{equation*}
            M_1 \backslash M_2 := \{ x: x \in M_1 \land x \notin M_2 \}
        \end{equation*}
    \item Das Kartesische Produkt von  $M_1$ und $M_2$ durch:
        \begin{equation*}
            M_1 \times M_2 := \{(a, b): a \in M_1 \land b \in M_2 \}
        \end{equation*}
    \item Das Kartesische Produkt von $M_1$ und $M_1$ durch;
        \begin{equation*}
            {(M_1)}^2 := M_1 \times M_1
        \end{equation*}
\end{enumerate}

\subsection{Potenzmenge}
Für eine Menge $M$ ist durch
\begin{equation*}
    P(M) := \{ A: A \subseteq M \}
\end{equation*}
die Potenzmenge definiert (Menge aller Teilmengen von M).

\subsubsection{Bemerkung}
Hier gilt $\emptyset \in P(M)$.

\subsection{Rechenregeln für Mengen}
Für bel. Mengen $M_1, M_2, M_3$ gilt:
\begin{enumerate}[label= (\alph*)]
    \item Kommutativität:
        \begin{equation*}
            M_1 \cup M_2 = M_2 \cup M_1 \text{ und }
            M_1 \cap M_2 = M_2 \cap M_1
        \end{equation*}
    \item Assoziativität:
        \begin{equation*}
            (M_1 \cup M_2) \cup M_3 = M_1 \cup (M_2 \cup M_3) \text{ und }
            (M_1 \cap M_2) \cap M_3 = M_1 \cap (M_2 \cap M_3)
        \end{equation*}
    \item Distributivgesetz:
        \begin{equation*}
            M_1 \cap (M_2 \cup M_3) = (M_1 \cap M_2) \cup (M_1 \cap M_3)
            \text{ und }
            M_1 \cup (M_2 \cap M_3) = (M_1 \cup M_2) \cap (M_1 \cup M_3)
        \end{equation*}
\end{enumerate}

\subsection{Komplement}
Ist $X$ eine feste Menge und $M \subseteq X$ beliebig, so heißt
\begin{equation*}
    M^c := X \backslash M
\end{equation*}
das Komplement von $M$ (bzgl, $X$).

\subsection{Bemerkung}
Die Schreibweise erfordert das $X$ aus dem Kontext bekannt sein muss.

\subsection{Verknüpfungen über mehrere Elemente}
Für Mengen $M_1, M_2, \ldots, M_n$ mit $n \in \N$ definieren wir die Notation:
\begin{enumerate}[label= (\alph*)]
    \item
        \begin{equation*}
            \setbigcup_{k=1}^n M_k = M_1 \cup M_2 \cup \ldots \cup M_n
        \end{equation*}
    \item
        \begin{equation*}
            \setbigcap_{k=1}^n M_k = M_1 \cap M_2 \cap \ldots \cap M_n
        \end{equation*}
    \item
        \begin{equation*}
            \setbigtimes_{k=1}^n M_k = M_1 \times M_2 \times \cdots \times M_n
        \end{equation*}
\end{enumerate}

\subsection{Wichtige Zusammenhänge}
\begin{enumerate}[label= (\alph*)]
    \item ${(M^c)}^c = M$
    \item $M_1 \subseteq M_2 \Rightarrow {M_2}^c \subseteq {M_1}^c$
    \item ${(M_1 \cup M_2)}^c = {M_1}^c \cap {M_2}^c$
\end{enumerate}

\section{Vollständige Induktion}

\subsection{Summen und Produktzeichen}
Für $m, n \in \Z, m \leq n$ und $a_m, a_{m+1}, \ldots a_n \in \R$ definieren
wir:

\begin{equation*}
    \sum_{k=n}^n a_k := a_m + a_{m+1} + \ldots + a_n
\end{equation*}
und
\begin{equation*}
    \prod_{k=m}^n a_k := a_m \cdot a_{m+1} \cdot \ldots \cdot a_n
\end{equation*}
Falls $m>n$ ist definieren wir $\sum_{k=m}^n a_k := 0$ und
$\prod_{k=m}^n a_k := 1$

\subsection{Prinzip der Vollständigen Induktion}
Gegen seien Aussagen $A(n)$ für $n \geq n_0$ mit $n_0, n \in \Z$
($n_0$ beliebig aber fest). Und es gelte:
\begin{enumerate}[label= (\alph*)]
    \item $A(n_0)$ ist wahr
    \item Für alle $n \geq n_0$ gilt: $A(n) \Rightarrow A(n+1)$
\end{enumerate}

\subsubsection{Bemerkung}
\begin{enumerate}[label= (\alph*)]
    \item $n_0$ wird als Induktionsanfang, $n$ als Induktionsschritt bezeichnet
    \item Nachteil: wir wissen nicht wieso etwas gilt, nur dass es gilt
\end{enumerate}

\subsection{Rechenregeln für Summen}
Für $m,n \in \Z$ und $a_k, b_k, c \in \R$ gilt:
\begin{enumerate}[label= (\alph*)]
    \item Indexverschiebung:
        \begin{equation*}
            \sum_{k=m}^n a_k = \sum_{k=m+l}^{n+l} a_{k-l}
        \end{equation*}
        für beliebiges $l \in \Z$
    \item Trennen von Summen:
        \begin{equation*}
            \sum_{k=m}^n (a_k + b_k) = \sum_{k=m}^n a_k + \sum_{k=m}^n b_k
        \end{equation*}
    \item Konstante Faktoren können aus der Summe ``gezogen'' werden:
        \begin{equation*}
            \sum_{k=m}^n c \cdot a_k = c \cdot \sum_{k=m}^n a_k
        \end{equation*}
    \item ``Teleskopsummen'':
        \begin{equation*}
            \sum_{k=m}^n (a_k - a_{k+1}) = a_m - a_{n+1}
        \end{equation*}
    \item Summe über Konstanten:
        \begin{equation*}
            \sum_{k=m}^n c = c \cdot (n-m+1)
        \end{equation*}
\end{enumerate}

\subsection{Doppelsummen}
Für $n \in \N$ und $a_{ij} \in \R$, $1 \leq i \leq j \leq n$ gilt:
\begin{equation*}
    \sum_{i=1}^n \sum_{j=i}^n a_{ij} = \sum_{j=1}^n \sum_{i=1}^j a_{ij}
\end{equation*}

\subsection{Fakultät und Binomialkoeffizient}
Für $n \in \N_0$ und ein $\alpha \in \R$ heißt
\begin{enumerate}[label= (\alph*)]
    \item die Fakultät von $n$
        \begin{equation*}
            n! := \begin{cases}
                n \cdot (n-1)!&;n \neq 0\\
                1&;n=0
            \end{cases}
        \end{equation*}
    \item den Binomielkoeffizienten
        \begin{equation*}
            {\alpha \choose n} := \frac{\prod\limits_{k=1}^n (\alpha-k+1)}{n!}
        \end{equation*}
\end{enumerate}

\subsection{Rechenregeln für den Binomialkoeffizienten}
Für $n, m \in \N_0$ mit $m \geq n$ und $\alpha \in \R$ gilt:
\begin{enumerate}[label= (\alph*)]
    \item
        \begin{equation*}
            {\alpha \choose n} + {\alpha \choose n+1} =
            {\alpha + 1 \choose n +1}
        \end{equation*}
     \item
        \begin{equation*}
            {m \choose n} = \frac{m!}{n!(m-n)!}
        \end{equation*}
\end{enumerate}

\subsection{Binomischer Lehrsatz}
Für $a, b \in \R$ und $n \in \N_0$ gilt:
\begin{equation*}
    {(a+b)}^n = \sum_{k=0}^n {n \choose k} a^k b^{n-k}
\end{equation*}

\subsection{Definition Betrag}
Für $x \in \R$ heißt
\begin{equation*}
    \abs{x} :=
        \begin{cases}
            x &,x\geq0\\
            -x &,x<0
        \end{cases}
\end{equation*}
der Betrag von $x$

\subsubsection{Bemerkung}
Es gilt:
\begin{enumerate}[label= (\alph*)]
    \item $\abs{x} \geq 0\ \forall x \in \R$
    \item $\abs{x \cdot y} = \abs{x} \cdot \abs{y}\ \forall x,y \in \R$
    \item $\abs{x-a} < \varepsilon \Leftrightarrow a-\varepsilon < x < a+\varepsilon$
    \item $\abs{x} = \max{\{x, -x\}}\ \forall x \in \R$
\end{enumerate}

\subsection{Dreiecksungleichung}
Für alle $x, y \in \R$ gilt:
\begin{enumerate}[label= (\alph*)]
    \item $\abs{x+y} \leq \abs{x} + \abs{y}$ (obere Dreiecksungleichung)
    \item $\abs{x+y} \geq \abs{\abs{x}-\abs{y}}$ (untere Dreiecksungleichung)
\end{enumerate}

\subsubsection{Bemerkung}
Es gilt $x \leq \abs{x}\ \forall x \in \R$.

\section{Funktion und Differentiation}
Eine Funktion (bzw. Abbildung, Operator) $f$ von $X$ nach $Y$ ist eine
Vorschrift, die jedem $x \in X$ ein eindeutig bestimmtes $y \in Y$ zuordnet.
Das $x \in X$ zugeordnete Element aus $Y$ wird mit $f(x)$ bezeichnet.

\subsubsection{Schreibweise}
\begin{equation*}
    f: X \rightarrow Y,\quad x \mapsto f(x)
\end{equation*}


\subsubsection{Bemerkung}
$X$ heißt Definitionsbereich, $Y := \{y \in Y: \exists x \in X \text{ mit }
y=f(x)\}$ die Zielmenge.

\subsection{Injektivität, Surjektivität, Bijektivität}
\begin{enumerate}[label= (\alph*)]
    \item Eine Funktion heißt injektiv, falls gilt:
        \begin{equation*}
            x \neq y \Rightarrow f(x) \neq f(y)\ forall x,y\in X
        \end{equation*}
    \item Eine Funktion heißt surjektiv, falls gilt:
         \begin{equation*}
            \forall y \in Y\ \exists x \in X: y=f(x)
         \end{equation*}
     \item Eine Funktion heißt bijektiv, wenn sie injektiv und
        surjektiv ist.
\end{enumerate}

\subsection{Verknüpfung von Funktionen}
Gegeben seien $f,g: X \rightarrow Y$ und $c \in \R$. Dann definieren wir
die Funktionen
\begin{eqnarray*}
    c \cdot f: &X \rightarrow Y, &x \mapsto (cf)(x):=c \cdot f(x)\\
    f+g: &X \rightarrow Y, &x \mapsto (f+g)(x) := f(x) + g(x)\\
    f \cdot g: &X \rightarrow Y, &x \mapsto (fg)(x) := f(x) \cdot g(x)\\
    \frac{f}{g}: &X \rightarrow Y, &x \mapsto (\frac{f}{g})(x) :=
    \frac{f(x)}{g(x)} \text{ für } x \text{ mit } g(x) \neq 0
\end{eqnarray*}

\subsection{Verkettung von Funktionen}
Seien $f: X \rightarrow Y$ und $g: Y \rightarrow Z$ gegeben, dann heißt
die Funktion
\begin{equation*}
    g \circ f: X \rightarrow Z, x \mapsto (g \circ f)(x):=g(f(x))
\end{equation*}
die Verkettung von $g$ mitt $f$ oder das Kompositum von $g$ mit $f$

\subsection{Stetigkeit und Differenzierbarkeit}
Sei $I \subseteq \R$ ein Intervall und $f: I \rightarrow \R$
\begin{enumerate}[label= (\alph*)]
    \item $f$ heißt stetig in $x_0$, wenn
        $\lim\limits_{x \rightarrow x_0} f(x) = f(x_0)$ gilt
    \item $f$ heißt stetig auf $I$, wenn $f$ in jedem $x_0 \in I$ stetig ist.
    \item $f$ heißt differenzierbar in $x_0$, wenn der Grenzwert
        $\lim\limits_{x \rightarrow x_o}\dfrac{f(x)-f(x_0)}{x-x_0}$
        existiert.
\end{enumerate}

\subsubsection{Bemerkung}
Im Fall der Differenzierbarkeit bezeichnen wir den Grenzwert mit
$f'(x_0)$ (Newton Notation) oder $\ddx f(x_0)$
(Leibniz Notation).

\subsection{Zusammenhang Differentierbarkeit --- Stetigkeit}
Eine differentierbare Funktion ist stets stetig.
\subsubsection{Bemerkung}
Die Ableitung einer differentierbaren Funktion muss hingegen nicht
stetig sein.

\subsection{Verkettung differentierbarer Funktionen}
Seien $g,f:I \rightarrow \R$ differentierbar, dann sind $cf$, $f+g$,
$f \cdot g$ und im Fall $g(x) \neq 0 \forall x \in I$ auch $\dfrac{f}{g}$
differntierbare Funktionen, und es gilt:
\begin{eqnarray*}
    \ddx (c \cdot f)(x) = &(c \cdot f)'(x) = &c \cdot f'(x)\\
    \ddx (f + g)(x) = &(f+g)'(x) = &f'(x) + g'(x)\\
    \ddx (f \cdot g)(x) = &(f \cdot g)'(x) =
        &f'(x)\cdot g(x) + f(x) \cdot g'(x)\\
    \ddx \left(\dfrac{f}{g}\right)(x) = &\left(\dfrac{f}{g}\right)'(x) =
    & \frac{f'(x)g(x)-g'(x)f(x)}{{(g(x))}^2}
\end{eqnarray*}

\subsection{Differentiation von Monomen}
Es sei $f: \R \rightarrow \R$ mit $x \mapsto x^n = f(x)$ und $n \in \Z$.
Dann ist $f$ differentierbar mit $f'(x) = n \cdot x^{n-1}$.

\subsection{Kettenregel}
Gegeben seien Intervalle $I, J \subseteq \R$ und differentierbare
Funktionen $f: I \rightarrow J, g: J \rightarrow \R$. Dann ist auch
$g \circ f$ differentierbar und es gilt:
\begin{equation*}
    \ddx(g \circ f)(x) = \ddx g(f(x)) = (g \circ f)'(x)= g'(f(x)) \cdot f'(x)
\end{equation*}

\subsection{Ableitung der Umkehrfunktion}
Sei $f: I \rightarrow J$ bijektiv und differentierbar dann ist die
Umkehrfunktion $f^{-1}: J \rightarrow I$ ebenfalls differentierbar und
es gilt:
\begin{equation*}
    \ddx f^{-1}(x) = \dfrac{1}{f'(f^{-1}(x))}
\end{equation*}

\section{Elementare Funktionen}%TODO #############################
\section{Integralrechnung}%TODO ##################################
\section{Komplexe Zahlen}%TODO ###################################

\section{Elementare Differentialgleichungen}
\subsection{Definition Rechteck}
\begin{enumerate}[label= (\alph*)]
    \item $I_1, I_2, \ldots, I_n \subseteq \R^n$ seien nicht leeren Intervalle.
        Dann heißt die Menge $M = I_1 \times I_2 \times \ldots \times I_n$
        ein (n-Dimensionales) Rechteck.
    \item Sei $M \subseteq \R^n$ ein Rechteck und $\varphi: M \rightarrow R$ stetig.
        Dann heißt eine Funktion $y: I \rightarrow \R$ die Lösung
        der Differentialgleichung (1. Ordnung)
        \begin{equation*}
            y' = \varphi(t;y)
        \end{equation*}
        wenn gilt:
        \begin{enumerate}[label=\roman*]
            \item $y$ ist stetig differentierbar
            \item $(t,y(t)) \in M \forall t \in I$
            \item $y'(t) = \varphi(t, y(t)) \forall t \in I$
        \end{enumerate}
    \item Sei $M \subseteq \R^2$ ein Rechteck, $\varphi: M \rightarrow \R$
        stetig und $(t_0, y_0) \in M$. Dann heißt $\varphi: I \rightarrow \R$
        eine Lösung des Anfangswertproblems (AWP)
        \begin{equation*}
            y' = \varphi(t,y);\ y(t_0)=y_0
        \end{equation*}
        wenn $y$ eine Lösung von $y'=f(t,y)$ ist und $y(t_0)=y_0$ gilt.
\end{enumerate}

\subsubsection{Bemerkung}
Eine DGL $n$-ter Ordnung mit $n \geq 2$ ist nicht direkt durch die
Definition beschrieben.

Wenn wir aber eine Funktion $\vec{y}: I \subseteq \R \rightarrow \R^2$ definiert
mit:
\begin{eqnarray*}
    y(t) &=& \left( \begin{array}{c}y_1(t)\\y_2(t)\end{array} \right)\\
    y_1(t) &=& x(t)\\
    y_2(t) &=& \dot{y}_1(t) - \dot{x}(t)\\
    \dot{y}_2(t) &=& \ddot{x}(t) = -\frac{a_1}{a_2}\dot{x}(t)-
        \dfrac{a_0}{a_2}x(t) = -\frac{a_1}{a_2}y_2(t)-
        \dfrac{a_0}{a_2}y_1(t)
\end{eqnarray*}

\subsection{Lineare DGL 1. Ordnung}
Sei $I \subseteq R$ ein Interval und $t_0$ ein Punkt in $I$ mit $t_0 - \delta;
t_0+\delta) \subseteq I$ (d.h. nicht auf dem Rand von $I$). Weiter seien
$f,g: I \rightarrow \R$ stetig.

Definiere
\begin{eqnarray*}
    y_0&:& I \rightarrow \R\\
    y_0(t) &=& \exp{\left(\int_{t_0}^t f(u) \text{d}u \right)}\\
    y&:&I \rightarrow \R\\
    y(t)&=&\left(y_0 \cdot \int_{t_0}^t \dfrac{g(u)}{y_0(u)} \text{d}u\right) \cdot
        y_0(t)
\end{eqnarray*}
Dann ist:
\begin{enumerate}[label= (\alph*)]
    \item $y_0$ eine Lösung von $y' = f(t)y;\ y(t_0)=1$
    \item $y$ eine Lösung von $y' = f(t)y + g(t);\ y(t_0)=y_0$
\end{enumerate}


    \chapter{Grenzwerte}
    \section{Gruppen und Körper}
\subsection{Gruppen}
Eine Gruppe ist definiert als ein Tuppel aus einer (nicht-leeren) Menge
und einer Verknüpfung.
Eine Gruppe erfüllt die folgenden Axiome (seien $a,b,c \in \mathbb{G}$):
\begin{align*}
    a \circ (b \circ c) &= (a \circ b) \circ c &&\text{(Assoziativität)}\\
    a \circ \varepsilon &= a &&\text{(Rechtsneutrales Element)}\\
    a \circ a' &= \varepsilon  &&\text{(Rechtsinverses Element)}
\end{align*}
Eine abelsche Gruppe erfüllt des weiteren:
\begin{align*}
    a \circ b &= b \circ a  &&\text{(Kommutativität)}
\end{align*}

\subsection{Körper}
Ein Körper ist definiert als eine Menge mit mindestens zwei Elementen
(0 und 1) und zwei Verknüfungen.
\begin{eqnarray*}
    +: \K \times \K &\rightarrow& \K \\
    \cdot: \K \times \K &\rightarrow& \K
\end{eqnarray*}
$\K$ ist bezüglich der Addition und der Multiplikation (genauer: $\K \backslash \{0\}$) ein abelscher
Körper, das heißt es gilt (seien $a,b,c \in \K$):
\begin{align*}
    a + (b + c) &= (a + b) + c &&\text{(Assoziativität bez.\ der Addition)}\\
    a + 0 &= a &&\text{(Existenz einer 0)}\\
    a + (-a) &= 0 &&\text{(Existenz eines Inversen bez.\ der Addition)}\\
    a + b &= b + a &&\text{(Kommutativität bez.\ der Addition)}\\
    a \cdot (b \cdot c) &= (a \cdot b) \cdot c &&\text{(Assoziativität bez.\ der Multiplikation)}\\
    a \cdot 1 &= a &&\text{(Existenz einer 1)}\\
    a \cdot a^{-1} &= 1\quad \forall a \neq 0 &&\text{(Existenz eines Inversen bez.\ der Multiplikation)}\\
    a \cdot b &= b \cdot a &&\text{(Kommutativität bezüglich der Multiplikation)}
\end{align*}
außerdem gilt:
\begin{align*}
    a \cdot (b + c) &= (a \cdot b) + (a \cdot c) &&\text{(Distributivgesetz)}
\end{align*}
\subsubsection{Bemerkung}
$\Q$, $\R$ und $\C$ sind Körper.
$\Z$ und $\N$ nicht (kein additiv inverses bei $\N$,
kein multiplikativ inverses bei beiden).


\subsection{Angeordnete Körper}
Ein Körper heißt angeordent wenn folgende Axiome erfüllt sind
(seien $a,b,c \in \K$):
\begin{eqnarray*}
    a<b\ \lor\ &b<a&\ \lor\ a=b\\
    a<b\ \land\ b<c\ &\Rightarrow&\ a<c\\
    a<b &\Rightarrow& a+c<b+c\\
    a<b \land c>0 &\Rightarrow& a*c<b*c
\end{eqnarray*}
\subsubsection{Bemerkung}
$\Q$ und $\R$ sind angeordnete Körper. Für $\C$
kann keine Ordnungsrelation definiert werden so das alle Axiome erfüllt
sind.

\subsubsection{Gebräuchliche Definition zu angeordenten Körpern}
Es gilt $0<1$, sonst Widerspruch in (O3).

 Die Ordnungsrelation wird dann definiert durch:
\begin{eqnarray*}
    2&:=& 1+1 \\
    3&:=& 2+1 \\
    4&:=&3+1 \\
    &\vdots&
\end{eqnarray*}

 Die Natürlichen Zahlen werden Induktiv definiert:
\begin{enumerate}
    \item $1 \in \N$
    \item $n \in \N \Rightarrow (n+1)\in\N$
\end{enumerate}

\subsubsection{Bemerkung}
Aus 2.\ lässt sich direkt ableiten das $\N$ nach oben unbeschränkt ist (Archimedisches Prinzip).

\subsubsection{Vollständig Angeordnete Körper}
Ein Körper heißt Vollständig, falls jede nach oben beschränkte, nicht-leere
Teilmenge ein Supremum besitzt.

$\Rightarrow$ $\R$ ist der einzige Vollständig angeordnete Körper.

\subsubsection{Bemerkung} $\Q$ ist nicht vollständig angeordnet, da
$A := \{x | x^2 \leq 2\} \subset \Q$ kein Supremum besitzt
(Supremum ist $\sqrt{2} \notin \Q$).


\subsection{Minimum und Maximum}
Sei $\K$ ein angeordnter Körper und $A \subset$ $\K$ dann heißt
$m$ Minimum falls gilt:
\begin{enumerate}
    \item $m \in \K$
    \item $a \geq m\ \forall a \in A$
\end{enumerate}
Analog ist das Maximum definiert:
Sei $\K$ ein angeordnter Körper und $A \subset$ $\K$ dann heißt
$m$ Maximum falls gilt:
\begin{enumerate}
    \item $m \in \K$
    \item $a \leq m\ \forall a \in A$
\end{enumerate}
\textbf{Schreibweisen:}
$m = \min{(A)}$ bzw.\ $m = \max{(A)}$\\
\subsubsection{Bemerkung}
Minimum und Maximum exisitieren nicht immer.\\
\textbf{Beispiel:} $A := \{x | x>0\}\subset\ \mathbb{R}$
hat nicht 0 als Minimum da $0 \notin A$ und kein beliebiges $m$ da $\tilde{m} := \frac{m}{2} < m\ \forall m \in A$

\subsection{Obere und untere Schranke}
Sei $\K$ ein angeordenter Körper und $A \subset$ $\K$ dann ist $s$ untere
Schranke falls gilt:
\begin{itemize}
    \item $s \leq a\ \forall a \in A$
\end{itemize}

 Analog ist die obere Schranke definiert:
Sei $\K$ ein angeordenter Körper und $A \subset$ $\K$ dann ist $s$ obere
Schranke falls gilt:
\begin{itemize}
    \item $s \geq a\ \forall a \in A$
\end{itemize}

\subsubsection{Bemerkung} Hat eine Menge eine obere (bzw.\ untere) Schranke
heißt er nach oben (bzw.\ unten) beschränkt. Ist eine Menge nach unten und
oben beschränkt bezeichnet man sie als beschränkt.

\subsection{Supremum und Infimum}
$s$ heißt Infimum (größte untere Schranke) falls gilt:
\begin{itemize}
    \item $s$ ist untere Schranke
    \item Falls $\tilde{s}$ ebenfalls untere Schranke ist gilt
    $s\geq\tilde{s}$
\end{itemize}

 Analog ist das Supremum definiert: $s$ heißt Supremum (kleinste obere Schranke) falls gilt:
\begin{itemize}
    \item $s$ ist obere Schranke
    \item Falls $\tilde{s}$ ebenfalls obere Schranke ist gilt
    $s\leq\tilde{s}$
\end{itemize}

\textbf{Schreibweise:}
$s = \inf{(A)}$ bzw. $s = \sup{(A)}$

\subsubsection{Bemerkung}
Wenn Minimum (bzw. Maximum) existieren sind diese gleich dem
Infimum (bzw. Supremum).

\section{Folgen}
Eine Folge $a_n$ ist definiert als eine Funktion:
\begin{equation*}
    a_n := \varphi: \mathbb{N} \rightarrow \mathbb{M} \subset \mathbb{R}
\end{equation*}
oder auch $\an$.

\subsection{Konvergenz}
Eine Folge $a_n$ heißt konvergent wenn gilt:
\begin{equation*}
    \forall \varepsilon>0\ \exists~n_0(\varepsilon):\ \abs{a_n - a} < \varepsilon\ \forall n > n_0(\varepsilon)
\end{equation*}

\subsubsection{Bemerkung}
Der Grenzwert ist eindeutig, d.h.\ es existiert nur ein Grenzwert.

\subsubsection{Schreibweise}
Falls $a_n$ gegen $a$ konvergiert schreibt man:
\begin{equation*}
    \lim_{n \rightarrow \infty} a_n = a
\end{equation*}

\subsection{Bestimmte Divergenz}
Eine Folge $a_n$ heißt bestimmt Divergent wenn gilt
\begin{equation*}
    \forall x \in \mathbb{R}\ \exists n(x):\ a_n>x \text{ bzw. } a_n<x
\end{equation*}
\textbf{Schreibweise:}
\begin{equation*}
    \lim_{n \rightarrow \infty} a_n = \infty \text{ bzw. } -\infty
\end{equation*}

\subsection{Beschränktheit}
Eine Folge heißt beschränkt wenn gilt:
\begin{equation*}
    \abs{a_n} < c\ \forall n
\end{equation*}

\subsubsection{Beschränktheit nach oben/unten}
Eine Folge heißt nach oben (bzw.~unten) beschränkt wenn gilt:
\begin{equation*}
    a_n < c\ \forall n \in \mathbb{N} \text{ bzw. }a_n > c\ \forall n \in \mathbb{N}
\end{equation*}

\subsection{Zusammenhang Konvergenz --- Beschränktheit}
Jede konvergente Folge ist beschränkt.

\subsection{Grenzwertrechenregeln}
Seien $\an$, ${(b_n)}_{n=1}^\infty$, ${(c_n)}_{n=1}^\infty$ Folgen
in $\C$ mit:
\begin{equation*}
    \lim_{n \rightarrow \infty} a_n = a \text{ und }
    \lim_{n \rightarrow \infty} b_n = b
\end{equation*}

Dann gilt:

\begin{itemize}
    \item $\lim\limits_{n \rightarrow \infty} \abs{a_n} = \abs{a}$
    \item $\lim\limits_{n \rightarrow \infty}(a_n + b_n) = a + b$
    \item $\lim\limits_{n \rightarrow \infty}(a_n \cdot b_n) = a \cdot b$
    \item Falls $b \neq 0$:
    $\lim\limits_{n \rightarrow \infty}\dfrac{a_n}{b_n} = \dfrac{a}{b}$
\end{itemize}

\subsection{Sandwich Theorem u.a.}
Seien $\an$, ${(b_n)}_{n=1}^\infty$, ${(c_n)}_{n=1}^\infty$
Folgen in $\R$ mit:
\begin{equation*}
    \lim_{n \rightarrow \infty}a_n = a \text{, }
    \lim_{n \rightarrow \infty}b_n = b \text{ und } \gamma \in \mathbb{R}
\end{equation*}

Dann gilt:

\begin{itemize}
    \item $a_n \leq \gamma\ \forall n \in \mathbb{N} \Rightarrow a \leq \gamma$
    \item $a_n \geq \gamma\ \forall n \in \mathbb{N} \Rightarrow a \geq \gamma$
    \item $a_n \leq b_n\ \forall n \in \mathbb{N} \Rightarrow a \leq b$
    \item $a_n \leq c_n \leq b_n\ \forall n \in \mathbb{N} \wedge a=b
    \Rightarrow c=\lim\limits_{n \rightarrow \infty} c_n = a = b$
\end{itemize}

\subsection{Monotonie}
Eine Folge $\an$ in $\R$ heißt:
\begin{itemize}
    \item Monoton wachsend falls: $a_{n+1} \geq {a_n}\ \forall n \in \mathbb{N}$ (Schreibweise: $a_n \nearrow$)
    \item Monoton fallend falls: $a_{n+1} \leq {a_n}\ \forall n \in \mathbb{N}$ (Schreibweise: $a_n \searrow$)
    \item Streng monoton wachsend falls: $a_{n+1} > {a_n}\ \forall n \in \mathbb{N}$ (Schreibweise: $a_n \uparrow$)
    \item Streng monoton fallend falls: $a_{n+1} < {a_n}\ \forall n \in \mathbb{N}$ (Schreibweise: $a_n \downarrow$)
\end{itemize}

\subsection{Zusammenhang Monotonie und Beschränktheit}
Jede Monotone und beschränkte Folge konvergiert.

\section{Häufungswerte}
Häufungswerte sind Grenzwerte einer Teilfolge.

\subsection{Teilfolgen}
Eine Folge ${(b_n)}_{n=1}^\infty$ heißt Teilfolge von $\an$, wenn
eine streng monotone Funktion $\varphi: \mathbb{N} \rightarrow \mathbb{N}$ exisitiert
mit $b_n = a_{\varphi(n)}$.

\subsection{Teilfolgen einer Konvergenten Folge}
Sei $\an$ eine konvergente Folge in $\C$ mit:
$\lim\limits_{n \rightarrow \infty} a_n = a$ und ${(b_n)}_{n=1}^\infty$
sei eine Teilfolge. Dann gilt $\lim\limits_{n \rightarrow \infty} b_n = a$.

\subsection{Häufungswerte} Sei $\an$ eine Folge in $\C$. Dann heißt
$a \in\C$\ ein Häufungswert einer Folge, falls eine Teilfolge gegen $a$ konvergiert.

\subsection{Limes superior/inferior}
Sei $\an$ eine reele Folge, dann heißt:
\begin{equation*}
    \lim_{n \rightarrow \infty} \sup a_n :=
    \varlimsup_{n \rightarrow \infty} a_n :=
    \sup \{ x \in \R, a_n > x\ \infty\text{-oft}\}
\end{equation*}
der Limes superior von $\an$ und
\begin{equation*}
    \lim_{n \rightarrow \infty} \inf a_n :=
    \varliminf_{n \rightarrow \infty} a_n :=
    \inf \{ x \in \R, a_n < x\ \infty\text{-oft}\}
\end{equation*}
der Limes inferior von $\an$.

\subsection{Charakterisierung limsup/liminf}
Sei $\an$ eine reelle Folge und $s \in \R$. Dann gilt:
\begin{enumerate}[label= (\alph*)]
    \item
        \begin{equation*}
            s = \varlimsup_{n \rightarrow \infty} a_n \Leftrightarrow
            \forall \varepsilon > 0 \text{ gilt: }
        \end{equation*}
        \begin{enumerate}[label=\roman*]
            \item $a_n < s + \varepsilon$ für fast alle $n$
            \item $a_n > s - \varepsilon$ für $\infty$-viele $n$
        \end{enumerate}
    \item
        \begin{equation*}
            s = \varliminf_{n \rightarrow \infty} a_n \Leftrightarrow
            \forall \varepsilon > 0 \text{ gilt: }
        \end{equation*}
        \begin{enumerate}[label=\roman*]
            \item $a_n > s - \varepsilon$ für fast alle $n$
            \item $a_n < s + \varepsilon$ für $\infty$-viele $n$
        \end{enumerate}
\end{enumerate}

\subsection{Konvergenz und limsup/liminf}
Eine beschränkte Folge $\an$ in $\R$ konvergiert $\Leftrightarrow$
\begin{equation*}
     \varlimsup_{n \rightarrow \infty} a_n
    = \varliminf_{n \rightarrow \infty} a_n
\end{equation*}

\subsection{Satz von Bolzano-Weierstraß}
Jede beschränkte Folge in $\C$ besitzt eine konvergente Teilfolge.

\subsection{Cauchy-Kriterium}
Sei $\an$ eine Folge in $\C$, dann gilt
\begin{equation*}
    \an \text{ konv.} \Leftrightarrow
    \forall \varepsilon > 0\ \exists n_0(\varepsilon):\
    \abs{a_n - a_m} < \varepsilon\ \forall n,m > n_0(\varepsilon)
\end{equation*}
\subsubsection{Bemerkung} Im Gegensatz zur Definition der Folgenkonvergenz muss der
Grenzwert nicht bekannt sein.

\section{Unendliche Reihen}
\subsection{Definition}
Sei $\an$ eine Folge in $\C$, dan  heißt die durch
\begin{equation*}
    s_n = \sum_{k=1}^n a_k
\end{equation*}
definiert Folge $\sn$ eine Folge von Partialsummen der
unendlichen Reihe:
\begin{equation*}
    \sum_{k=1}^\infty a_k
\end{equation*}
Falls die Folge $\sn$ konvergiert setzten wir:
\begin{equation*}
    \lim_{n \rightarrow \infty} s_n =: \sum_{k=1}^\infty a_k
\end{equation*}

\subsection{Cauchy-Kriterium für unendliche Reihen}
Sei $\sum_{k=1}^\infty a_k$ eine $\infty$-Reihe, dann gilt:
\begin{equation*}
    \sum_{k=1}^\infty a_k \text{ konv.} \Leftrightarrow
    \forall \varepsilon > 0\ \exists n_0(\varepsilon):\
    \abs{\sum_{k=m}^n a_k} < \varepsilon\
    \forall n, m>n_0(\varepsilon)
\end{equation*}
und:
\begin{equation*}
    \sum_{k=1}^\infty a_k \text{ konv.} \Rightarrow
    \lim_{n \rightarrow \infty} a_n = 0
\end{equation*}

\subsection{Grenzwertrechenregeln für unendliche Reihen}
Seien
\begin{equation*}
    \sum_{n=1}^\infty a_k \text{ und } \sum_{n=1}^\infty b_k
    \text{ gegeben und } \alpha,\beta \in \C
\end{equation*}
dann gilt:
\begin{enumerate}[label= (\alph*)]
    \item
        \begin{eqnarray*}
            \sum_{n=1}^\infty a_k \text{ und } \sum_{n=1}^\infty b_k \text{ konv.:}\\
            \Rightarrow \sum_{k=1}^\infty(\alpha a_k &+& \beta b_k) \text{ konv.}\\
            \text{und: }
            \sum_{k=1}^\infty(\alpha a_k + \beta b_k) &=& \alpha \sum_{n=1}^\infty a_k
            + \beta \sum_{n=1}^\infty b_k
        \end{eqnarray*}\\
    \item
        \begin{equation*}
            \sum_{k=1}^\infty a_k \text{ konv.} \Leftrightarrow
            \sum_{k=1}^\infty \ReP(a_k) \text{ und }
            \sum_{k=1}^\infty \ImP(a_k) \text{ konv.}
        \end{equation*}
    \item
        \begin{equation*}
            \sum_{k=1}^\infty a_k \text{ konv.} \Leftrightarrow
            \text{ die Restreihe } R_n := \sum_{k=n}^\infty a_k \text{ konv.\ gegen }0\\
            \Rightarrow \lim_{n \rightarrow \infty} R_n = 0
        \end{equation*}
\end{enumerate}

\subsection{Positive Folgen}
Es sei $\an$ eine Folge mit $\an \in \left[0, \infty\right)$ dann gilt:
\begin{equation*}
    \sum_{k=1}^\infty a_k \text{ konv. } \Leftrightarrow
    \text{ Folge der Partialsummen }\sum_{k=1}^n a_k \text{ ist beschr.}
\end{equation*}

\subsection{Leibniz-Kriterium}
Sei $\an$ eine monoton fallende, reele Folge. Dann gilt falls
$\lim\limits_{n \rightarrow \infty} a_n = 0$ ist, konv.\ die sogennante
alternierende Reihe
\begin{equation*}
    \sum_{k=1}^\infty {(-1)}^k a_k
\end{equation*}


\subsection{Absolute Konvergenz}
Eine Reihe $\sum_{k=1}^\infty a_k$ heißt absolut konvergent, wenn
\begin{equation*}
    \sum_{k=1}^\infty \abs{a_k}
\end{equation*}
konvergiert.

\subsubsection{Bemerkung} Jede absolut konvergente Reihe ist auch konvergent.

\subsection{Majorantenkriterium}
Seien $\sum_{k=1}^\infty a_k$ und $\sum_{k=1}^\infty b_k$ mit $b_k \geq 0$
gegeben.

Wenn $\sum_{k=1}^\infty b_k$ konv.\ und ein $c>0$ ex.\ mit
$$\abs{a_k} \leq c \cdot \abs{b_k}$$ für fast alle k, dann konv.
$\sum_{k=1}^\infty a_k$ absolut.

\subsection{Minorantenkriterium}
Falls ein $c > 0$ ex.\ mit $a_k \geq c \cdot b_k > 0$ für fast alle k,
dann:
\begin{equation*}
    \sum_{k=1}^\infty b_k \text{ div. }
    \Rightarrow \sum_{k=1}^\infty a_k \text{ div.}
\end{equation*}

\subsection{Wurzel- und Quotientenkriterium}
Sei $\sum_{k=1}^\infty a_k$ gegeben. Dann gilt:
\begin{enumerate}[label= (\alph*)]
    \item Wenn
        \begin{equation*}
            \varlimsup_{n \rightarrow \infty}
            \nthSqrt{n}{\abs{a_n}} < 1
        \end{equation*}
        gilt, dann konv. $\sum_{k=1}^\infty a_k$ absolut.

        Wenn
        \begin{equation*}
            \varlimsup_{n \rightarrow \infty}
            \nthSqrt{n}{\abs{a_n}} > 1
        \end{equation*}
        gilt, dann div. $\sum_{k=1}^\infty a_k$.
    \item Wenn $a_n \neq 0\ \forall n$ und
        \begin{equation*}
            \varlimsup_{n \rightarrow \infty}
            \abs{\frac{a_{n+1}}{a_n}} < 1
        \end{equation*}
        gilt, dann konv. $\sum_{k=1}^\infty a_k$ absolut.

        Wenn $a_n \neq 0\ \forall n$ und
        \begin{equation*}
            \varlimsup_{n \rightarrow \infty}
            \abs{\frac{a_{n+1}}{a_n}} > 1
        \end{equation*}
        gilt, dann divergiert. $\sum_{k=1}^\infty a_k$.
\end{enumerate}
\subsubsection{Bemerkung} Wenn das Wurzelkriterium keine Aussage macht, kann das
Quotientenkriterium trotzdem eine Aussage machen.

\subsection{Umordnung einer Reihe}
Eine Reihe $\sum_{k=1}^\infty b_k$ heißt Umordnung der Reihe
$\sum_{k=1}^\infty a_k$,
wenn eine bij. Abb $\varphi: \N \rightarrow \N$ ex.\ mit $b_k = a_{\varphi(k)}$.

\subsubsection{Bemerkung}
Die Reihe konvergiert nur gegen den selben Wert, wenn $\sum_{k=1}^\infty a_k$
absolut konvergent ist.

\subsection{Cauchy-Produkt}
Die Reihen  $\sum_{k=1}^\infty b_k$ und $\sum_{k=1}^\infty a_k$ seien absolut
konv.. Dann gilt:
\begin{equation*}
    \left(\sum_{k=0}^\infty a_k\right) \cdot \left(\sum_{k=0}^\infty b_k\right) =
    \sum_{k=0}^\infty \left(\sum_{j=0}^k a_j \cdot b_{k-j}\right) =
    \sum_{k=0}^\infty c_k
\end{equation*}
und $\sum_{k=0}^\infty c_k$ konv.\ ebenfalls absolut.

\subsection{Cauchy-Verdichtungssatz}
\begin{equation*}
    \sum_{n=1}^\infty a_n \text{ konv. } \Leftrightarrow
    \sum_{k=1}^\infty 2^k a_{2^k} \text{ konv.}
\end{equation*}


\section{Potenzreihen}

\subsection{Definition}
Sei $\an$ eine Folge in $\C$ und $z_0 \in \C$. Dann heißt
\begin{equation*}
    \sum_{k=0}^\infty a_k \cdot {(z - z_0)}^k
\end{equation*}
eine Potenzreihe mit Entwicklungspunkt $z_0$ und Koeffizienten
$a_n$.

\subsubsection{Bemerkung} Viele wichtige Funktionen können als Potenzreihen dargestellt
werden.

\subsection{Hadamard (Konvergenzradius mit Wurzelkriterium)}
Sei $\sum_{k=0}^\infty a_k {(z-z_o)}^k$ eine PR.\ Definiere
\begin{equation*}
    R := \frac{1}{\varlimsup\limits_{n \rightarrow \infty} \nthSqrt{n}{\abs{a_n}}}
\end{equation*}

Dabei sei $R:=\infty$, falls
$\varlimsup\limits_{n \rightarrow \infty} \nthSqrt{n}{\abs{a_n}} = 0$ und
$R=0$ falls
$\varlimsup\limits_{n \rightarrow \infty} \nthSqrt{n}{\abs{a_n}} = \infty$.

Dann konv.\ die PR absolut, falls $\abs{z - z_0} < R$ und divergiert falls
$\abs{z - z_0} > R$.

\subsubsection{Bemerkung I} Für $\abs{z - z_0} = R$ wird keine Aussage gemacht.

\subsubsection{Bemerkung II} $R$ heißt der Konvergenzradius der Potenzreihe.

\subsection{Konvergenzradius mit Quotientenkriterium}
Sei $\sum_{k=0}^\infty a_k {(z - z_0)}^k$ eine PR.\@ Der Potenzradius kann
ebenfalls berechnet werden durch:
\begin{equation*}
    R = \varlimsup_{n \rightarrow \infty} \abs{\frac{a_n}{a_{n+1}}}
\end{equation*}

\subsection{Hinweis}
Es gilt:
\begin{equation*}
    \lim_{n \rightarrow \infty} \nthSqrt{n}{n} = 1
\end{equation*}

\subsection{Integration und Differentiation von Potenzreihen}
Sei $\sum_{k=0}^\infty a_k {(z - z_0)}^k$ mit Konvergenzradius $R$. Dann besitzen
auch die Potenzreihen
\begin{equation*}
    \sum_{k=0}^\infty k\: a_k {(z - z_0)}^{k-1} \text{ und }
    \sum_{k=0}^\infty \frac{a_k}{k+1} {(z-z_0)}^{k+1}
\end{equation*}
den Konvergenzradius R.

\subsection{Cauchy-Produkt für Potenzreihen}
Seien $\sum_{k=0}^\infty a_k {(z-z_0)}^k$ und
$\sum_{k=0}^\infty b_k {(z-z_0)}^k$ Potenzreihen, die den Konvergenzradius
$R_1$ bzw.\ $R_2$ besitzen. Dann besitzt
\begin{equation*}
        \sum_{k=0}^\infty c_k {(z-z_0)}^k \text{ mit }
        c_k = \sum_{l=0}^k a_l \cdot b_{k-l}
\end{equation*}
den Konvergenzradius $R = \min \{R_1, R_2\}$.

\subsection{Wichtige Potenzreihen}
\begin{enumerate}[label= (\alph*)]
    \item Die Expontentialfunktion ist definiert durch:
        \begin{equation*}
            \exp: \C \rightarrow \C\quad z \mapsto \exp(z) :=
            \sum_{k=0}^\infty \frac{z^k}{k!}
        \end{equation*}
    \item Die Trigonometrischen Funktionen sind definiert durch:
        \begin{eqnarray*}
            \sin: \C \rightarrow \C\quad z \mapsto \sin(z) &:=&
            \sum_{k=0}^\infty \frac{{(-1)}^k}{(2k+1)!}z^{2k+1}\\
            \cos: \C \rightarrow \C\quad z \mapsto \cos(z) &:=&
            \sum_{k=0}^\infty \frac{{(-1)}^k}{(2k)!}z^{2k}
        \end{eqnarray*}
    \item Tangens und Cotangens sind dann definiert als:
        \begin{eqnarray*}
            \tan: \{z \in \C:\ \cos(z) \neq 0 \} \rightarrow \C\quad
            z \mapsto \tan(z)&:=&\frac{\sin(z)}{\cos(z)}\\
            \cot: \{z \in \C:\ \sin(z) \neq 0 \} \rightarrow \C\quad
            z \mapsto \cot(z)&:=&\frac{\cos(z)}{\sin(z)}
        \end{eqnarray*}
\end{enumerate}

\subsection{Alternative Definiton der Exponentialfunktion}
\begin{equation*}
    \forall z \in \C \text{ gilt }
    \lim_{n \rightarrow \infty} {\left(1 + \frac{z}{n}\right)}^n = \exp{(z)}
\end{equation*}

\section{Funktionsgrenzwerte}

\subsection{Bemerkung}
In diesem Intervall bezeichnet $I$ stets ein offenes Intervall und
$\overline{I}$ dessen sog.\ Abschluss z.B.:
\begin{enumerate}[label= (\alph*)]
    \item $I = (a, b)$ und $\overline{I} = [a, b]$
    \item $I = (-\infty, b)$ und $\overline{I} = (-\infty, b]$
    \item $I = (a, \infty)$ und $\overline{I} = [a, \infty)$
    \item $I = (\infty, \infty)$ und $\overline{I} = (\infty, \infty)$
\end{enumerate}

\subsection{Epsilon-Umgebung}
Für $x_0 \in \R$ und $\varepsilon > 0$ heißt
\begin{equation*}
    U_e(x_0) := \{ x \in \R: \abs{x - x_0}<\varepsilon \} =
    (x_0 - \varepsilon, x_0+\varepsilon)
\end{equation*}
die $\varepsilon$-Umgebung von $x_0$. Und
\begin{equation*}
    \dot{U}_e(x_0) := U_e(x_0)\backslash \{0\} =
    (x_0-\varepsilon, x_0) \cup (x_0, x_0+\varepsilon)
\end{equation*}
die punktierte $\varepsilon$-Umgebung von $x_0$.

\subsection{Funktionsgrenzwerte (über Delta-Epsilon-Kriterium)}
Sei $f: I \rightarrow \R$ und $x_0 \in I$
\begin{enumerate}[label= (\alph*)]
    \item $f$ konv.\ gegen ein $a\in \R$ für $x \rightarrow x_0$
        (kurz: $\lim\limits_{x \rightarrow x_0} f(x)=a$) wenn gilt
        \begin{equation*}
            \forall \varepsilon > 0\ \exists \delta(\varepsilon):\
            \abs{f(x) - a} < \varepsilon\ \forall x \text{ mit }
            \abs{x- x_0} < \delta(\varepsilon) \text{ und } x \neq x_0
        \end{equation*}
        Schreibweise:
        \begin{equation*}
            \lim_{x \rightarrow x_0} f(x) = a
            \text{ oder } f(x)=a \text{ für } x \rightarrow x_0
        \end{equation*}
    \item Sei $x_o \in I$, dann konv.\ f einseitig von links
        gegen $a\in\R$ wenn gilt:
        \begin{equation*}
            \forall \varepsilon > 0\  \exists \delta(\varepsilon):\
            \abs{f(x) -a} < \varepsilon\ \forall x \in
            (x_0 - \delta{\varepsilon}, x_0)
         \end{equation*}
        Schreibweise:
        \begin{equation*}
            \lim_{x \rightarrow x_{0^-}} f(x) = a
        \end{equation*}
    \item Sei $x_o \in I$, dann konv.\ f einseitig von rechts
        gegen $a\in\R$ wenn gilt:
        \begin{equation*}
            \forall \varepsilon > 0\  \exists \delta(\varepsilon):\
            \abs{f(x) -a} < \varepsilon\ \forall x \in
            (x_0 , x_0 + \delta{\varepsilon})
         \end{equation*}
        Schreibweise:
        \begin{equation*}
            \lim_{x \rightarrow x_{0^+}} f(x) = a
        \end{equation*}
    \item Sei $I = (\alpha, \infty)$ (bzw. $I = (-\infty, \beta)$) dann
        konv. $f$ gegen $a$ für $x \rightarrow \infty$
        (bzw. $x \rightarrow -\infty$) wenn gilt:
        \begin{equation*}
            \forall \varepsilon > 0\ \exists x_1(\varepsilon):\
            \abs{f(x) - a}<\varepsilon\ \forall x \in I:\ x > x_1(\varepsilon)
            \text{ (bzw. } x < x_1(\varepsilon) \text{)}
        \end{equation*}
\end{enumerate}

\subsection{Folgenkriterium}
Sei $f: I \rightarrow \R$ und $x_0 \in \overline{I}, u \in \R$ dann gilt
$\lim\limits_{x \rightarrow \infty} f(x) = a \Leftrightarrow$

\begin{equation*}
    \begin{cases}
        \text{Für eine beliebe Folge } {(x_n)}_{n=1}^\infty \text{ mit}\\
        \text{(i)} x_n \neq x_0 \forall n\\
        \text{(ii)} \lim\limits_{n \rightarrow \infty} x_n = x_0\\
        \text{gilt stets:}\\
        \lim\limits_{n \rightarrow \infty} f(x_n) = a
    \end{cases}
\end{equation*}

\subsection{Rechenregeln für Funktionsgrenzwerte}
Seien $f, g: I \rightarrow \R$ und $x_0 \in I$ und gelte
\begin{equation*}
    \lim_{x \rightarrow x_0} f(x) = a \text{, }
    \lim_{x \rightarrow x_0} g(x) = b
\end{equation*}
Dann gilt:
\begin{enumerate}[label= (\alph*)]
    \item
        \begin{equation*}
            \lim_{x \rightarrow x_0} (\alpha \cdot f(x)) = \alpha \cdot a
        \end{equation*}
    \item
        \begin{equation*}
            \lim_{x \rightarrow x_0} (g(x) + f(x)) = a + b
        \end{equation*}
    \item
        \begin{equation*}
            \lim_{x \rightarrow x_0} (g(x) \cdot f(x)) = a \cdot b
        \end{equation*}
    \item
        \begin{equation*}
            \lim_{x \rightarrow x_0} \left(\frac{f(x)}{g(x)}\right) = \frac{a}{b}
            \qquad\text{falls } b \neq 0
        \end{equation*}
\end{enumerate}

\subsection{Cauchy-Kriterium für Funktionsgrenzwerte}
Sei $f: I \rightarrow \R$ und $x_0 \in I$ dann ex.
$\lim\limits_{x \rightarrow x_0} f(x) \Leftrightarrow$
\begin{equation*}
    \forall \varepsilon > 0\ \exists \delta(\varepsilon):\
    \abs{f(x) - f(y)}<\varepsilon\ \forall x,y \in I \text{ mit }
    0 < \abs{x-x_o} < \delta(\varepsilon) \text{ und }
    0 < \abs{y-x_0} < \delta(\varepsilon)
\end{equation*}

\subsection{Bestimmte Divergenz}
Sei $f: I \rightarrow \R ,\ x_0 \in I$ dann definieren wir die bestimmte
Divergenz (uneigentliche Konvergenz) von ($f \rightarrow \infty$) durch
\begin{equation*}
    \lim_{x \rightarrow x_0} f(x) = \infty \Leftrightarrow
    \forall c>0\ \exists \delta(c): f(x) > c\ \forall x \text{ mit }
    0 < \abs{x - x_0} < \delta(c)
\end{equation*}
Analog definieren man links- und rechtsseitig Divergenz gegen $\infty$ bzw.
$-\infty$.

\subsection{Monotone Funktionen}
Sei $f: I \rightarrow \R$ dann heißt (auf $I$)
\begin{enumerate}[label= (\alph*)]
    \item monoton wachsend ($f \nearrow$), falls gilt
        \begin{equation*}
            x < y \Rightarrow f(x) \leq f(y)
        \end{equation*}
    \item streng monoton wachsend ($f \uparrow$), falls gilt
        \begin{equation*}
            x < y \Rightarrow f(x) < f(y)
        \end{equation*}
    \item monoton fallend ($f \searrow$), falls gilt
        \begin{equation*}
            x < y \Rightarrow f(x) \geq f(y)
        \end{equation*}
    \item streng monoton fallend ($f \downarrow$)
        \begin{equation*}
            x < y \Rightarrow f(x) > f(y)
        \end{equation*}
    \item monoton falls $f$ monoton fallend oder monoton steigend ist
    \item streng monoton falls $f$ streng monoton fallend oder streng monoton
        steigend ist
    \item Beschränkt falls gilt:
        \begin{equation*}
            \exists c:\ \abs{f(x)} < c\ \forall x \in I
        \end{equation*}
\end{enumerate}

\subsection{Grenzwerte an Intervallgrenzen}
Sei $a \leq b$ und $f: (a, b) \rightarrow \R$ monoton und beschränkt, dann
ex.
\begin{equation*}
    \lim_{x \rightarrow b^-} f(x) \text{ und } \lim_{x \rightarrow a^+} f(x)
\end{equation*}

\section{Stetigkeit}

\subsection{Anschaulich}
Graph einer Funktion kann ohne Absetzen gezeichnet werden $\Leftrightarrow$

Es gibt keine Sprünge $\Leftrightarrow$

$f: I \rightarrow \R$ an keiner Stelle $x_0 \in I$ ist ein Sprung $\Leftrightarrow$

$\forall x_0 \in I:\ \lim\limits_{x \rightarrow x_0} f(x) = f(x_0)$


\subsection{Stetigkeit: Delta-Epsilon-Kriterium}
Sei $f: I \rightarrow \R$ und $x_0 \in I$, dann ist $f$ in $x_0$ stetig
falls gilt:
\begin{equation*}
    \forall \varepsilon > 0\ \exists \delta(\varepsilon):\
    \abs{f(x)-f(x_0)}<\varepsilon\ \forall x \in I \text{ mit }
    \abs{x-x_0}<\delta(\varepsilon)
\end{equation*}
Und f ist stetig (auf $I$), wenn $f$ in jedem $x_0 \in I$ stetig ist.

\subsection{Bemerkungen}
\begin{enumerate}[label= (\alph*)]
    \item $f$ ist stetig in $x_0 \Leftrightarrow$
        \begin{equation*}
            \lim_{x \rightarrow x_0} f(x) = f(x_0)
        \end{equation*}
        gilt.
    \item $f$ ist stetig in $x_0$ dann gilt:
        \begin{equation*}
            \lim_{n \rightarrow \infty} x_n = x_0 \Rightarrow
            \lim_{n \rightarrow \infty} f(x_n) = f(x_0)
        \end{equation*}
\end{enumerate}

\subsection{Rechenregeln für Stetigkeit}
Sind $f,g: I \rightarrow \R$ stetig, dann sind auch die Funktionen
\begin{enumerate}[label= (\alph*)]
    \item $c \cdot f$ (für $c \in \R$)
    \item $f + g$
    \item $f \cdot g$
    \item und falls $g(x) \neq 0 \forall x \in I \frac{f}{g}$
\end{enumerate}
stetig.

Ist $f: I \rightarrow J, g: I \rightarrow \R$ und beide stetig dann
ist auch $g \circ f$ stetig.

\subsection{Stetigkeit von Potenzreihen}
Sei $f(x) = \sum_{k=0}^\infty a_k {(x - x_0)}^k$ eine Potenzereihe mit
Konvergenzradius $R > 0$, dann gilt für $x_1 \in U_R(x_0)$, dass
$\lim\limits_{x \rightarrow x_1} f(x) = f(x_1)$ (d.h. Potenzreihen
sind innerhalb des Konvergenzradius stetig).

\subsection{Umgebung positiver Funktionswerte}
Sei $f: I \rightarrow \R$ stetig in $x_0$, dann gilt:
\begin{equation*}
    f(x_0) > 0 \Rightarrow \exists \delta > 0:\ f(x)>0\
    \forall x \in I \text{ mit } \abs{x - x_0} < \delta
\end{equation*}

\subsection{Zwischenwertsatz}
Sei $D = [a, b]$ (also abgeschlossen) und $f: D \rightarrow \R$ stetig dann
ex.\ zu jedem $y$ zwischen $f(a)$ und $f(b)$ ein $x \in [a,b]$ mit $f(x) =y$.

\textbf{Genauer:}
\begin{equation*}
    \forall y \in [m, M]\ \exists x \in [a,b] \text{ mit } f(x)=y
\end{equation*}
Wobei $m = \min \{ f(a), f(b) \}$ und $M = \max \{ f(a), f(b) \}$.

\subsubsection{Bemerkung}
Bei einer Funktion ist das Bild eines Intervals wieder ein Interval. D.h.
\begin{equation*}
    f([a,b]) = [c, d]
\end{equation*}

\subsection{Existenz des Logarithmus}
Die Exponentialfunktion $\exp: \R \rightarrow \left( 0, \infty \right)$ ist
bijektiv. Das heißt es existiert eine Umkehrfunktion, diese wird
$\log: \left( 0, \infty \right) \rightarrow \R$ genannt.

\subsection{Maximum/Minimum/Infimum/Supremum einer Funktion}
Sei $f: D \rightarrow \R$ mit $D \subseteq \R$, dann heißt im Fall der
Existenz:
\begin{enumerate}[label= (\alph*)]
    \item
        \begin{equation*}
            \max\limits_{x \in D} f(x)
            := \max\limits_{D} f(x) := \max \{ f(x)\ x \in D \}
        \end{equation*}
        das Maximum von $f$ auf $D$.
    \item
        \begin{equation*}
            \min\limits_{x \in D} f(x) := \min\limits_{D} f(x)
            := \min \{ f(x)\ x \in D \}
        \end{equation*}
        das Minimum von $f$ auf $D$.
    \item
        \begin{equation*}
            \sup\limits_{x \in D} f(x)
            := \sup\limits_{D} f(x) := \sup \{ f(x)\ x \in D \}
        \end{equation*}
        das Supremum von $f$ auf $D$.
    \item
        \begin{equation*}
            \inf\limits_{x \in D} f(x) := \inf\limits_{D} f(x)
            := \inf \{ f(x)\ x \in D \}
        \end{equation*}
        das Infimum von $f$ auf $D$.
\end{enumerate}

\subsection{Beschränktheit einer stetigen Funktion}
Seien $a, b \in \R$ mit $a < b$ und eine stetige Funktion
$f: [a, b] \rightarrow \R$ gegeben, dann ist $f$ beschränkt.
(d.h. $\sup\limits_{[a, b]} (f) < \infty$ und $\inf\limits_{[a, b]}  (f)> \infty$).

\subsection{Weierstraß: Existenz von Min und Max}
Seien $a, b \in \R$ mit $a < b$ und $f: [a, b] \rightarrow \R$ stetig,
dann ex.:
\begin{equation*}
    \min_{[a, b]} f \text{ und } \max_{[a, b]} f
\end{equation*}

\subsection{Zusammenhang Injektivität --- Stetigkeit}
Sei $f: I \rightarrow \R$ stetig auf einem Intervall $I \subseteq \R$. Dann
gilt:
\begin{equation*}
    f \text{ inj.\ auf }I \Leftrightarrow
    f \text{ ist streng monoton}
\end{equation*}

\subsection{Existenz und Monotonie der Umkehrfunktion}
Sei $f: I \rightarrow \R$ stetig und streng monoton auf einem Intervall $I$.
Dann ex.\ auf $J := f(I)$ die Umkehrfunktion $f^{-1}: J \rightarrow I$
und diese ist  im gleichen Sinn wie $f$ streng Monoton und stetig.

\subsection{Gleichmäßige Stetigkeit}
Eine Funktion $f: I \rightarrow \R$ heißt gleichmäßig stetig auf $I$, wenn gilt:
\begin{equation*}
    \forall \varepsilon > 0\ \exists \delta(\varepsilon)\
    \abs{f(x_1) - f(x_2)} < \varepsilon\ \forall x_1, x_2 \in I \text{ mit }
    \abs{x_1 - x_2} < \delta(\varepsilon)
\end{equation*}

\subsubsection{Bemerkung}
Im Gegensatz zur normalen Stetigkeit wird bei der gleichmäßigen Stetigkeit eine
Funktion $\delta(\varepsilon)$ für die ganze Funktion bestimmt und nicht nur
für jeden Punkt einzeln (also $\delta(x_0, \varepsilon$)). Es wird also
zwischen Stetigkeit in einem Punkt und Stetigkeit auf einem Intervall
unterschieden.


    \chapter{Differentialrechnung}
    \section{Ableitung}

\subsection{Definition Differenzen-Quotient}
Sei $f: D \subseteq \R \rightarrow \R $. Dann heißt $f$ in
$x_0 \in D$ differentierbar, falls
\begin{equation*}
    f'(x_0) := \lim_{x \rightarrow x_0}
        \frac{f(x) - f(x_0)}{x - x_0}
\end{equation*}
für alle $x_0 \in D$ existiert.

\subsection{Rechtsseitige und linksseitige Ableitung}
Im Fall der Existenz heißen
\begin{eqnarray*}
    f'(x_0^+) &:=& \lim_{x \rightarrow x_0^+}
        \frac{f(x) - f(x_0)}{x-x_0} \text{ bzw.}\\
    f'(x_0^-) &:=& \lim_{x \rightarrow x_0^-}
        \frac{f(x) - f(x_0)}{x-x_0}
\end{eqnarray*}
die rechts- bzw.\ linksseitige Ableitung in $x_0$

\subsubsection{Bemerkung}
\begin{equation*}
    f'(x_0) \text{ ex.} \Leftrightarrow
    f'(x_0^+) \text{ und } f(x_0^-) \text{ ex.\ und }
    f'(x_0^+) = f'(x_0^-)
\end{equation*}

\subsection{Ableitungsrechenregeln}
Seien $f, g: D \rightarrow \R$ differentierbar in $x_0 \in D$, dann gilt:
\begin{enumerate}[label= (\alph*)]
    \item $(f+g)'(x_0) = f'(x_0) + g'(x_0)$
    \item $(f \cdot g)'(x_0) = f'(x_0) \cdot g(x_0) +
                                f(x_0) \cdot g'(x_0)$
    \item Falls $g(x_0) \neq 0$: $\left( \dfrac{f}{g} \right) '(x_0) =
            \dfrac{f'(x_0) g(x_0) - f(x_0) g'(x_0)}{{g(x_0)}^2}$
\end{enumerate}

\subsection{Alternative Definition der Ableitung}
Sei $f: D \subseteq \R \rightarrow \R$ und $x_0 \in D$. Dann gilt:
f differenzierbar in $x_0 \Leftrightarrow$
\begin{equation*}
    \exists A \in \R \text{ und } r: D \rightarrow \R \text{ mit }
    \lim_{x \rightarrow x_0} r(x) = 0 \text{ so dass gilt: }
    f(x) = f(x_0) + A \cdot (x - x_0)  + r(x) \cdot (x-x_0)
\end{equation*}

\subsection{Zusammenhang Differentierbarkeit --- Stetigkeit}
Ist $f: D \rightarrow \R$ differentierbar in $x_0 \in D \Rightarrow$
$f$ stetig in $x_0$

\subsection{Differentiation von Potenzreihen}
Sei $f(x) = \sum_{k=0}^\infty a_k {(x-x_0)}^k$ eine Potenzreihe mit $R>0$, dann
ist $f$ für $x$ mit $\abs{x-x_0}<R$ differentierbar, und es gilt:
\begin{equation*}
    f'(x) = \sum_{k=1}^\infty a_k \cdot k \cdot {(x - x_0)}^{k-1}
\end{equation*}

\subsubsection{Bemerkung}
Der Konvergenzradius von $f'(x)$ ist ebenfalls $R$.

\subsection{Ableitung der Umkehrfunktion}
Seien $I, J \subseteq \R$ Intervalle und $f: I \rightarrow J$ sei differentiarbar
und bijektiv, dann ist auch $f^{-1}: J \rightarrow I$ differentierbar und es
gilt:
\begin{equation*}
    {(f^{-1})}'(y_0) = \frac{d}{dx} f^{-1}(y_0) =
    \frac{1}{f'(f^{-1}(y_0))} \forall y_0 \in J \text{ für ein } y_0=f(x_0)
    \text{ und } f'(y_0) \neq 0
\end{equation*}

\subsection{Ketternregel}
Seien $f: A \rightarrow B$, $g: B \rightarrow \R$ mit $A,B \subseteq \R$
differentierbar auf $A$ bzw. $B$, dann ist auch $g \circ f$ auf $A$
differentierbar und es gilt:
\begin{equation*}
    (g \circ f)'(x_0) = g'(f(x_0)) \cdot f'(x_0)\ \forall x_0 \in A
\end{equation*}

\section{Mittelwertsätze}

\subsection{Satz von Rolle}
Sei $f: [a, b] \rightarrow \R$ stetig und auf $(a,b)$ differentierbar. Falls
$f(a) = f(b)$ gilt, existiert ein $x_0 \in (a,b)$ mit $f'(x_0)=0$

\subsection{Definition lokaler Extrempunkt}
Sei $f: D \rightarrow \R$ und $x_0 \in D$. Dann besitzt$f$ in $x_0$ ein
lokales Maximum (bzw. Minimum)$:\Leftrightarrow$
\begin{equation*}
    \exists \delta > 0: f(x) \leq f(x_0)
    \text{ (bzw. } f(x) \geq f(x_0) \text{) } \forall x \in D \cap U_\delta(x_0)
\end{equation*}

\subsection{Notwendige Bedingung für lokale Extrema}
Sei $f: D \rightarrow \R$ differentierbar in $x_0 \in D$ und $x_0$ sei kein
Randpunkt, dann gilt:

Liegt bei $x_0$ ein lokales Maximum/Minimum $\Rightarrow f'(x_0)=0$.

\subsection{2. Mittelwertsatz}
Seien $f,g: [a,b] \rightarrow \R$ stetig und auf $(a,b)$ differentierbar
dann existiert ein $x_0 \in (a,b)$ mit
\begin{equation*}
    f'(x_0) \cdot (g(b) - g(a)) = g'(x_0) \cdot (f(b)-f(a))
\end{equation*}
Bzw.\ falls nicht durch Null geteilt wird:
\begin{equation*}
 \frac{f'(x_0)}{g'(x_0)} = \frac{f(b)-f(a)}{g(b)-g(a)}
\end{equation*}

\subsection{1. Mittelwertsatz (Folgerung aus 2. Mittelwertsatz)}
Sei $f: [a,b] \rightarrow \R$ stetig und auf $(a,b)$ differentierbar
\begin{equation*}
    \Rightarrow \exists x_0 \in (a,b) \text{ mit } f'(x_0)=\frac{f(b)-f(a)}{b-a}
\end{equation*}

\subsection{L'Hospital}
Seien $f,g: [a,b) \rightarrow \R (a < b, b \in (\R \cup {\infty}))$
differentierbar auf $(a,b)$ mit $g'(x) \neq 0\ \forall x \in (a,b)$. Falls
der Grenzwert $\alpha = \lim\limits_{n \rightarrow b^-} \dfrac{f'(x)}{g'(x)}$
ex.\ und:
\begin{enumerate}[label= (\alph*)]
    \item $\lim\limits_{n \rightarrow b^-} f(x) =
        \lim\limits_{n \rightarrow b^-} g(x) = 0$ oder
    \item $\lim\limits_{x \rightarrow b^-} g(x) = \infty$
\end{enumerate}
dann gilt:
\begin{equation*}
    \lim_{x \rightarrow b^-} \frac{f(x)}{g(x)} =
    \lim_{x \rightarrow b^-} \frac{f'(x)}{g'(x)}
\end{equation*}

\subsection{Satz von Taylor}
Sei $f: [a,b] \rightarrow \R$ $n+1$ mal differentierbar auf $(a,b)$ und
$x_0 \in (a,b)$. Dann gilt für ein $\xi \in (x_0, x)$:
\begin{equation*}
    f(x) = \sum_{k=0}^n \frac{f^{(k)}(x_0)}{k!} \cdot {(x-x_0)}^k +
    \frac{f^{(n+1)}(\xi)}{(n+1)!} \cdot {(x-x_0)}^{n+1}
\end{equation*}


    \part{HM 2 --- Zusammenfassung}
    \chapter{Integration}
    \section{Integration}

\subsection{Definition Zerlegung, Zwischenwerte}
Eine Teilmenge $T$ von $[a,b]$ mit $a, b \in T$ nennt man eine
Unterteilung, Zerlegung oder Partitionierung von $[a, b]$ wenn
gilt:
\begin{eqnarray*}
    T = \{ x_0, x_1, \ldots , x_n\} \text{ mit}\\
    a = x_0 < x_1 < \ldots < x_n = b
\end{eqnarray*}

Schreibweise für diese Menge $T$ sei:
\begin{equation*}
    T: a = x_0 < x_1 < \ldots < x_n = b
\end{equation*}

Ist T eine Zerlegung, dann heißt:
\begin{enumerate}[label= (\alph*)]
    \item Die Zahl $\mu(T) :=
        \max{ \{\ \abs{x_k - x_{k+1}}, k = 1, \ldots, n \} }$
        das Feinheitsmaß von $T$.
    \item Ein Vektor $\xi = (\xi_1, \ldots, \xi_n) \in \R^n$ heißt
        ein Zwischenwertvektor zu $T$, wenn gilt
        \begin{equation*}
            x_{k-1} \leq \xi_k \leq x_k \text{ für } k = 1, \ldots, n
        \end{equation*}
        Dann heißt die Komponente $\xi_k$ ein Zwischenwert von
        $x_{k-1}$ und $x_k$.
\end{enumerate}

\subsection{Definition Riemannsumme}
Ist $f: [a,b] \rightarrow \R$ eine Funktion, $T: a=x_0<\ldots<x_n=b$
eine Zerlegung von $[a, b]$ und $\xi = (\xi_1, \ldots, \xi_n)$ ein
Zwischenwertevektor zu $T$, dann nennen wir die Summe
\begin{equation*}
    S(f; T, \xi) = S_f(T, \xi) = \sum_{k=1}^n f(\xi_k)(x_k - x_{k-1})
\end{equation*}
die Riemansumme von $f$ bezüglich $T$ und $\xi$.

\subsection{Definition Riemann-Integral}
Eine Funktion $f: [a, b] \rightarrow \R$ heißt Riemann-Integrierbar
unter $[a, b]$ wenn für jede Folge ${(T_N)}_{N=1}^\infty$ von Zerlegungen
von $[a,b]$ mit $\mu(T_N) \rightarrow 0$ für $N \rightarrow \infty$
und jede Folge ${(\xi_N)}_{N=1}^\infty$ von Zwischenpunktvektoren
der Grenzwert
\begin{equation*}
    \lim_{N \rightarrow \infty} S(f; T_N, \xi_N) \text{ existiert.}
\end{equation*}

\subsubsection{Behauptung}
Der Grenzwert ist im Fall der Existenz für jede Folge identisch.

\subsubsection{Bemerkung}
\begin{enumerate}[label= (\alph*)]
    \item Im Fall der Existenz bezeichnet man den Grenzwert durch:
        \begin{equation*}
            \int_{a}^b f(x) \dx = \lim_{N \rightarrow \infty} S(f; T_N, \xi_N)
        \end{equation*}
    \item Zu ${(T_N)}_{N=1}^\infty$, also $T_1, T_2, T_3, \ldots$:
        \begin{eqnarray*}
            &T_1:& a = x_0^{(1)} < \ldots < x_n^{(1)} = b\\
            &T_2:& a = x_0^{(2)} < \ldots < x_n^{(2)} = b\\
            &T_3:& a = x_0^{(3)} < \ldots < x_n^{(3)} = b\\
            &\vdots&\\
            &T_l:& a = x_0^{(l)} < \ldots < x_n^{(l)} = b
        \end{eqnarray*}
    \item Zu ${(\xi_N)}_{N=1}^\infty$, also $\xi_1, \xi_2, \xi_3, \ldots$:
        \begin{eqnarray*}
            &\xi_1& = (\xi_1^{(1)}, \ldots, \xi_n^{(1)})
            \text{ mit } x_{k-1}^{(1)} \leq \xi_k^{(2)} \leq x_k^{(1)}
            \text{ mit } 1 \leq k \leq n_1\\
            &\xi_2& = (\xi_1^{(2)}, \ldots, \xi_n^{(2)})
            \text{ mit } x_{k-1}^{(2)} \leq \xi_k^{(2)} \leq x_k^{(2)}
            \text{ mit } 1 \leq k \leq n_2\\
            &\xi_3& = (\xi_1^{(3)}, \ldots, \xi_n^{(3)})
            \text{ mit } x_{k-1}^{(3)} \leq \xi_k^{3} \leq x_k^{(3)}
            \text{ mit } 1 \leq k \leq n_3\\
            &\vdots&\\
            &\xi_l& = (\xi_1^{(l)}, \ldots, \xi_n^{(l)})
            \text{ mit } x_{k-1}^{(l)} \leq \xi_k^{l} \leq x_k^{(l)}
            \text{ mit } 1 \leq k \leq n_l\\
        \end{eqnarray*}
    \item Sei $f$ integrierbar und ${(T_N)}_{N=1}^\infty$ und
        ${(\xi_N)}_{N=1}^\infty$ sowie ${(\tilde{T}_N)}_{N=1}^\infty$ und
            ${(\tilde{\xi}_N)}_{N=1}^\infty$ entsprechende Folgen,
            d.h. $\mu(T_N) \rightarrow 0, \mu(\tilde{T}_N) \rightarrow 0$
            für $N \rightarrow \infty$. Dann gilt gilt für ${(\hat{T}_N)}_{N=1}^\infty$
            und ${(\hat{\xi}_N)}_{N=1}^\infty$ mit
            \begin{equation*}
                \hat{T}_N :=
                \begin{cases}
                    T_N &\text{ für } N \text{ gerade}\\
                    \tilde{T}_N &\text{ für } N \text{ ungerade}
                \end{cases}
            \end{equation*}
            und
            \begin{equation*}
                \hat{\xi}_N :=
                \begin{cases}
                    \xi_N &\text{ für } N \text{ gerade}\\
                    \tilde{\xi}_N &\text{ für } N \text{ ungerade}
                \end{cases}
            \end{equation*}
            dass
            \begin{equation*}
                \lim_{N \rightarrow \infty} S(f; \hat{T}_N, \hat{S}_N)
            \end{equation*}
            existiert, da $f$ integrierbar ist.

            Dann stimmt der Grenzwert von
            $\lim_{N \rightarrow \infty} S(f; \tilde{T}_N, \tilde{S}_N)$ und
            $\lim_{N \rightarrow \infty} S(f; T_N, S_N)$ überein.
\end{enumerate}

\subsection{Menge der Riemann-Integrierbaren Funktionen}
Mit $R [a, b]$ oder $R ([a, b])$ bezeichnen wir die Menge von Funktionen
$f: [a, b] \rightarrow \R$ die auf $[a, b]$ Riemann integrierbar sind.

\subsection{Kriterien für Riemann-Integrierbarkeit}
\begin{enumerate}[label= (\alph*)]
    \item
        \begin{equation*}
            f \in R[a, b] \Rightarrow f \text{ ist auf } [a,b] \text{beschränkt}
        \end{equation*}
    \item Ist $f, g \in R [a, b]$ und $c \in \R$ dann sind auch die Funktionen
        \begin{eqnarray*}
            f&+&g\\
            f&-&g\\
            c &\cdot& f
        \end{eqnarray*}
        Riemann integrierbar auf $[a, b]$.
    \item Ist $f, g \in R[a, b]$, dann ist auch
        \begin{equation*}
            f \cdot g \in R[a, b]
        \end{equation*}
    \item Ist $f, g \in R[a, b]$ und falls
        $\abs{g(x)} > \delta > 0\ \forall x \in [a, b]$ dann ist auch
        \begin{equation*}
            \frac{f}{g} \in R[a, b]
        \end{equation*}
    \item Für beliebiges $c \in [a, b]$  gilt:
        \begin{equation*}
            f \in R[a, b] \Leftrightarrow f \in R[a, c] \land f \in R[c, b]
        \end{equation*}
        und weiter gilt:
        \begin{equation*}
            \int_a^b f(x) \dx = \int_a^c f(x) \dx + \int_c^b f(x) \dx
        \end{equation*}
    \item
        \begin{equation*}
        f \in R[a, b] \Rightarrow \abs{f} \in R[a, bv]
        \end{equation*}
        und
        \begin{equation*}
            \abs{\int_a^b f(x)\dx} \leq \int_a^b \abs{f(x)} \dx
        \end{equation*}
\end{enumerate}

\subsection{Änderung von Funktionen}
Wenn $f \in R[a, b]$ ist und durch endlich viele Änderungen daraus
$g: [a,b] \rightarrow \R$ konstruiert werden kann, d.h.
\begin{equation*}
    g(x) =
    \begin{cases}
        f(x) &\text{ falls } x \notin \{x_1, \ldots, x_n \} \\
        y_1 &\text{ falls} x=x_1\\
        \vdots
    \end{cases}
\end{equation*}
dann gilt $g \in R[a, b]$ und
\begin{equation*}
    \int_a^b f(x) \dx = \int_a^b g(x) \dx
\end{equation*}

\subsection{Zusammenhang Stetigkeit und Integrierbarkeit}
Es gilt:
\begin{equation*}
    f \in C[a, b] \Rightarrow f \in R[a, b]
\end{equation*}

\subsection{Stückweise Integration}
Falls $f: [a, b] \rightarrow \R$ stückweise stetig ist, d.h.\ es existieren
endlich viele Intervall-Stücke auf denen $f$ stetig ist, dann ist
$f \in R[a,b]$ und es gilt:
\begin{equation*}
    \int_a^b f(x) \dx = \int_{x_0}^{x_1} f(x) \dx + \int_{x_1}^{x_2} f(x)
    + \ldots + \int_{x_{n-1}}^{x_n} f(x)
\end{equation*}

\subsection{1. Mittelwertsatz der Integralrechnung}
Seien $f, g \in R[a,b]$ und $g \geq 0$ auf $[a, b]$. Dann gibt es ein $\mu \in \R$
mit $\inf\limits_{[a,b]} f(x) \leq \mu \leq \sup\limits_{[a,b]} f(x)$ sodass gilt:
\begin{equation*}
    \int_a^b f(x)g(x)\dx = \mu \int_a^b g(x) \dx
\end{equation*}

Ist $f$ stetig auf $[a, b]$, dann existiert ein $\xi \in [a, b]$ mit
\begin{equation*}
    \int_a^b f(x)g(x)\dx = f(\xi) \int_a^b g(x) \dx
\end{equation*}

\subsubsection{Bemerkung}
Für $g(x)=1$ und $f$ stetig lautet die Aussage also:
\begin{equation*}
    \int_a^b f(x) \dx = f(\xi) \cdot (b - a)
\end{equation*}

\subsection{Existenz der Stammfunktion}
Sei $f \in R[a, b]$, dann ist für jedes $c \in [a, b]$ durch:
\begin{equation*}
    F(x) := \int_c^x f(t) dt
\end{equation*}
eine stetige Funktion definiert. Und für jedes $x_0 \in (a,b)$ gilt:
\begin{equation*}
    f \text{ stetig in } x_0 \Rightarrow F \text{ ist differentierbar in } x_0
    \land F'(x_0)=f(x_0)
\end{equation*}

\subsection{Definition Stammfunktion}
Gilt $F'(x) = f(x) \forall x \in [a,b]$ dann wird $F$ als Stammfunktion von
$f$ bezeichnet.

\subsection{Eindeutigkeit der Stammfunktion}
Sind $F$ und $G$ Stammfunktionen von $f$, dann existiert ein $c \in \R$ mit
\begin{equation*}
    F(x) = G(x) + c\ \forall x \in [a, b]
\end{equation*}

\subsection{Hauptsatz der Differential und Integralrechnung}
Sei $f: [a,b] \rightarrow \R$ gegeben dann gilt:
\begin{enumerate}[label= (\alph*)]
    \item Ist $f \in R[a,b]$ und $F$ eine Stammfunktion, dann gilt:
        \begin{equation*}
            \int_a^b f(t) dt = F(b) - F(a) =: {\left[ F(x) \right]}_a^b
        \end{equation*}
    \item Ist $f \in C[a, b]$ dann existiert eine Stammfunktion und zwar
        \begin{equation*}
            F(x) := \int_c^x f(t) dt
        \end{equation*}
\end{enumerate}

\subsubsection{Bemerkung}
Aus dem Hauptsatz folgen Integrationstechniken wie partielles Integrieren oder
die Subsitutionsregel.

\subsection{Zusammenhang Monotonie und Riemann-Integrierbarkeit}
Ist $f: [a, b] \rightarrow \R$ auf $[a,b]$ monoton, dann ist $f \in R[a,b]$.

\subsection{Zweiter Mittelwertsatz der Integralrechnung}
Ist $f$ monoton auf $[a,b]$, $g$ integrierbar auf $[a,b]$, dann exisitiert ein
$\xi \in [a,b]$ mit:
\begin{equation*}
    \int_a^b f(x)g(x) \dx = f(a) \int_a^\xi g(x) \dx +
        \int_\xi^b g(x) \dx
\end{equation*}

\subsection{Definition uneigentliches Integral}
Eine Funktion $f: [a, b) \rightarrow \R$ mit $a < b \leq \infty$ heißt über
$[a,b)$ uneigentlich Riemann integrierbar, wenn gilt:
\begin{enumerate}[label= (\alph*)]
    \item $\forall c$ mit $a \leq c < b$ ist $f \in R[a, c]$
    \item Der Grenzwert
        \begin{equation*}
            \alpha = \lim_{c \rightarrow \infty} \int_a^c f(x) \dx
        \end{equation*}
        existiert. In dem Fall schreiben wir
        \begin{equation*}
            \alpha = \int_a^b f(x) \dx
        \end{equation*}
        und sagen das uneigentliche Integral
        \begin{equation*}
            \int_a^b f(x) \dx
        \end{equation*}
        konvergiert gegen $\alpha$ oder hat den Wert $\alpha$.
\end{enumerate}

Andernfalls divergiert das uneigentliche Integral.
Analog geht man für Funktionen
\begin{itemize}
    \item $f: (a, b] \rightarrow \R$ mit $-\infty \leq a < b$ und
    \item $f: (a, b) \rightarrow \R$ mit $-\infty \leq a < b \leq \infty$
\end{itemize}
vor.

\subsection{Cauchy-Kriterium}
Sei $f \in R[a,b]\ \forall c \in (a,b), a < c \leq \infty$ Dann konv.
\begin{equation*}
    \int_a^b f(x) \dx
\end{equation*}
\begin{enumerate}[label= (\alph*)]
    \item Im Fall $b<\infty$ genau dann, wenn gilt:
        \begin{equation*}
            \forall \varepsilon > 0\ \exists \delta > 0: \abs{\int_{T_1}^{T_2} f(x) \dx} < \varepsilon\ \forall~T_1, T_2 \in [b - \delta, b)
        \end{equation*}
    \item Im Fall $b = \infty$, wenn gilt:
        \begin{equation*}
                \forall \varepsilon > 0\ \exists K \geq a: \abs{\int_{T_1}^{T_2} f(x) \dx} < \varepsilon\ \forall~T_1, T_2 \geq K
        \end{equation*}
\end{enumerate}

\subsection{Majorantenkriterium}
Seien $f, g \in R[a, c] \forall c \in (a,b), a < b \leq \infty$ oder
$f, g \in R[c, b] \forall c \in (a, b) -\infty \leq a < b$.
Außerdem $\abs{f(x)} \leq g(x)$. Und
\begin{equation*}
    \int_a^b g(x) \dx
\end{equation*}
konvergiert, dann konvergiert auch
\begin{equation*}
    \int_a^b f(x) \dx
\end{equation*}

\subsection{Absolute Konvergenz}
Ist $f \in R[T_1, T_2]$ für $a < T_1 \leq T_2 < b \leq \infty$ so heißt
\begin{equation*}
    \int_a^b f(x) \dx
\end{equation*}
absolut konvergent, wenn
\begin{equation*}
    \int_a^b \abs{f(x)} \dx
\end{equation*}
konvergent ist.

\subsection{Minorantenkriterium}
\begin{equation*}
    f(x) \geq g(x) \geq 0 \land \int_a^b g(x) \dx = \infty
    \Rightarrow \int_a^\infty f(x) \dx = \infty
\end{equation*}

\subsection{Integralkriterium für Reihen}
Sei $f: [a, \infty) \rightarrow [0, \infty)$ und $f \searrow (a \in \Z)$. Dann
gilt:
\begin{equation*}
    \sum_{n=m}^\infty f(n) < \infty \Leftrightarrow
    \int_a^\infty f(x) \dx < \infty
\end{equation*}


    \chapter{Gleichmäßige Konvergenz}
    %!TEX root = ../main.tex
\section{Gleichmäßige Konvergenz}
\subsection{Definition Funktionenfolge und Funktionenreihe}
Sei $M$ eine Menge und $m \in \Z$. Ist jedem $n \in \{m, m+1, \ldots\}$ eine
Funktion $f_n: M \rightarrow \R$ zugeordnet, so nennt man:
\begin{enumerate}[label= (\alph*)]
    \item Die Folge ${(f_n)}_{n=m}^\infty$ eine Funktionenfolge auf $M$
    \item Die Reihe $\sum_{n=m}^\infty f_n(x)$ eine Funktionenreihe auf $M$
\end{enumerate}

konvergiert ${(f_n)}_{n \geq m}$ (bzw. $\sum_{n=m}^\infty f_n(x)$) für alle
$x \in \tilde{M} \subseteq M$ so heißt die durch $f(x) = \lim\limits_{n \rightarrow \infty}
f_n(x)$ (bzw. $f(x) = \sum_{n=m}^\infty f_n(x)$) definierte Funktion $f:
\tilde{M} \rightarrow \R$ die Grenzfunktion von ${(f_n)}_{n=m}^\infty$
(bzw. $\sum_{n=m}^\infty f_n$).

\subsection{Gleichmäßige Konvergenz}
Sei $M$ eine Menge und sei $f: M \rightarrow \R$ eine Funktion.
\begin{enumerate}[label= (\alph*)]
    \item Eine Funktionenfolge ${(f_n)}_{n=1}^\infty$ heißt auf $M$ gleichmäßig
        konvergent gegen $f$ wenn gilt:
        \begin{equation*}
            \forall \varepsilon > 0\  \exists\ n_0(\varepsilon):
                \abs{f_n(x) - f(x)} < \varepsilon\
                \forall x \in M \text{ und } n \geq n_0(\varepsilon)
        \end{equation*}
    \item Eine Funktionenfolge $\sum_{n=m}^\infty f_n$ konvergiert gleichmäßig
        auf $M$ wenn gilt:
            \begin{equation*}
                \forall \varepsilon > 0\ \exists\ n_0(\varepsilon): \abs{\sum_{k=m}^n f_k(x) - f(x)}
                    < \varepsilon\ \forall x \in M \text{ und } n \geq n_0(\varepsilon)
            \end{equation*}
\end{enumerate}

\subsubsection{Bemerkung}
Offensichtlich gilt:
\begin{equation*}
    \text{Gleichmäßig konvergent} \Rightarrow \text{Punktweise Konvergent}
\end{equation*}


\subsection{Stetigkeit der Grenzfunktion}
Sei ${(f_n)}_{n=1}^\infty$ (bzw. $\sum_{n=1}^\infty f_n(x)$) gleichmäßig konvergent gegen
$f$ auf einem Intervall $I$ und alle $f_n$ stetig auf $I$. Dann ist auch die
Grenzfunktion $f$ stetig.

\subsection{Integration der Grenzfunktion}
Sei ${(f_n)}_{n=1}^\infty$ eine Folge von integrierbaren Funktionen auf $[a,b]$
\begin{enumerate}[label= (\alph*)]
    \item Falls ${(f_n)}_{n=1}^\infty$ gleichmäßig gegn $f$ konvergiert, dann ist
    auch $f$ auf $[a,b]$ integrierbar und es gilt:
        \begin{equation*}
            \lim_{n \rightarrow \infty} \int_a^b f_n(x) \dx =
            \int_a^b \lim_{n \rightarrow \infty} f_n(x) \dx
        \end{equation*}
    \item Analog für Funktionenreihen
\end{enumerate}

\subsection{Cauchy Kriterium für gleichmäßige Konvergenz}
\begin{enumerate}[label= (\alph*)]
    \item Eine Funktionenfolge ${(f_n)}_{n=1}^\infty$ konvergiert genau dann
        gleichmäßig auf einer Menge $M$ ($\subseteq$ Definitionsbereich), wenn
        gilt:
        \begin{equation*}
            \forall \varepsilon > 0\ \exists n(\varepsilon):
            \abs{f_n(x) - f_m(x)} < \varepsilon\ \forall n \geq n(\varepsilon) \forall x \in M
        \end{equation*}
    \item Analog für Funktionenreihen
\end{enumerate}

\subsection{Differentiation der Grenzfunktion}
Sei ${(f_n)}_{n=1}^\infty$ eine auf dem Intervall $I$ differentierbare Folge von
Funktionen.
\begin{enumerate}[label= (\alph*)]
    \item Konvergiert die Folge ${(f_n')}_{n=1}^\infty$ gleichmäßig auf $I$ und konvergiert
    für ein beliebiges, festes $x_0 \in I$ die reele Folge ${(f_n(x_0))}_{n=1}^\infty$
    dann ist auch die Grenzfunktion $f$ von ${(f_n)}_{n=1}^\infty$ differentierbar
    und es gilt:
    \begin{equation*}
        \lim_{n \rightarrow \infty} \ddx f_n(x) = \ddx \lim_{n \rightarrow \infty} f_n(x)
    \end{equation*}
    \item Analog für Funktionenreihen
\end{enumerate}

\subsubsection{Bemerkung}
Außerdem gilt dass ${(f_n)}_{n=1}^\infty$ (bzw. $\sum_{n=1}^\infty f_n$) auf jedem
beschränkten Teilintervall von $I$ gleichmäßig konvergiert.

\subsection{Majorantenkriterium auf Potenzreihen anwenden}
Für eine reele Potenzreihe $f(x) = \sum_{k=0}^\infty a_k {(x-x_0)}^k$ mit
Konvergenzradius $R > 0$ gilt:
\begin{enumerate}[label= (\alph*)]
    \item $f$ ist stetig auf $(x_0 - R, x_0 + R) =: I$
    \item $f$ ist differentierbar auf $I$ und
        \begin{equation*}
            f'(x) = \sum_{k=0}^\infty a_k \cdot k \cdot {(x-x_0)}^{k-1}
        \end{equation*}
    \item $f$ ist integrierbar auf $I$ und hat die Stammfunktion
        \begin{equation*}
            F(x) = \sum_{k=0}^\infty \frac{a_k}{k+1} {(x-x_0)}^{k+1}
        \end{equation*}
\end{enumerate}

\subsubsection{Bemerkung}
Wurde alles schon in HM1 gezeigt aber mühsam.

\subsection{Majorantenkriterium für Funktionenreihen}
Falls $\abs{f_n(x)} \leq a_n$ und $\sum_{n=1}^\infty a_n$ konvergiert
$\Rightarrow \sum_{n=1}^\infty f_n(x)$ ist gleichmäßig konvergent.


    \chapter{Differentialrechung mit mehreren Variablen}
    \section{Der n-dimensionale Euklidische Raum}

\subsection{Definitionen}
Sind $n, m \in \N$, so gelten folgende Bezeichungen:
\begin{eqnarray*}
    \R^n &:=& \Bigg\{
        \begin{pmatrix}
            x_1\\
            \vdots \\
            x_n
        \end{pmatrix}
        \text{ für }
        x_1, \ldots, x_n \in \R^n
    \Bigg\} \\
    \R^{m \times n} &:=& \Bigg\{
        \begin{pmatrix}
            a_{11} & \cdots & a_{1n}\\
            \vdots & \ddots\\
            a_{m1} & \ldots & a_{mn}
        \end{pmatrix}
        \text{ für }
        a_{ij} \in \R, 1 \leq i \leq m, 1 \leq j \leq n
    \Bigg\}\\
    <x, y> &:=& x \cdot y := x^T y := \sum_{k=1}^n x_k y_k \text{ (Skalarprodukt)}\\
    \norm{x} &:=& {\Vert x \Vert}_2 := \abs{x} := \sqrt{\sum_{k=1}^n x_k^2}
    \text{ euklidische Norm des }\R^n\text{ /Betrag in }\R^n
\end{eqnarray*}

\subsection{Folgerungen}
\begin{enumerate}
    \item
        \begin{equation*}
            \norm{x}_\infty = \max_{k=1\ldots n} \abs{x_k}
            \leq \norm{x}_2 \leq \sqrt{n} \max_{k=1 \ldots n} \abs{x_k}
            \ \forall x \in \R^n
        \end{equation*}
    \item
        \begin{equation*}
            \norm{x}_1 = \sum_{k=1}^n \abs{x_k}
        \end{equation*}
        und
        \begin{equation*}
            \norm{x}_2 \leq \norm{x}_1
        \end{equation*}
    \item $\norm{x}_1$, $\norm{x}_2$, $\norm{x}_\infty$ sind drei mögliche Festlegungen
        für Vektornormen. Allgemein hat eine Norm $\norm{\cdot}_2$
        ($\norm{\cdot}_2: \R^2 \rightarrow \R$) folgende Eigenschaften:
        \begin{eqnarray*}
            &\norm{x} \geq 0&\ \forall x \in \R^n \land \norm{x}=0 \Leftrightarrow
            x = \begin{pmatrix}
            0 \\ 0
            \end{pmatrix}
            = \vec{0} \\
            &\norm{\alpha \cdot x} = \abs{\alpha} \cdot \norm{x}&\ \forall \alpha
            \in \R\land \forall x \in \R^n \\
            &\norm{x + y} \leq \norm{x} + \norm{y}&\ \forall x, y \in \R^n
        \end{eqnarray*}
    \item Der Einheitskreis ist bezüglich verschiedener Normen nicht immer ein Kreis
    \item p-Norm:
        \begin{equation*}
            \norm{x}_p = \nthSqrt{p}{\sum_{k=1}^n \abs{x_k}^p}
        \end{equation*}
     \item $x \cdot y$ im $\R^2$ hat die anschauliche Bedeutung
        \begin{equation*}
            <x,y> = x \cdot y = \norm{x}_2 \cdot \norm{y}_2 \cdot \cos(\alpha)
        \end{equation*}
        Daraus folgt die Cauchy-Schwarzsche-Ungleichung (CSU)
        \begin{equation*}
            <x,y> \leq \norm{x}_2 \cdot \norm{y}_2
        \end{equation*}
\end{enumerate}

\subsection{Konventionen}
\begin{enumerate}[label= (\alph*)]
    \item In $\R^n$ sei stets $A^c := \R^n \ A$ für eine Menge $A \subseteq \R^n$
    \item Mit $\norm{\cdot}$ bezeichnen wir die euklidische Norm $\norm{\cdot}_2$.
    Außer es wird explizit gesagt, dass $\norm{\cdot}$ eine allgemeine Norm ist
    (z.B. \glqq{} Sei $\norm{\cdot}$ eine Norm   auf $\R^n$)
\end{enumerate}

\subsection{Definition Epsilon-Umgebung}
Sei $a \in \R^n, \varepsilon > 0$ dann heißt
\begin{eqnarray*}
    U_\varepsilon(a) &:=& \{ x \in \R^n |\ \norm{x-a} < \varepsilon \} \text{ die }
    \varepsilon \text{-Umgebung von }a\\
    \dot{U}_\varepsilon(a) &:=& U_\varepsilon(a) \backslash \{a \}
    ( =\{ x \in \R^n |\ 0<\norm{x-a}<\varepsilon \} ) \text{ die punktierte }
    \varepsilon \text{-Umgebung von }a
\end{eqnarray*}

\subsection{Definition Topologische Begriffe}
Sei $A \subseteq \R^n$. Ein Punkt $a \in \R^n$ heißt:
\begin{enumerate}[label= (\alph*)]
    \item Innerer Punt von $A$, falls ein $\varepsilon > 0$ existiert, sodass
        $U_\varepsilon(a) \subseteq A$
        Kurz:
        \begin{equation*}
            a \text{ innerer Punkt von }A :\Leftrightarrow \exists \varepsilon>0
            : U_\varepsilon(a) \subseteq A
        \end{equation*}

        Die Menge $\overset{\circ}{A}$ ist die Menge aller innerer Punkte von $A$
        \begin{equation*}
            \overset{\circ}{A} := \{ a \in \R^n | \exists \varepsilon > 0 \text{ mit }
            U_\varepsilon(a) \subseteq A \}
        \end{equation*}
    \item Berührungspunkt von $A$, wenn jede $\varepsilon$-Umgebung von $a$ mindestens
        einen Punkt aus $A$ enthält.
        Kurz:
        \begin{equation*}
            a \in \R^n \text{ ist Berührpunkt von } A :\Leftrightarrow
            \forall \varepsilon>0: U_\varepsilon(a) \cap A \neq \emptyset
        \end{equation*}

        Die Menge aller Berührpunkte von
        \begin{equation*}
            \bar{A} := \{ x \in \R^n | \forall \varepsilon > 0 \text{ ist }
            U_\varepsilon(a) \cap A \neq \emptyset \}
        \end{equation*}
        heißt der Abschluss oder abgeschlossene Hülle von $A$.
    \item Häufungspunkt von $A$, wenn jede punktierte $\varepsilon$-Umgebung von
        $a$ ein Element von $A$ enthält.
        Kurz:
        \begin{equation*}
            a \in \R^n \text{ ist Häufungspunkt } :\Leftrightarrow
            \forall\ \varepsilon > 0 \dot{U}_\varepsilon(a) \cap A \neq \emptyset
        \end{equation*}
    \item Randpunkt von $A$, wenn jede $\varepsilon$-Umgebung Elemente aus $A$
        und $A^c$ enthält.
        Kurz:
        \begin{equation*}
            a \in \R^n \text{ ist Randpunkt von }A :\Leftrightarrow
            \forall \varepsilon > 0\
            (U_\varepsilon(a) \hat A \neq \emptyset) \land
            (U_\varepsilon(a) \hat A^c \neq \emptyset)
        \end{equation*}
        Die Menge
        \begin{equation*}
            \partial A := \{ a \in \R^n | a \text{ ist Randpunkt von }A \}
        \end{equation*}
        heißt der Rand von $A$.
\end{enumerate}

\subsubsection{Bemerkung}
Man kann zeigen:
\begin{equation*}
    \bar{A} = A \cup \partial A =  \overset{\circ}{A} \cup \partial A
\end{equation*}


    \chapter{Integration in mehreren Veränderlichen}
    %!TEX root = ../main.tex
\section{Parameterintegrale}
\subsection{Eigentliche Parameterintegrale}
Sei $f(x,t)$ reel und stetig in $[\alpha, \beta] \times [a,b]$ (also $x\in[\alpha, \beta],
t \in [a,b]$). Dann gilt für
\begin{equation*}
    F(x) := \int_a^b f(x,t) \dt
\end{equation*}
\begin{enumerate}[label= (\alph*)]
    \item $F$ ist stetig auf $[\alpha, \beta]$
    \item Ist $f_x$ stetig in $[\alpha, \beta] \times [a,b]$, so ist
        $F \in C^1([\alpha, \beta])$ und $F'(x) = \int_a^b f_x(x,t) \dt$
    \item Satz von Fubini:
        \begin{equation*}
            \int_\alpha^\beta F(x) \dx = \int_\alpha^\beta \int_a^b f(x,t) \dt \dx
            = \int_a^b \int_\alpha^\beta f(x,t) \dx \dt
        \end{equation*}
\end{enumerate}

\subsection{Leibniz Regel}
Seien $f(x,t), f_x(x,t)$ stetig in $[\alpha, \beta] \times [a,b]$ und
$u, v \in C^1 ( [a,b])$. Dann ist
\begin{equation*}
    F(x) = \int_{u(x)}^{v(x)} f(x,t) \dt \in C^1([a,b])
\end{equation*}
und
\begin{equation*}
    F'(x) = \int_{u(x)}^{v(x)} f_x(x,t) \dt +
    f(x, v(x)) v'(x) - f(x, u(x)) u'(x)
\end{equation*}

\subsection{Uneigentliche Parameterintegrale}
Ist für jedes $x \in M \subseteq \R$ ein uneigentliches Integral
\begin{equation*}
    \int_a^b f(x,t) \dt
\end{equation*}
mit kritischem Punkt $a$ oder $b$ gegeben, so heißt dieses gleichmäßig konvergent
in $M$, wenn gilt:
\begin{equation*}
    \forall \varepsilon > 0\ \exists L \in (a,b): \abs{\int_{T_1}^{T_2} f(x,t) \dt}
    <\varepsilon \ \forall x \in M \forall T_1, T_2 \in (a,L) (\text{bzw.}
    \forall T_1, T_2 \in (L,b))
\end{equation*}

\subsection{Majorantenkriterium}
Ein uneigentliches Integral $\int_a^b f(x,t) \dt$ konvergiert gleichmäßig in $M$ wenn ein konvergentes
Integral
\begin{equation*}
    \int_a^b g(t) \dt \text{ ex.\ mit} \abs{f(x,t)} \leq g(t)
\end{equation*}

\subsection{Fubini für uneigentliche Parameterintegrale}
Ist $f(x,t)$ stetig in $[\alpha, \beta] \times [a,b]$ und konvergiert
\begin{equation*}
    F(x) = \int_a^b f(x,t) \dt
\end{equation*}
gleichmäßig auf $[\alpha, \beta]$ dann ist $F$ stetig und
\begin{equation*}
    \int_\alpha^\beta \int_a^b f(x,t) \dt \dx =
    \int_a^b \int_\alpha^\beta f(x,t) \dx \dt
\end{equation*}

\subsection{Konvergenzkriterien}
Sind $f(x,t), f_x(x,t)$ stetig auf $[\alpha, \beta] \times [a,b]$ und ist
\begin{equation*}
    \int_a^b f(x, t) \dt
\end{equation*}
für ein $x_0 \in [\alpha, \beta]$ konvergent und ist
\begin{equation*}
    \int_a^b f_x(x,t) \dt
\end{equation*}
gleichmäßig konvergent. Dann gilt:
\begin{equation*}
    F(x) = \int_a^b f(x,t) \dt\ \forall x \in [\alpha, \beta]
\end{equation*}
und
\begin{equation*}
    F'(x) = \int_a^b f_x(x,t) \dt\ \forall x \in (\alpha, \beta)
\end{equation*}
existiert und ist stetig.

\section{Kurvenintegrale}
\subsection{Äquivalenz für Kurven}
Zwei stetige Funktionen $x: [a, b] \subseteq \R \rightarrow \R^n, y:[\alpha, \beta]
\subseteq \R \rightarrow \R^n$ heißen Äquivalent (schreibweise $x \sim y$), wenn eine
streng monoton wachsende Funktion
\begin{equation*}
    \phi: [a,b] \rightarrow [\alpha, \beta]
\end{equation*}
gibt mit
\begin{equation*}
    x(t) = y(\phi(t))\ \forall t \in [a,b]
\end{equation*}

\subsubsection{Bemerkung}
Es gilt:
\begin{enumerate}[label= (\alph*)]
    \item $x \sim x$ (Reflexivität)
    \item $x \sim y \Rightarrow y \sim x$ (Symmetrie)
    \item $x \sim y \land y \sim z \Rightarrow x \sim z$ (Transitivität)
\end{enumerate}

\subsection{Kurven im $\R^n$}
Ist $x: [a,b] \subseteq \R \rightarrow \R^n$ stetig, so nennt man die Menge
\begin{equation*}
    \K := \{y: [\alpha, \beta] \subseteq \R \rightarrow \R^n \text{ mit } x \sim y\}
\end{equation*}
die Kurve $\K$ mit Parameterdarstellung $x$ und den Punkt $x(a)$ Anfangspunkt und
$x(b)$ Endpunkt.

\subsubsection{Schreibweise}
\begin{equation*}
    \K: x(t), a \leq t \leq b
\end{equation*}
Die Menge
\begin{equation*}
    T(\K) := \{x(t): t \in [a,b]\} = x([a,b])
\end{equation*}
nennt man den Träger der Kurve $\K$.

\subsubsection{Bemerkung}
Verschieden Kurven können also den gleichen Träger haben.

Man nennt $K$:
\begin{enumerate}[label= (\alph*)]
    \item Geschlossen, wenn $x(a) = x(b)$
    \item Einfach oder Jordankurve, wenn $x(t) \neq x(s)\ \forall t,s: a \leq t < s <b$
\end{enumerate}

\subsection{Eigenschaften von Parameterdarstellungen}
\begin{enumerate}[label= (\alph*)]
    \item Eine Parameterdarstellung $x:[a,b] \rightarrow \R^n$ einer Kurve heißt
        stückweise stetig differentierbar, wenn eine Zerlegung
        \begin{equation}
            T: a=t_0 < \ldots < t_k = b
        \end{equation}
        existiert und $x$ auf $(t_l, t_{l+1})\ l\in\{0,\ldots,k-1\}$
        differentierbar ist.
    \item Besitzt eine Kurve $\K$ eine (stückweise) stetig differentierbare
        Parameterdarstellung $x(t), t\in [a,b]$ mit $\dot{x}(t) \neq \vec{0}$
        für $t \in [a,b]$ so heißt $\K$ stückweise glatt oder stückweise regulär.
    \item Ist eine Parameterdarstellung $x$ von $\K$ differentierbar und glatt,
        so heißt
        \begin{equation*}
            T(t) := \frac{\dot{x}(t)}{\norm{\dot{x}(t)}}
        \end{equation*}
        der Tangential (einheits) vektor von $x$ und $\K$
    \item Ist auch $T$ differentierbar und glatt (also $\dot{T}(k) \neq \vec{0}$)
        so heißt
        \begin{equation*}
            N(t) := \frac{\dot{T}(t)}{\norm{\dot{T}(t)}}
        \end{equation*}
        der (Haupt-) Normalen (einheits) vektor von $\K$ und $x$ bei $t$
    \item Und falls $n=3$
        \begin{equation*}
            B(t) = T(t) \times N(t)
        \end{equation*}
        der Binormalen (einheits) vektor von $\K$ und $x$ bei $t$
        (Man nennt dann $T(t), N(t), B(t)$ ein begleitendes Dreibein von $\K$)
    \item Existiert $T(t)$, so nennt man die Gerade
        \begin{equation*}
            \{x(t) + \lambda \dot{x}(t): \lambda \in \R\}
        \end{equation*}
        die Tangente von $\K$ bei $t$
    \item Existiert auch $N(t)$ so nennt man die Ebene
        \begin{equation*}
            \{ x(t) + \lambda \dot{x}(t) + \mu \ddot{x}(t): \lambda, \mu \in \R\}
        \end{equation*}
        die Schmiegeebene von $\K$ bei $t$.
\end{enumerate}

\subsubsection{Bemerkung}
Sei $x(t) = y(\phi(t))$ mit $a\leq t \leq b$ zwei Parameterdarstellungen von $x$.
Dann gilt:
\begin{equation*}
    T(t) = \frac{\dot{x}(t)}{\norm{\dot{x}(t)}} = \frac{\dot{y}(\varphi(t)) \cdot \dot{\varphi}(t)}
    {\norm{\dot{y}(\phi(t)) \cdot \dot{\varphi}(t)}} = \frac{\dot{y}(\varphi(t))}
    {\norm{\dot{y}(\varphi(t))}}
\end{equation*}
Das heißt die Berechnung von $T$ ist unabhängig von der konkreten Parameterdarstellung

Existiert $N(t)$ dann gilt:
\begin{equation*}
    N(t) \bot T(t)
\end{equation*}
Existiert auch $B(t)$ (im $\R^3$), dann gilt:
$N(t), T(t), B(t)$ sind paarweise Orthogonal.

\subsection{Weitere Definitionen zu Kurven}
\begin{enumerate}[label= (\alph*)]
    \item Ist $\K: x(t), a \leq t \leq b$ eine Kurve, so heißt:
        \begin{equation*}
            - \K: y(t), a \leq t \leq b \text{ mit } y(t)=x(a+b-1)
        \end{equation*}
        die zu $\K$ entgegengesetzte Kurve
    \item Sind $\K: x(t), a \leq t \leq b$ und $\mathbb{L}: y(t), \alpha \leq
        t \leq \beta$ zwei Kurven und gilt $x(b) = y(\alpha)$ dann ist
        \begin{equation*}
            \K + \mathbb{L}: z(t), a \leq t \leq (\beta-\alpha) + b
        \end{equation*}
        und
        \begin{equation*}
            z(t) =
            \begin{cases}
                x(t) &, a \leq t \leq b\\
                y(t-b+\alpha) &, b \leq t \leq (\beta-\alpha) + b
            \end{cases}
        \end{equation*}
        die Aus $\K$ und $\mathbb{L}$ zusammengesetzte Kurve.
\end{enumerate}

\subsection{Kurventintegrale 2. Art}
Sei $\K$ eine Kurve im $\R^n$ und
\begin{equation*}
    f: T(\K) \rightarrow \R^n
\end{equation*}
\begin{enumerate}[label= (\alph*)]
    \item Sei $x: [a,b]  \rightarrow \R^n$ eine Parameterdarstellung von $\K$
        \begin{enumerate}[label = (\roman*)]
            \item Für eine Zerlegung $T: a = t_0 < \cdots < t_n = b$,
                Zwischenpunte $Z: (\xi_1, \ldots, \xi_n)$ mit
                $t_{k-1} \leq \xi_k \leq t_k$
                heißt
                \begin{equation*}
                    S(f,x,T,Z) := \sum_{k=1}^n f(x(\xi_k)) \cdot (x(t_k) - x(t_{k-1}))
                \end{equation*}
                die Riemann-Summe von $f, T, Z$ bezüglich $x$.
            \item Exitiert eine Zahl $I \in \R$ derart, dass für jede Folge von
                Zerlegungen $T_n$ mit
                \begin{equation*}
                    \lim_{n \rightarrow \infty} \mu(T_n) = 0
                \end{equation*}
                stets
                \begin{equation*}
                    \lim_{n \rightarrow \infty} S(f,x,T_n, Z_n) = I
                \end{equation*}
                folgt, so heißt $I$ das Kurvenintegral (2. Art) von
                $f$ längs $\K$ bzgl. $x$.
        \end{enumerate}
    \item Gibt es stets ein $I$ wie in (a) so heißt $f$ längs $\K$ (Riemann-)
        integrierbar und man nennt $I$ das (unbestimmte) Kurvenintegral von $f$
        längs $\K$ und schreibt:
        \begin{equation*}
            I = \int_\K f = \int_\K f(x) \cdot \dx = \int_\K f_1(x) \dx_1 + \cdots
            + f_n(x) \dx_n
        \end{equation*}
\end{enumerate}

\subsection{Substitutionsregel}
Ist $\K: x(t), a \leq t \leq b$ eine Kurve im $\R^n$ und $x(t)$ stückweise
differentierbar, sowie $f: T(\K) \rightarrow \R^n$ stetig, so ist $f$ längs
$\K$ integrierbar und es gilt:
\begin{equation*}
    \int_\K f(x) \dx = \int_a^b f(x(t)) \text{d}x(t) = \int_a^b f(x(t)) \cdot
    \dot{x}(t) \dt
\end{equation*}

\subsection{Definition Wegunabhängigkeit}
Sei $f\in C(G, \R^n)$ mit $G \subseteq \R^n$ ein Gebiet:
\begin{enumerate}[label= (\alph*)]
    \item Gilt für zwei Wege $\K$ und $\mathbb{L}$ mit gleichem Anfangs- und
        Endpunkt stets
        \begin{equation*}
            \int_\K f = \int_{\mathbb{L}} f
        \end{equation*}
        dann heißen die Kurvenintegrale Wegunabängig in $G$.
    \item Eine Funktion $F \in C^1(G,\R)$ heißt Stammfunktion von $f$ in $G$, wenn
        \begin{equation*}
            \nabla F(x) = f(x)\ \forall x \in G
        \end{equation*}
        gilt.
    \item Man nennt
        \begin{equation*}
            P := -F
        \end{equation*}
        das Potential von $f$.
    \item Man nennt $f$ konservativ in $G$ oder ein Potentialfeld oder Gradienentenfeld
        in $G$, wenn $f$ eine Stammfunktion hat.
\end{enumerate}

\subsection{1. Hauptsatz für Kurvenintegral}
Sei $f$ konservativ in $G$ mit Stammfunktion $F$ und Potential $P$ dann gilt
für jeden Weg $\K$ in $G$ mit Anfangspunkt $A \in G$ und Endpunkt $B \in G$:
\begin{equation*}
    \int_\K f = F(B) - F(A) = P(A) - P(B)
\end{equation*}
insbesondere ist also das Integral wegunabhängig.

\subsection{Äquivalente Aussagen zu Stammfunktionen}
\begin{enumerate}[label= (\alph*)]
    \item
        \begin{equation*}
            \int_\K f \text{ ist wegunabhängig in } G
        \end{equation*}
    \item
        $f$ besitzt eine Stammfunktion
    \item
        \begin{equation*}
            \int_\K f = 0 \text{ für jede geschlossene Kurve } \K
        \end{equation*}
\end{enumerate}

\subsubsection{Bemerkung}
Rechenregeln für zwei Kurven $\K$ und $\mathbb{L}$:
\begin{enumerate}[label= (\alph*)]
    \item
        \begin{equation*}
            \int_{\K + \mathbb{L}} f = \int_\K f + \int_\mathbb{L} f
        \end{equation*}
    \item
        \begin{equation*}
            \int_{-\K} f = - \int_\K f
        \end{equation*}
\end{enumerate}

\subsection{Definition einfach zusammenhängende Gebiete}
Ein Gebiet $G \subseteq \R^n$ heißt einfach zusammenhängend, wenn sich jede
geschlossene Kurve in $G$ innerhalb von $G$ \glqq{}auf einen beliebigen
Punkt zusammenziehen lässt\grqq{}.

\subsection{Sternförmige Gebiete}
Eine Menge $G \subseteq \R^n$ heißt Sternförmig bezüglich $x_0 \in G$, wenn für
alle $x \in G$ gilt, dass $\overline{x_0x} \subseteq G$ (d.h.\ jedes $x$ ist von
$x_0$ durch einen Streckenzug erreichbar). $G$ ist ein sternförmiges Gebiet,
wenn $G$ offen und sternförmig ist.

\subsubsection{Bemerkung}
\begin{equation*}
    G \text{ sternförmig} \Rightarrow G \text{ einfach zusammenhängend}
\end{equation*}

\subsection{2. Hauptsatz für Kurvenintegrale}
Sei $f \in C^1(G, \R^n), G \subseteq \R^n$ ein Gebiet, dann gilt:
\begin{enumerate}[label= (\alph*)]
    \item Besitzt $f$ eine Stammmfunktion in $G$, so erfüllt $f$ in $G$ die
        Integrabilitätsbedingung:
        \begin{equation*}
            \frac{\partial f_l}{\partial x_k} = \frac{\partial f_k}{\partial x_l}\
            k,l \in \{ 1, \ldots, n\}
        \end{equation*}
        D.h.\ die Jacobi-Matrix von $f$ ist symetrisch.

        Kurz:
        \begin{equation*}
            f \text{ hat Stammfunktion} \Rightarrow f' = (f')^T
        \end{equation*}
    \item Ist $G$ einfach zusammenhängend und erfüllt $f$ die Integrabilitätsbedingung
        dann besitzt $f$ eine Stammfunktion.

        Kurz:
        \begin{equation*}
            G \text{ einfach zusammenhängend} \land f' = (f')^T \Rightarrow
                \exists F: \nabla F = f
        \end{equation*}
\end{enumerate}

\subsection{Definition Rotation}
Sei $G \subseteq \R^3$ offen und $f: G \rightarrow R^3$ partiell differentierbar,
dann heißt die Funktion $\rot f: G \rightarrow \R$ mit
\begin{equation*}
    \rot f(x) :=
    \begin{pmatrix}
        \dfrac{\partial f_3}{\partial x_2} - \dfrac{\partial f_2}{\partial x_3} \\
        \dfrac{\partial f_1}{\partial x_3} - \dfrac{\partial f_3}{\partial x_1} \\
        \dfrac{\partial f_2}{\partial x_1} - \dfrac{\partial f_1}{\partial x_2} \\
    \end{pmatrix}
\end{equation*}
die Rotation von $f$ in $G$.

\subsubsection{Bemerkung}
Im Fall $f: G \subseteq \R^2 \rightarrow \R^2$ definiert man
\begin{equation*}
    \rot f(x_1, x_2) = \frac{\partial f_2}{\partial x_1} - \frac{\partial f_1}{\partial x_2}
\end{equation*}
Formal betrachtet man die Hilfsfunktion
\begin{equation*}
    \tilde{f}(x,y,z) :=
    \begin{pmatrix}
        f_1(x,y)\\
        f_2(x,y)\\
        0
    \end{pmatrix}
\end{equation*}

\subsection{Zusammenhang Rotation und Integrabilitätsbedingung}
Ist $f \in C^1(G, \R^3), G$ ein Gebiet, dann gilt
\begin{enumerate}[label= (\alph*)]
    \item $f$ besitzt eine Stammfunktion $\Rightarrow \rot f = \vec{0}$
    \item $G$ einfach zusammenhängend und $\rot f = \vec{0} \Rightarrow f$
        hat Stammfunktion.
\end{enumerate}

\subsection{Definition Linienintegral/Kurvenintegral 1. Art}
Sei $\K: x(t), a \leq t \leq b$ ein Weg, und $x$ stückweise differentierbar.
Für ein $\phi \in  C(T(\R), \R)$ heißt
\begin{equation*}
    \int_\K \phi \ds := \int_a^b \phi(x(t)) \norm{ \dot{x}(t)} \dt
\end{equation*}
ein Linienintegral oder Kurvenintegral 1. Art von $\phi$ längs $\K$.

\subsubsection{Bemerkung}
\begin{enumerate}[label= (\alph*)]
    \item
        Mit $\phi \equiv 1$:
        \begin{equation*}
            \int_\K 1 \ds = \int_a^b \phi(x,t) \norm{\dot{x}(t)} \dt
            \int_a^b  \norm{\dot{x}(t)} \dt = l(\K)
        \end{equation*}
        d.h.\ mit Linienintegralen können auch Weglängen berechnet werden,
        bzw. Weglängen berechnet man mit $\phi = 1$.
    \item
        $\phi: [a,b] \rightarrow \R$ wähle $\K: x(t) = a + t \cdot (b-a)\
        t \in [0,1]$:
        \begin{equation*}
            \int_\K \phi \ds = \int_0^1 \phi (a + t \cdot (b-a)) \norm{b-a} \dt
            = \int_a^b \phi(t) \dt
        \end{equation*}
    \item Linienintegrale hängen nicht von der Parameterdarstellung ab.
    \item Man schreibt (falls Parameter-Darstellung bekannt ist) oft
        \begin{equation*}
            \ds = \norm{\dot{x}(t)} \dt
        \end{equation*}
        und nennt $\ds$ Bogensegment oder Liniensegment.
    \item Ist $f \in C(T(\K), \R^n)$ und $\dot{x}(t) \neq \vec{0}\ \forall
        t \in [a,b]$, dann ist:
        \begin{eqnarray*}
            \int_\K f &=& \int_\K f(x) \dx = \int_a^b f(x(t)) \dot{x}(t) \dt \\
            &=& \int_a^b \frac{f(x(t)) \dot{x}(t)}{\norm{\dot{x}(t)}} \norm{\dot{x}(t)}
            \dt \\ &=& \int_a^b f(x(t)) \cdot T(t) \norm{\dot{x}(t)} \dt  \\ &=&
            \int_\K \phi \ds \text{ mit } \phi(t) = f(x(t)) \cdot T(t)
        \end{eqnarray*}
\end{enumerate}

\section{Bereichsintegrale}
Hier: $f: G \subseteq \R^n \rightarrow \R$ und
\begin{equation*}
    \int_G f = \int_G f(x_1, \ldots, x_n) \intd{(x_1, \ldots x_n)}
\end{equation*}
sollen anschaulich bedeuten:

Welches Volumen schließt der Graph von $f$ mit der Grundfläche $G$ ein.

\subsection{Intervalle im $\R^n$}
Für $a,b \in \R^n$ bezeichnet die Menge
\begin{equation*}
    [a,b] := [a_1, b_1] \times \cdots \times [a_n, b_n]
\end{equation*}
einen (kompakten) Quader oder (kompaktes) Intervall im $\R^n$. Die Zahl
\begin{equation*}
    V([a,b]) =
    \begin{cases}
        \prod_{k=1}^n (b_k - a_k) &, \text{falls }  b_k > a_k \text{ für } k=1,\ldots\\
        0 &,\text{sonst}
    \end{cases}
\end{equation*}
bezeichnet das Volumen, und die Zahlen $b_1 - a_1, \ldots b_n - a_n$ als
Kantenlängen.

\subsection{Definition Zerlegung}
Ist $[a,b] = [a_1, b_1] \times \cdots \times [a_n, b_n]$ und ist für jedes
$k \in \{1, \ldots, n\}$ mit
\begin{equation*}
    T^{(k)} : a_k = x_0 < \cdots < x_{l_k} = b_k
\end{equation*}
eine Zerlegung von $[a_k, b_k]$ dann heißt die Menge
\begin{equation*}
    I_{l_1, \ldots, l_n} = [x_{l_1 - 1}^{(1)} - x_{l_1}^{(1)}] \times \cdots \times
    [x_{l_1 - 1}^{(n)} - x_{l_1}^{(n)}]
\end{equation*}
mit $l_k \in \{1, \ldots, l_k\}$ für $k \in \{1, \ldots, n\}$ eine Zerlegung $T$
von $[a,b]$.

Das Feinheitsmaß von $T$ ist
\begin{equation*}
    \mu(T) = \max_{l_1, \ldots, l_n} V(I_{l_1, \ldots, l_n})
\end{equation*}

Allgemein ist ein Intervall von der Form
\begin{equation*}
    [x_i^{(1)}, x_{i+1}^{(1)}] \times [x_j^{(2)}, x_{j+1}^{(2)}]
\end{equation*}
mit $i \in \{0, \ldots, l_1-1\}$ und $j \in \{0, \ldots l_2-1\}$.

\subsection{Definition Riemann-Summe}
Sei $T$ eine Zerlegung eines kompakten Quaders $I \subseteq \R^n$ mit Teilquadern
$I_1, \ldots, I_l$ mit $l=l_1 \cdot \cdots \cdot l_n$ (entstehen indem man die
Zerlegungsintervalle fortlaufend durchnummeriert) und Zwischenpunkte
$\xi = (\xi_1, \ldots, \xi_l)$ mit $\xi_i \in I_i (i \in \{l, \ldots, n\})$
und $f: I \to \R$ (d.h. Skalarwertige Funktion). Dann heißt
\begin{equation*}
    S(f, T, \xi) = \sum_{i=1}^l f(\xi_i) f(\xi_i) \mu(I_1)
\end{equation*}
die Riemann-Summe von $f$ bezüglich $T$ und $\xi$.

\subsection{Riemann integrierbare Bereichsintegrale}
Sei $f: I \to \R$ eine Funktion, $I \subseteq \R^n$ ein Quader. Gibt es eine Zahl
$\alpha \in \R$, so dass für jede Folge von Zerlegungen ${(T_k)}_{k=1}^\infty$ mit
Zwischenpunkten ${(\xi_k)}_{k=1}^\infty$ mit $\lim\limits_{k \to \infty} \mu(T_k)=0$
die Riemann-Summe $S(f, T_k, \xi_k)$ gegen $\alpha$ konvergiert für $k \to \infty$
dann heißt $f$ Riemann integrierbar über $I$ und $\alpha$ nennen wir das
Bereichsintegral von $f$ über $I$.

\subsubsection{Schreibweise}
\begin{equation*}
    \alpha = \int_I f(x) \dx
\end{equation*}
Zur Schreibweise: z.B. $n=2$ auch:
\begin{equation*}
    \alpha = \iint_I f(x,y) \intd{(x,y)} := \int_I f(x,y) \intd{(x,y)}
\end{equation*}
oder Angabe von $I$ an dem Integral:
\begin{equation*}
    \alpha = \int_{[a_1, b_1]\times[a_2,b_2]} f(x,y) \intd{(x,y)}
\end{equation*}

\subsection{Bereichsintegrale über beschränkte Mengen}
Sei $M \subseteq \R^n$ beschränkt und $I = [a_1, b_1] \times \cdots \times [a_n,
b_n]$ ein Quader mit $M \subseteq I$. Dann heißt $f: M \to \R$ über $M$ integrierbar
wenn die Funktion
\begin{equation*}
    \tilde{f}: I \to \R \text{ mit } \tilde{f}(x) =
    \begin{cases}
        f(x) &, x\in M\\
        0 &, \text{ sonst}
    \end{cases}
\end{equation*}
über $I$ Bereichs-Riemann integrierbar ist. Wir definieren:
\begin{equation*}
    \int_M f(x) \dx = \int_I \tilde{f}(x) \dx
\end{equation*}

\subsection{Cavalieri}
Sei $M \subseteq \R^n (n >1)$ und bezeichne
\begin{equation*}
    M' = \{ x \in \R: {(x,y)}^T \in M \text{ für ein } y \in \R^{n-1} \}
\end{equation*}
und für $x \in M'$
\begin{equation*}
    M(x) = \{ y \in \R^{n-1}: {(x,y)}^ \in M \}
\end{equation*}
dann gilt für $f \in C(\bar{M})$ (falls $M, M', M(x)$ sogenannte messbare Mengen
sind, d.h $\mu(M), \mu{M'}, \mu{M(x)}$ sind definiert)
\begin{equation*}
    \int_M f(x,y) \intd{(x,y)} = \int_{M'} \left[ \int_{M(x)} f(x,y) \dy \right]
        \dx
\end{equation*}
mit $x \in \R$ und $y \in \R^{n-1}$.

\subsection{Fubini}
Im Fall $n=2$ steht nach Cavalieri ein Parameterintegral und mit Fubini gilt:
\begin{equation*}
    \int_{M'} \int_{M(x)} f(x,y) \dy \dx =
    \int_{\tilde{M}'} \int{\tilde{M}(y)} f(x,y) \dx \dy
\end{equation*}
wobei $\tilde{M}', \tilde{M}(y)$ analog zu $M', M(x)$ bezüglich $y$ definiert
sind.

\subsection{Definition Meßbare-Mengen}
Eine beschränkte Menge $M \subseteq \R^n$ heißt (Jordan-) meßbar, wenn
\begin{equation*}
    \int_M 1 \dx
\end{equation*}
existiert, in diesem Fall nennt man
\begin{equation*}
    \mu(M) := \int_M 1 \dx
\end{equation*}
das Volumen von $M$. Ist $\mu{M} =0$, so nenntn man $M$ eine Nullmenge.

\subsection{Definition $2 \times 2$ Determinante}
Für
\begin{equation*}
    A =
    \begin{pmatrix}
        a & c \\
        b & d \\
    \end{pmatrix}
    \in \R^{2 \times 2}
\end{equation*}
definieren wir die Funktion
\begin{equation*}
    \det: \R^{2 \times 2} \to \R
\end{equation*}
durch $A \mapsto \det(A) = a \cdot d -  c \cdot b$ und nennen die
Funktionsauswertung die Determinante von $A$.

\subsection{Mehrdimensionale Substitutionsregel}
Sei $M \subseteq \R^n$ meßbar und $G \supseteq M$ ein Gebiet. Ist $T\in C^1(G, \R^m)$
und gilt $\det(T'(x)) \neq 0\ \forall x \in M \setminus N$ für eine Nullmenge $N$,
dann gilt:
\begin{equation*}
    \int_{T(M)} f(x_1, \ldots, x_n) \intd{(x_1, \ldots, x_n)} =
    \int_M f(T(u_1, \ldots, u_n)) \cdot \abs{\det(T'(u_1, \ldots, u_n))} \intd{(u_1, \ldots
    u_n)}
\end{equation*}

\section{Integralsätze in der Ebene}
\subsection{Positiv berandete Menge}
Eine beschränkte Menge $B \subseteq \R^2$ heißt positiv berandet durch einen Weg,
(Randkurve) $\K$, wenn $T(\K) = \partial B$ ist und wenn $\K$ eine stückweise
stetig differentierbare Parameterdarstellung $x: [a,b] \to \R^2$ hat mit:
\begin{enumerate}[label = (\roman*)]
    \item $\dot{x}(t) \neq 0$ für fast alle $t \in [a,b]$
    \item der Normalenvektor von $x(t)$ zeigt nach außen
\end{enumerate}

\subsection{Satz von Green}
Ist $B \subseteq \R^2$ positiv berandet, dann gilt für alle $f \in C^1(B, \R^2)$
\begin{equation*}
    \iint_B \frac{\partial f_2}{\partial x} - \frac{\partial f_1}{\partial y}
    \intd{(x,y)} =
    \int_{\partial B} f(x,y) \intd{(x,y)}
\end{equation*}

\subsection{Definition Normalbereiche}
Eine Menge $B \subseteq \R^2$ heißt Normalbereich bezüglich der $x$-Achse
(bzw. $y$-Achse), wenn es ein Intervall $[a,b]$ gibt und die Funktion
$\varphi, \psi$ mit
\begin{equation*}
    B = \{ {(x,y)}^T: a \leq x \leq b, \phi(x) \leq y \leq \psi(x) \}
\end{equation*}

\subsection{Gauß'sche Integralsätze in der Ebene}
Sei $B \subseteq \R^2$ ein positiv berandeter Bereich und $f \in C^1(B, \R^2)$
bzw. $f \in C^2(B, \R^2)$ und bezeichne $\nu$ die nach außen gerichtete Normale
auf $\partial B$. Dann gelten die Integralsätze:
\begin{enumerate}[label = (\roman*)]
    \item
        \begin{equation*}
            \iint_B (\vdiv f)(x,y) \intd{(x,y)} = \int_{\partial B} f \cdot \nu \ds
        \end{equation*}
    \item
        \begin{equation*}
            \iint_B f_1(x,y) \Delta f_2(x,y) - f_2(x,y) \Delta f_1(x,y) \intd{(x,y)}
            = \int_{\partial B} f_1 \frac{\partial f_2}{\partial \nu} - f_2
            \frac{\partial f_1}{\partial \nu}
        \end{equation*}
\end{enumerate}

\section{Oberflächenintegrale und Integralsätze im $\R^3$}
\subsection{Definition Reguläre Flächen}
Sei $B \subseteq \R^2$ und $x: B \to \R^3$,
\begin{equation*}
    x(u,v) =
    \begin{pmatrix}
        x_1(u,v)\\
        x_2(u,v)\\
        x_3(u,v)
    \end{pmatrix}
\end{equation*}
eine stetig diffbare Funktion, für die $x_u = \frac{\partial x}{\partial u}$ und
$x_v = \frac{\partial x}{\partial v}$ linear unabhängig sind (d.h.\ die Vektoren
$x_u$ und $x_v$ zeigen nicht in die gleiche oder entgegengesetzte Richtung) (für
fast alle ${(u,v)}^T \in B$) die Menge der Ausnahmen muss $\tilde{B} \subseteq B$
muss $\mu{\tilde{B}} = 0$ erfüllen.

Das Bild einer solchen Funktion, d.h.\ die Menge
\begin{equation*}
    A = x(B) := \{ x(u,v) \vert {(u,v)}^T \in B \}
\end{equation*}
heißt dann eine reguläre Fläche im $\R^3$ und die Funktion $x$ heißt die
Parametrisierung von $A$.

Man nennt
\begin{enumerate}[label = (\roman*)]
    \item $x_u(u,v), x_v(u,v)$ die Tangentialvektoren in ${(u,v)}^T$
    \item $n(u,v) := \frac{(x_u \times x_v)(u,v)}{\norm{(x_u \times x_v)}(u,v)}$
        \begin{itemize}
            \item Vektor mit Länge $1$ der Senkrecht auf den Tangentialvektoren
                steht
            \item Rechnerisch zu enthalten durch das Kreuzprodukt der
                Tangentialvektoren
        \end{itemize}
\end{enumerate}
der (Flächen-) Normalenvektor in ${(u,v)}^T$ falls $x_u(u,v)$ und $x_v(u,v)$
linear unabhängig sind.

Ist $B$ positiv berandet durch $\K: y(t), a \leq t \leq b$ so nennt man $A$ positiv
berandet durch Kurve mit Parameterdarstellung $x(y(t)), a \leq t \leq b$.

\subsection{Defintion Oberflächenintegral}
Sei $A$ eine reguläre Fläche im $\R^3$ mit Parameterdarstellung $x: B \to \R^3,
B \subseteq \R^2$ meßbar und $x$ injektiv auf $B \setminus N$ für eine Nullmenge
$N$.
\begin{enumerate}[label = (\alph*)]
    \item Für jedes $f \in C(A, \R)$ heißt
        \begin{equation*}
            \iint_A f \cdot \ido= \iint_B f(x(u,v)) \cdot \norm{(x_u \times x_v)
            (u,v)} \intd{(u,v)}
        \end{equation*}
        das Oberflächenintegral von $f$ über $A$ und man nennt
        \begin{equation*}
            \ido= \norm{(x_u \times x_v)
            (u,v)} \intd{(u,v)}
        \end{equation*}
        das Oberflächenelement.
    \item $O(A) := \iint_A 1 \do$ heißt Oberflächeninhalt von $A$.
\end{enumerate}

\subsubsection{Bemerkung}
\begin{enumerate}
    \item Das Oberflächenintegral hängt nicht von der Parameterdarstellung ab.
    \item Ein Summand des Obefĺöchenintegrals sieht so aus:
        \begin{equation*}
            f(x(u,v)) \cdot \norm{(x_u \times x_v)(u,v)} \cdot \Delta u \Delta v
        \end{equation*}
\end{enumerate}

\subsection{Satz von Stokes}
Sei $A$ eine reguläre Fläche im $\R^3$ und $\partial A$ positiv berandet. Dann
gilt für $f \in C^1(A, \R^3)$
\begin{equation*}
    \iint_A \rot f \cdot n \ido= \int_{\partial A} f
\end{equation*}
Mit $n$:
\begin{enumerate}[label = (\roman*)]
    \item Normalenvektor
    \item Länge 1
    \item Senkrecht auf Fläche
    \item Immer auf der gleichen Seite von $A$
\end{enumerate}
also
\begin{equation*}
    \iint_B \rot(f(x(u,v))) \cdot n(x(u,v)) \cdot \norm{(x_u \times x_v)(u,v)}
    \intd{(u,v)} = \int_{\partial A} f
\end{equation*}
mit
\begin{equation*}
    n(x(u,v)) = \pm \frac{(x_u \times x_v)(u,v)}{\norm{(x_u \times x_v)(u,v)}}
\end{equation*}

\subsection{Divergenzsatz von Gauß}
Sei $M \subseteq \R^3$ kompakt und $\partial M$ ergebe sich als endliche
Vereinigung von regulären Flächen, deren Normale $n$ (normiert) nach Außen
zeigt. Dann gilt für jedes $f \in C^1(M, \R^3)$
\begin{equation*}
    \iiint_M \vdiv f = \iint_{\partial M} f \cdot n \ido
\end{equation*}


    \chapter{Lineare Algebra}
    \section{Der Begriff Vektorraum}
\subsection{Definition Vektorraum}
Gegeben sei eine abelsche Gruppe $V$ und ein Körper $\K$ (bei uns wird $\K=\R$
oder $\K=\C$ gelten) und eine Abbildung:
\begin{equation*}
    \cdot: \K \times V \to V, \cdot(\alpha, x) \mapsto \alpha \cdot x =:
    \alpha x \text{ (Skalierung)}
\end{equation*}
Dann nennt man $V$ einen Vektorraum über $\K$, wenn die folgenden Vektorraumaxiome
erfüllt sind:
\begin{eqnarray*}
    &\text{(V1)}& \alpha \cdot (\beta \cdot v) = (\alpha \cdot \beta) \cdot v
    \text{ (Assoziativgesetz)}\\
    &\text{(V2)}& \alpha \cdot (x+y) = (\alpha \cdot x) + (\alpha \cdot y) =
    \alpha x + \alpha y \\
    && (\alpha + \beta) \cdot x = \alpha x + \beta x
    \text{ (Distributivgesetzte)}\\
    &\text{(V3)}& 1 \cdot x = x \text{ für die } 1 \in \K \text{ (Gesetz der Eins)}
\end{eqnarray*}

In einem Vektorraum $V$ über $\K$ nennt man Elemente aus $V$ Vektoren, die Elemente
aus $\K$ Skalare, $\K$ den Skalarkörper und \glqq{}$\cdot$\grqq{} die
Multiplikation mit Skalaren. Die \glqq{}$+$\grqq{} Verknüpfung in $V$ die $V$ die
Vektoraddition und das neutrale Element $\vec{0} \in V$ den Nullvektor.

\subsection{Rechenregeln}
Ist $V$ ein Vektorraum über $\K$, so gilt für $\alpha, \beta \in \K$ und
$x, y \in V$:
\begin{enumerate}
    \item
        \begin{enumerate}[label = (\alph*)]
            \item $0 \cdot x = \vec{0} = \alpha \vec{0}$
            \item $\alpha \cdot x = \vec{0} \Rightarrow \alpha = 0 \lor x=\vec{0}$
        \end{enumerate}
     \item
        \begin{equation*}
            \alpha (-x) = (-\alpha) x = - (\alpha x)
        \end{equation*}
\end{enumerate}

\section{Unterräume}
\subsection{Definition Unterraum}
Eine Teilmenge $U$ eines Vektorraums $V$ über $\K$ heißt Unterraum von $V$, wenn
$U$ bezüglich der in $V$ definierten Vektoraddition und Skalierung ein Vektorraum
ist.

\subsection{Unterraumkriterien}
Für $U \subseteq V$ und $U \neq \emptyset$ sind folgende Aussagen äquivalent
\begin{enumerate}[label = (\alph*)]
    \item $U$ ist ein Unterraum von $V$
    \item
        \begin{equation*}
            x,y \in U, \alpha, \beta \in \K \Rightarrow
            \alpha x + \beta y \in U
        \end{equation*}
    \item
        \begin{equation*}
            (x,y \in U \Rightarrow x+y \in U) \land
            (\alpha \in K, x \in U \Rightarrow \alpha x \in U)
        \end{equation*}
\end{enumerate}

\subsection{Durchschnitt von Unterräumen}
Der Durchschnitt von Unterräumen ist wieder ein Unterraum, d.h.:
\begin{equation*}
    U_i\ i \in J\ (J\text{ eine Indexmenge}) \text{ sind Unterräume}
    \Rightarrow \bigcap_{i \in J} U_i \text{ ist Unterraum}
\end{equation*}

\subsection{Defintion lineare Hülle}
\begin{itemize}
    \item Ist $M$ eine beliebige Teilemenge eines Vektorraums. Dann heißt
        \begin{equation*}
            \vspan (M) := \bigcap_{U \in S} U \text{ mit }
            S := \{ U \subseteq V: U \text{ ist Unterraum}, U \supseteq M \}
        \end{equation*}
        der von $M$ aufgespannte Unterraum oder die lineare Hülle von M.
    \item Ist $U$ ein Unterraum und $M \subseteq V$ mit $\vspan(M) = U$, dann heißt
        $M$ ein erzeugendes System von $U$.
\end{itemize}

\subsubsection{Bemerkung}
\begin{enumerate}
    \item $\vspan(M)$ ist der kleinste Unterraum, der $M$ enthält
    \item $\vspan(\emptyset) = \vec{0}$
    \item $M \subseteq N \Rightarrow \vspan(M) \subseteq \vspan(N)$
    \item Ist $U$ ein Unterraum, dann gilt $U=\vspan(U)=\vspan(U \setminus \{\vec{0}\})$
\end{enumerate}

\subsection{Definition Linearkombination}
Ist $V$ ein Vektorraum über $\K$ und $x_1, \ldots x_n \in V, \alpha_1, \ldots,
\alpha_n \in \K$ dann heißt
\begin{equation*}
    \sum_{k=1}^n a_k x_k \in V
\end{equation*}
eine Linearkombination von $x_1, \ldots, x_n$ (mit Koeffizienten $\alpha_1, \ldots
\alpha_n$).

\subsection{Zusammenhang lineare Hülle --- Linearkombination}
Sei $V$ ein Vektorraum über $\K$ und $M \subseteq V$, dann gilt $\vspan{M}$ ist
die Menge aller Linearkombinationen, d.h.
\begin{equation*}
    \vspan(M) = \{\alpha_1 x_1 + \cdots + \alpha_n x_n \vert x \in \N,
    x_1, \ldots, x_n \in M, \alpha_1, \ldots, \alpha_n \in \K \}
\end{equation*}
im Fall $M=\{x_1, \ldots, x_n\}$ gilt:
\begin{equation*}
    \vspan(M) =  \{\alpha_1 x_1 + \ldots \alpha_n x_n \vert \alpha_1, \ldots
    \alpha_n \in \K \}
\end{equation*}

\section{Lineare Unabhängigkeit}
\subsection{Definition Lineare Unabhängigkeit}
Sei $V$ ein Vektorraum über $\K$
\begin{enumerate}[label = (\alph*)]
    \item Eine endliche Liste $a_1, \ldots, a_n \in V$ heißt linear unabhängig
        (l.u.), wenn gilt
        \begin{equation*}
            \alpha_1 a_1 + \cdots + \alpha_n a_n = \vec{0}
            \Rightarrow \alpha_1 = \cdots = \alpha_n  = 0
        \end{equation*}
        Andernfalls heißen $a_1, \ldots, a_n$ linear abhängig (l.a.).
    \item Eine beliebige Teilmenge $M \subseteq V$ heißt linear unabhängig, wenn
        für eine beliebige endliche Liste $a_1, \ldots, a_n \in M$ gilt, dass
        diese linear unabhängig sind. Andernfalls ist $M$ linear abhängig.
\end{enumerate}

\subsection{Rechenregeln für lineare Unabhängigkeit}
Für Vektoren $a, a_1, \ldots, a_n, b_1, \ldots, b_n$ eines Vektorraumes $V$ gilt:
\begin{enumerate}[label = (\alph*)]
        \item
            \begin{equation*}
                a \text{l.u.} \Leftrightarrow \{a\} \text{ l.u.} \Leftrightarrow
                a \neq \vec{0}
            \end{equation*}
            Bemerkung:

            $a_1, a_2$ mit $a_1 = a_2$ ist linear unabhängig, aber $M=\{a, a\} =
            \{a\}$ ist nur dann linear abhängig wenn $a = \vec{0}$.
        \item
            $a_1, \ldots, a_n$ linear abhängig $\Rightarrow$ $a_1, \ldots a_n,
            b_1, \ldots, b_k$ sind linear abhängig für $k \geq 0$.
        \item $a_1, \ldots, a_n$ linear unabhängig $\Rightarrow a_1, \ldots, a_k$
            linear unabhängig für $k \leq n$
        \item $a_1, \ldots, a_n$ linear unabhängig $\Rightarrow a_1, \ldots a_n$
            sind paarweise verschieden
        \item Für $n \geq 2$ sind $a_1, \ldots, a_n$ genau dann linear abhängig,
            wenn ein Vektor als Linearkombination darstellbar ist. D.h.:
            \begin{equation*}
                \exists i \in \{1, \ldots, n\}: a_i = \sum_{k=1, k \neq i}^n
                \alpha_k a_k \text{ für } \alpha_1, \ldots, \alpha_{i-1},
                \alpha_{i+1}, \ldots, \alpha_n \in \K
            \end{equation*}
        \item Sind $a_1, \ldots, a_n$ linear unabhängig und $a_1, \ldots, a_n, a$
            linear abhängig, so ist $a$ die linear Kombination von $a_1, \ldots,
            a_n$ und die Koeffizienten sind eindeutig.
        \item Ist $a$ eine Linearkombination von $a_1, \ldots, a_n$ und jeder
            Vektor $a_k$ eine Linearkombination von $b_1, \ldots, b_m$ so ist
            $a$ eine Linearkombination von $b_1, \ldots, b_m$
\end{enumerate}

\subsubsection{Bemerkung}
Für Teilmengen $M, N$ eines Vektorraums $V$ gilt:
\begin{enumerate}[label= (\alph*)]
    \item $M$ l.a. $M \subseteq N \Rightarrow N$ l.a.
    \item $M=\emptyset \Rightarrow M$ l.u.
    \item $\vec{0} \in M \Rightarrow M$ l.a.
\end{enumerate}

\paragraph{Für $V = \R^3$}
\begin{enumerate}[label= (\alph*)]
    \item $a_1, a_2, a_3$ seien linear abhängig und $a_1, a_2$ linear unabhängig

        $\Leftrightarrow$ also $a_3$ ist in der von $a_1$ und $a_2$ aufgespannten
        Ebene

        $\Rightarrow$ Spat mit Kanten $a_1, a_2, a_3$ hat Volumen $0$

        $\Leftrightarrow \det(a_1, a_2, a_3) = 0$
    \item $a_1, a_2$ linear abhängig $\Rightarrow a_2$ ist auf der von $a_1$
        aufgespannten Gerade

        $\Rightarrow \det(a_1, a_2) = 0$
\end{enumerate}

\section{Basis und Dimension}
\subsection{Definition Hamel-Basis}
\begin{enumerate}[label= (\alph*)]
    \item Eine Teilmenge $B$ eines Vektorraums $V$ heißt (Hamel-) Basis von $V$,
        wenn gilt
        \begin{enumerate}[label= (\roman*)]
            \item $B$ ist linear unabhängig
            \item $V = \vspan(B)$
        \end{enumerate}
        Kurz:
        \begin{equation*}
            B \text{ ist ein linear unabhängiges Erzeuger-System
            von }V
        \end{equation*}
    \item Man sagt Vektoren $b_1, \ldots, b_n$ bilden eine Basis von $V$, wenn
        gilt $B=\{b_1, \ldots, b_n\}$ ist eine Basis von $V$.
\end{enumerate}


    \part{HM 3 --- Zusammenfassung}
	\chapter{Exkurs Funktionanlanalysis}
    \section{Normen und innere Produkte}
\subsection{Definition Vektornorm}
Sei $V$ ein Vektorraum über $\K$ eine Abbildung
\begin{equation*}
	\norm{\cdot}: V \rightarrow \R
\end{equation*}

    \part{Beweisansätze}
    \chapter{HM 1}
\section{Grenzwerte}
\subsection{Eindeutigkeit des Grenzwert einer Folge }
 Zeige, dass Grenzwert a = Grenzwert b, nahrhafte 0
\subsection{Konvergente Folgen sind beschränkt }
 Nahrhafte 0, Dreiecks-ugl.
\subsection{Grenzwertrechenregeln }
 Nahrhafte 0, Dreiecks-ugl.
$a_n \leq \gamma\ \forall n \Rightarrow a \leq \gamma$
 Ausgehend von a über nahrh. 0 zu Def Konvergenz
$a_n \leq b_n\ \forall n \Rightarrow a\leq b$
 Definiere Hilfsfolge, argumentiere nach s.o
Sandwich-Theorem
 Zeige, dass $-\varepsilon < c_n < \varepsilon$  (Quasi Epsilon-Schlauch)
\subsection{Monotoniekriterium }
 Da $\abs{a_n} < c\ \forall n$, argumentiere über das Supremum der Menge, die aus $a_n$ besteht
\subsection{Grenzwert einer konv. Folge = Grenzwert jeder Teilfolge }
 Def. Konvergenz + Def Teilfolge
\subsection{Charakterisierung $\varlimsup$ und $\varliminf$ }
 Argumentiere über Eigneschaften sup und inf
\subsection{Folge konv.\ $\varlimsup = \varliminf$ }
 Hin: Eindeutigkeit des Grenzwert;\@Rück: Charakterisierung limSup und limInf
\subsection{Bolzano-Weierstraß }
 Zunächst für reelle Folge (trivial), dann für komplex: Realteil ist klar, Imaginärteil: Teilfolge konstruieren
\subsection{Cauchykriterium }
 Hin: nahrhafte 0; Rück: zeige Beschränktheit, dann folge daraus, dass ein Häufungswert existiert und benutze diesen als Grenzwert-Kandidat
\subsection{Reihe konv.  Folge ist Nullfolge }
 Cauchy für Reihen
\subsection{GrenzwertRR für Reihen }
 GrenzwertRR für Folgen
\subsection{Reihe konv g. 0 }
 Restreihe als Differenz darstellen
\subsection{Leibniz }
 Cauchy für Reihen
\subsection{Absolut konv.\ $\Rightarrow$  konv. }
 Cauchy und Dreiecks-ugl.
\subsection{Majorantenkriterium }
 Cauchy
\subsection{Minorantenkriterium }
 Kontradiktion von Majorantenkriterium
\subsection{Wurzelkriterium }
 Majorantenkrit: geom. Summe über $Q:=q+\varepsilon<1$, in $q$ das Wurzelkriteriumeinsetzen, Charakterisierung $\varlimsup$
\subsection{Quotientenkriterium }
 Majorantenkrit: setze in $q$ das Quotientenkriteriumein und Argumentation über $\varlimsup$
\subsection{Hadamard }
 Wurzelkriterium+ Fallunterscheidung für Sonderfälle
\subsection{Differenzieren / Integrieren von Potenzreihen}
 Wurzelkriterium
\subsection{Lemma zu sin, cos und exp }
 Cauchy-Produkt + Definitionen
\subsection{$e^z \neq 0$ und $e^{-z} = \frac{1}{e^{z}}$ }
 Inverses Element der Multiplikation
\subsection{Pythagoras }
 3.\ binomische Formel
\subsection{$e^x > 0\ \forall x \in \R$ }
 Betrachte $x \geq 0$, angeordneter Körper
\subsection{$1+x \leq e^x\ \forall x \in \R$ }
 Bernoulli
\subsection{$x<y \Rightarrow e^x < e^y$ }
 nahrhafte 0
\subsection{Folgenkriterium }
 Hin: Def. Folgenkonv.\ und dann Def Funktionsgrenzwert einsetzen; \@Rück: Wähle versch. $\delta$ und zeige Widerspruch
\subsection{Cauchy für Funktionen }
 Hin: Def. FunktionsGrenzwert + nahrhafte 0; \@Rück: Cauchy für Folgen
\subsection{Grenzwerte an Intervallgrenzen }
 Argumentiere über Supremum / Infimum
\subsection{Verknüpfungen stetiger Funktionen stetig }
 Folgenkriterium
\subsection{Potenzreihen sind innerhalb des Konvergenzradius stetig }
 Abschätzung: $\exists r>0 : \abs{x-x_0 \text{ bzw. }
 x_1} \leq r$, dann einfach $\abs{f(x)-f(x_1)}$ nach oben abschätzen
\subsection{Umgebung pos. Funktionswerte }
 Wähle $\varepsilon = \frac{f(x_0)}{2}$, Def. Stetigkeit
\subsection{Zwischenwertsatz }
 Definiere $x_0 := \sup \{x \in [a,b] : f(x) \leq y \}$ und zwei Hilfsfolgen, die gegen $x_0$ konvergieren
\subsection{Existenz $\log$ }
 Zeigen $\exp$ ist bijektiv (Zwischenwertsatz)
\subsection{Beschränktheit stetiger Funktionen}
 Annahme $f$ nicht beschränkt Folgenkriterium
\subsection{Weierstraß existenz min bzw.\ max }
 Zeigen das $\sup=\max$

 \chapter{HM 2}
 \section{Integration}
 \subsection{Riemann int\grq bar impliziert Beschränktheit}
 Betrachte Riemannsumme
 \subsection{Rechenregeln für Integrale (Verkettung usw.)}
 Betrachte Riemannsumme
 \subsection{Transitivität}
 Betrachte Riemannsumme
 \subsection{1. MWS der Integralrechnung}\label{HM2Int:1MWS}
 Benutze die Tranisitivität des Integrals und folgende Abschätzung:
 \begin{eqnarray*}
     m \cdot g(x) &\leq& f(x)g(x) \leq M \cdot g(x)\\
     \text{mit}\\
     m &:=&\inf\limits_{[a,b]}(f)\\
     M &:=&\sup\limits_{[a,b]}(f)
 \end{eqnarray*}
 \subsection{Eine Stammfunktion einer Funktion ist stetig und diff \grq bar}\label{HM2Int:StFkt}
 Stetigkeit mit $\delta\varepsilon$-Krit nachrechen und dabei die Beschränktheit von
 f ausnutzen

 Diff \grq barkeit mit Differenzenquotient prüfen (f muss stetig in $x_0$ sein)

 \subsection{Hauptsatz der DI}
 Schreibe $F(b) - F(a)$ als Teleskopsumme und nutze den 1. MWS aus HM1

 Zweiter Teil folgt aus \ref{HM2Int:StFkt}

 \subsection{Monotonie impliziert Riemann Int\grq barkeit}
 Betrachte Riemannsumme und nutze SWT (mithilfe der Randpunkte, die $\xi$ einschließen)

 \subsection{2. MWS der Integralrechnung}
 Nur für den vereinfachten Fall ($f \in C^1[a,b], g \in C[a,b]$):

 Definiere passende Stammfunktion für g. Löse das Integral $\int_a^b f(x)g(x) dx$
 über partielle Integration. Weiterhin wird \ref{HM2Int:1MWS} benötigt.

 \section{Integration in mehreren Veränderlichen}
 \subsection{Ableitung in Integral ziehen}
 Betrachte
 \begin{equation*}
     F'(x) = \lim_{h \to 0} \frac{F(x + h) + F(x)}{h}
 \end{equation*}
 sowie
 \begin{equation*}
     F'(x) - \int_a^b f_x(x,t) \dt
 \end{equation*}
 $F'(x)$ einsetzten, Integral in Grenzwert einsetzten, MWS,
 zeigen das Differenz im Grenzwert $0$.
 \subsection{Fubini}
 Hilfsfunktion:
 \begin{equation*}
     g(u) = \int_\alpha^\beta \int_a^b f(x,t) \dt \dx - \int_a^b \int_\alpha^\beta
     f(x,t) \dx \dt
 \end{equation*}
 zeigen dass $g'(u) \equiv 0$ und $g(x) = g(a) = 0$.

 \subsection{Leibniz Regel}
 Hilfsfunktion:
 \begin{equation*}
     G(x,a,b) = \int_a^b f(x,t) \dt
 \end{equation*}
 $\nabla G$ berechenen, innere Ableitung.

 \subsection{Beweis-Idee Kurvenintegrale (Subsitutionsregel)}
 Riemann Summe, Mittelwertsatz, Abschätzung für verschiedene $\xi$, da $f$ stetig.

 \subsection{1. Hauptsatz für Kurvenintegrale}
 Kurvenintegral mit Parametrisierung, integrant als Ableitung darstellen.

 \subsection{Äquivalente Aussagen für Kurvenintegrale}
 Kurven kombinieren/aufteilen um aus mehreren Kurven eine geschlossene bzw.\ aus
 einer geschlossenen Kurven mehrer mit gleichem Anfangs-/Endpunkt zu erzeugen.

 \subsection{2. Hauptsatz für Kurvenintegrale}
 \begin{enumerate}
     \item $f$ stetig, $F \in C^2$, Satz von Schwarz
     \item Nur für Sternförmiges Gebiet.

        $F$ als Integral von $x_0$ (Mittelpunkt von Sternförmigem Gebiet) zu $x$
        darstellen und Weg Parametrisieren.

        Ableitung von $F$ nach $x_k$ berechnen, Skalarprodukt als Summe schreiben,
        Produktregel, Integrabilitätsbedinung anwenden, als Ableitung nach $t$
        darstellen.
 \end{enumerate}

 \subsection{Gauß'sche Integralsätze in der Ebene}
 \begin{enumerate}
     \item Hilfsfunktion:
        \begin{equation*}
            h(x,y) =
            \begin{pmatrix}
                -f_2(x,y)\\
                f_1(x,y)
            \end{pmatrix}
        \end{equation*}
        zeigen dass $\iint \vdiv h = - \iint \rot f$, Stokes anwenden, $f$ durch
        $h$ darstellen, Normalenvektor normieren, Linienintegral.
    \item Hilfsfunktion:
        \begin{equation*}
            h(x,y) = f_1(x,y) \nabla f_2(x,y) - f_2(x,y) \nabla f_1(x,y)
        \end{equation*}
        $\vdiv h$ und $h \nu$ ausrechnen und Gleichheit über ersten Teil von Gauß.
 \end{enumerate}


    \part{Klausurvorbereitung}
    Hier findest du eine kurze Übersicht über alle Themen, die du für die jeweilige
Klausur beherrschen solltest:
Wichtige Definitionen und Beweise, die man gut in der Klausur abfragen kann,
besonders trickreiche Aufgaben, die mehrmals in der Vorlesung oder in der Übung
besprochen wurden und generelle Kompetenzen, die höchstwahrscheinlich von dir
verlangt werden.
\chapter{HM1}
%@TODO
\chapter{HM2}
\section{Integration}

\subsection{Wichtige Beweise}
\begin{itemize}
  \item 1. und 2. Mittelwertsatz der Integralrechnung
  \item Hauptsatz der Differential- und Integralrechnung
\end{itemize}

\subsection{Typische Aufgaben}
Berechne den GW von z.B. folgender Reihe (hast du also das Prinzip der Riemann-Summen verstanden?)
\begin{equation*}
    \lim\limits_{n \rightarrow \infty}{\frac{1}{n}\sum_{k=0}^n \sin{\frac{k\pi}{n}}}
\end{equation*}

Untersuche Reihen auf Konvergenz (wende das Integralkriterium an)
\begin{equation*}
    \sum_{n=-m}^\infty \frac{1}{1+n^2} \quad (m \in \N)
\end{equation*}
Oder diese hier (Tipp: Eulersche Gammafunktion)
\begin{equation}
    \sum_{n=0}^\infty n^3 e^{-n^2}
\end{equation}

Untersuche uneigentliche Integrale auf Konvergenz

\subsection{Trickreiche Aufgaben}
Schwierige uneigentliche Integrale. Konvergiert beispielsweise dieses Integral? (Ja, tut es)
\begin{equation*}
    \int_1^\infty \frac{\sin{x}}{x}\dx
\end{equation*}

\subsection{Weitere hilfreiche Dinge}
Schau dir uneigentliche Integrale an, die man gut als Majorante oder Minorante
verwenden kann, z.B.:
\begin{equation*}
    \int_0^1 \frac{1}{x^\alpha}\dx
\end{equation*}

\section{Gleichmäßige Konvergenz}
\subsection{Wichtige Beweise}
\begin{itemize}
  \item Stetigkeit der Grenzfunktion
  \item Satz von Dini (ziemlich tricky, aber die Idee sollte man im Kopf haben)
\end{itemize}

\subsection{Typische Aufgaben}
Untersuche Reihen auf gleichmäßige Konvergenz, z.B.:
\begin{equation*}
    \sum_{k=0}^\infty x(1-x)^k,\quad \forall x \in [0,1] \quad \text{bzw} \quad \forall x \in [a,1] \quad \text{mit} \quad 0 < a \leq 1
\end{equation*}

\subsection{Trickreiche Aufgaben}
Auf welchem Intervall konvergiert die Riemannsche Zeta-Funktion gleichmäßig?
\begin{equation*}
    \zeta(s):=\sum_{n=1}^\infty \frac{1}{n^s}
\end{equation*}


    \part{Appendix}
    \chapter{Grenzwerte}
\section{Konvergenzkriterien}
Zusammenfassung verschiedener Konvergenzkriterien nach Wikipedia (Seite: Konvergenzkriterium):
\begin{center}
    \begin{tabular}{lcccccccp{2cm}}
         \toprule
         Kriterium & {nur f.\ mon. F.} & Konv. & Div. & abs. Konv. & Absch. & Fehlerabsch.\\
         \midrule
         Nullfolgenkriterium &  &  & x &  &  & \\
         Monotoniekriterium &  & x &  & x &  & \\
         Leibniz-Kriterium & x & x &  &  & x & x\\
         Cauchy-Kriterium &  & x & x &  &  & \\
         Abel-Kriterium & x & x &  &  &  & \\
         Dirichlet-Kriterium & x & x &  &  &  & \\
         Majorantenkriterium &  & x &  & x &  & \\
         Minorantenkriterium &  &  & x &  &  & \\
         Wurzelkriterium &  & x & x & x &  & x\\
         Integralkriterium & x & x & x & x & x & \\
         Cauchy-Kriterium & x & x & x & x &  & \\
         Grenzwertkriterium &  & x & x &  &  & \\
         Quotientenkriterium &  & x & x & x &  & x\\
         Gauß-Kriterium &  & x & x & x &  & \\
         Raabe-Kriterium &  & x & x & x &  & \\
         Kummer-Kriterium &  & x & x & x &  & \\
         Bertrand-Kriterium &  & x & x & x &  & \\
         Ermakoff-Kriterium & x & x & x & x &  & \\
         \bottomrule
    \end{tabular}
\end{center}

\pagebreak
\section{Beweis-Ansätze}
\begin{center}
    \begin{longtable}{lp{6cm}}
        \toprule
            Lemma / Satz & Beweisansatz\\
        \midrule
            Eindeutigkeit des GW einer Folge & Zeige, dass GW a = GW b, nahrhafte 0\\
            Konvergente Folgen sind beschränkt & Nahrhafte 0, Dreiecks-ugl.\\
            Grenzwertrechenregeln & Nahrhafte 0, Dreiecks-ugl. \\
            $a_n \leq \gamma\ \forall n \Rightarrow a \leq \gamma$  & Ausgehend von a über nahrh. 0 zu Def Konvergenz \\
            $a_n \leq b_n\ \forall n \Rightarrow a\leq b$ & Definiere Hilfsfolge, argumentiere nach s.o \\
            Sandwich-Theorem & Zeige, dass $-\varepsilon < c_n < \varepsilon$  (Quasi Epsilon-Schlauch) \\
            Monotoniekriterium & Da $\abs{a_n} < c\ \forall n$, argumentiere über das Supremum der Menge, die aus $a_n$ besteht \\
            GW einer konv. Folge = GW jeder Teilfolge & Def. Konvergenz + Def Teilfolge \\
            Charakterisierung limSup und limInf & Argumentiere über Eigneschaften sup und inf \\
            Folge konv.\ $\varlimsup = \varliminf$ & Hin: Eindeutigkeit des GW;\@Rück: Charakterisierung limSup und limInf \\
            Bolzano-Weierstraß & Zunächst für reelle Folge (trivial), dann für komplex: Realteil ist klar, Imaginärteil: Teilfolge konstruieren \\
            Cauchykriterium & Hin: nahrhafte 0; Rück: zeige Beschränktheit, dann folge daraus, dass ein HW ex und benutze diesen als GW-Kandidat \\
            Reihe konv.  Folge ist Nullfolge & Cauchy für Reihen \\
            GWRR für Reihen & GWRR für Folgen \\
            Reihe konv g. 0 & Restreihe als Differenz darstellen \\
            Leibniz & Cauchy für Reihen \\
            Absolut konv.\ $\Rightarrow$  konv. & Cauchy und Dreiecks-ugl.\\
            Majorantenkrit. & Cauchy\\
            Minorantenkrit. & Kontradiktion von Majorantenkrit.\\
            Wurzelkriterium & Majorantenkrit: geom. Summe über $Q:=q+\varepsilon<1$, in $q$ das Wurzelkrit einsetzen, Char. LimSup\\
            Quotientenkrit. & Majorantenkrit: setze in $q$ das Quotientenkrit ein u.\ arg. über LimSup \\
            Hadamard & Wurzelkrit + Fallunterscheidung für Sonderfälle\\
            Differenzieren / Integrieren von PR & Wurzelkriterium\\
            Lemma zu sin, cos und exp & Cauchy-Produkt + Definitionen\\
            $e^z \neq 0$ und $e^{-z} = \frac{1}{e^{z}}$ & Inverses Element der Multiplikation\\
            Pythagoras & 3.\ binomische Formel\\
            $e^x > 0\ \forall x \in \R$ & Betrachte $x \geq 0$, angeordneter Körper\\
            $1+x \leq e^x\ \forall x \in \R$ & Bernoulli\\
            $x<y \Rightarrow e^x < e^y$ & nahrhafte 0\\
            Folgenkriterium & Hin: Def. Folgenkonv.\ und dann Def FunktionsGW einsetzen; \@Rück: Wähle versch. $\delta$ und zeige Widerspruch\\
            Cauchy für Funktionen & Hin: Def. FunktionsGW + nahrhafte 0; \@Rück: Cauchy für Folgen\\
            Grenzwerte an Intervallgrenzen & Argumentiere über Supremum / Infimum\\
            Verknüpfungen stetiger Fnkt.\ stetig & Folgenkriterium\\
            Potenzreihen sind innerhalb des KR stetig & Abschätzung: $\exists r>0 : \abs{x-x_0 \text{ bzw. } x_1} \leq r$, dann einfach $\abs{f(x)-f(x_1)}$ nach oben abschätzen\\
            Umgebung pos. Funktionswerte & Wähle $\varepsilon = \frac{f(x_0)}{2}$, Def. Stetigkeit\\
            Zwischenwertsatz & Definiere $x_0 := \sup \{x \in [a,b] : f(x) \leq y \}$ und zwei Hilfsfolgen, die gegen $x_0$ konvergieren\\
            Existenz $\log$ & Zeigen $\exp$ ist bijektiv (Zwischenwertsatz)\\
            Beschr.\ stet.\ Fkt.\ & Annahme $f$ nicht beschr. Folgenkriterium\\
            Weierstraß ex.\ min bzw.\ max & Zeigen das $\sup=\max$\\
        \bottomrule
    \end{longtable}
\end{center}

\end{document}
