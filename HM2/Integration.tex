\section{Integration}

\subsection{Definition Zerlegung, Zwischenwerte}
Eine Teilmenge $T$ von $[a,b]$ mit $a, b \in T$ nennt man eine
Unterteilung, Zerlegung oder Partitionierung von $[a, b]$ wenn
gilt:
\begin{eqnarray*}
    T = \{ x_0, x_1, \ldots , x_n\} \text{ mit}\\
    a = x_0 < x_1 < \ldots < x_n = b
\end{eqnarray*}

Schreibweise für diese Menge $T$ sei:
\begin{equation*}
    T: a = x_0 < x_1 < \ldots < x_n = b
\end{equation*}

Ist T eine Zerlegung, dann heißt:
\begin{enumerate}[label= (\alph*)]
    \item Die Zahl $\mu(T) :=
        \max{ \{\ \abs{x_k - x_{k+1}}, k = 1, \ldots, n \} }$
        das Feinheitsmaß von $T$.
    \item Ein Vektor $\xi = (\xi_1, \ldots, \xi_n) \in \R^n$ heißt
        ein Zwischenwertvektor zu $T$, wenn gilt
        \begin{equation*}
            x_{k-1} \leq \xi_k \leq x_k \text{ für } k = 1, \ldots, n
        \end{equation*}
        Dann heißt die Komponente $\xi_k$ ein Zwischenwert von
        $x_{k-1}$ und $x_k$.
\end{enumerate}

\subsection{Definition Riemannsumme}
Ist $f: [a,b] \rightarrow \R$ eine Funktion, $T: a=x_0<\ldots<x_n=b$
eine Zerlegung von $[a, b]$ und $\xi = (\xi_1, \ldots, \xi_n)$ ein
Zwischenwertevektor zu $T$, dann nennen wir die Summe
\begin{equation*}
    S(f; T, \xi) = S_f(T, \xi) = \sum_{k=1}^n f(\xi_k)(x_k - x_{k-1})
\end{equation*}
die Riemansumme von $f$ bezüglich $T$ und $\xi$.

\subsection{Definition Riemann-Integral}
Eine Funktion $f: [a, b] \rightarrow \R$ heißt Riemann-Integrierbar
unter $[a, b]$ wenn für jede Folge ${(T_N)}_{N=1}^\infty$ von Zerlegungen
von $[a,b]$ mit $\mu(T_N) \rightarrow 0$ für $N \rightarrow \infty$
und jede Folge ${(\xi_N)}_{N=1}^\infty$ von Zwischenpunktvektoren
der Grenzwert
\begin{equation*}
    \lim_{N \rightarrow \infty} S(f; T_N, \xi_N) \text{ existiert.}
\end{equation*}

\subsubsection{Behauptung}
Der Grenzwert ist im Fall der Existenz für jede Folge identisch.

\subsubsection{Bemerkung}
\begin{enumerate}[label= (\alph*)]
    \item Im Fall der Existenz bezeichnet man den Grenzwert durch:
        \begin{equation*}
            \int_{a}^b f(x) \dx = \lim_{N \rightarrow \infty} S(f; T_N, \xi_N)
        \end{equation*}
    \item Zu ${(T_N)}_{N=1}^\infty$, also $T_1, T_2, T_3, \ldots$:
        \begin{eqnarray*}
            &T_1:& a = x_0^{(1)} < \ldots < x_n^{(1)} = b\\
            &T_2:& a = x_0^{(2)} < \ldots < x_n^{(2)} = b\\
            &T_3:& a = x_0^{(3)} < \ldots < x_n^{(3)} = b\\
            &\vdots&\\
            &T_l:& a = x_0^{(l)} < \ldots < x_n^{(l)} = b
        \end{eqnarray*}
    \item Zu ${(\xi_N)}_{N=1}^\infty$, also $\xi_1, \xi_2, \xi_3, \ldots$:
        \begin{eqnarray*}
            &\xi_1& = (\xi_1^{(1)}, \ldots, \xi_n^{(1)})
            \text{ mit } x_{k-1}^{(1)} \leq \xi_k^{(2)} \leq x_k^{(1)}
            \text{ mit } 1 \leq k \leq n_1\\
            &\xi_2& = (\xi_1^{(2)}, \ldots, \xi_n^{(2)})
            \text{ mit } x_{k-1}^{(2)} \leq \xi_k^{(2)} \leq x_k^{(2)}
            \text{ mit } 1 \leq k \leq n_2\\
            &\xi_3& = (\xi_1^{(3)}, \ldots, \xi_n^{(3)})
            \text{ mit } x_{k-1}^{(3)} \leq \xi_k^{3} \leq x_k^{(3)}
            \text{ mit } 1 \leq k \leq n_3\\
            &\vdots&\\
            &\xi_l& = (\xi_1^{(l)}, \ldots, \xi_n^{(l)})
            \text{ mit } x_{k-1}^{(l)} \leq \xi_k^{l} \leq x_k^{(l)}
            \text{ mit } 1 \leq k \leq n_l\\
        \end{eqnarray*}
    \item Sei $f$ integrierbar und ${(T_N)}_{N=1}^\infty$ und
        ${(\xi_N)}_{N=1}^\infty$ sowie ${(\tilde{T}_N)}_{N=1}^\infty$ und
            ${(\tilde{\xi}_N)}_{N=1}^\infty$ entsprechende Folgen,
            d.h. $\mu(T_N) \rightarrow 0, \mu(\tilde{T}_N) \rightarrow 0$
            für $N \rightarrow \infty$. Dann gilt gilt für ${(\hat{T}_N)}_{N=1}^\infty$
            und ${(\hat{\xi}_N)}_{N=1}^\infty$ mit
            \begin{equation*}
                \hat{T}_N :=
                \begin{cases}
                    T_N &\text{ für } N \text{ gerade}\\
                    \tilde{T}_N &\text{ für } N \text{ ungerade}
                \end{cases}
            \end{equation*}
            und
            \begin{equation*}
                \hat{\xi}_N :=
                \begin{cases}
                    \xi_N &\text{ für } N \text{ gerade}\\
                    \tilde{\xi}_N &\text{ für } N \text{ ungerade}
                \end{cases}
            \end{equation*}
            dass
            \begin{equation*}
                \lim_{N \rightarrow \infty} S(f; \hat{T}_N, \hat{S}_N)
            \end{equation*}
            existiert, da $f$ integrierbar ist.

            Dann stimmt der Grenzwert von
            $\lim_{N \rightarrow \infty} S(f; \tilde{T}_N, \tilde{S}_N)$ und
            $\lim_{N \rightarrow \infty} S(f; T_N, S_N)$ überein.
\end{enumerate}

\subsection{Menge der Riemann-Integrierbaren Funktionen}
Mit $R [a, b]$ oder $R ([a, b])$ bezeichnen wir die Menge von Funktionen
$f: [a, b] \rightarrow \R$ die auf $[a, b]$ Riemann integrierbar sind.

\subsection{Kriterien für Riemann-Integrierbarkeit}
\begin{enumerate}[label= (\alph*)]
    \item
        \begin{equation*}
            f \in R[a, b] \Rightarrow f \text{ ist auf } [a,b] \text{beschränkt}
        \end{equation*}
    \item Ist $f, g \in R [a, b]$ und $c \in \R$ dann sind auch die Funktionen
        \begin{eqnarray*}
            f&+&g\\
            f&-&g\\
            c &\cdot& f
        \end{eqnarray*}
        Riemann integrierbar auf $[a, b]$.
    \item Ist $f, g \in R[a, b]$, dann ist auch
        \begin{equation*}
            f \cdot g \in R[a, b]
        \end{equation*}
    \item Ist $f, g \in R[a, b]$ und falls
        $\abs{g(x)} > \delta > 0\ \forall x \in [a, b]$ dann ist auch
        \begin{equation*}
            \frac{f}{g} \in R[a, b]
        \end{equation*}
    \item Für beliebiges $c \in [a, b]$  gilt:
        \begin{equation*}
            f \in R[a, b] \Leftrightarrow f \in R[a, c] \land f \in R[c, b]
        \end{equation*}
        und weiter gilt:
        \begin{equation*}
            \int_a^b f(x) \dx = \int_a^c f(x) \dx + \int_c^b f(x) \dx
        \end{equation*}
    \item
        \begin{equation*}
        f \in R[a, b] \Rightarrow \abs{f} \in R[a, bv]
        \end{equation*}
        und
        \begin{equation*}
            \abs{\int_a^b f(x)\dx} \leq \int_a^b \abs{f(x)} \dx
        \end{equation*}
\end{enumerate}

\subsection{Änderung von Funktionen}
Wenn $f \in R[a, b]$ ist und durch endlich viele Änderungen daraus
$g: [a,b] \rightarrow \R$ konstruiert werden kann, d.h.
\begin{equation*}
    g(x) =
    \begin{cases}
        f(x) &\text{ falls } x \notin \{x_1, \ldots, x_n \} \\
        y_1 &\text{ falls} x=x_1\\
        \vdots
    \end{cases}
\end{equation*}
dann gilt $g \in R[a, b]$ und
\begin{equation*}
    \int_a^b f(x) \dx = \int_a^b g(x) \dx
\end{equation*}

\subsection{Zusammenhang Stetigkeit und Integrierbarkeit}
Es gilt:
\begin{equation*}
    f \in C[a, b] \Rightarrow f \in R[a, b]
\end{equation*}

\subsection{Stückweise Integration}
Falls $f: [a, b] \rightarrow \R$ stückweise stetig ist, d.h.\ es existieren
endlich viele Intervall-Stücke auf denen $f$ stetig ist, dann ist
$f \in R[a,b]$ und es gilt:
\begin{equation*}
    \int_a^b f(x) \dx = \int_{x_0}^{x_1} f(x) \dx + \int_{x_1}^{x_2} f(x)
    + \ldots + \int_{x_{n-1}}^{x_n} f(x)
\end{equation*}

\subsection{1. Mittelwertsatz der Integralrechnung}
Seien $f, g \in R[a,b]$ und $g \geq 0$ auf $[a, b]$. Dann gibt es ein $\mu \in \R$
mit $\inf\limits_{[a,b]} f(x) \leq \mu \leq \sup\limits_{[a,b]} f(x)$ sodass gilt:
\begin{equation*}
    \int_a^b f(x)g(x)\dx = \mu \int_a^b g(x) \dx
\end{equation*}

Ist $f$ stetig auf $[a, b]$, dann existiert ein $\xi \in [a, b]$ mit
\begin{equation*}
    \int_a^b f(x)g(x)\dx = f(\xi) \int_a^b g(x) \dx
\end{equation*}

\subsubsection{Bemerkung}
Für $g(x)=1$ und $f$ stetig lautet die Aussage also:
\begin{equation*}
    \int_a^b f(x) \dx = f(\xi) \cdot (b - a)
\end{equation*}

\subsection{Existenz der Stammfunktion}
Sei $f \in R[a, b]$, dann ist für jedes $c \in [a, b]$ durch:
\begin{equation*}
    F(x) := \int_c^x f(t) dt
\end{equation*}
eine stetige Funktion definiert. Und für jedes $x_0 \in (a,b)$ gilt:
\begin{equation*}
    f \text{ stetig in } x_0 \Rightarrow F \text{ ist differentierbar in } x_0
    \land F'(x_0)=f(x_0)
\end{equation*}

\subsection{Definition Stammfunktion}
Gilt $F'(x) = f(x) \forall x \in [a,b]$ dann wird $F$ als Stammfunktion von
$f$ bezeichnet.

\subsection{Eindeutigkeit der Stammfunktion}
Sind $F$ und $G$ Stammfunktionen von $f$, dann existiert ein $c \in \R$ mit
\begin{equation*}
    F(x) = G(x) + c\ \forall x \in [a, b]
\end{equation*}

\subsection{Hauptsatz der Differential und Integralrechnung}
Sei $f: [a,b] \rightarrow \R$ gegeben dann gilt:
\begin{enumerate}[label= (\alph*)]
    \item Ist $f \in R[a,b]$ und $F$ eine Stammfunktion, dann gilt:
        \begin{equation*}
            \int_a^b f(t) dt = F(b) - F(a) =: {\left[ F(x) \right]}_a^b
        \end{equation*}
    \item Ist $f \in C[a, b]$ dann existiert eine Stammfunktion und zwar
        \begin{equation*}
            F(x) := \int_c^x f(t) dt
        \end{equation*}
\end{enumerate}

\subsubsection{Bemerkung}
Aus dem Hauptsatz folgen Integrationstechniken wie partielles Integrieren oder
die Subsitutionsregel.

\subsection{Zusammenhang Monotonie und Riemann-Integrierbarkeit}
Ist $f: [a, b] \rightarrow \R$ auf $[a,b]$ monoton, dann ist $f \in R[a,b]$.

\subsection{Zweiter Mittelwertsatz der Integralrechnung}
Ist $f$ monoton auf $[a,b]$, $g$ integrierbar auf $[a,b]$, dann exisitiert ein
$\xi \in [a,b]$ mit:
\begin{equation*}
    \int_a^b f(x)g(x) \dx = f(a) \int_a^\xi g(x) \dx +
        \int_\xi^b g(x) \dx
\end{equation*}
