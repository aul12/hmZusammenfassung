\section{Parameterintegrale}
\subsection{Eigentliche Parameterintegrale}
Sei $f(x,t)$ reel und stetig in $[\alpha, \beta] \times [a,b]$ (also $x\in[\alpha, \beta],
t \in [a,b]$). Dann gilt für
\begin{equation*}
    F(x) := \int_a^b f(x,t) \dt
\end{equation*}
\begin{enumerate}[label= (\alph*)]
    \item $F$ ist stetig auf $[\alpha, \beta]$
    \item Ist $f_x$ stetig in $[\alpha, \beta] \times [a,b]$, so ist
        $F \in C^1([\alpha, \beta])$ und $F'(x) = \int_a^b f_x(x,t) \dt$
    \item Satz von Fubini:
        \begin{equation*}
            \int_\alpha^\beta F(x) \dx = \int_\alpha^\beta \int_a^b f(x,t) \dt \dx
            = \int_a^b \int_\alpha^\beta f(x,t) \dx \dt
        \end{equation*}
\end{enumerate}

\subsection{Leibniz Regel}
Seien $f(x,t), f_x(x,t)$ stetig in $[\alpha, \beta] \times [a,b]$ und
$u, v \in C^1 ( [a,b])$. Dann ist
\begin{equation*}
    F(x) = \int_{u(x)}^{v(x)} f(x,t) \dt \in C^1([a,b])
\end{equation*}
und
\begin{equation*}
    F'(x) = \int_{u(x)}^{v(x)} f_x(x,t) \dt +
    f(x, v(x)) v'(x) - f(x, u(x)) u'(x)
\end{equation*}

\subsection{Uneigentliche Parameterintegrale}
Ist für jedes $x \in M \subseteq \R$ ein uneigentliches Integral
\begin{equation*}
    \int_a^b f(x,t) \dt
\end{equation*}
mit kritischem Punkt $a$ oder $b$ gegeben, so heißt dieses gleichmäßig konvergent
in $M$, wenn gilt:
\begin{equation*}
    \forall \varepsilon > 0\ \exists L \in (a,b): \abs{\int_{T_1}^{T_2} f(x,t) \dt}
    <\varepsilon \ \forall x \in M \forall T_1, T_2 \in (a,L) (\text{bzw.}
    \forall T_1, T_2 \in (L,b))
\end{equation*}

\subsection{Majorantenkriterium}
Ein uneigentliches Integral $\int_a^b f(x,t) \dt$ konvergiert gleichmäßig in $M$ wenn ein konvergentes
Integral
\begin{equation*}
    \int_a^b g(t) \dt \text{ ex.\ mit} \abs{f(x,t)} \leq g(t)
\end{equation*}

\subsection{Fubini für uneigentliche Parameterintegrale}
Ist $f(x,t)$ stetig in $[\alpha, \beta] \times [a,b]$ und konvergiert
\begin{equation*}
    F(x) = \int_a^b f(x,t) \dt
\end{equation*}
gleichmäßig auf $[\alpha, \beta]$ dann ist $F$ stetig und
\begin{equation*}
    \int_\alpha^\beta \int_a^b f(x,t) \dt \dx =
    \int_a^b \int_\alpha^\beta f(x,t) \dx \dt
\end{equation*}

\subsection{Konvergenzkriterien}
Sind $f(x,t), f_x(x,t)$ stetig auf $[\alpha, \beta] \times [a,b]$ und ist
\begin{equation*}
    \int_a^b f(x, t) \dt
\end{equation*}
für ein $x_0 \in [\alpha, \beta]$ konvergent und ist
\begin{equation*}
    \int_a^b f_x(x,t) \dt
\end{equation*}
gleichmäßig konvergent. Dann gilt:
\begin{equation*}
    F(x) = \int_a^b f(x,t) \dt\ \forall x \in [\alpha, \beta]
\end{equation*}
und
\begin{equation*}
    F'(x) = \int_a^b f_x(x,t) \dt\ \forall x \in (\alpha, \beta)
\end{equation*}
existiert und ist stetig.

\section{Kurvenintegrale}
\subsection{Äquivalenz für Kurven}
Zwei stetige Funktionen $x: [a, b] \subseteq \R \rightarrow \R^n, y:[\alpha, \beta]
\subseteq \R \rightarrow \R^n$ heißen Äquivalent (schreibweise $x \sim y$), wenn eine
streng monoton wachsende Funktion
\begin{equation*}
    \phi: [a,b] \rightarrow [\alpha, \beta]
\end{equation*}
gibt mit
\begin{equation*}
    x(t) = y(\phi(t))\ \forall t \in [a,b]
\end{equation*}

\subsubsection{Bemerkung}
Es gilt:
\begin{enumerate}[label= (\alph*)]
    \item $x \sim x$ (Reflexivität)
    \item $x \sim y \Rightarrow y \sim x$ (Symmetrie)
    \item $x \sim y \land y \sim z \Rightarrow x \sim z$ (Transitivität)
\end{enumerate}

\subsection{Kurven im $\R^n$}
Ist $x: [a,b] \subseteq \R \rightarrow \R^n$ stetig, so nennt man die Menge
\begin{equation*}
    \K := \{y: [\alpha, \beta] \subseteq \R \rightarrow \R^n \text{ mit } x \sim y\}
\end{equation*}
die Kurve $\K$ mit Parameterdarstellung $x$ und den Punkt $x(a)$ Anfangspunkt und
$x(b)$ Endpunkt.

\subsubsection{Schreibweise}
\begin{equation*}
    \K: x(t), a \leq t \leq b
\end{equation*}
Die Menge
\begin{equation*}
    T(\K) := \{x(t): t \in [a,b]\} = x([a,b])
\end{equation*}
nennt man den Träger der Kurve $\K$.

\subsubsection{Bemerkung}
Verschieden Kurven können also den gleichen Träger haben.

Man nennt $K$:
\begin{enumerate}[label= (\alph*)]
    \item Geschlossen, wenn $x(a) = x(b)$
    \item Einfach oder Jordankurve, wenn $x(t) \neq x(s)\ \forall t,s: a \leq t < s <b$
\end{enumerate}

\subsection{Eigenschaften von Parameterdarstellungen}
\begin{enumerate}[label= (\alph*)]
    \item Eine Parameterdarstellung $x:[a,b] \rightarrow \R^n$ einer Kurve heißt
        stückweise stetig differentierbar, wenn eine Zerlegung
        \begin{equation}
            T: a=t_0 < \ldots < t_k = b
        \end{equation}
        existiert und $x$ auf $(t_l, t_{l+1})\ l\in\{0,\ldots,k-1\}$
        differentierbar ist.
    \item Besitzt eine Kurve $\K$ eine (stückweise) stetig differentierbare
        Parameterdarstellung $x(t), t\in [a,b]$ mit $\dot{x}(t) \neq \vec{0}$
        für $t \in [a,b]$ so heißt $\K$ stückweise glatt oder stückweise regulär.
    \item Ist eine Parameterdarstellung $x$ von $\K$ differentierbar und glatt,
        so heißt
        \begin{equation*}
            T(t) := \frac{\dot{x}(t)}{\norm{\dot{x}(t)}}
        \end{equation*}
        der Tangential (einheits) vektor von $x$ und $\K$
    \item Ist auch $T$ differentierbar und glatt (also $\dot{T}(k) \neq \vec{0}$)
        so heißt
        \begin{equation*}
            N(t) := \frac{\dot{T}(t)}{\norm{\dot{T}(t)}}
        \end{equation*}
        der (Haupt-) Normalen (einheits) vektor von $\K$ und $x$ bei $t$
    \item Und falls $n=3$
        \begin{equation*}
            B(t) = T(t) \times N(t)
        \end{equation*}
        der Binormalen (einheits) vektor von $\K$ und $x$ bei $t$
        (Man nennt dann $T(t), N(t), B(t)$ ein begleitendes Dreibein von $\K$)
    \item Existiert $T(t)$, so nennt man die Gerade
        \begin{equation*}
            \{x(t) + \lambda \dot{x}(t): \lambda \in \R\}
        \end{equation*}
        die Tangente von $\K$ bei $\lambda$
    \item Existiert auch $N(t)$ so nennt man die Ebene
        \begin{equation*}
            \{ x(t) + \lambda \dot{x}(t) + \mu \ddot{x}(t): \lambda, \mu \in \R\}
        \end{equation*}
        die Schmiegeebene von $\K$ bei $t$.
\end{enumerate}

\subsubsection{Bemerkung}
Sei $x(t) = y(\phi(t))$ mit $a\leq t \leq b$ zwei Parameterdarstellungen von $x$.
Dann gilt:
\begin{equation*}
    T(t) = \frac{\dot{x}(t)}{\norm{\dot{x}(t)}} = \frac{\dot{y}(\varphi(t)) \cdot \dot{\varphi}(t)}
    {\norm{\dot{y}(\phi(t)) \cdot \dot{\varphi}(t)}} = \frac{\dot{y}(\varphi(t))}
    {\norm{\dot{y}(\varphi(t))}}
\end{equation*}
Das heißt die Berechnung von $T$ ist unabhängig von der konkreten Parameterdarstellung

Existiert $N(t)$ dann gilt:
\begin{equation*}
    N(t) \bot T(t)
\end{equation*}
Existiert auch $B(t)$ (im $\R^3$), dann gilt:
$N(t), T(t), B(t)$ sind paarweise Orthogonal.

\subsection{Weitere Definitionen zu Kurven}
\begin{enumerate}[label= (\alph*)]
    \item Ist $\K: x(t), a \leq t \leq b$ eine Kurve, so heißt:
        \begin{equation*}
            - \K: y(t), a \leq t \leq b \text{ mit } y(t)=x(a+b-1)
        \end{equation*}
        die zu $\K$ entgegengesetzte Kurve
    \item Sind $\K: x(t), a \leq t \leq b$ und $\mathbb{L}: y(t), \alpha \leq
        t \leq \beta$ zwei Kurven und gilt $x(b) = y(\alpha)$ dann ist
        \begin{equation*}
            \K + \mathbb{L}: z(t), a \leq t \leq (\beta-\alpha) + b
        \end{equation*}
        und
        \begin{equation*}
            z(t) =
            \begin{cases}
                x(t) &, a \leq t \leq b\\
                y(t-b+\alpha) &, b \leq t \leq (\beta-\alpha) + b
            \end{cases}
        \end{equation*}
        die Aus $\K$ und $\mathbb{L}$ zusammengesetzte Kurve.
\end{enumerate}

\subsection{Kurventintegrale 2. Art}
Sei $\K$ eine Kurve im $\R^n$ und
\begin{equation*}
    f: T(\K) \rightarrow \R^n
\end{equation*}
\begin{enumerate}[label= (\alph*)]
    \item Sei $x: [a,b]  \rightarrow \R^n$ eine Parameterdarstellung von $\K$
        \begin{enumerate}[label = (\roman*)]
            \item Für eine Zerlegung $T: a = t_0 < \cdots < t_n = b$,
                Zwischenpunte $Z: (\xi_1, \ldots, \xi_n)$ mit
                $t_{k-1} \leq \xi_k \leq t_k$
                heißt
                \begin{equation*}
                    S(f,x,T,Z) := \sum_{k=1}^n f(x(\xi_k)) \cdot (x(t_k) - x(t_{k-1}))
                \end{equation*}
                die Riemann-Summe von $f, T, Z$ bezüglich $x$.
            \item Exitiert eine Zahl $I \in \R$ derart, dass für jede Folge von
                Zerlegungen $T_n$ mit
                \begin{equation*}
                    \lim_{n \rightarrow \infty} \mu(T_n) = 0
                \end{equation*}
                stets
                \begin{equation*}
                    \lim_{n \rightarrow \infty} S(f,x,T_n, Z_n) = I
                \end{equation*}
                folgt, so heißt $I$ das Kurvenintegral (2. Art) von
                $f$ längs $\K$ bzgl. $x$.
        \end{enumerate}
    \item Gibt es stets ein $I$ wie in (a) so heißt $f$ längs $\K$ (Riemann-)
        integrierbar und man nennt $I$ das (unbestimmte) Kurvenintegral von $f$
        längs $\K$ und schreibt:
        \begin{equation*}
            I = \int_\K f = \int_\K f(x) \cdot \dx = \int_\K f_1(x) \dx_1 + \cdots
            + f_n(x) \dx_n
        \end{equation*}
\end{enumerate}

\subsection{Substitutionsregel}
Ist $\K: x(t), a \leq t \leq b$ eine Kurve im $\R^n$ und $x(t)$ stückweise
differentierbar, sowie $f: T(\K) \rightarrow \R^n$ stetig, so ist $f$ längs
$\K$ integrierbar und es gilt:
\begin{equation*}
    \int_K f(x) \dx = \int_a^b f(x(t)) \text{d}x(t) = \int_a^b f(x(t)) \cdot
    \dot{x}(t) \dt
\end{equation*}

\subsection{Definition Wegunabhängigkeit}
Sei $f\in C(G, \R^n)$ mit $G \subseteq \R^n$ ein Gebiet:
\begin{enumerate}[label= (\alph*)]
    \item Gilt für zwei Wege $\K$ und $\mathbb{L}$ mit gleichem Anfangs- und
        Endpunkt stets
        \begin{equation*}
            \int_\K f = \int_{\mathbb{L}} f
        \end{equation*}
        dann heißen die Kurvenintegrale Wegunabängig in $G$.
    \item Eine Funktion $F \in C^1(G,\R)$ heißt Stammfunktion von $f$ in $G$, wenn
        \begin{equation*}
            \nabla F(x) = f(x)\ \forall x \in G
        \end{equation*}
        gilt.
    \item Man nennt
        \begin{equation*}
            P := -F
        \end{equation*}
        das Potential von $f$.
    \item Man nennt $f$ konservativ in $G$ oder ein Potentialfeld oder Gradienentenfeld
        in $G$, wenn $f$ eine Stammfunktion hat.
\end{enumerate}

\subsection{1. Hauptsatz für Kurvenintegral}
Sei $f$ konservativ in $G$ mit Stammfunktion $F$ und Potential $P$ dann gilt
für jeden Weg $\K$ in $G$ mit Anfangspunkt $A \in G$ und Endpunkt $B \in G$:
\begin{equation*}
    \int_\K f = F(B) - F(A) = P(A) - P(B)
\end{equation*}
insbesondere ist also das Integral wegunabhängig.

\subsection{Äquivalente Aussagen zu Stammfunktionen}
\begin{enumerate}[label= (\alph*)]
    \item
        \begin{equation*}
            \int_\K f \text{ ist wegunabhängig in } G
        \end{equation*}
    \item
        $f$ besitzt eine Stammfunktion
    \item
        \begin{equation*}
            \int_\K f = 0 \text{ für jede geschlossene Kurve } \K
        \end{equation*}
\end{enumerate}

\subsubsection{Bemerkung}
Rechenregeln für zwei Kurven $\K$ und $\mathbb{L}$:
\begin{enumerate}[label= (\alph*)]
    \item
        \begin{equation*}
            \int_{\K + \mathbb{L}} f = \int_\K f + \int_\mathbb{L} f
        \end{equation*}
    \item
        \begin{equation*}
            \int_{-\K} f = - \int_\K f
        \end{equation*}
\end{enumerate}

\subsection{Definition einfach zusammenhängende Gebiete}
Ein Gebiet $G \subseteq \R^n$ heißt einfach zusammenhängend, wenn sich jede
geschlossene Kurve in $G$ innerhalb von $G$ \glqq{}auf einen beliebigen
Punkt zusammenziehen lässt\grqq{}.

\subsection{Sternförmige Gebiete}
Eine Menge $G \subseteq \R^n$ heißt Sternförmig bezüglich $x_0 \in G$, wenn für
alle $x \in G$ gilt, dass $\overline{x_0x} \subseteq G$ (d.h.\ jedes $x$ ist von
$x_0$ durch einen Streckenzug erreichbar). $G$ ist ein sternförmiges Gebiet,
wenn $G$ offen und sternförmig ist.

\subsubsection{Bemerkung}
\begin{equation*}
    G \text{ sternförmig} \Rightarrow G \text{ einfach zusammenhängend}
\end{equation*}

\subsection{2. Hauptsatz für Kurvenintegrale}
Sei $f \in C^1(G, \R^n), G \subseteq \R^n$ ein Gebiet, dann gilt:
\begin{enumerate}[label= (\alph*)]
    \item Besitzt $f$ eine Stammmfunktion in $G$, so erfüllt $f$ in $G$ die
        Integrabilitätsbedingung:
        \begin{equation*}
            \frac{\partial f_l}{\partial x_k} = \frac{\partial f_k}{\partial x_l}\
            k,l \in \{ 1, \ldots, n\}
        \end{equation*}
        D.h.\ die Jacobi-Matrix von $f$ ist symetrisch.

        Kurz:
        \begin{equation*}
            f \text{ hat Stammfunktion} \Rightarrow f' = (f')^T
        \end{equation*}
    \item Ist $G$ einfach zusammenhängend und erfüllt $f$ die Integrabilitätsbedingung
        dann besitzt $f$ eine Stammfunktion.

        Kurz:
        \begin{equation*}
            G \text{ einfach zusammenhängend} \land f' = (f')^T \Rightarrow
                \exists F: \nabla F = f
        \end{equation*}
\end{enumerate}

\subsection{Definition Rotation}
Sei $G \subseteq \R^3$ offen und $f: G \rightarrow R^3$ partiell differentierbar,
dann heißt die Funktion $\rot f: G \rightarrow \R$ mit
\begin{equation*}
    \rot f(x) :=
    \begin{pmatrix}
        \dfrac{\partial f_3}{\partial x_2} - \dfrac{\partial f_2}{\partial x_3} \\
        \dfrac{\partial f_1}{\partial x_3} - \dfrac{\partial f_3}{\partial x_1} \\
        \dfrac{\partial f_2}{\partial x_1} - \dfrac{\partial f_1}{\partial x_2} \\
    \end{pmatrix}
\end{equation*}
die Rotation von $f$ in $G$.

\subsubsection{Bemerkung}
Im Fall $f: G \subseteq \R^2 \rightarrow \R^2$ definiert man
\begin{equation*}
    \rot f(x_1, x_2) = \frac{\partial f_2}{\partial x_1} - \frac{\partial f_1}{\partial x_2}
\end{equation*}
Formal betrachtet man die Hilfsfunktion
\begin{equation*}
    \tilde{f}(x,y,z) :=
    \begin{pmatrix}
        f_1(x,y)\\
        f_2(x,y)\\
        0
    \end{pmatrix}
\end{equation*}

\subsection{Zusammenhang Rotation und Integrabilitätsbedingung}
Ist $f \in C^1(G, \R^3), G$ ein Gebiet, dann gilt
\begin{enumerate}[label= (\alph*)]
    \item $f$ besitzt eine Stammfunktion $\Rightarrow \rot f = \vec{0}$
    \item $G$ einfach zusammenhängend und $\rot f = \vec{0} \Rightarrow f$
        hat Stammfunktion.
\end{enumerate}

\subsection{Definition Linienintegral/Kurvenintegral 1. Art}
Sei $\K: x(t), a \leq t \leq b$ ein Weg, und $x$ stückweise differentierbar.
Für ein $\phi \in  C(T(\R), \R)$ heißt
\begin{equation*}
    \int_\K \phi \ds := \int_a^b \phi(x(t)) \norm{ \dot{x}(t)} \dt
\end{equation*}
ein Linienintegral oder Kurvenintegral 1. Art von $\phi$ längs $\K$.

\subsubsection{Bemerkung}
\begin{enumerate}[label= (\alph*)]
    \item
        Mit $\phi \equiv 1$:
        \begin{equation*}
            \int_\K 1 \ds = \int_a^b \phi(x,t) \norm{\dot{x}(t)} \dt
            \int_a^b  \norm{\dot{x}(t)} \dt = l(\K)
        \end{equation*}
        d.h.\ mit Linienintegralen können auch Weglängen berechnet werden,
        bzw. Weglängen berechnet man mit $\phi = 1$.
    \item
        $\phi: [a,b] \rightarrow \R$ wähle $\K: x(t) = a + t \cdot (b-a)\
        t \in [0,1]$:
        \begin{equation*}
            \int_\K \phi \ds = \int_0^1 \phi (a + t \cdot (b-a)) \norm{b-a} \dt
            = \int_a^b \phi(t) \dt
        \end{equation*}
    \item Linienintegrale hängen nicht von der Parameterdarstellung ab.
    \item Man schreibt (falls Parameter-Darstellung bekannt ist) oft
        \begin{equation*}
            \ds = \norm{\dot{x}(t)} \dt
        \end{equation*}
        und nennt $\ds$ Bogensegment oder Liniensegment.
    \item Ist $f \in C(T(\K), \R^n)$ und $\dot{x}(t) \neq \vec{0}\ \forall
        t \in [a,b]$, dann ist:
        \begin{eqnarray*}
            \int_\K f &=& \int_\K f(x) \dx = \int_a^b f(x(t)) \dot{x}(t) \dt \\
            &=& \int_a^b \frac{f(x(t)) \dot{x}(t)}{\norm{\dot{x}(t)}} \norm{\dot{x}(t)}
            \dt \\ &=& \int_a^b f(x(t)) \cdot T(t) \norm{\dot{x}(t)} \dt  \\ &=&
            \int_\K \phi \ds \text{ mit } \phi(t) = f(x(t)) \cdot T(t)
        \end{eqnarray*}
\end{enumerate}

\section{Bereichsintegrale}
Hier: $f: G \subseteq \R^n \rightarrow \R$ und
\begin{equation*}
    \int_G f = \int_G f(x_1, \ldots, x_n) \intd{(x_1, \ldots x_n)}
\end{equation*}
sollen anschaulich bedeuten:

Welches Volumen schließt der Graph von $f$ mit der Grundfläche $G$ ein.

\subsection{Intervalle im $\R^n$}
Für $a,b \in \R^n$ bezeichnet die Menge
\begin{equation*}
    [a,b] := [a_1, b_1] \times \cdots \times [a_n, b_n]
\end{equation*}
einen (kompakten) Quader oder (kompaktes) Intervall im $\R^n$. Die Zahl
\begin{equation*}
    V([a,b]) =
    \begin{cases}
        \prod_{k=1}^n (b_k - a_k) &, \text{falls }  b_k > a_k \text{ für } k=1,\ldots\\
        0 &,\text{sonst}
    \end{cases}
\end{equation*}
bezeichnet das Volumen, und die Zahlen $b_1 - a_1, \ldots b_n - a_n$ als
Kantenlängen.

\subsection{Definition Zerlegung}
Ist $[a,b] = [a_1, b_1] \times \cdots \times [a_n, b_n]$ und ist für jedes
$k \in \{1, \ldots, n\}$ mit
\begin{equation*}
    T^{(k)} : a_k = x_0 < \cdots < x_{l_k} = b_k
\end{equation*}
eine Zerlegung von $[a_k, b_k]$ dann heißt die Menge
\begin{equation*}
    I_{l_1, \ldots, l_n} = [x_{l_1 - 1}^{(1)} - x_{l_1}^{(1)}] \times \cdots \times
    [x_{l_1 - 1}^{(n)} - x_{l_1}^{(n)}]
\end{equation*}
mit $l_k \in \{1, \ldots, l_k\}$ für $k \in \{1, \ldots, n\}$ eine Zerlegung $T$
von $[a,b]$.

Das Feinheitsmaß von $T$ ist
\begin{equation*}
    \mu(T) = \max_{l_1, \ldots, l_n} V(I_{l_1, \ldots, l_n})
\end{equation*}

Allgemein ist ein Intervall von der Form
\begin{equation*}
    [x_i^{(1)}, x_{i+1}^{(1)}] \times [x_j^{(2)}, x_{j+1}^{(2)}]
\end{equation*}
mit $i \in \{0, \ldots, l_1-1\}$ und $j \in \{0, \ldots l_2-1\}$.

\subsection{Definition Riemann-Summe}
Sei $T$ eine Zerlegung eines kompakten Quaders $I \subseteq \R^n$ mit Teilquadern
$I_1, \ldots, I_l$ mit $l=l_1 \cdot \cdots \cdot l_n$ (entstehen indem man die
Zerlegungsintervalle fortlaufend durchnummeriert) und Zwischenpunkte
$\xi = (\xi_1, \ldots, \xi_l)$ mit $\xi_i \in I_i (i \in \{l, \ldots, n\})$
und $f: I \to \R$ (d.h. Skalarwertige Funktion). Dann heißt
\begin{equation*}
    S(f, T, \xi) = \sum_{i=1}^l f(\xi_i) f(\xi_i) \mu(I_1)
\end{equation*}
die Riemann-Summe von $f$ bezüglich $T$ und $\xi$.

\subsection{Riemann integrierbare Bereichsintegrale}
Sei $f: I \to \R$ eine Funktion, $I \subseteq \R^n$ ein Quader. Gibt es eine Zahl
$\alpha \in \R$, so dass für jede Folge von Zerlegungen ${(T_k)}_{k=1}^\infty$ mit
Zwischenpunkten ${(\xi_k)}_{k=1}^\infty$ mit $\lim\limits_{k \to \infty} \mu(T_k)=0$
die Riemann-Summe $S(f, T_k, \xi_k)$ gegen $\alpha$ konvergiert für $k \to \infty$
dann heißt $f$ Riemann integrierbar über $I$ und $\alpha$ nennen wir das
Bereichsintegral von $f$ über $I$.

\subsubsection{Schreibweise}
\begin{equation*}
    \alpha = \int_I f(x) \dx
\end{equation*}
Zur Schreibweise: z.B. $n=2$ auch:
\begin{equation*}
    \alpha = \iint_I f(x,y) \intd{(x,y)} := \int_I f(x,y) \intd{(x,y)}
\end{equation*}
oder Angabe von $I$ an dem Integral:
\begin{equation*}
    \alpha = \int_{[a_1, b_1]\times[a_2,b_2]} f(x,y) \intd{(x,y)}
\end{equation*}

\subsection{Bereichsintegrale über beschränkte Mengen}
Sei $M \subseteq \R^n$ beschränkt und $I = [a_1, b_1] \times \cdots \times [a_n,
b_n]$ ein Quader mit $M \subseteq I$. Dann heißt $f: M \to \R$ über $M$ integrierbar
wenn die Funktion
\begin{equation*}
    \tilde{f}: I \to \R \text{ mit } \tilde{f}(x) =
    \begin{cases}
        f(x) &, x\in M\\
        0 &, \text{ sonst}
    \end{cases}
\end{equation*}
über $I$ Bereichs-Riemann integrierbar ist. Wir definieren:
\begin{equation*}
    \int_M f(x) \dx = \int_I \tilde{f}(x) \dx
\end{equation*}

\subsection{Cavalieri}
Sei $M \subseteq \R^n (n >1)$ und bezeichne
\begin{equation*}
    M' = \{ x \in \R: {(x,y)}^T \in M \text{ für ein } y \in \R^{n-1} \}
\end{equation*}
und für $x \in M'$
\begin{equation*}
    M(x) = \{ y \in \R^{n-1}: {(x,y)}^ \in M \}
\end{equation*}
dann gilt für $f \in C(\bar{M})$ (falls $M, M', M(x)$ sogenannte messbare Mengen
sind, d.h $\mu(M), \mu{M'}, \mu{M(x)}$ sind definiert)
\begin{equation*}
    \int_M f(x,y) \intd{(x,y)} = \int_{M'} \left[ \int_{M(x)} f(x,y) \dy \right]
        \dx
\end{equation*}
mit $x \in \R$ und $y \in \R^{n-1}$.

\subsection{Fubini}
Im Fall $n=2$ steht nach Cavalieri ein Parameterintegral und mit Fubini gilt:
\begin{equation*}
    \int_{M'} \int_{M(x)} f(x,y) \dy \dx =
    \int_{\tilde{M}'} \int{\tilde{M}(y)} f(x,y) \dx \dy
\end{equation*}
wobei $\tilde{M}', \tilde{M}(y)$ analog zu $M', M(x)$ bezüglich $y$ definiert
sind.

\subsection{Definition Meßbare-Mengen}
Eine beschränkte Menge $M \subseteq \R^n$ heißt (Jordan-) meßbar, wenn
\begin{equation*}
    \int_M 1 \dx
\end{equation*}
existiert, in diesem Fall nennt man
\begin{equation*}
    \mu(M) := \int_M 1 \dx
\end{equation*}
das Volumen von $M$. Ist $\mu{M} =0$, so nenntn man $M$ eine Nullmenge.

\subsection{Definition $2 \times 2$ Determinante}
Für
\begin{equation*}
    A =
    \begin{pmatrix}
        a & c \\
        b & d \\
    \end{pmatrix}
    \in \R^{2 \times 2}
\end{equation*}
definieren wir die Funktion
\begin{equation*}
    \det: \R^{2 \times 2} \to \R
\end{equation*}
durch $A \mapsto \det(A) = a \cdot d -  c \cdot b$ und nennen die
Funktionsauswertung die Determinante von $A$.

\subsection{Mehrdimensionale Substitutonsregel}
Sei $M \subseteq \R^n$ meßbar und $G \supseteq M$ ein Gebiet. Ist $T\in C^1(G, \R^m)$
und gilt $\det(T'(x)) \neq 0\ \forall x \in M \setminus N$ für eine Nullmenge $N$,
dann gilt:
\begin{equation*}
    \int_{T(M)} f(x_1, \ldots, x_n) \intd{(x_1, \ldots, x_n)} =
    \int_M f(T(u_1, \ldots, u_n)) \cdot \abs{\det(T'(u_1, \ldots, u_n))} \intd{(u_1, \ldots
    u_n)}
\end{equation*}

\section{Integralsätze in der Ebene}
\subsection{Positiv berandete Menge}
Eine beschränkte Menge $B \subseteq \R^2$ heißt positiv berandet durch einen Weg,
(Randkurve) $\K$, wenn $T(\K) = \partial B$ ist und wenn $\K$ eine stückweise
stetig differentierbare Parameterdarstellung $x: [a,b] \to \R^2$ hat mit:
\begin{enumerate}[label = (\roman*)]
    \item $\dot{x}(t) \neq 0$ für fast alle $t \in [a,b]$
    \item der Normalenvektor von $x(t)$ zeigt nach außen
\end{enumerate}

\subsection{Satz von Green}
Ist $B \subseteq \R^2$ positiv berandet, dann gilt für alle $f \in C^1(B, \R^2)$
\begin{equation*}
    \iint_B \frac{\partial f_2}{\partial x} - \frac{\partial f_1}{\partial y}
    \intd{(x,y)} =
    \int_{\partial B} f(x,y) \intd{(x,y)}
\end{equation*}

\subsection{Definition Normalbereiche}
Eine Menge $B \subseteq \R^2$ heißt Normalbereich bezüglich der $x$-Achse
(bzw. $y$-Achse), wenn es ein Intervall $[a,b]$ gibt und die Funktion
$\varphi, \psi$ mit
\begin{equation*}
    B = \{ {(x,y)}^T: a \leq x \leq b, \phi(x) \leq y \leq \psi(x) \}
\end{equation*}

\subsection{Gauß'sche Integralsätze in der Ebene}
Sei $B \subseteq \R^2$ ein positiv berandeter Bereich und $f \in C^1(B, \R^2)$
bzw. $f \in C^2(B, \R^2)$ und bezeichne $\nu$ die nach außen gerichtete Normale
auf $\partial B$. Dann gelten die Integralsätze:
\begin{enumerate}[label = (\roman*)]
    \item
        \begin{equation*}
            \iint_B (\vdiv f)(x,y) \intd{(x,y)} = \int_{\partial B} f \cdot \nu \ds
        \end{equation*}
    \item
        \begin{equation*}
            \iint_B f_1(x,y) \Delta f_2(x,y) - f_2(x,y) \Delta f_1(x,y) \intd{(x,y)}
            = \int_{\partial B} f_1 \frac{\partial f_2}{\partial \nu} - f_2
            \frac{\partial f_1}{\partial \nu}
        \end{equation*}
\end{enumerate}

\section{Oberflächenintegrale und Integralsätze im $\R^3$}
\subsection{Definition Reguläre Flächen}
Sei $B \subseteq \R^2$ und $x: B \to \R^3$,
\begin{equation*}
    x(u,v) =
    \begin{pmatrix}
        x_1(u,v)\\
        x_2(u,v)\\
        x_3(u,v)
    \end{pmatrix}
\end{equation*}
eine stetig diffbare Funktion, für die $x_u = \frac{\partial x}{\partial u}$ und
$x_v = \frac{\partial x}{\partial v}$ linear unabhängig sind (d.h.\ die Vektoren
$x_u$ und $x_v$ zeigen nicht in die gleiche oder entgegengesetzte Richtung) (für
fast alle ${(u,v)}^T \in B$) die Menge der Ausnahmen muss $\tilde{B} \subseteq B$
muss $\mu{\tilde{B}} = 0$ erfüllen.

Das Bild einer solchen Funktion, d.h.\ die Menge
\begin{equation*}
    A = x(B) := \{ x(u,v) \vert {(u,v)}^T \in B \}
\end{equation*}
heißt dann eine reguläre Fläche im $\R^3$ und die Funktion $x$ heißt die
Parametrisierung von $A$.

Man nennt
\begin{enumerate}[label = (\roman*)]
    \item $x_u(u,v), x_v(u,v)$ die Tangentialvektoren in ${(u,v)}^T$
    \item $n(u,v) := \frac{(x_u \times x_v)(u,v)}{\norm{(x_u \times x_v)}(u,v)}$
        \begin{itemize}
            \item Vektor mit Länge $1$ der Senkrecht auf den Tangentialvektoren
                steht
            \item Rechnerisch zu enthalten durch das Kreuzprodukt der
                Tangentialvektoren
        \end{itemize}
\end{enumerate}
der (Flächen-) Normalenvektor in ${(u,v)}^T$ falls $x_u(u,v)$ und $x_v(u,v)$
linear unabhängig sind.

Ist $B$ positiv berandet durch $\K: y(t), a \leq t \leq b$ so nennt man $A$ positiv
berandet durch Kurve mit Parameterdarstellung $x(y(t)), a \leq t \leq b$.

\subsection{Defintion Oberflächenintegral}
Sei $A$ eine reguläre Fläche im $\R^3$ mit Parameterdarstellung $x: B \to \R^3,
B \subseteq \R^2$ meßbar und $x$ injektiv auf $B \setminus N$ für eine Nullmenge
$N$.
\begin{enumerate}[label = (\alph*)]
    \item Für jedes $f \in C(A, \R)$ heißt
        \begin{equation*}
            \iint_A f \cdot \ido= \iint_B f(x(u,v)) \cdot \norm{(x_u \times x_v)
            (u,v)} \intd{(u,v)}
        \end{equation*}
        das Oberflächenintegral von $f$ über $A$ und man nennt
        \begin{equation*}
            \ido= \norm{(x_u \times x_v)
            (u,v)} \intd{(u,v)}
        \end{equation*}
        das Oberflächenelement.
    \item $O(A) := \iint_A 1 \do$ heißt Oberflächeninhalt von $A$.
\end{enumerate}

\subsubsection{Bemerkung}
\begin{enumerate}
    \item Das Oberflächenintegral hängt nicht von der Parameterdarstellung ab.
    \item Ein Summand des Obefĺöchenintegrals sieht so aus:
        \begin{equation*}
            f(x(u,v)) \cdot \norm{(x_u \times x_v)(u,v)} \cdot \Delta u \Delta v
        \end{equation*}
\end{enumerate}

\subsection{Satz von Stokes}
Sei $A$ eine reguläre Fläche im $\R^3$ und $\partial A$ positiv berandet. Dann
gilt für $f \in C^1(A, \R^3)$
\begin{equation*}
    \iint_A \rot f \cdot n \ido= \int_{\partial A} f
\end{equation*}
Mit $n$:
\begin{enumerate}[label = (\roman*)]
    \item Normalenvektor
    \item Länge 1
    \item Senkrecht auf Fläche
    \item Immer auf der gleichen Seite von $A$
\end{enumerate}
also
\begin{equation*}
    \iint_B \rot(f(x(u,v))) \cdot n(x(u,v)) \cdot \norm{(x_u \times x_v)(u,v)}
    \intd{(u,v)} = \int_{\partial A} f
\end{equation*}
mit
\begin{equation*}
    n(x(u,v)) = \pm \frac{(x_u \times x_v)(u,v)}{\norm{(x_u \times x_v)(u,v)}}
\end{equation*}

\subsection{vdivgenzsatz von Gauß}
Sei $M \subseteq \R^3$ kompakt und $\partial M$ ergebe sich als endliche
Vereinigung von regulären Flächen, deren Normale $n$ (normiert) nach Außen
zeigt. Dann gilt fpr jedes $f \in C^1(M, \R^3)$
\begin{equation*}
    \iiint_M \div f = \iint_{\partial M} f \cdot n \ido
\end{equation*}
