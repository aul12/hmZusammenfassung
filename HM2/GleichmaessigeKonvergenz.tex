\section{Gleichmäßige Konvergenz}
\subsection{Definition Funktionenfolge und Funktionenreihe}
Sei $M$ eine Menge und $m \in \Z$. Ist jedem $n \in \{m, m+1, \ldots\}$ eine
Funktion $f_n: M \rightarrow \R$ zugeordnet, so nennt man:
\begin{enumerate}[label= (\alph*)]
    \item Die Folge ${(f_n)}_{n=m}^\infty$ eine Funktionenfolge auf $M$
    \item Die Reihe $\sum_{n=m}^\infty f_n(x)$ eine Funktionenreihe auf $M$
\end{enumerate}

konvergiert ${(f_n)}_{n \geq m}$ (bzw. $\sum_{n=m}^\infty f_n(x)$) für alle
$x \in \tilde{M} \subseteq M$ so heißt die durch $f(x) = \lim\limits_{n \rightarrow \infty}
f_n(x)$ (bzw. $f(x) = \sum_{n=m}^\infty f_n(x)$) definierte Funktion $f:
\tilde{M} \rightarrow \R$ die Grenzfunktion von ${(f_n)}_{n=m}^\infty$
(bzw. $\sum_{n=m}^\infty f_n$).

\subsection{Gleichmäßige Konvergenz}
Sei $M$ eine Menge und sei $f: M \rightarrow \R$ eine Funktion.
\begin{enumerate}[label= (\alph*)]
    \item Eine Funktionenfolge ${(f_n)}_{n=1}^\infty$ heißt auf $M$ gleichmäßig
        konvergent gegen $f$ wenn gilt:
        \begin{equation*}
            \forall \varepsilon > 0\  \exists\ n_0(\varepsilon):
                \abs{f_n(x) - f(x)} < \varepsilon\
                \forall x \in M \text{ und } n \geq n_0(\varepsilon)
        \end{equation*}
    \item Eine Funktionenfolge $\sum_{n=m}^\infty f_n$ konvergiert gleichmäßig
        auf $M$ wenn gilt:
            \begin{equation*}
                \forall \varepsilon > 0\ \exists\ n_0(\varepsilon): \abs{\sum_{k=m}^n f_k(x) - f(x)}
                    < \varepsilon\ \forall x \in M \text{ und } n \geq n_0(\varepsilon)
            \end{equation*}
\end{enumerate}

\subsubsection{Bemerkung}
Offensichtlich gilt:
\begin{equation*}
    \text{Gleichmäßig konvergent} \Rightarrow \text{Punktweise Konvergent}
\end{equation*}


\subsection{Stetigkeit der Grenzfunktion}
Sei ${(f_n)}_{n=1}^\infty$ (bzw. $\sum_{n=1}^\infty f_n(x)$) gleichmäßig konvergent gegen
$f$ auf einem Intervall $I$ und alle $f_n$ stetig auf $I$. Dann ist auch die
Grenzfunktion $f$ stetig.

\subsection{Integration der Grenzfunktion}
Sei ${(f_n)}_{n=1}^\infty$ eine Folge von integrierbaren Funktionen auf $[a,b]$
\begin{enumerate}[label= (\alph*)]
    \item Falls ${(f_n)}_{n=1}^\infty$ gleichmäßig gegn $f$ konvergiert, dann ist
    auch $f$ auf $[a,b]$ integrierbar und es gilt:
        \begin{equation*}
            \lim_{n \rightarrow \infty} \int_a^b f_n(x) \dx =
            \int_a^b \lim_{n \rightarrow \infty} f_n(x) \dx
        \end{equation*}
    \item Analog für Funktionenreihen
\end{enumerate}

\subsection{Cauchy Kriterium für gleichmäßige Konvergenz}
\begin{enumerate}[label= (\alph*)]
    \item Eine Funktionenfolge ${(f_n)}_{n=1}^\infty$ konvergiert genau dann
        gleichmäßig auf einer Menge $M$ ($\subseteq$ Definitionsbereich), wenn
        gilt:
        \begin{equation*}
            \forall \varepsilon > 0\ \exists n(\varepsilon):
            \abs{f_n(x) - f_m(x)} < \varepsilon\ \forall n \geq n(\varepsilon) \forall x \in M
        \end{equation*}
    \item Analog für Funktionenreihen
\end{enumerate}

\subsection{Differentiation der Grenzfunktion}
Sei ${(f_n)}_{n=1}^\infty$ eine auf dem Intervall $I$ differentierbare Folge von
Funktionen.
\begin{enumerate}[label= (\alph*)]
    \item Konvergiert die Folge ${(f_n')}_{n=1}^\infty$ gleichmäßig auf $I$ und konvergiert
    für ein beliebiges, festes $x_0 \in I$ die reele Folge ${(f_n(x_0))}_{n=1}^\infty$
    dann ist auch die Grenzfunktion $f$ von ${(f_n)}_{n=1}^\infty$ differentierbar
    und es gilt:
    \begin{equation*}
        \lim_{n \rightarrow \infty} \ddx f_n(x) = \ddx \lim_{n \rightarrow \infty} f_n(x)
    \end{equation*}
    \item Analog für Funktionenreihen
\end{enumerate}

\subsubsection{Bemerkung}
Außerdem gilt dass ${(f_n)}_{n=1}^\infty$ (bzw. $\sum_{n=1}^\infty f_n$) auf jedem
beschränkten Teilintervall von $I$ gleichmäßig konvergiert.

\subsection{Majorantenkriterium auf Potenzreihen anwenden}
Für eine reele Potenzreihe $f(x) = \sum_{k=0}^\infty a_k {(x-x_0)}^k$ mit
Konvergenzradius $R > 0$ gilt:
\begin{enumerate}[label= (\alph*)]
    \item $f$ ist stetig auf $(x_0 - R, x_0 + R) =: I$
    \item $f$ ist differentierbar auf $I$ und
        \begin{equation*}
            f'(x) = \sum_{k=0}^\infty a_k \cdot k \cdot {(x-x_0)}^{k-1}
        \end{equation*}
    \item $f$ ist integrierbar auf $I$ und hat die Stammfunktion
        \begin{equation*}
            F(x) = \sum_{k=0}^\infty \frac{a_k}{k+1} {(x-x_0)}^{k+1}
        \end{equation*}
\end{enumerate}

\subsubsection{Bemerkung}
Wurde alles schon in HM1 gezeigt aber mühsam.

\subsection{Majorantenkriterium für Funktionenreihen}
Falls $\abs{f_n(x)} \leq a_n$ und $\sum_{n=1}^\infty f_n(x)$ konvergiert
$\Rightarrow \sum_{n=1}^\infty f_n(x)$ ist gleichmäßig konvergent.
