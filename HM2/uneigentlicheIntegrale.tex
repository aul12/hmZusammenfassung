\section{Definition uneigentliches Integral}
Eine Funktion $f: [a, b) \rightarrow \R$ mit $a < b \leq \infty$ heißt über
$[a,b)$ uneigentlich Riemann integrierbar, wenn gilt:
\begin{enumerate}[label= (\alph*)]
    \item $\forall c$ mit $a \leq c < b$ ist $f \in R[a, c]$
    \item Der Grenzwert
        \begin{equation*}
            \alpha = \lim_{c \rightarrow \infty} \int_a^c f(x) \dx
        \end{equation*}
        existiert. In dem Fall schreiben wir
        \begin{equation*}
            \alpha = \int_a^b f(x) \dx
        \end{equation*}
        und sagen das uneigentliche Integral
        \begin{equation*}
            \int_a^b f(x) \dx
        \end{equation*}
        konvergiert gegen $\alpha$ oder hat den Wert $\alpha$.
\end{enumerate}

Andernfalls divergiert das uneigentliche Integral.
Analog geht man für Funktionen
\begin{itemize}
    \item $f: (a, b] \rightarrow \R$ mit $-\infty \leq a < b$ und
    \item $f: (a, b) \rightarrow \R$ mit $-\infty \leq a < b \leq \infty$
\end{itemize}
vor.

\subsection{Cauchy-Kriterium}
Sei $f \in R[a,b]\ \forall c \in (a,b), a < c \leq \infty$ Dann konv.
\begin{equation*}
    \int_a^b f(x) \dx
\end{equation*}
\begin{enumerate}[label= (\alph*)]
    \item Im Fall $b<\infty$ genau dann, wenn gilt:
        \begin{equation*}
            \forall \varepsilon > 0\ \exists \delta > 0: \abs{\int_{T_1}^{T_2} f(x) \dx} < \varepsilon\ \forall~T_1, T_2 \in [b - \delta, b)
        \end{equation*}
    \item Im Fall $b = \infty$, wenn gilt:
        \begin{equation*}
                \forall \varepsilon > 0\ \exists K \geq a: \abs{\int_{T_1}^{T_2} f(x) \dx} < \varepsilon\ \forall~T_1, T_2 \geq K
        \end{equation*}
\end{enumerate}
