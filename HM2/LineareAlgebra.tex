\section{Der Begriff Vektorraum}
\subsection{Definition Vektorraum}
Gegeben sei eine abelsche Gruppe $V$ und ein Körper $\K$ (bei uns wird $\K=\R$
oder $\K=\C$ gelten) und eine Abbildung:
\begin{equation*}
    \cdot: \K \times V \to V, \cdot(\alpha, x) \mapsto \alpha \cdot x =:
    \alpha x \text{ (Skalierung)}
\end{equation*}
Dann nennt man $V$ einen Vektorraum über $\K$, wenn die folgenden Vektorraumaxiome
erfüllt sind:
\begin{eqnarray*}
    &\text{(V1)}& \alpha \cdot (\beta \cdot v) = (\alpha \cdot \beta) \cdot v
    \text{ (Assoziativgesetz)}\\
    &\text{(V2)}& \alpha \cdot (x+y) = (\alpha \cdot x) + (\alpha \cdot y) =
    \alpha x + \alpha y \\
    && (\alpha + \beta) \cdot x = \alpha x + \beta x
    \text{ (Distributivgesetzte)}\\
    &\text{(V3)}& 1 \cdot x = x \text{ für die } 1 \in \K \text{ (Gesetz der Eins)}
\end{eqnarray*}

In einem Vektorraum $V$ über $\K$ nennt man Elemente aus $V$ Vektoren, die Elemente
aus $\K$ Skalare, $\K$ den Skalarkörper und \glqq{}$\cdot$\grqq{} die
Multiplikation mit Skalaren. Die \glqq{}$+$\grqq{} Verknüpfung in $V$ die $V$ die
Vektoraddition und das neutrale Element $\vec{0} \in V$ den Nullvektor.

\subsection{Rechenregeln}
Ist $V$ ein Vektorraum über $\K$, so gilt für $\alpha, \beta \in \K$ und
$x, y \in V$:
\begin{enumerate}
    \item
        \begin{enumerate}[label = (\alph*)]
            \item $0 \cdot x = \vec{0} = \alpha \vec{0}$
            \item $\alpha \cdot x = \vec{0} \Rightarrow \alpha = 0 \lor x=\vec{0}$
        \end{enumerate}
     \item
        \begin{equation*}
            \alpha (-x) = (-\alpha) x = - (\alpha x)
        \end{equation*}
\end{enumerate}

\section{Unterräume}
\subsection{Definition Unterraum}
Eine Teilmenge $U$ eines Vektorraums $V$ über $\K$ heißt Unterraum von $V$, wenn
$U$ bezüglich der in $V$ definierten Vektoraddition und Skalierung ein Vektorraum
ist.

\subsection{Unterraumkriterien}
Für $U \subseteq V$ und $U \neq \emptyset$ sind folgende Aussagen äquivalent
\begin{enumerate}[label = (\alph*)]
    \item $U$ ist ein Unterraum von $V$
    \item
        \begin{equation*}
            x,y \in U, \alpha, \beta \in \K \Rightarrow
            \alpha x + \beta y \in U
        \end{equation*}
    \item
        \begin{equation*}
            (x,y \in U \Rightarrow x+y \in U) \land
            (\alpha \in K, x \in U \Rightarrow \alpha x \in U)
        \end{equation*}
\end{enumerate}

\subsection{Durchschnitt von Unterräumen}
Der Durchschnitt von Unterräumen ist wieder ein Unterraum, d.h.:
\begin{equation*}
    U_i\ i \in J\ (J\text{ eine Indexmenge}) \text{ sind Unterräume}
    \Rightarrow \bigcap_{i \in J} U_i \text{ ist Unterraum}
\end{equation*}

\subsection{Definition lineare Hülle}
\begin{itemize}
    \item Ist $M$ eine beliebige Teilemenge eines Vektorraums. Dann heißt
        \begin{equation*}
            \vspan (M) := \bigcap_{U \in S} U \text{ mit }
            S := \{ U \subseteq V: U \text{ ist Unterraum}, U \supseteq M \}
        \end{equation*}
        der von $M$ aufgespannte Unterraum oder die lineare Hülle von M.
    \item Ist $U$ ein Unterraum und $M \subseteq V$ mit $\vspan(M) = U$, dann heißt
        $M$ ein erzeugendes System von $U$.
\end{itemize}

\subsubsection{Bemerkung}
\begin{enumerate}
    \item $\vspan(M)$ ist der kleinste Unterraum, der $M$ enthält
    \item $\vspan(\emptyset) = \vec{0}$
    \item $M \subseteq N \Rightarrow \vspan(M) \subseteq \vspan(N)$
    \item Ist $U$ ein Unterraum, dann gilt $U=\vspan(U)=\vspan(U \setminus \{\vec{0}\})$
\end{enumerate}

\subsection{Definition Linearkombination}
Ist $V$ ein Vektorraum über $\K$ und $x_1, \ldots x_n \in V, \alpha_1, \ldots,
\alpha_n \in \K$ dann heißt
\begin{equation*}
    \sum_{k=1}^n a_k x_k \in V
\end{equation*}
eine Linearkombination von $x_1, \ldots, x_n$ (mit Koeffizienten $\alpha_1, \ldots
\alpha_n$).

\subsection{Zusammenhang lineare Hülle --- Linearkombination}
Sei $V$ ein Vektorraum über $\K$ und $M \subseteq V$, dann gilt $\vspan{M}$ ist
die Menge aller Linearkombinationen, d.h.
\begin{equation*}
    \vspan(M) = \{\alpha_1 x_1 + \cdots + \alpha_n x_n \vert x \in \N,
    x_1, \ldots, x_n \in M, \alpha_1, \ldots, \alpha_n \in \K \}
\end{equation*}
im Fall $M=\{x_1, \ldots, x_n\}$ gilt:
\begin{equation*}
    \vspan(M) =  \{\alpha_1 x_1 + \ldots \alpha_n x_n \vert \alpha_1, \ldots
    \alpha_n \in \K \}
\end{equation*}

\section{Lineare Unabhängigkeit}
\subsection{Definition Lineare Unabhängigkeit}
Sei $V$ ein Vektorraum über $\K$
\begin{enumerate}[label = (\alph*)]
    \item Eine endliche Liste $a_1, \ldots, a_n \in V$ heißt linear unabhängig
        (l.u.), wenn gilt
        \begin{equation*}
            \alpha_1 a_1 + \cdots + \alpha_n a_n = \vec{0}
            \Rightarrow \alpha_1 = \cdots = \alpha_n  = 0
        \end{equation*}
        Andernfalls heißen $a_1, \ldots, a_n$ linear abhängig (l.a.).
    \item Eine beliebige Teilmenge $M \subseteq V$ heißt linear unabhängig, wenn
        für eine beliebige endliche Liste $a_1, \ldots, a_n \in M$ gilt, dass
        diese linear unabhängig sind. Andernfalls ist $M$ linear abhängig.
\end{enumerate}

\subsection{Rechenregeln für lineare Unabhängigkeit}
Für Vektoren $a, a_1, \ldots, a_n, b_1, \ldots, b_n$ eines Vektorraumes $V$ gilt:
\begin{enumerate}[label = (\alph*)]
        \item
            \begin{equation*}
                a \text{l.u.} \Leftrightarrow \{a\} \text{ l.u.} \Leftrightarrow
                a \neq \vec{0}
            \end{equation*}
            Bemerkung:

            $a_1, a_2$ mit $a_1 = a_2$ ist linear unabhängig, aber $M=\{a, a\} =
            \{a\}$ ist nur dann linear abhängig wenn $a = \vec{0}$.
        \item
            $a_1, \ldots, a_n$ linear abhängig $\Rightarrow$ $a_1, \ldots a_n,
            b_1, \ldots, b_k$ sind linear abhängig für $k \geq 0$.
        \item $a_1, \ldots, a_n$ linear unabhängig $\Rightarrow a_1, \ldots, a_k$
            linear unabhängig für $k \leq n$
        \item $a_1, \ldots, a_n$ linear unabhängig $\Rightarrow a_1, \ldots a_n$
            sind paarweise verschieden
        \item Für $n \geq 2$ sind $a_1, \ldots, a_n$ genau dann linear abhängig,
            wenn ein Vektor als Linearkombination darstellbar ist. D.h.:
            \begin{equation*}
                \exists i \in \{1, \ldots, n\}: a_i = \sum_{k=1, k \neq i}^n
                \alpha_k a_k \text{ für } \alpha_1, \ldots, \alpha_{i-1},
                \alpha_{i+1}, \ldots, \alpha_n \in \K
            \end{equation*}
        \item Sind $a_1, \ldots, a_n$ linear unabhängig und $a_1, \ldots, a_n, a$
            linear abhängig, so ist $a$ die linear Kombination von $a_1, \ldots,
            a_n$ und die Koeffizienten sind eindeutig.
        \item Ist $a$ eine Linearkombination von $a_1, \ldots, a_n$ und jeder
            Vektor $a_k$ eine Linearkombination von $b_1, \ldots, b_m$ so ist
            $a$ eine Linearkombination von $b_1, \ldots, b_m$
\end{enumerate}

\subsubsection{Bemerkung}
Für Teilmengen $M, N$ eines Vektorraums $V$ gilt:
\begin{enumerate}[label= (\alph*)]
    \item $M$ l.a. $M \subseteq N \Rightarrow N$ l.a.
    \item $M=\emptyset \Rightarrow M$ l.u.
    \item $\vec{0} \in M \Rightarrow M$ l.a.
\end{enumerate}

\paragraph{Für $V = \R^3$}
\begin{enumerate}[label= (\alph*)]
    \item $a_1, a_2, a_3$ seien linear abhängig und $a_1, a_2$ linear unabhängig

        $\Leftrightarrow$ also $a_3$ ist in der von $a_1$ und $a_2$ aufgespannten
        Ebene

        $\Rightarrow$ Spat mit Kanten $a_1, a_2, a_3$ hat Volumen $0$

        $\Leftrightarrow \det(a_1, a_2, a_3) = 0$
    \item $a_1, a_2$ linear abhängig $\Rightarrow a_2$ ist auf der von $a_1$
        aufgespannten Gerade

        $\Rightarrow \det(a_1, a_2) = 0$
\end{enumerate}

\section{Basis und Dimension}
\subsection{Definition Hamel-Basis}
\begin{enumerate}[label= (\alph*)]
    \item Eine Teilmenge $B$ eines Vektorraums $V$ heißt (Hamel-) Basis von $V$,
        wenn gilt
        \begin{enumerate}[label= (\roman*)]
            \item $B$ ist linear unabhängig
            \item $V = \vspan(B)$
        \end{enumerate}
        Kurz:
        \begin{equation*}
            B \text{ ist ein linear unabhängiges Erzeuger-System
            von }V
        \end{equation*}
    \item Man sagt Vektoren $b_1, \ldots, b_n$ bilden eine Basis von $V$, wenn
        gilt $B=\{b_1, \ldots, b_n\}$ ist eine Basis von $V$.
\end{enumerate}

\subsection{Äquivalente Aussagen zu Basen}
Sein $B \subseteq V, V$ ein Vektorraum, dann sind folgende Aussagen äquivalent
\begin{enumerate}
    \item $B$ ist eine Basis
    \item $B$ ist ein minimales Erzeugersystem
        \begin{enumerate}
            \item $V = \vspan(B)$
            \item $V \neq \vspan(\tilde{B})$ mit $\tilde{B} \subsetneq B$
        \end{enumerate}
    \item $B$ ist eine maximale, linear Unabhängige Teilmenge von $V$, d.h
        \begin{enumerate}
            \item $B$ ist linear unabhängig
            \item $B \cup \{ x \}$ ist linear abhängig für $x \in V \setminus B$
                beliebig. D.h. $x \in \vspan(B)$
        \end{enumerate}
    \item Falls $B \neq \emptyset$ ist dies auch äquivalent zu:

        Jedes $x \in V$ ist eine Linearkombination von endlich vielen Vektoren
        aus $B$ und die koeffizienten sind eindeutig bestimmt.
\end{enumerate}

\subsection{Existenz einer Basis}
Jeder Vektorraum $V$ besitzt eine Basis. Genauer gilt: ist $M$ ein
Erzeugendensystem von $V$, so gibt es eine Basis $B$ von $V$ mit
$B \subseteq V$.

\subsection{Eigenschaften der Basis}
Sei $V$ ein Vektorraum, dann gilt
\begin{enumerate}
    \item Ist $M \subseteq V$ endlich und linear unabhängig so existiert eine
        Basis $B$ von $V$ mit $M \subseteq B$. D.h. $M$ kann zu einer Basis von
        $V$ erweitert werden.
    \item Ist $B$ und $\tilde{B}$ jeweils eine Basis von $V$ dann haben beide
        gleich viele Elemente. Also $\abs{B} = \abs{\tilde{B}}$ (also auch im
        Fall $\infty = \infty$).
\end{enumerate}

\subsection{Definition Dimension}
Ist $V$ ein Vektorraum und $B$ eine Basis, dann ist
\begin{equation*}
    \dim V := \abs{B}
\end{equation*}
die Dimension von $V$. Im Fall $\abs{B} < \infty$ also $\dim V \in \N_0$ heißt
$V$ endlich dimensional.

\subsection{Beziehung von Dimensionen}
Ist $U$ ein Unterraum von $V$, so gilt
\begin{enumerate}[label = (\alph*)]
    \item $\dim U \leq \dim V$
    \item $\dim U = \dim V \Leftrightarrow U = V$ für $\dim V < \infty$
\end{enumerate}

\subsection{Lineare unabängigkeit im $n$-Dimensionalen}
Ist $V$ ein $n$-Dimensionaler Vektorraum, so gilt:
\begin{enumerate}[label = (\alph*)]
    \item $n$ linear unabhängige Vektoren bilden eine Basis
    \item $n+1$ Vektoren $V$ sind linear abhängig.
\end{enumerate}

\section{Lineare Gleichungssysteme}
\subsection{Definition lineares Gleichungssystem}
Sei $m, n \in \K, \K$ ein Körper. $A \in \K^{m \times n}, b \in \K^m$ dann nennt
man die Gleichung
\begin{equation*}
    A \cdot x = b \Leftrightarrow
    \begin{cases}
        a_{11} x_1 + a_{12} x_2 + \cdots + a_{1n} x_n &= b_1\\
        a_{21} x_1 + a_{22} x_2 + \cdots + a_{2n} x_n &= b_2\\
        \vdots & \vdots \\
        a_{m1} x_1 + a_{m2} x_2 + \cdots + a_{mn} x_n &= b_m
    \end{cases}
\end{equation*}
ein Lineares Gleichungssystem mit Koeffizienten $A$, rechter Seite $b$ und
Unbekannten $x$.

Etwas allgemeiner mit $A \in \K^{m \times n}, B \in \K^{m \times r},
X \in \K^{n \times r}$.

Dann ist
\begin{equation*}
    A \cdot X = B \Leftrightarrow
    \begin{cases}
        A \cdot x_1 & = b_1\\
        \vdots & \vdots \\
        A \cdot x_r & = b_r
    \end{cases}
\end{equation*}
ein System von Linearengleichungen mit Koeffizienten $X$, rechten Seiten $B$
und Unbekannten $X$.

Im Fall $b=\vec{0}$ (bzw. $B = \vec{0}$) heißt das LGS homogen.

\subsubsection{Bemerkung}
Im Fall $A x = b$ nennt man
\begin{enumerate}[label= (\alph*)]
    \item $\mathbb{L}(A, b) := \{ x \in \K^n: A x = b \}$ die Lösungsmenge des
        LGS
    \item $\Kern(A) := \{ x \in \K^n: A \cdot x = \vec{0}\}$ den Kern von $A$.
        Es gilt $\Kern(A) = \mathbb{L}(A,\vec{0})$.
    \item $\Image(A) := \{ y \in \K^m:\ \exists x \in \K^n \text{ mit } A x = y\}$
        das Bild von $A$.
\end{enumerate}

\subsection{Zusammenhang Kern und Lösung eines LGS}
Sei $A \in \K^{m \times n}$, dann gilt
\begin{enumerate}[label= (\alph*)]
    \item $\Kern(A)$ ist Unterraum von $\K^n$
    \item Ist $b \in \K^m$ und $x_0 \in \K^n$ eine Lösung von $Ax=n$, dann ist
        die Lösungsgesamtheit
        \begin{equation*}
            x_0 + \Kern(A) := \{ x_0 + y: y \in \Kern(A) \}
        \end{equation*}
\end{enumerate}

\subsubsection{Bemerkung}
Außerdem ist $\mathbb{L}(A,b)$ und $\Kern(A) = \mathbb{L}(A,\vec{0})$
ein Unterraum von $\K^n$.

\subsection{Definition Affiner Unterraum, lineare Mannigfaltigkeit}
Ist $x_0$ ein Vektor und $U$ ein Unterraum, dann ist
\begin{equation*}
    x_0 + U := \{ x_0 + u: u \in U\}
\end{equation*}
ein affiner Unterraum und $\dim(x_0 + U) := \dim(U)$ die Dimension von $x_o + U$.

\subsection{Definition Zeilen-/Spaltenrang}
Sei
\begin{equation*}
    A = (a_1, \ldots, a_n) =
    \begin{pmatrix}
        \tilde{a}_1^T \\ \vdots \\ \tilde{a}_m^T
    \end{pmatrix}
\end{equation*}
. Dann heißt
\begin{enumerate}[label= (\alph*)]
    \item $\dim(\vspan(a_1, \ldots, a_n))$ der Spaltenrang
    \item $\dim(\vspan(\tilde{a}_1, \ldots, \tilde{a}_m))$ der Zeilenrang
\end{enumerate}

\subsubsection{Bemerkung}
Also Spaltenrang $\widehat{=}$ Anzahl der linear unabhängigen Spalten.

Und Zeilenrang $\widehat{=}$ Anzahl der linear unabhängigen Zeilen.

\subsection{Elementare Zeilen-/Stufenoperationen}
Mittels elementarer Zeilen- bzw. Stufenoperationen kann man eine Matrix auf
Zeilenstufenform bringen.
\begin{enumerate}
    \item Vertauschen von Zeilen bzw. Spalten

        Spalten-Operation:
        \begin{equation*}
            A \cdot E =
            \begin{pmatrix}
                1 & 2 & 3\\
                4 & 5 & 6\\
                7 & 8 & 9
            \end{pmatrix}
            \begin{pmatrix}
                0 & 0 & 1\\
                0 & 1 & 0\\
                1 & 0 & 0
            \end{pmatrix}
            =
            \begin{pmatrix}
                3 & 2 & 1\\
                6 & 5 & 4\\
                9 & 8 & 7
            \end{pmatrix}
        \end{equation*}
        Zeilen-Operationen
        \begin{equation*}
            E \cdot A =
            \begin{pmatrix}
                0 & 0 & 1\\
                0 & 1 & 0\\
                1 & 0 & 0
            \end{pmatrix}
            \begin{pmatrix}
                1 & 2 & 3\\
                4 & 5 & 6\\
                7 & 8 & 9
            \end{pmatrix}
            =
            \begin{pmatrix}
                7 & 8 & 9\\
                4 & 5 & 6\\
                1 & 2 & 3\\
            \end{pmatrix}
        \end{equation*}
    \item Das $\alpha$-fache der zweiten Zeile/Spalte zur ersten Zeile/Spalte
        addieren

        Spalten-Operation:
        \begin{equation*}
            A \cdot E =
            \begin{pmatrix}
                1 & 2 & 3\\
                4 & 5 & 6\\
                7 & 8 & 9
            \end{pmatrix}
            \begin{pmatrix}
                1 & 0 & 0\\
                \alpha & 1 & 0\\
                0 & 0 & 1
            \end{pmatrix}
            =
            \begin{pmatrix}
                1 + 2 \alpha & 2 & 3\\
                4 + 5 \alpha & 5 & 6\\
                7 + 8 \alpha & 8 & 9
            \end{pmatrix}
        \end{equation*}
        Zeilen-Operationen
        \begin{equation*}
            E \cdot A =
            \begin{pmatrix}
                1 & 0 & 0\\
                \alpha & 1 & 0\\
                0 & 0 & 1
            \end{pmatrix}
            \begin{pmatrix}
                1 & 2 & 3\\
                4 & 5 & 6\\
                7 & 8 & 9
            \end{pmatrix}
            =
            \begin{pmatrix}
                1 + 4 \alpha & 2+5\alpha & 3+6\alpha \\
                4 & 5 & 6\\
                7 & 8 & 9
            \end{pmatrix}
        \end{equation*}
    \item Zeile/Spalte 2 mit $\alpha$ multiplizieren

        Spalten-Operation:
        \begin{equation*}
            A \cdot E =
            \begin{pmatrix}
                1 & 2 & 3\\
                4 & 5 & 6\\
                7 & 8 & 9
            \end{pmatrix}
            \begin{pmatrix}
                1 & 0 & 0\\
                0 & \alpha & 0\\
                0 & 0 & 1
            \end{pmatrix}
            =
            \begin{pmatrix}
                1 & 2 \alpha & 3\\
                4 & 5 \alpha & 6\\
                7 & 8 \alpha & 9
            \end{pmatrix}
        \end{equation*}
        Zeilen-Operationen:
        \begin{equation*}
            E \cdot A =
            \begin{pmatrix}
                1 & 0 & 0\\
                0 & \alpha & 0\\
                0 & 0 & 1
            \end{pmatrix}
            \begin{pmatrix}
                1 & 2 & 3\\
                4 & 5 & 6\\
                7 & 8 & 9
            \end{pmatrix}
            =
            \begin{pmatrix}
                1 & 2 & 3\\
                4 \alpha & 5 \alpha & 6 \alpha \\
                7 & 8 & 9
            \end{pmatrix}
        \end{equation*}
\end{enumerate}
Formal  bedeutet die Übeführung von $A$ in Zeilenstufenform $\tilde{A}$ also:
\begin{enumerate}[label= (\alph*)]
    \item mit elementaren Zeilenoperationen (Gauß) überführt ist
        $\tilde{A} = E_r \cdot \cdots \cdot E_1 \cdot A$ mit Elementarmatrizen
        $E_1, \ldots, E_r$.
    \item Mit elementaren Spaltenoperationen ist $\tilde{\tilde{A}} = A_1 \cdot
        E_1 \cdot \cdots \cdot E_s$ mit Elementarmatrizen $E_1, \ldots, E_s$.
\end{enumerate}

\subsubsection{Bemerkung}
Statt in Zeilenstufenform kann man $A$ auch in die Form der Pseudo-Einheitsmatrix
\begin{equation*}
    \hat{I} =
    \begin{pmatrix}
        1  \\
        & 1 & \vec{0} &  \\
        & \vec{0} & \ddots \\
        &  & & 0
    \end{pmatrix} = \diag(1, 1, \ldots, 0, 0)
\end{equation*}

Man kann zeigen:
\begin{enumerate}
    \item Wird eine Zeilenoperation durchgeführt ändert sich nicht der
        Zeilenrang
    \item Wir eine Spaltenoperation durchgeführt ändert sich nicht der
        Spaltenrang
\end{enumerate}

\subsection{Beziehung Spalten-/Zeilenrang}
Zeilen- und Spaltenrang sind gleich.

\subsection{Definition Rang einer Matrix}
Seu $A\in\K^{m\times n}$, dann ist der Rang von $A$ definiert aös der Zeilenrang
von $A$ und wir schreiben dafür:
\begin{equation*}
    \rg(A) := \text{ Zeilenrang}
\end{equation*}

\subsection{Gauß-Algorithmus}
Sei für $A \in \K^{m \times n}, b\in\K^m, x\in\K^n$ $Ax=b$ ein LGS mit gewissen
elementaren Zeilenoperationen wird $A$ in eine Zeilenstufenform $\tilde{A}$
überführt ($\tilde{A} = E_r \cdot \cdots \cdot E_1 \cdot A$). Gleichzeitig werden
diese auf $b$ angewende, wobei man ein $\tilde{b}$ erhält ($\tilde{b} = E_r \cdot
\cdots \cdot E_1 \cdot b$).

Dann gilt für $x$:
\begin{equation*}
    Ax = b \Leftrightarrow \tilde{A}x = \tilde{b}
\end{equation*}

\subsection{Lösbarkeit eines LGS}
Ein LGS $AX = b$ ist lösbar, genau dann wenn
\begin{equation*}
    \rg(A) = \rg(A\vert b)
\end{equation*}
gilt.

\subsection{Lösung eines LGS}
Falls ein LGS $Ax = b$ lösbar ist, dann ist $\mathbb{L}(A\vert b)$ ein affiner
Unterraum und es gilt:
\begin{equation*}
    \dim(\L(A\vert b)) = n - \rg(A)
\end{equation*}
