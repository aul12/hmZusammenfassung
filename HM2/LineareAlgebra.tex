\section{Der Begriff Vektorraum}
\subsection{Definition Vektorraum}
Gegeben sei eine abelsche Gruppe $V$ und ein Körper $\K$ (bei uns wird $\K=\R$
oder $\K=\C$ gelten) und eine Abbildung:
\begin{equation*}
    \cdot: \K \times V \to V, \cdot(\alpha, x) \mapsto \alpha \cdot x =:
    \alpha x \text{ (Skalierung)}
\end{equation*}
Dann nennt man $V$ einen Vektorraum über $\K$, wenn die folgenden Vektorraumaxiome
erfüllt sind:
\begin{eqnarray*}
    &\text{(V1)}& \alpha \cdot (\beta \cdot v) = (\alpha \cdot \beta) \cdot v
    \text{ (Assoziativgesetz)}\\
    &\text{(V2)}& \alpha \cdot (x+y) = (\alpha \cdot x) + (\alpha \cdot y) =
    \alpha x + \alpha y \\
    && (\alpha + \beta) \cdot x = \alpha x + \beta x
    \text{ (Distributivgesetzte)}\\
    &\text{(V3)}& 1 \cdot x = x \text{ für die } 1 \in \K \text{ (Gesetz der Eins)}
\end{eqnarray*}

In einem Vektorraum $V$ über $\K$ nennt man Elemente aus $V$ Vektoren, die Elemente
aus $\K$ Skalare, $\K$ den Skalarkörper und \glqq{}$\cdot$\grqq{} die
Multiplikation mit Skalaren. Die \glqq{}$+$\grqq{} Verknüpfung in $V$ die $V$ die
Vektoraddition und das neutrale Element $\vec{0} \in V$ den Nullvektor.

\subsection{Rechenregeln}
Ist $V$ ein Vektorraum über $\K$, so gilt für $\alpha, \beta \in \K$ und
$x, y \in V$:
\begin{enumerate}
    \item
        \begin{enumerate}[label = (\alph*)]
            \item $0 \cdot x = \vec{0} = \alpha \vec{0}$
            \item $\alpha \cdot x = \vec{0} \Rightarrow \alpha = 0 \lor x=\vec{0}$
        \end{enumerate}
     \item
        \begin{equation*}
            \alpha (-x) = (-\alpha) x = - (\alpha x)
        \end{equation*}
\end{enumerate}

\section{Unterräume}
\subsection{Definition Unterraum}
Eine Teilmenge $U$ eines Vektorraums $V$ über $\K$ heißt Unterraum von $V$, wenn
$U$ bezüglich der in $V$ definierten Vektoraddition und Skalierung ein Vektorraum
ist.

\subsection{Unterraumkriterien}
Für $U \subseteq V$ und $U \neq \emptyset$ sind folgende Aussagen äquivalent
\begin{enumerate}[label = (\alph*)]
    \item $U$ ist ein Unterraum von $V$
    \item
        \begin{equation*}
            x,y \in U, \alpha, \beta \in \K \Rightarrow
            \alpha x + \beta y \in U
        \end{equation*}
    \item
        \begin{equation*}
            (x,y \in U \Rightarrow x+y \in U) \land
            (\alpha \in K, x \in U \Rightarrow \alpha x \in U)
        \end{equation*}
\end{enumerate}

\subsection{Durchschnitt von Unterräumen}
Der Durchschnitt von Unterräumen ist wieder ein Unterraum, d.h.:
\begin{equation*}
    U_i\ i \in J\ (J\text{ eine Indexmenge}) \text{ sind Unterräume}
    \Rightarrow \bigcap_{i \in J} U_i \text{ ist Unterraum}
\end{equation*}

\subsection{Definition lineare Hülle}
\begin{itemize}
    \item Ist $M$ eine beliebige Teilemenge eines Vektorraums. Dann heißt
        \begin{equation*}
            \vspan (M) := \bigcap_{U \in S} U \text{ mit }
            S := \{ U \subseteq V: U \text{ ist Unterraum}, U \supseteq M \}
        \end{equation*}
        der von $M$ aufgespannte Unterraum oder die lineare Hülle von M.
    \item Ist $U$ ein Unterraum und $M \subseteq V$ mit $\vspan(M) = U$, dann heißt
        $M$ ein erzeugendes System von $U$.
\end{itemize}

\subsubsection{Bemerkung}
\begin{enumerate}
    \item $\vspan(M)$ ist der kleinste Unterraum, der $M$ enthält
    \item $\vspan(\emptyset) = \vec{0}$
    \item $M \subseteq N \Rightarrow \vspan(M) \subseteq \vspan(N)$
    \item Ist $U$ ein Unterraum, dann gilt $U=\vspan(U)=\vspan(U \setminus \{\vec{0}\})$
\end{enumerate}

\subsection{Definition Linearkombination}
Ist $V$ ein Vektorraum über $\K$ und $x_1, \ldots x_n \in V, \alpha_1, \ldots,
\alpha_n \in \K$ dann heißt
\begin{equation*}
    \sum_{k=1}^n a_k x_k \in V
\end{equation*}
eine Linearkombination von $x_1, \ldots, x_n$ (mit Koeffizienten $\alpha_1, \ldots
\alpha_n$).

\subsection{Zusammenhang lineare Hülle --- Linearkombination}
Sei $V$ ein Vektorraum über $\K$ und $M \subseteq V$, dann gilt $\vspan{M}$ ist
die Menge aller Linearkombinationen, d.h.
\begin{equation*}
    \vspan(M) = \{\alpha_1 x_1 + \cdots + \alpha_n x_n \vert x \in \N,
    x_1, \ldots, x_n \in M, \alpha_1, \ldots, \alpha_n \in \K \}
\end{equation*}
im Fall $M=\{x_1, \ldots, x_n\}$ gilt:
\begin{equation*}
    \vspan(M) =  \{\alpha_1 x_1 + \ldots \alpha_n x_n \vert \alpha_1, \ldots
    \alpha_n \in \K \}
\end{equation*}

\section{Lineare Unabhängigkeit}
\subsection{Definition Lineare Unabhängigkeit}
Sei $V$ ein Vektorraum über $\K$
\begin{enumerate}[label = (\alph*)]
    \item Eine endliche Liste $a_1, \ldots, a_n \in V$ heißt linear unabhängig
        (l.u.), wenn gilt
        \begin{equation*}
            \alpha_1 a_1 + \cdots + \alpha_n a_n = \vec{0}
            \Rightarrow \alpha_1 = \cdots = \alpha_n  = 0
        \end{equation*}
        Andernfalls heißen $a_1, \ldots, a_n$ linear abhängig (l.a.).
    \item Eine beliebige Teilmenge $M \subseteq V$ heißt linear unabhängig, wenn
        für eine beliebige endliche Liste $a_1, \ldots, a_n \in M$ gilt, dass
        diese linear unabhängig sind. Andernfalls ist $M$ linear abhängig.
\end{enumerate}

\subsection{Rechenregeln für lineare Unabhängigkeit}
Für Vektoren $a, a_1, \ldots, a_n, b_1, \ldots, b_n$ eines Vektorraumes $V$ gilt:
\begin{enumerate}[label = (\alph*)]
        \item
            \begin{equation*}
                a \text{l.u.} \Leftrightarrow \{a\} \text{ l.u.} \Leftrightarrow
                a \neq \vec{0}
            \end{equation*}
            Bemerkung:

            $a_1, a_2$ mit $a_1 = a_2$ ist linear unabhängig, aber $M=\{a, a\} =
            \{a\}$ ist nur dann linear abhängig wenn $a = \vec{0}$.
        \item
            $a_1, \ldots, a_n$ linear abhängig $\Rightarrow$ $a_1, \ldots a_n,
            b_1, \ldots, b_k$ sind linear abhängig für $k \geq 0$.
        \item $a_1, \ldots, a_n$ linear unabhängig $\Rightarrow a_1, \ldots, a_k$
            linear unabhängig für $k \leq n$
        \item $a_1, \ldots, a_n$ linear unabhängig $\Rightarrow a_1, \ldots a_n$
            sind paarweise verschieden
        \item Für $n \geq 2$ sind $a_1, \ldots, a_n$ genau dann linear abhängig,
            wenn ein Vektor als Linearkombination darstellbar ist. D.h.:
            \begin{equation*}
                \exists i \in \{1, \ldots, n\}: a_i = \sum_{k=1, k \neq i}^n
                \alpha_k a_k \text{ für } \alpha_1, \ldots, \alpha_{i-1},
                \alpha_{i+1}, \ldots, \alpha_n \in \K
            \end{equation*}
        \item Sind $a_1, \ldots, a_n$ linear unabhängig und $a_1, \ldots, a_n, a$
            linear abhängig, so ist $a$ die linear Kombination von $a_1, \ldots,
            a_n$ und die Koeffizienten sind eindeutig.
        \item Ist $a$ eine Linearkombination von $a_1, \ldots, a_n$ und jeder
            Vektor $a_k$ eine Linearkombination von $b_1, \ldots, b_m$ so ist
            $a$ eine Linearkombination von $b_1, \ldots, b_m$
\end{enumerate}

\subsubsection{Bemerkung}
Für Teilmengen $M, N$ eines Vektorraums $V$ gilt:
\begin{enumerate}[label= (\alph*)]
    \item $M$ l.a. $M \subseteq N \Rightarrow N$ l.a.
    \item $M=\emptyset \Rightarrow M$ l.u.
    \item $\vec{0} \in M \Rightarrow M$ l.a.
\end{enumerate}

\paragraph{Für $V = \R^3$}
\begin{enumerate}[label= (\alph*)]
    \item $a_1, a_2, a_3$ seien linear abhängig und $a_1, a_2$ linear unabhängig

        $\Leftrightarrow$ also $a_3$ ist in der von $a_1$ und $a_2$ aufgespannten
        Ebene

        $\Rightarrow$ Spat mit Kanten $a_1, a_2, a_3$ hat Volumen $0$

        $\Leftrightarrow \det(a_1, a_2, a_3) = 0$
    \item $a_1, a_2$ linear abhängig $\Rightarrow a_2$ ist auf der von $a_1$
        aufgespannten Gerade

        $\Rightarrow \det(a_1, a_2) = 0$
\end{enumerate}

\section{Basis und Dimension}
\subsection{Definition Hamel-Basis}
\begin{enumerate}[label= (\alph*)]
    \item Eine Teilmenge $B$ eines Vektorraums $V$ heißt (Hamel-) Basis von $V$,
        wenn gilt
        \begin{enumerate}[label= (\roman*)]
            \item $B$ ist linear unabhängig
            \item $V = \vspan(B)$
        \end{enumerate}
        Kurz:
        \begin{equation*}
            B \text{ ist ein linear unabhängiges Erzeuger-System
            von }V
        \end{equation*}
    \item Man sagt Vektoren $b_1, \ldots, b_n$ bilden eine Basis von $V$, wenn
        gilt $B=\{b_1, \ldots, b_n\}$ ist eine Basis von $V$.
\end{enumerate}
