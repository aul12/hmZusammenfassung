\section{Der n-dimensionale Euklidische Raum}

\subsection{Definitionen}
Sind $n, m \in \N$, so gelten folgende Bezeichungen:
\begin{eqnarray*}
    \R^n &:=& \Bigg\{
        \begin{pmatrix}
            x_1\\
            \vdots \\
            x_n
        \end{pmatrix}
        \text{ für }
        x_1, \ldots, x_n \in \R^n
    \Bigg\} \\
    \R^{m \times n} &:=& \Bigg\{
        \begin{pmatrix}
            a_{11} & \cdots & a_{1n}\\
            \vdots & \ddots\\
            a_{m1} & \ldots & a_{mn}
        \end{pmatrix}
        \text{ für }
        a_{ij} \in \R, 1 \leq i \leq m, 1 \leq j \leq n
    \Bigg\}\\
    <x, y> &:=& x \cdot y := x^T y := \sum_{k=1}^n x_k y_k \text{ (Skalarprodukt)}\\
    \norm{x} &:=& {\Vert x \Vert}_2 := \abs{x} := \sqrt{\sum_{k=1}^n x_k^2}
    \text{ euklidische Norm des }\R^n\text{ /Betrag in }\R^n
\end{eqnarray*}

\subsection{Folgerungen}
\begin{enumerate}
    \item
        \begin{equation*}
            \norm{x}_\infty = \max_{k=1\ldots n} \abs{x_k}
            \leq \norm{x}_2 \leq \sqrt{n} \max_{k=1 \ldots n} \abs{x_k}
            \ \forall x \in \R^n
        \end{equation*}
    \item
        \begin{equation*}
            \norm{x}_1 = \sum_{k=1}^n \abs{x_k}
        \end{equation*}
        und
        \begin{equation*}
            \norm{x}_2 \leq \norm{x}_1
        \end{equation*}
    \item $\norm{x}_1$, $\norm{x}_2$, $\norm{x}_\infty$ sind drei mögliche Festlegungen
        für Vektornormen. Allgemein hat eine Norm $\norm{\cdot}_2$
        ($\norm{\cdot}_2: \R^2 \rightarrow \R$) folgende Eigenschaften:
        \begin{eqnarray*}
            &\norm{x} \geq 0&\ \forall x \in \R^n \land \norm{x}=0 \Leftrightarrow
            x = \begin{pmatrix}
            0 \\ 0
            \end{pmatrix}
            = \vec{0} \\
            &\norm{\alpha \cdot x} = \abs{\alpha} \cdot \norm{x}&\ \forall \alpha
            \in \R\land \forall x \in \R^n \\
            &\norm{x + y} \leq \norm{x} + \norm{y}&\ \forall x, y \in \R^n
        \end{eqnarray*}
    \item Der Einheitskreis ist bezüglich verschiedener Normen nicht immer ein Kreis
    \item p-Norm:
        \begin{equation*}
            \norm{x}_p = \nthSqrt{p}{\sum_{k=1}^n \abs{x_k}^p}
        \end{equation*}
     \item $x \cdot y$ im $\R^2$ hat die anschauliche Bedeutung
        \begin{equation*}
            <x,y> = x \cdot y = \norm{x}_2 \cdot \norm{y}_2 \cdot \cos(\alpha)
        \end{equation*}
        Daraus folgt die Cauchy-Schwarzsche-Ungleichung (CSU)
        \begin{equation*}
            <x,y> \leq \norm{x}_2 \cdot \norm{y}_2
        \end{equation*}
\end{enumerate}

\subsection{Konventionen}
\begin{enumerate}[label= (\alph*)]
    \item In $\R^n$ sei stets $A^c := \R^n \ A$ für eine Menge $A \subseteq \R^n$
    \item Mit $\norm{\cdot}$ bezeichnen wir die euklidische Norm $\norm{\cdot}_2$.
    Außer es wird explizit gesagt, dass $\norm{\cdot}$ eine allgemeine Norm ist
    (z.B. \glqq{} Sei $\norm{\cdot}$ eine Norm   auf $\R^n$)
\end{enumerate}

\subsection{Definition Epsilon-Umgebung}
Sei $a \in \R^n, \varepsilon > 0$ dann heißt
\begin{eqnarray*}
    U_\varepsilon(a) &:=& \{ x \in \R^n |\ \norm{x-a} < \varepsilon \} \text{ die }
    \varepsilon \text{-Umgebung von }a\\
    \dot{U}_\varepsilon(a) &:=& U_\varepsilon(a) \backslash \{a \}
    ( =\{ x \in \R^n |\ 0<\norm{x-a}<\varepsilon \} ) \text{ die punktierte }
    \varepsilon \text{-Umgebung von }a
\end{eqnarray*}

\subsection{Definition Topologische Begriffe}
Sei $A \subseteq \R^n$. Ein Punkt $a \in \R^n$ heißt:
\begin{enumerate}[label= (\alph*)]
    \item Innerer Punt von $A$, falls ein $\varepsilon > 0$ existiert, sodass
        $U_\varepsilon(a) \subseteq A$
        Kurz:
        \begin{equation*}
            a \text{ innerer Punkt von }A :\Leftrightarrow \exists \varepsilon>0
            : U_\varepsilon(a) \subseteq A
        \end{equation*}

        Die Menge $\overset{\circ}{A}$ ist die Menge aller innerer Punkte von $A$
        \begin{equation*}
            \overset{\circ}{A} := \{ a \in \R^n | \exists \varepsilon > 0 \text{ mit }
            U_\varepsilon(a) \subseteq A \}
        \end{equation*}
    \item Berührungspunkt von $A$, wenn jede $\varepsilon$-Umgebung von $a$ mindestens
        einen Punkt aus $A$ enthält.
        Kurz:
        \begin{equation*}
            a \in \R^n \text{ ist Berührpunkt von } A :\Leftrightarrow
            \forall \varepsilon>0: U_\varepsilon(a) \cap A \neq \emptyset
        \end{equation*}

        Die Menge aller Berührpunkte von
        \begin{equation*}
            \bar{A} := \{ x \in \R^n | \forall \varepsilon > 0 \text{ ist }
            U_\varepsilon(a) \cap A \neq \emptyset \}
        \end{equation*}
        heißt der Abschluss oder abgeschlossene Hülle von $A$.
    \item Häufungspunkt von $A$, wenn jede punktierte $\varepsilon$-Umgebung von
        $a$ ein Element von $A$ enthält.
        Kurz:
        \begin{equation*}
            a \in \R^n \text{ ist Häufungspunkt } :\Leftrightarrow
            \forall\ \varepsilon > 0 \dot{U}_\varepsilon(a) \cap A \neq \emptyset
        \end{equation*}
    \item Randpunkt von $A$, wenn jede $\varepsilon$-Umgebung Elemente aus $A$
        und $A^c$ enthält.
        Kurz:
        \begin{equation*}
            a \in \R^n \text{ ist Randpunkt von }A :\Leftrightarrow
            \forall \varepsilon > 0\
            (U_\varepsilon(a) \hat A \neq \emptyset) \land
            (U_\varepsilon(a) \hat A^c \neq \emptyset)
        \end{equation*}
        Die Menge
        \begin{equation*}
            \partial A := \{ a \in \R^n | a \text{ ist Randpunkt von }A \}
        \end{equation*}
        heißt der Rand von $A$.
\end{enumerate}

\subsubsection{Bemerkung}
Man kann zeigen:
\begin{equation*}
    \bar{A} = A \cup \partial A =  \overset{\circ}{A} \cup \partial A
\end{equation*}
