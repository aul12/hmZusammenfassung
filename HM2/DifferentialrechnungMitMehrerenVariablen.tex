\section{Der n-dimensionale Euklidische Raum}

\subsection{Definitionen}
Sind $n, m \in \N$, so gelten folgende Bezeichungen:
\begin{eqnarray*}
    \R^n &:=& \Bigg\{
        \begin{pmatrix}
            x_1\\
            \vdots \\
            x_n
        \end{pmatrix}
        \text{ für }
        x_1, \ldots, x_n \in \R^n
    \Bigg\} \\
    \R^{m \times n} &:=& \Bigg\{
        \begin{pmatrix}
            a_{11} & \cdots & a_{1n}\\
            \vdots & \ddots \\
            a_{m1} & \cdots & a_{mn}
        \end{pmatrix}
        \text{ für }
        a_{ij} \in \R, 1 \leq i \leq m, 1 \leq j \leq n
    \Bigg\}\\
    <x, y> &:=& x \cdot y := x^T y := \sum_{k=1}^n x_k y_k \text{ (Skalarprodukt)}\\
    \norm{x} &:=& {\Vert x \Vert}_2 := \abs{x} := \sqrt{\sum_{k=1}^n x_k^2}
    \text{ euklidische Norm des }\R^n\text{ /Betrag in }\R^n
\end{eqnarray*}

\subsection{Folgerungen}
\begin{enumerate}
    \item
        \begin{equation*}
            \norm{x}_\infty = \max_{k=1\ldots n} \abs{x_k}
            \leq \norm{x}_2 \leq \sqrt{n} \max_{k=1 \ldots n} \abs{x_k}
            \ \forall x \in \R^n
        \end{equation*}
    \item
        \begin{equation*}
            \norm{x}_1 = \sum_{k=1}^n \abs{x_k}
        \end{equation*}
        und
        \begin{equation*}
            \norm{x}_2 \leq \norm{x}_1
        \end{equation*}
    \item $\norm{x}_1$, $\norm{x}_2$, $\norm{x}_\infty$ sind drei mögliche Festlegungen
        für Vektornormen. Allgemein hat eine Norm $\norm{\cdot}_2$
        ($\norm{\cdot}_2: \R^2 \rightarrow \R$) folgende Eigenschaften:
        \begin{eqnarray*}
            &\norm{x} \geq 0&\ \forall x \in \R^n \land \norm{x}=0 \Leftrightarrow
            x = \begin{pmatrix}
            0 \\ 0
            \end{pmatrix}
            = \vec{0} \\
            &\norm{\alpha \cdot x} = \abs{\alpha} \cdot \norm{x}&\ \forall \alpha
            \in \R\land \forall x \in \R^n \\
            &\norm{x + y} \leq \norm{x} + \norm{y}&\ \forall x, y \in \R^n
        \end{eqnarray*}
    \item Der Einheitskreis ist bezüglich verschiedener Normen nicht immer ein Kreis
    \item p-Norm:
        \begin{equation*}
            \norm{x}_p = \nthSqrt{p}{\sum_{k=1}^n \abs{x_k}^p}
        \end{equation*}
     \item $x \cdot y$ im $\R^2$ hat die anschauliche Bedeutung
        \begin{equation*}
            <x,y> = x \cdot y = \norm{x}_2 \cdot \norm{y}_2 \cdot \cos(\alpha)
        \end{equation*}
        Daraus folgt die Cauchy-Schwarzsche-Ungleichung (CSU)
        \begin{equation*}
            <x,y> \leq \norm{x}_2 \cdot \norm{y}_2
        \end{equation*}
\end{enumerate}

\subsection{Konventionen}
\begin{enumerate}[label= (\alph*)]
    \item In $\R^n$ sei stets $A^c := \R^n \ A$ für eine Menge $A \subseteq \R^n$
    \item Mit $\norm{\cdot}$ bezeichnen wir die euklidische Norm $\norm{\cdot}_2$.
    Außer es wird explizit gesagt, dass $\norm{\cdot}$ eine allgemeine Norm ist
    (z.B. \glqq{} Sei $\norm{\cdot}$ eine Norm   auf $\R^n$)
\end{enumerate}

\subsection{Definition Epsilon-Umgebung}
Sei $a \in \R^n, \varepsilon > 0$ dann heißt
\begin{eqnarray*}
    U_\varepsilon(a) &:=& \{ x \in \R^n |\ \norm{x-a} < \varepsilon \} \text{ die }
    \varepsilon \text{-Umgebung von }a\\
    \dot{U}_\varepsilon(a) &:=& U_\varepsilon(a) \backslash \{a \}
    ( =\{ x \in \R^n |\ 0<\norm{x-a}<\varepsilon \} ) \text{ die punktierte }
    \varepsilon \text{-Umgebung von }a
\end{eqnarray*}

\subsection{Definition Topologische Begriffe}
Sei $A \subseteq \R^n$. Ein Punkt $a \in \R^n$ heißt:
\begin{enumerate}[label= (\alph*)]
    \item Innerer Punt von $A$, falls ein $\varepsilon > 0$ existiert, sodass
        $U_\varepsilon(a) \subseteq A$
        Kurz:
        \begin{equation*}
            a \text{ innerer Punkt von }A :\Leftrightarrow \exists \varepsilon>0
            : U_\varepsilon(a) \subseteq A
        \end{equation*}

        Die Menge $\overset{\circ}{A}$ ist die Menge aller innerer Punkte von $A$
        \begin{equation*}
            \overset{\circ}{A} := \{ a \in \R^n | \exists \varepsilon > 0 \text{ mit }
            U_\varepsilon(a) \subseteq A \}
        \end{equation*}
    \item Berührungspunkt von $A$, wenn jede $\varepsilon$-Umgebung von $a$ mindestens
        einen Punkt aus $A$ enthält.
        Kurz:
        \begin{equation*}
            a \in \R^n \text{ ist Berührpunkt von } A :\Leftrightarrow
            \forall \varepsilon>0: U_\varepsilon(a) \cap A \neq \emptyset
        \end{equation*}

        Die Menge aller Berührpunkte von
        \begin{equation*}
            \bar{A} := \{ x \in \R^n | \forall \varepsilon > 0 \text{ ist }
            U_\varepsilon(a) \cap A \neq \emptyset \}
        \end{equation*}
        heißt der Abschluss oder abgeschlossene Hülle von $A$.
    \item Häufungspunkt von $A$, wenn jede punktierte $\varepsilon$-Umgebung von
        $a$ ein Element von $A$ enthält.
        Kurz:
        \begin{equation*}
            a \in \R^n \text{ ist Häufungspunkt } :\Leftrightarrow
            \forall\ \varepsilon > 0 \dot{U}_\varepsilon(a) \cap A \neq \emptyset
        \end{equation*}
    \item Randpunkt von $A$, wenn jede $\varepsilon$-Umgebung Elemente aus $A$
        und $A^c$ enthält.
        Kurz:
        \begin{equation*}
            a \in \R^n \text{ ist Randpunkt von }A :\Leftrightarrow
            \forall \varepsilon > 0\
            (U_\varepsilon(a) \hat A \neq \emptyset) \land
            (U_\varepsilon(a) \hat A^c \neq \emptyset)
        \end{equation*}
        Die Menge
        \begin{equation*}
            \partial A := \{ a \in \R^n | a \text{ ist Randpunkt von }A \}
        \end{equation*}
        heißt der Rand von $A$.
\end{enumerate}

\subsubsection{Bemerkung}
Man kann zeigen:
\begin{equation*}
    \bar{A} = A \cup \partial A =  \overset{\circ}{A} \cup \partial A
\end{equation*}

\subsection{Definition offene und abgeschlossene Menge}
Eine Menge $A \subseteq \R^n$ heißt:
\begin{enumerate}[label= (\alph*)]
    \item offen, wenn $A = \overset{\circ}{A}$ gilt (also $A$ besteht nur aus
        innerern Punkten)
    \item abgeschlossen, wenn $\partial A \subseteq A$ (Rand gehört zu $A$)
\end{enumerate}

\section{Folgen}
\subsection{Definition}
Eine Folge
\begin{equation*}
    a_k =
    \begin{pmatrix}
        a_1^{(k)} \\
        \vdots \\
        a_n^{(k)}
    \end{pmatrix}
\end{equation*}
in $\R^n$ heißt:
\begin{enumerate}[label= (\alph*)]
    \item Konvegent gegen einen Grenzwert $a$, wenn gilt:
        \begin{equation*}
            \forall \varepsilon > 0\ \exists n(\varepsilon): \norm{a_k - a}<\varepsilon
            \ \forall k \geq n(\varepsilon)
        \end{equation*}
        Schreibweise:
        \begin{equation*}
            \lim_{k \rightarrow \infty} a_k = a \text{ oder }
            a_k \rightarrow a\ (k \rightarrow \infty)
        \end{equation*}
    \item Beschränkt, wenn gilt:
        \begin{equation*}
            \exists c > 0:\ \norm{a_k} < c\ \forall k \in \N
        \end{equation*}
\end{enumerate}

\subsubsection{Bemerkung}
\begin{enumerate}[label= (\alph*)]
    \item Die Norm $\norm{\cdot}$ sei hier die euklidische Norm $\norm{\cdot}_2$. Wir
        werden aber sehen: Jede Norm auf $\R^n$ wäre ok.
    \item
        \begin{equation*}
            a_k \rightarrow a\ (k \rightarrow \infty) \Rightarrow
            \text{ Jede Komponente von }a_k\text{ konvergiert gegen entsprechende
            Komponente von }a
        \end{equation*}
    \item Cauchy-Kriterium:
        \begin{equation*}
            {(a_k)}_{k=1}^\infty \text{ konv. } \Leftrightarrow \forall \varepsilon > 0\ \exists
            n(\varepsilon): \norm{a_k - a_l} < \varepsilon
            \forall k,l \geq n(\varepsilon)
        \end{equation*}
\end{enumerate}

\subsection{Bolzano-Weierstraß}
Jede beschränkte Folge im $\R^n$ hat eine konvergente Teilfolge.

\subsection{Grenzwertrechenregeln}
Die Grenzwertrechenregeln übertragen sich auch auf Folgen im $\R^n$.

\subsection{Weitere Bemerkungen}
Sei $A \subseteq  \R^n$ und $a \in \R^n$, dann gilt:
\begin{enumerate}[label= (\alph*)]
    \item
        \begin{equation*}
            a \in \bar{A} \Leftrightarrow \exists {(a_k)}_{k=1}^\infty
            \text{ mit } a_k \in A \forall k \text{ mit } \lim_{k \rightarrow \infty}
            a_k = a
        \end{equation*}
    \item $a$ ist ein Häufungspunkt von $A$
        \begin{equation*}
            \exists {(a_k)}_{k=1}^\infty \text{ mit } a_k \in A \backslash \{a\}
            \text{ mit } \lim_{k \rightarrow \infty} a_k = a
        \end{equation*}
    \item $A$ ist abgeschlossen $\Leftrightarrow$ für jede konvergente Folge
        ${(a_k)}_{k=1}^\infty$ mit $a_k \in A \forall k$ gilt
        $\lim\limits_{k \rightarrow \infty} a_k \in A$.
    \item $A$ ist kompakt $\Leftrightarrow$ Jede Folge in $A$ besitzt einen
        Häufungspunt in $A$.
\end{enumerate}

\section{Funktionsgrenzwerte und Stetigkeit}

\subsection{Definition Funktion}
Eine Funktione $f: A \subseteq \R^n \rightarrow \R^m$ nennt man eine Funktion in
$n$ Veränderlichen (oder Vektorfeld).
Im Fall $m=1$ nennt man $f$ eine reele Funktion (oder Skalarfeld).

\subsubsection{Schreibweise}
\begin{equation*}
    f(x_1, x_2, \ldots, x_n) :=
    f\left(\begin{pmatrix}
        x_1\\
        \vdots \\
        x_n
    \end{pmatrix} \right) :=
    \begin{pmatrix}
        f_1(x_1, \ldots, x_n)\\
        \vdots \\
        f_n(x_1, \ldots, x_n)
    \end{pmatrix}
\end{equation*}

\subsection{Definition Funktionsgrenzwert}
Sei $f: A \subseteq \R^n \rightarrow \R^m$ und $a \in \bar{A}$ dann heißt ein
$b \in \R^m$ mit:
\begin{equation*}
    \forall \varepsilon>0\ \exists \delta(\varepsilon):
    \norm{f(x) - b}<\varepsilon\
    \forall x \in U_{\delta(\varepsilon)}(a) \cap A
\end{equation*}
der Grenzwert von $f$ für x gegen a. Kurz:
\begin{equation*}
    \lim_{x \rightarrow a} f(x) = b
\end{equation*}

\subsection{Definitionen aus HM 1 im Mehrdimensionalen}
Sei $f: A \subseteq \R^n \rightarrow \R^m$, $a \in \bar{A}$ und $b \in \R^m$
\begin{enumerate}[label= (\alph*)]
    \item Folgende Aussagen sind äquivalent:
        \begin{enumerate}
            \item $f(x) \rightarrow  b\ (x \rightarrow a)$
            \item $\norm{f(x) - b} \rightarrow 0\ (x \rightarrow a, x \in A)$
            \item Für jede Komponente
                \begin{equation*}
                    f_l(x) \text{ von } f(x) =
                    \begin{pmatrix}
                        f_1(x)\\
                        \vdots \\
                        f_m(x)
                    \end{pmatrix}
                    \text{ gilt }
                    f_l(x) \rightarrow b_l\ (x \rightarrow a)
                \end{equation*}
            \item Für eine Folge ${(x_k)}_{k=1}^\infty$ in A mit
                $\lim\limits_{k \rightarrow \infty}$ und $x_k \neq a\ \forall k$ folgt:
                \begin{equation*}
                        f(x_k) \rightarrow b\ (k \rightarrow \infty)
                \end{equation*}
        \end{enumerate}
    \item Falls $\lim\limits_{x \rightarrow a} f(x)$ existiert ist dieser
        Eindeutig.
    \item Cauchy-Kriterium:
        \begin{equation*}
            \forall \varepsilon > 0\ \exists \delta(\varepsilon):
            \norm{f(x) -f(y)}<\varepsilon\ \forall x,y \in
            \circ{U}_{\delta{\varepsilon}}(a) \cap A
        \end{equation*}
    \item Grenzwertrechenregeln gelten analog zu HM 1
    \item Sei $B \subseteq A$ mit $a \in \bar{B}$ dann gilt:
        \begin{equation*}
            \lim_{x \rightarrow a \text{ mit } x \in B} f(x) = b \Leftrightarrow
            \lim_{x \rightarrow a \text{ mit } x \in A} f(x) = b
        \end{equation*}
\end{enumerate}

\subsection{Definition Stetigkeit}
Sei $f: A \subseteq \R^n \rightarrow \R^m$ und $a \in A$, dann ist $f$ in $a$
stetig wenn gilt $\lim\limits_{x \rightarrow a} f(x) = f(a)$. Das heißt:
\begin{equation*}
    \forall\varepsilon\ \exists \delta(\varepsilon):
    \norm{f(x) - f(a)} < \varepsilon\ \forall x \in U_{\delta(\varepsilon)}(a)
    \cap A
\end{equation*}

\subsection{Grenzwerte von verketteten Funktionen}
Sei $A \subseteq \R^n, B \subseteq \R^m, a \in \bar{A}$ und $f: A \rightarrow B,
g: B \rightarrow \R^l$. Existiert $\lim\limits_{x \rightarrow a} f(x) = b$ so
gilt $b \in \bar{B}$ und es gilt:
\begin{equation*}
    \lim_{x \rightarrow a} g(f(x)) = \lim_{y \rightarrow b} g(y)
\end{equation*}
sofern der Grenzwert $\lim\limits_{y \rightarrow b} g(y)$ existiert.

\subsection{Grenzwertrechenregeln}
Für $f, g: A \rightarrow \R^n$ gilt: Falls $\lim\limits_{x \rightarrow a} f(x)
= \alpha$ und $\lim\limits_{x \rightarrow a} g(x) = \beta$ existiert, dann gilt:
\begin{enumerate}[label= (\alph*)]
    \item
        \begin{equation*}
            \lim_{x \rightarrow a} f(x) + g(x) = \alpha + \beta
        \end{equation*}
    \item
        \begin{equation*}
            \lim_{x \rightarrow a} {f(x)}^T g(x) = \alpha^T \beta
        \end{equation*}
\end{enumerate}
