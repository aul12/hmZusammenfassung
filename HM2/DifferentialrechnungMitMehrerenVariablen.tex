\section{Der n-dimensionale Euklidische Raum}

\subsection{Definitionen}
Sind $n, m \in \N$, so gelten folgende Bezeichungen:
\begin{eqnarray*}
    \R^n &:=& \Bigg\{
        \begin{pmatrix}
            x_1\\
            \vdots \\
            x_n
        \end{pmatrix}
        \text{ für }
        x_1, \ldots, x_n \in \R^n
    \Bigg\} \\
    \R^{m \times n} &:=& \Bigg\{
        \begin{pmatrix}
            a_{11} & \cdots & a_{1n}\\
            \vdots & \ddots\\
            a_{m1} & \ldots & a_{mn}
        \end{pmatrix}
        \text{ für }
        a_{ij} \in \R, 1 \leq i \leq m, 1 \leq j \leq n
    \Bigg\}\\
    <x, y> &:=& x \cdot y := x^T y := \sum_{k=1}^n x_k y_k \text{ (Skalarprodukt)}\\
    \norm{x} &:=& {\Vert x \Vert}_2 := \abs{x} := \sqrt{\sum_{k=1}^n x_k^2}
    \text{ euklidische Norm des }\R^n\text{ /Betrag in }\R^n
\end{eqnarray*}

\subsection{Folgerungen}
\begin{enumerate}
    \item
        \begin{equation*}
            \norm{x}_\infty = \max_{k=1\ldots n} \abs{x_k}
            \leq \norm{x}_2 \leq \sqrt{n} \max_{k=1 \ldots n} \abs{x_k}
            \ \forall x \in \R^n
        \end{equation*}
    \item
        \begin{equation*}
            \norm{x}_1 = \sum_{k=1}^n \abs{x_k}
        \end{equation*}
        und
        \begin{equation*}
            \norm{x}_2 \leq \norm{x}_1
        \end{equation*}
    \item $\norm{x}_1$, $\norm{x}_2$, $\norm{x}_\infty$ sind drei mögliche Festlegungen
        für Vektornormen. Allgemein hat eine Norm $\norm{\cdot}_2$
        ($\norm{\cdot}_2: \R^2 \rightarrow \R$) folgende Eigenschaften:
        \begin{eqnarray*}
            &\norm{x} \geq 0&\ \forall x \in \R^n \land \norm{x}=0 \Leftrightarrow
            x = \begin{pmatrix}
            0 \\ 0
            \end{pmatrix}
            = \vec{0} \\
            &\norm{\alpha \cdot x} = \abs{\alpha} \cdot \norm{x}&\ \forall \alpha
            \in \R\land \forall x \in \R^n \\
            &\norm{x + y} \leq \norm{x} + \norm{y}&\ \forall x, y \in \R^n
        \end{eqnarray*}
    \item Der Einheitskreis ist bezüglich verschiedener Normen nicht immer ein Kreis
    \item p-Norm:
        \begin{equation*}
            \norm{x}_p = \nthSqrt{p}{\sum_{k=1}^n \abs{x_k}^p}
        \end{equation*}
     \item $x \cdot y$ im $\R^2$ hat die anschauliche Bedeutung
        \begin{equation*}
            <x,y> = x \cdot y = \norm{x}_2 \cdot \norm{y}_2 \cdot \cos(\alpha)
        \end{equation*}
        Daraus folgt die Cauchy-Schwarzsche-Ungleichung (CSU)
        \begin{equation*}
            <x,y> \leq \norm{x}_2 \cdot \norm{y}_2
        \end{equation*}
\end{enumerate}

\subsection{Konventionen}
\begin{enumerate}[label= (\alph*)]
    \item In $\R^n$ sei stets $A^c := \R^n \ A$ für eine Menge $A \subseteq \R^n$
    \item Mit $\norm{\cdot}$ bezeichnen wir die euklidische Norm $\norm{\cdot}_2$.
    Außer es wird explizit gesagt, dass $\norm{\cdot}$ eine allgemeine Norm ist
    (z.B. \glqq{} Sei $\norm{\cdot}$ eine Norm auf $\R^n$)
\end{enumerate}
