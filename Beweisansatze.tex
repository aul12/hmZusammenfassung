\chapter{HM 1}
\section{Grenzwerte}
\subsection{Ableitung in Integral ziehen}
Betrachte
\begin{equation*}
    F'(x) = \lim_{h \to 0} \frac{F(x + h) + F(x)}{h}
\end{equation*}
sowie
\begin{equation*}
    F'(x) - \int_a^b f_x(x,t) \dt
\end{equation*}
$F'(x)$ einsetzten, Integral in Grenzwert einsetzten, MWS,
zeigen das Differenz im Grenzwert $0$.

\subsection{Eindeutigkeit des Grenzwert einer Folge }
 Zeige, dass Grenzwert a = Grenzwert b, nahrhafte 0
\subsection{Konvergente Folgen sind beschränkt }
 Nahrhafte 0, Dreiecks-ugl.
\subsection{Grenzwertrechenregeln }
 Nahrhafte 0, Dreiecks-ugl.
$a_n \leq \gamma\ \forall n \Rightarrow a \leq \gamma$
 Ausgehend von a über nahrh. 0 zu Def Konvergenz
$a_n \leq b_n\ \forall n \Rightarrow a\leq b$
 Definiere Hilfsfolge, argumentiere nach s.o
Sandwich-Theorem
 Zeige, dass $-\varepsilon < c_n < \varepsilon$  (Quasi Epsilon-Schlauch)
\subsection{Monotoniekriterium }
 Da $\abs{a_n} < c\ \forall n$, argumentiere über das Supremum der Menge, die aus $a_n$ besteht
\subsection{Grenzwert einer konv. Folge = Grenzwert jeder Teilfolge }
 Def. Konvergenz + Def Teilfolge
\subsection{Charakterisierung $\varlimsup$ und $\varliminf$ }
 Argumentiere über Eigneschaften sup und inf
\subsection{Folge konv.\ $\varlimsup = \varliminf$ }
 Hin: Eindeutigkeit des Grenzwert;\@Rück: Charakterisierung limSup und limInf
\subsection{Bolzano-Weierstraß }
 Zunächst für reelle Folge (trivial), dann für komplex: Realteil ist klar, Imaginärteil: Teilfolge konstruieren
\subsection{Cauchykriterium }
 Hin: nahrhafte 0; Rück: zeige Beschränktheit, dann folge daraus, dass ein Häufungswert existiert und benutze diesen als Grenzwert-Kandidat
\subsection{Reihe konv.  Folge ist Nullfolge }
 Cauchy für Reihen
\subsection{GrenzwertRR für Reihen }
 GrenzwertRR für Folgen
\subsection{Reihe konv g. 0 }
 Restreihe als Differenz darstellen
\subsection{Leibniz }
 Cauchy für Reihen
\subsection{Absolut konv.\ $\Rightarrow$  konv. }
 Cauchy und Dreiecks-ugl.
\subsection{Majorantenkriterium }
 Cauchy
\subsection{Minorantenkriterium }
 Kontradiktion von Majorantenkriterium
\subsection{Wurzelkriterium }
 Majorantenkrit: geom. Summe über $Q:=q+\varepsilon<1$, in $q$ das Wurzelkriteriumeinsetzen, Charakterisierung $\varlimsup$
\subsection{Quotientenkriterium }
 Majorantenkrit: setze in $q$ das Quotientenkriteriumein und Argumentation über $\varlimsup$
\subsection{Hadamard }
 Wurzelkriterium+ Fallunterscheidung für Sonderfälle
\subsection{Differenzieren / Integrieren von Potenzreihen}
 Wurzelkriterium
\subsection{Lemma zu sin, cos und exp }
 Cauchy-Produkt + Definitionen
\subsection{$e^z \neq 0$ und $e^{-z} = \frac{1}{e^{z}}$ }
 Inverses Element der Multiplikation
\subsection{Pythagoras }
 3.\ binomische Formel
\subsection{$e^x > 0\ \forall x \in \R$ }
 Betrachte $x \geq 0$, angeordneter Körper
\subsection{$1+x \leq e^x\ \forall x \in \R$ }
 Bernoulli
\subsection{$x<y \Rightarrow e^x < e^y$ }
 nahrhafte 0
\subsection{Folgenkriterium }
 Hin: Def. Folgenkonv.\ und dann Def Funktionsgrenzwert einsetzen; \@Rück: Wähle versch. $\delta$ und zeige Widerspruch
\subsection{Cauchy für Funktionen }
 Hin: Def. FunktionsGrenzwert + nahrhafte 0; \@Rück: Cauchy für Folgen
\subsection{Grenzwerte an Intervallgrenzen }
 Argumentiere über Supremum / Infimum
\subsection{Verknüpfungen stetiger Funktionen stetig }
 Folgenkriterium
\subsection{Potenzreihen sind innerhalb des Konvergenzradius stetig }
 Abschätzung: $\exists r>0 : \abs{x-x_0 \text{ bzw. }
 x_1} \leq r$, dann einfach $\abs{f(x)-f(x_1)}$ nach oben abschätzen
\subsection{Umgebung pos. Funktionswerte }
 Wähle $\varepsilon = \frac{f(x_0)}{2}$, Def. Stetigkeit
\subsection{Zwischenwertsatz }
 Definiere $x_0 := \sup \{x \in [a,b] : f(x) \leq y \}$ und zwei Hilfsfolgen, die gegen $x_0$ konvergieren
\subsection{Existenz $\log$ }
 Zeigen $\exp$ ist bijektiv (Zwischenwertsatz)
\subsection{Beschränktheit stetiger Funktionen}
 Annahme $f$ nicht beschränkt Folgenkriterium
\subsection{Weierstraß existenz min bzw.\ max }
 Zeigen das $\sup=\max$

 \chapter{HM 2}
 \section{Integration in mehreren Veränderlichen}
 \subsection{Fubini}
 Hilfsfunktion:
 \begin{equation*}
     g(u) = \int_\alpha^\beta \int_a^b f(x,t) \dt \dx - \int_a^b \int_\alpha^\beta
     f(x,t) \dx \dt
 \end{equation*}
 zeigen dass $g'(u) \equiv 0$ und $g(x) = g(a) = 0$.

 \subsection{Leibniz Regel}
 Hilfsfunktion:
 \begin{equation*}
     G(x,a,b) = \int_a^b f(x,t) \dt
 \end{equation*}
 $\nabla G$ berechenen, innere Ableitung.

 \subsection{Beweis-Idee Kurvenintegrale (Subsitutionsregel)}
 Riemann Summe, Mittelwertsatz, Abschätzung für verschiedene $\xi$, da $f$ stetig.

 \subsection{1. Hauptsatz für Kurvenintegrale}
 Kurvenintegral mit Parametrisierung, integrant als Ableitung darstellen.

 \subsection{Äquivalente Aussagen für Kurvenintegrale}
 Kurven kombinieren/aufteilen um aus mehreren Kurven eine geschlossene bzw.\ aus
 einer geschlossenen Kurven mehrer mit gleichem Anfangs-/Endpunkt zu erzeugen.

 \subsection{2. Hauptsatz für Kurvenintegrale}
 \begin{enumerate}
     \item $f$ stetig, $F \in C^2$, Satz von Schwarz
     \item Nur für Sternförmiges Gebiet.

        $F$ als Integral von $x_0$ (Mittelpunkt von Sternförmigem Gebiet) zu $x$
        darstellen und Weg Parametrisieren.

        Ableitung von $F$ nach $x_k$ berechnen, Skalarprodukt als Summe schreiben,
        Produktregel, Integrabilitätsbedinung anwenden, als Ableitung nach $t$
        darstellen.
 \end{enumerate}

 \subsection{Gauß'sche Integralsätze in der Ebene}
 \begin{enumerate}
     \item Hilfsfunktion:
        \begin{equation*}
            h(x,y) =
            \begin{pmatrix}
                -f_2(x,y)\\
                f_1(x,y)
            \end{pmatrix}
        \end{equation*}
        zeigen dass $\iint \vdiv h = - \iint \rot f$, Stokes anwenden, $f$ durch
        $h$ darstellen, Normalenvektor normieren, Linienintegral.
    \item Hilfsfunktion:
        \begin{equation*}
            h(x,y) = f_1(x,y) \nabla f_2(x,y) - f_2(x,y) \nabla f_1(x,y)
        \end{equation*}
        $\vdiv h$ und $h \nu$ ausrechnen und Gleichheit über ersten Teil von Gauß.
 \end{enumerate}
