\chapter{Grenzwerte}
\section{Konvergenzkriterien}
Zusammenfassung verschiedener Konvergenzkriterien nach Wikipedia (Seite: Konvergenzkriterium):
\begin{center}
    \begin{tabular}{lcccccccp{2cm}}
         \toprule
         Kriterium & {nur f.\ mon. F.} & Konv. & Div. & abs. Konv. & Absch. & Fehlerabsch.\\
         \midrule
         Nullfolgenkriterium &  &  & X &  &  & \\
         Monotoniekriterium &  & X &  & X &  & \\
         Leibniz-Kriterium & X & X &  &  & X & X\\
         Cauchy-Kriterium &  & X & X &  &  & \\
         Abel-Kriterium & X & X &  &  &  & \\
         Dirichlet-Kriterium & X & X &  &  &  & \\
         Majorantenkriterium &  & X &  & X &  & \\
         Minorantenkriterium &  &  & X &  &  & \\
         Wurzelkriterium &  & X & X & X &  & X\\
         Integralkriterium & X & X & X & X & X & \\
         Cauchy-Kriterium & X & X & X & X &  & \\
         Grenzwertkriterium &  & X & X &  &  & \\
         Quotientenkriterium &  & X & X & X &  & X\\
         Gauß-Kriterium &  & X & X & X &  & \\
         Raabe-Kriterium &  & X & X & X &  & \\
         Kummer-Kriterium &  & X & X & X &  & \\
         Bertrand-Kriterium &  & X & X & X &  & \\
         Ermakoff-Kriterium & X & X & X & X &  & \\
         \bottomrule
    \end{tabular}
\end{center}

\chapter{Integration}
\section{Riemann-Integrierbarkeit}
\begin{center}
    \begin{tabular}{lcc}
        \toprule
        Kriterium & Integrierbar & Nicht Integrierbar \\
        \midrule
        Funktion nicht beschränkt & & X \\
        Verknüpfung Riemann-Integrierbarer Funktionen & X \\
        Stetige Funktion & X\\
        Endliche vielen Änderungen zu Riemann-Int.barer Funktion & X\\
        Monotone Funktion & X\\
        \bottomrule
    \end{tabular}
\end{center}

\chapter{Integration in mehreren Veränderlichen}
\section{Häufige Additionstheoreme}
\begin{eqnarray*}
    \sin^2(t) &=& \frac{1}{2} (1 - \cos(2t))\\
    \cos^2(t) &=& \frac{1}{2} (1 + \cos(2t))\\
    \sin(t) \cos(t) &=& \frac{1}{2} \sin(2t)
\end{eqnarray*}
