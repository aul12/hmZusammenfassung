\chapter{Grenzwerte}
\section{Konvergenzkriterien}
Zusammenfassung verschiedener Konvergenzkriterien nach Wikipedia (Seite: Konvergenzkriterium):
\begin{center}
    \begin{tabular}{lcccccccp{2cm}}
         \toprule
         Kriterium & {nur f.\ mon. F.} & Konv. & Div. & abs. Konv. & Absch. & Fehlerabsch.\\
         \midrule
         Nullfolgenkriterium &  &  & x &  &  & \\
         Monotoniekriterium &  & x &  & x &  & \\
         Leibniz-Kriterium & x & x &  &  & x & x\\
         Cauchy-Kriterium &  & x & x &  &  & \\
         Abel-Kriterium & x & x &  &  &  & \\
         Dirichlet-Kriterium & x & x &  &  &  & \\
         Majorantenkriterium &  & x &  & x &  & \\
         Minorantenkriterium &  &  & x &  &  & \\
         Wurzelkriterium &  & x & x & x &  & x\\
         Integralkriterium & x & x & x & x & x & \\
         Cauchy-Kriterium & x & x & x & x &  & \\
         Grenzwertkriterium &  & x & x &  &  & \\
         Quotientenkriterium &  & x & x & x &  & x\\
         Gauß-Kriterium &  & x & x & x &  & \\
         Raabe-Kriterium &  & x & x & x &  & \\
         Kummer-Kriterium &  & x & x & x &  & \\
         Bertrand-Kriterium &  & x & x & x &  & \\
         Ermakoff-Kriterium & x & x & x & x &  & \\
         \bottomrule
    \end{tabular}
\end{center}

\pagebreak
