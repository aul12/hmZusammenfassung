\chapter{Grenzwerte}
\section{Konvergenzkriterien}
Zusammenfassung verschiedener Konvergenzkriterien nach Wikipedia (Seite: Konvergenzkriterium):
\begin{center}
    \begin{tabular}{lcccccccp{2cm}}
         \toprule
         Kriterium & {nur f.\ mon. F.} & Konv. & Div. & abs. Konv. & Absch. & Fehlerabsch.\\
         \midrule
         Nullfolgenkriterium &  &  & x &  &  & \\
         Monotoniekriterium &  & x &  & x &  & \\
         Leibniz-Kriterium & x & x &  &  & x & x\\
         Cauchy-Kriterium &  & x & x &  &  & \\
         Abel-Kriterium & x & x &  &  &  & \\
         Dirichlet-Kriterium & x & x &  &  &  & \\
         Majorantenkriterium &  & x &  & x &  & \\
         Minorantenkriterium &  &  & x &  &  & \\
         Wurzelkriterium &  & x & x & x &  & x\\
         Integralkriterium & x & x & x & x & x & \\
         Cauchy-Kriterium & x & x & x & x &  & \\
         Grenzwertkriterium &  & x & x &  &  & \\
         Quotientenkriterium &  & x & x & x &  & x\\
         Gauß-Kriterium &  & x & x & x &  & \\
         Raabe-Kriterium &  & x & x & x &  & \\
         Kummer-Kriterium &  & x & x & x &  & \\
         Bertrand-Kriterium &  & x & x & x &  & \\
         Ermakoff-Kriterium & x & x & x & x &  & \\
         \bottomrule
    \end{tabular}
\end{center}

\pagebreak
\section{Beweis-Ansätze}
\begin{center}
    \begin{longtable}{lp{6cm}}
        \toprule
            Lemma / Satz & Beweisansatz\\
        \midrule
            Eindeutigkeit des GW einer Folge & Zeige, dass GW a = GW b, nahrhafte 0\\
            Konvergente Folgen sind beschränkt & Nahrhafte 0, Dreiecks-ugl.\\
            Grenzwertrechenregeln & Nahrhafte 0, Dreiecks-ugl. \\
            $a_n \leq \gamma\ \forall n \Rightarrow a \leq \gamma$  & Ausgehend von a über nahrh. 0 zu Def Konvergenz \\
            $a_n \leq b_n\ \forall n \Rightarrow a\leq b$ & Definiere Hilfsfolge, argumentiere nach s.o \\
            Sandwich-Theorem & Zeige, dass $-\varepsilon < c_n < \varepsilon$  (Quasi Epsilon-Schlauch) \\
            Monotoniekriterium & Da $\abs{a_n} < c\ \forall n$, argumentiere über das Supremum der Menge, die aus $a_n$ besteht \\
            GW einer konv. Folge = GW jeder Teilfolge & Def. Konvergenz + Def Teilfolge \\
            Charakterisierung limSup und limInf & Argumentiere über Eigneschaften sup und inf \\
            Folge konv.\ $\varlimsup = \varliminf$ & Hin: Eindeutigkeit des GW;\@Rück: Charakterisierung limSup und limInf \\
            Bolzano-Weierstraß & Zunächst für reelle Folge (trivial), dann für komplex: Realteil ist klar, Imaginärteil: Teilfolge konstruieren \\
            Cauchykriterium & Hin: nahrhafte 0; Rück: zeige Beschränktheit, dann folge daraus, dass ein HW ex und benutze diesen als GW-Kandidat \\
            Reihe konv.  Folge ist Nullfolge & Cauchy für Reihen \\
            GWRR für Reihen & GWRR für Folgen \\
            Reihe konv g. 0 & Restreihe als Differenz darstellen \\
            Leibniz & Cauchy für Reihen \\
            Absolut konv.\ $\Rightarrow$  konv. & Cauchy und Dreiecks-ugl.\\
            Majorantenkrit. & Cauchy\\
            Minorantenkrit. & Kontradiktion von Majorantenkrit.\\
            Wurzelkriterium & Majorantenkrit: geom. Summe über $Q:=q+\varepsilon<1$, in $q$ das Wurzelkrit einsetzen, Char. LimSup\\
            Quotientenkrit. & Majorantenkrit: setze in $q$ das Quotientenkrit ein u.\ arg. über LimSup \\
            Hadamard & Wurzelkrit + Fallunterscheidung für Sonderfälle\\
            Differenzieren / Integrieren von PR & Wurzelkriterium\\
            Lemma zu sin, cos und exp & Cauchy-Produkt + Definitionen\\
            $e^z \neq 0$ und $e^{-z} = \frac{1}{e^{z}}$ & Inverses Element der Multiplikation\\
            Pythagoras & 3.\ binomische Formel\\
            $e^x > 0\ \forall x \in \R$ & Betrachte $x \geq 0$, angeordneter Körper\\
            $1+x \leq e^x\ \forall x \in \R$ & Bernoulli\\
            $x<y \Rightarrow e^x < e^y$ & nahrhafte 0\\
            Folgenkriterium & Hin: Def. Folgenkonv.\ und dann Def FunktionsGW einsetzen; \@Rück: Wähle versch. $\delta$ und zeige Widerspruch\\
            Cauchy für Funktionen & Hin: Def. FunktionsGW + nahrhafte 0; \@Rück: Cauchy für Folgen\\
            Grenzwerte an Intervallgrenzen & Argumentiere über Supremum / Infimum\\
            Verknüpfungen stetiger Fnkt.\ stetig & Folgenkriterium\\
            Potenzreihen sind innerhalb des KR stetig & Abschätzung: $\exists r>0 : \abs{x-x_0 \text{ bzw. } x_1} \leq r$, dann einfach $\abs{f(x)-f(x_1)}$ nach oben abschätzen\\
            Umgebung pos. Funktionswerte & Wähle $\varepsilon = \frac{f(x_0)}{2}$, Def. Stetigkeit\\
            Zwischenwertsatz & Definiere $x_0 := \sup \{x \in [a,b] : f(x) \leq y \}$ und zwei Hilfsfolgen, die gegen $x_0$ konvergieren\\
            Existenz $\log$ & Zeigen $\exp$ ist bijektiv (Zwischenwertsatz)\\
            Beschr.\ stet.\ Fkt.\ & Annahme $f$ nicht beschr. Folgenkriterium\\
            Weierstraß ex.\ min bzw.\ max & Zeigen das $\sup=\max$\\
        \bottomrule
    \end{longtable}
\end{center}
