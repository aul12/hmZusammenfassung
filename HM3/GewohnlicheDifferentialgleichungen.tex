%!TEX root = ../main.tex
\section{Einführung}
\subsection{Explizite DGL 1. Ordnung}
Gegeben sei ein Gebiet $G \subseteq \R^2$ und eine stetige Funktion $f: G \rightarrow \R$, dann heißt
eine Funktion $y: I \rightarrow \R$ (mit $I \subseteq \R$ ein Intervall) eine Lösung der (expliziten) Differentialgleichung (DGL)
\begin{equation*}
	y' = f(x, y)
\end{equation*}
wenn gilt:
\begin{enumerate}[label= (\alph*)]
	\item 
		\begin{equation*}
			\begin{pmatrix}
				x \\ y(x)
			\end{pmatrix}
			\in G\ \forall x \in I
		\end{equation*}
	\item 
		\begin{equation*}
			y'(x) = f(x,y(x))\ \forall x \in I		
		\end{equation*}
\end{enumerate}
Ferner heißt $y$ eine Lösung des Anfangswertproblems
\begin{equation*}
	(\text{AWP})  =
	\begin{cases}
		y' &= f(x,y) \\
		y(x_0) &= y_0
	\end{cases}
\end{equation*}
wenn $y$ eine Lösung von $y'=f(x,y)$ ist und $y(x_0) = y_0$ gilt.

\subsection{Implizite DGL 1. Ordnung}
\begin{enumerate}[label= (\alph*)]
	\item Eine DGL der Form
		\begin{equation*}
			g(x,y,y') = 0
		\end{equation*}
		mit $g: M \subseteq \R^3 \rightarrow \R$ heißt eine implizite DGL 1. Ordnung und
		\begin{equation*}
			(\text{AWP}) =
			\begin{cases}
				g(x,y,y') &= 0 \\
				y(x_0) &= y_0
			\end{cases}
		\end{equation*}
		ist das dazugehörige AWP.
	\item Das lässt sich (manchmal) auf eine explizite DGL zurückführen:
		\begin{equation*}
			g(x,y,z) = 0
		\end{equation*}
		hat bei $(x_0, y_0)$ eine Auflösung nach $z=f(x,y)$ falls die Determinate der Jakobi-Matrix
		$J = \frac{\partial}{\partial z} g(x,y,z)$ nicht null ist.
\end{enumerate}

\subsection{DGL-System 1. Ordnung}
Sei $G \subseteq \R^{n+1}$ und für $k\in\{1, \ldots, n\}$ sei $f_k: G \rightarrow \R$ stetig.
$y: I \rightarrow \R^n$ mit $y(x) = {(y_1(x), \ldots, y_n(x))}^T$ heißt Lösung des Systems
\begin{equation*}
	\begin{cases}
		y_1' &= f_1(x, y_1, \ldots, y_n) \\
		\vdots \\
		y_n' &= f_n(x, y_1, \ldots, y_n)
	\end{cases}
\end{equation*}
kurz:
\begin{equation*}
	y' = f(x,y)
\end{equation*}
von Differentialgleichungen 1. Ordnung, wenn gilt:
\begin{enumerate}
	\item 
		\begin{equation*}
			\begin{pmatrix}
				x \\ y(x)
			\end{pmatrix}
			=
			\begin{pmatrix}
				x \\ y_1() \\ \vdots \\ y_n(x)
			\end{pmatrix}
			\in G\ \forall x \in I
		\end{equation*}
	\item $y_k$ ist stetig differentierbar auf $I$
	\item $y_k'(x) = f_k(x, y_1(x), \ldots, y_n(x)) = f_k(x, y(x))$
\end{enumerate}
Außerdem heißt $y$ eine Lösung des zugehörigen AWP
\begin{equation*}
	(\text{AWP}) = 
	\begin{cases}
		y' = f(x,y) \\
		y(x_0) = 
		\begin{pmatrix}
			y_1(x_0) \\ \vdots \\ y_n(x_0)
		\end{pmatrix} = 
		\begin{pmatrix}
			\tilde{y_1} \\ \vdots \\ \tilde{y_n}
		\end{pmatrix} = 
		\tilde{y} = y_0
	\end{cases}
\end{equation*}
wenn gilt: $y$ ist eine Lösung von $y' = f(x,y)$ und $y(x_0) = y_0$.

\subsection{Umschreiben von DGLen}
Eine sogenannte DGL $n$-ter Ordnung
\begin{equation*}
	y^{(n+1)} = f(x, y, y', \ldots, y^{(n+1)})
\end{equation*}
kann man stets in ein System 1. Ordnung umschreiben.

\begin{enumerate}
	\item $y_1, \ldots, y_n$ definieren als:
		\begin{eqnarray*}
			y_1 &:=& y\\
			y_2 &:=& y_1'\\
			y_n &:=& y_{n-1}'
		\end{eqnarray*}
	\item Das ursprüngliche Problem einsetzten:
		\begin{equation*}
			y_n' = y^{(n)} = f(x,y, y', \ldots, y^{(n-1)}) =
			f(x, y_1, \ldots, y_n)
		\end{equation*}
	\item Als System formulieren:
		\begin{equation*}
			{\begin{pmatrix}
				y_1 \\ \vdots \\ y_n
			\end{pmatrix}}'
			=
			\begin{pmatrix}
				f_1(x, y_1, \ldots, y_n) \\
				\vdots \\
				f_n(x, y_1, \ldots, y_n) \\
			\end{pmatrix}
			=
			\begin{pmatrix}
				y_2 \\ y_3 \\ \vdots \\ f(x, y_1, \ldots, y_n)
			\end{pmatrix}
		\end{equation*}
	\item Die Anfangsbedinung formulieren:
		\begin{eqnarray*}
			y(x_0) &=& \tilde{y_1} \\
			y'(x_0) &=& \tilde{y_2} \\
			& \vdots & \\
			y^{(n-1)}(x_0) &=& \tilde{y_n}
		\end{eqnarray*}
		ergibt den Vektor
		\begin{equation*}
			\tilde{y} := 
			\begin{pmatrix}
				\tilde{y_1} \\ \vdots \\ \tilde{y_n}
			\end{pmatrix}
		\end{equation*}
\end{enumerate}