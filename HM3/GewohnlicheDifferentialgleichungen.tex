%!TEX root = ../main.tex
\section{Einführung}
\subsection{Explizite DGL 1. Ordnung}
Gegeben sei ein Gebiet $G \subseteq \R^2$ und eine stetige Funktion $f: G \rightarrow \R$, dann heißt
eine Funktion $y: I \rightarrow \R$ (mit $I \subseteq \R$ ein Intervall) eine Lösung der (expliziten) Differentialgleichung (DGL)
\begin{equation*}
	y' = f(x, y)
\end{equation*}
wenn gilt:
\begin{enumerate}[label= (\alph*)]
	\item 
		\begin{equation*}
			\begin{pmatrix}
				x \\ y(x)
			\end{pmatrix}
			\in G\ \forall x \in I
		\end{equation*}
	\item 
		\begin{equation*}
			y'(x) = f(x,y(x))\ \forall x \in I		
		\end{equation*}
\end{enumerate}
Ferner heißt $y$ eine Lösung des Anfangswertproblems
\begin{equation*}
	(\text{AWP})  =
	\begin{cases}
		y' &= f(x,y) \\
		y(x_0) &= y_0
	\end{cases}
\end{equation*}
wenn $y$ eine Lösung von $y'=f(x,y)$ ist und $y(x_0) = y_0$ gilt.

\subsection{Implizite DGL 1. Ordnung}
\begin{enumerate}[label= (\alph*)]
	\item Eine DGL der Form
		\begin{equation*}
			g(x,y,y') = 0
		\end{equation*}
		mit $g: M \subseteq \R^3 \rightarrow \R$ heißt eine implizite DGL 1. Ordnung und
		\begin{equation*}
			(\text{AWP}) =
			\begin{cases}
				g(x,y,y') &= 0 \\
				y(x_0) &= y_0
			\end{cases}
		\end{equation*}
		ist das dazugehörige AWP.
	\item Das lässt sich (manchmal) auf eine explizite DGL zurückführen:
		\begin{equation*}
			g(x,y,z) = 0
		\end{equation*}
		hat bei $(x_0, y_0)$ eine Auflösung nach $z=f(x,y)$ falls die Determinate der Jakobi-Matrix
		$J = \frac{\partial}{\partial z} g(x,y,z)$ nicht null ist.
\end{enumerate}

\subsection{DGL-System 1. Ordnung}
Sei $G \subseteq \R^{n+1}$ und für $k\in\{1, \ldots, n\}$ sei $f_k: G \rightarrow \R$ stetig.
$y: I \rightarrow \R^n$ mit $y(x) = {(y_1(x), \ldots, y_n(x))}^T$ heißt Lösung des Systems
\begin{equation*}
	\begin{cases}
		y_1' &= f_1(x, y_1, \ldots, y_n) \\
		\vdots \\
		y_n' &= f_n(x, y_1, \ldots, y_n)
	\end{cases}
\end{equation*}
kurz:
\begin{equation*}
	y' = f(x,y)
\end{equation*}
von Differentialgleichungen 1. Ordnung, wenn gilt:
\begin{enumerate}
	\item 
		\begin{equation*}
			\begin{pmatrix}
				x \\ y(x)
			\end{pmatrix}
			=
			\begin{pmatrix}
				x \\ y_1() \\ \vdots \\ y_n(x)
			\end{pmatrix}
			\in G\ \forall x \in I
		\end{equation*}
	\item $y_k$ ist stetig differentierbar auf $I$
	\item $y_k'(x) = f_k(x, y_1(x), \ldots, y_n(x)) = f_k(x, y(x))$
\end{enumerate}
Außerdem heißt $y$ eine Lösung des zugehörigen AWP
\begin{equation*}
	(\text{AWP}) = 
	\begin{cases}
		y' = f(x,y) \\
		y(x_0) = 
		\begin{pmatrix}
			y_1(x_0) \\ \vdots \\ y_n(x_0)
		\end{pmatrix} = 
		\begin{pmatrix}
			\tilde{y_1} \\ \vdots \\ \tilde{y_n}
		\end{pmatrix} = 
		\tilde{y} = y_0
	\end{cases}
\end{equation*}
wenn gilt: $y$ ist eine Lösung von $y' = f(x,y)$ und $y(x_0) = y_0$.

\subsection{Umschreiben von DGLen}
Eine sogenannte DGL $n$-ter Ordnung
\begin{equation*}
	y^{(n+1)} = f(x, y, y', \ldots, y^{(n+1)})
\end{equation*}
kann man stets in ein System 1. Ordnung umschreiben.

\begin{enumerate}
	\item $y_1, \ldots, y_n$ definieren als:
		\begin{eqnarray*}
			y_1 &:=& y\\
			y_2 &:=& y_1'\\
			y_n &:=& y_{n-1}'
		\end{eqnarray*}
	\item Das ursprüngliche Problem einsetzten:
		\begin{equation*}
			y_n' = y^{(n)} = f(x,y, y', \ldots, y^{(n-1)}) =
			f(x, y_1, \ldots, y_n)
		\end{equation*}
	\item Als System formulieren:
		\begin{equation*}
			{\begin{pmatrix}
				y_1 \\ \vdots \\ y_n
			\end{pmatrix}}'
			=
			\begin{pmatrix}
				f_1(x, y_1, \ldots, y_n) \\
				\vdots \\
				f_n(x, y_1, \ldots, y_n) \\
			\end{pmatrix}
			=
			\begin{pmatrix}
				y_2 \\ y_3 \\ \vdots \\ f(x, y_1, \ldots, y_n)
			\end{pmatrix}
		\end{equation*}
	\item Die Anfangsbedinung formulieren:
		\begin{eqnarray*}
			y(x_0) &=& \tilde{y_1} \\
			y'(x_0) &=& \tilde{y_2} \\
			& \vdots & \\
			y^{(n-1)}(x_0) &=& \tilde{y_n}
		\end{eqnarray*}
		ergibt den Vektor
		\begin{equation*}
			\tilde{y} := 
			\begin{pmatrix}
				\tilde{y_1} \\ \vdots \\ \tilde{y_n}
			\end{pmatrix}
		\end{equation*}
\end{enumerate}

\section{Existenz- und Eindeutigkeitsaussagen}
\subsection{Voltera'sche Integralgleichung}
Sei $G \subseteq \R^2$ ein Gebiet und $f:G \rightarrow \R$ stetig, $I \subseteq \R$ ein Intervall und
$x_0 \in I$. Dann gilt:
\begin{enumerate}
	\item $y: I \rightarrow \R$ löst das AWP.
		\begin{equation*}
			y' = f(x,y),\ y(x_0) = y_0
		\end{equation*}
		genau dann, wenn die Integralgleichung
	\item
		\begin{equation*}
			y(x) = y(x_0) \int_{x_0}^x f(t, y(t)) \dt\ \forall x \in I
		\end{equation*}
		für alle $y$ erfüllt ist.
\end{enumerate}

Von der Form $y=\Phi(y)$ mit
\begin{equation*}
	\Phi(y) := y_0 + \int_{x_0}^x f(t, y(t)) \dt
\end{equation*}
genauer: $\Phi: \operatorname{C}^1(I) \rightarrow \operatorname{C}^1(I)$ also $y$ muss Fixpunkt von $\Phi$ sein.

\subsection{Bemerkung: Newton-Verfahren zur Bestimmung von Nullstellen}
Sei $f: \R \rightarrow \R$. Tangente bestimmen:
\begin{eqnarray*}
	T(x) &=& f(x_0) + f'(x_0) (x-x_0) = 0 \\
	\Rightarrow x_1 &=& x_0 - \frac{f(x_0)}{f'(x_0)} = \Phi(x_0) \\
	x_{n+1} &=& x_n - \frac{f(x_n)}{f'(x_n)} = \Phi(x_n)
\end{eqnarray*}
Hoffnung die so erzeugte Folge ${(x_n)}_{n=1}^\infty$ konvertiert gegen ein $\overline{x}$, 
das heißt es gilt (falls $\Phi$ stetig):
\begin{equation*}
	\overline{x} = \lim_{n \to \infty} x{n+1} = \lim_{n \to \infty} \Phi(x_n) = \Phi(\lim_{n \to \infty} x_n) = \Phi(\overline{x})
\end{equation*}
Das heißt, falls alles gut geht ist $\overline{x}$ der Fixpunkt von $\Phi$ 
(muss aber nicht immer so sein, z.B $f(x)=\sin(x)$).

\subsection{Picard-Iterator}
Der sogenannte Picard-Iterator bildet Funktionen auf Funktionen ab und ist definiert durch:
\begin{equation*}
	\Phi: y \rightarrow y_0 + \int_{x_0}^x f(t, y(t)) \dt
\end{equation*}

\subsection{Definition Lipschitz-Bedingung}
Sei $G \subseteq \R^2$ ein Gebiet. Eine Funktion $f: G \rightarrow \R$ erfüllt eine Lipschitz-Bedingung
(L-Bedingung) bezüglich der zweiten Komponente wenn gilt:
\begin{equation*}
	\exists L > 0:\ \abs{f(x,y_1) - f(x, y_2)} < L \cdot \abs{y_1 - y_2}
\end{equation*}

\subsubsection{Bemerkung}
Für $x$ fest, ist die L-Bedingung äquivalent zur Lipschitz-Stetigkeit.

\subsection{Partielle Ableitung und Lipschitz-Bedingung}
Ist bei $f$ die partielle Ableitung nach der zweiten Komponente beschränkt, dann ist die L-Bedingung erfüllt:
\begin{equation*}
	\abs{f_y(x,y)} < L\ \forall
	\begin{pmatrix}
		x \\ y
	\end{pmatrix} \in G \Rightarrow \abs{f(x, y_1) - f(x,y_2)} = \abs{f_y(x, \xi)(y_1-y_2)} \leq L \abs{y_1 - y_2}
\end{equation*}

\subsection{Picard-Lindelöf (Existenz und Eindeutigkeit einer Lösung)}
Seien $r, s > 0, x_0; y_0 \in \R; M := [x_0, x_0 + r] \times [y_0-s, y_0+s]$ sei $G$ ein Gebiet mit
$M \subseteq G \subseteq \R^2$. Außerdem erfülle $f$ auf $G$ eine L-Bedingung bezüglich der zweiten 
Komponente mit $L>0$. Außerdem gelte $\abs{f(x,y)} \leq c \ \forall (x,y)^T \in M$. Dann besitzt das AWP
\begin{equation*}
	(\text{AWP}) = \begin{cases}
		y' = f(x,y) \\ y(x_0) = y_0
	\end{cases}
\end{equation*}
auf dem Intervall $I := [x_0, x_0 + \alpha]$ mit $\alpha = \min \{r, \frac{s}{c} \}$ genau eine Lösung.

\subsection{Peano (Existenz einer Lösung)}
Gelten alle Voraussetzungen von Picard-Lindelöf bis auf die L-Bedingung, dann besitzt
\begin{equation*}
	(\text{AWP}) = \begin{cases}
		y' = f(x,y) \\ y(x_0) = y_0
	\end{cases}
\end{equation*}
mindestens eine Lösung $y: I \rightarrow \R$ mit $I = [x_0, x_0 + \alpha]$ und $\alpha = \min \{ r, \frac{s}{c} \}$.

\subsection{Gronwallsche Ungleichung}
Sei $c > 0; f,g: [a,b] \rightarrow [0, \infty)$ stetig gilt:
\begin{equation*}
	f(x) \leq c + \int_a^x f(t) g(t) \dt\ \forall x \in [a,b]
\end{equation*}
dann ist
\begin{equation*}
	f(x) \leq c \cdot \exp{\left(\int_a^x g(t) \dt\right)}\ \forall x \in [a,b]
\end{equation*}

\subsection{Banach'scher Fixpunktsatz}
Sei $V = \R^n$ und $M \subseteq \R^n$ abgeschlossen (bezüglich $\norm{\cdot}_\infty$) weiter sei:
$\Phi: M \to M$ eine Kontradiktion, das heißt $\Phi$ ist L-stetig mit $L < 1$.

Dann besitzt $\Phi$ genau einen Fixpunkt $\overline{x} \in M$ (das heißt $\Phi(\overline{x}) = \overline{x}$) und
$\overline{x}$ ist der Grenzwert der Folge ${(x_n)}_{n=1}^\infty$ mit:
\begin{equation*}
	x_k = \begin{cases}
		\text{beliebig} & k = 0 \\
		\Phi({x_{k-1}}) & \text{sonst}
	\end{cases}
\end{equation*}

\subsection{Norm Äquivalenz im $\R^n$}
Sei $\norm{\cdot}$ eine beliebige Norm im $\R^n$, dann existieren $c_1, c_2 > 0$ mit
\begin{equation*}
 	c_1 \norm{x}_\infty \leq \norm{x} \leq c_2 \norm{x}_\infty\ \forall x \in \R^n
\end{equation*}

\subsection{Stabilität}
Sei $G \subseteq \R^2$ ein Gebiet, $f_1, f_2: G \to \R$ stetige Funktionen die auf $G$ eine L-Bedingung erfüllen
mit $L > 0$. Zusätzlich gelte $\abs{f_1(x,y)-f_2(x,y)} < c\ \forall (x,y)^T \in G$ (Modellierungsfehler).

Dann gilt ist $y_1$ eine Lösung des AWP $y' = f_1(x,y), y(x_0) = y_0$ und $y_2$ eine Lösung des AWP 
$y' = f_2(x,y), y(x_0) = \tilde{y_0}$ so folgt:
\begin{equation*}
	\abs{y_1(x) - y_2(x)} \leq (\abs{y_0 - \tilde{y_0}} + c  \cdot (x-x_0)) \exp{(L (x-x_0))}
\end{equation*} 