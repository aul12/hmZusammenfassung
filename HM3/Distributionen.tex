%!TEX root = ../main.tex
\section{Äquivalenzrelation und Äquivalenzklassen}
\subsection{Definition}
Sei $X$ eine beliebige Menge, mit $\tilde{}$ wird eine Eigenschaft zwischen zwei
Elementen definiert (Formal: $\tilde{}: X \times X \to \{\text{Wahr}, \text{Falsch}\}$).
Diese Relation heißt Äquivalenzrelation wenn gilt:
\begin{enumerate}
    \item $a \tilde{} a \ \forall a \in X$ 
    \item $a \tilde{} b \Rightarrow b \tilde{} a\ \forall a,b \in X$
    \item $a \tilde{} b \land b \tilde{} c \Rightarrow a \tilde{} c\ \forall a,b,c \in X$
\end{enumerate}
Mit so einer Relation kann man $X$ in Äquivalenzklassen $\hat{x}$ zerlegen:
\begin{equation*}
    \hat{x} := \{A \subseteq X: \text{Für} a,b \in X \text{ gilt } a \tilde{} b \land \text{für kein} y \in A^C  \text{gilt} y \tilde{} a \text{ für } a \in A\}
\end{equation*}
$A$ ist die größte Teilmenge von $X$ in der alle Elemente in Relation stehen.

\section{Distributionen}
\subsection{Testfunktionen}
\begin{enumerate}
    \item Eine Funktion $\Phi: \R \to \R$ heißt Testfunktion, wenn $\Phi \in C_0^\infty$ gilt, d.h $\Phi \in C^\infty$
        und
        \begin{equation*}
            \exists c > 0:\ \Phi(x)=0\ \forall x \notin [-c, c]
        \end{equation*}
        die Menge aller Testfunktionen bezeichnen wir mit $D$ d.h. $D = C_0^\infty$.
    \item Eine Folge ${(\Phi_n)}_{n=1}^\infty \subseteq D$ heißt konvergent, wenn ein $\Phi \in D$ existiert mit
        $\Phi_n$ konvergiert gleichmäßig gegen
        \begin{equation*}
            \Phi \Leftrightarrow \forall \varepsilon > 0: \exists n_0:\ \abs{\Phi(x) - \Phi_n(x)} < \varepsilon\ 
            \forall x \in \R, n \geq n_0
        \end{equation*}
\end{enumerate}

\subsection{Distributionen}
Eine Funktion $f: D \to \R$ heißt Distribution, wenn
\begin{enumerate}
    \item $f$ ein lineares Funktional ist, das heißt es gilt:
        \begin{enumerate}
            \item Der Wertebereich von $f$ ist $\R$ (oder $\C$)
            \item $\forall \Phi, \Psi \in D, \alpha, \beta \in \R$ gilt
                \begin{equation*}
                    f(\alpha \Phi + \beta \Psi) = \alpha f(\Phi) + \beta f(\Psi)
                \end{equation*}
        \end{enumerate}
    \item $f$ ist stetig, d.h.
        \begin{equation}
            \Phi_n \to \Phi (n \to \infty) \Rightarrow f(\Phi_n) \to f(\Phi) (n \to \infty)
        \end{equation}
\end{enumerate}

\subsubsection{Bemerkung:}
Die Menge aller Distribution nennen wir $D'$

\subsection{Duale Paarung und Repräsentanten}
\begin{enumerate}
    \item Ist $f \in D'$ und $\Phi \in D$ dann schreibt man
        \begin{equation*}
            <f, \Phi> = f(\Phi)
        \end{equation*}
    \item Eine Funktion $\tilde{f}: \R \to \R$ heißt Repräsentant von $f \in D'$, wenn gilt
        \begin{equation*}
            <f, \Phi> = \int_{-\infty}^\infty \tilde{f}(x) \Phi(x) \dx = <\tilde{f}, \Phi>
        \end{equation*}
    das heißt $f$ kann man sich vorstellen.
\end{enumerate}
\subsubsection{Bemerkung:}
Wenn eine Distribution einen Repräsentanten besitzt ist dieser nicht eindeutig.

\subsection{Ableitung einer Distribution}
Sei $f\in D'$, dann heißt $f'$ die schwache Ableitung von
$f$ wenn gilt:
\begin{equation*}
    <f', \Phi> = -<f, \Phi'>\ \forall \Phi \in D
\end{equation*}

\subsection{Kriterium für die Konvergenz einer Funktionenfolge gegen $\delta$}
In der Praxis gibt es ein einfaches Kriterium, um zu prüfen ob eine Folge $\delta_n$ gegen die Delta-Distribution im Sinne von
\begin{equation*}
    <\delta, \Phi> = \lim_{n \to \infty} <\delta_n, \Phi> 
\end{equation*}
konvergiert:
\begin{enumerate}
    \item  
        \begin{equation*}
            \delta_n(x) \geq 0 \forall x 
        \end{equation*}
    \item
        \begin{equation*}
            \int_{-\infty}^\infty \delta_n(x) \dx = 1      
        \end{equation*}
    \item
        \begin{equation*}
            \forall \varepsilon > 0: \lim_{n \to \infty} \int_{-\varepsilon}^\varepsilon \delta_n(x) \dx = 1 
        \end{equation*}
\end{enumerate}

\subsubsection{Bemerkung:}
Das Kriterium ist allerdings nur hinreichend, nicht notwendig. Das heißt eine Folge $\delta_n$ kann gegen $\delta$ konvergieren, und das Kriterium nicht erfüllen.

\section{Fouriertransformation}
\subsection{Definition Fourier-Trafo}
Sei $f: \R \to \C$ stückweise stetig und gelte
\begin{equation*}
    \int_{-\infty}^\infty \abs{f(t)} \dt < \infty
\end{equation*}
dann heißt
\begin{equation*}
    \hat{f}(x) = F_f(x) = \infty_{-\infty}^\infty
    f(t) \exp(-itx) \dt
\end{equation*}
die Fourier-Transformierte von $f$.

\subsection{Stetigkeit der Fourier-Transformierten}
Sei $f: \R \to \C$ stückweise stetig und $\int_{-\infty}^\infty \abs{f(t)} \dt < \infty$
dann ist $F_f(x)$ beschränkt und stetig.

\subsection{Zeitliche Verschiebung und Skalierung der Fouriertransformierten}
Seien $f_1, f_2, f: \R \to \C$ stückweise stetig und $\int_{-\infty}^\infty \abs{f(t)} \dt < \infty$ sowie
$\int_{-\infty}^\infty \abs{f_k(t)} \dt\ k \in \{1,2\}$.

Definiere $g: \R \to \C$ mit
\begin{equation*}
    g(t) = f(a*t + b)
\end{equation*}
dann gilt
\begin{enumerate}
    \item 
        \begin{equation*}
            F_g(x) = \frac{1}{\abs{x}} \exp(i x \frac{b}{a}) \cdot F_f(\frac{x}{a})\ \text{für } a \neq 0
        \end{equation*}
    \item
        \begin{equation*}
            F_{\alpha f_1 + \beta f_2} (x) = \alpha F_{f_1}(x) + \beta F_{f_2}(x)
        \end{equation*}
\end{enumerate}

\subsection{Ableitung der Fouriertransformierten}
Ist $f: \R \to \C$ stückweise stetig, $\int_{-\infty}^\infty \abs{f(t)} \dt < \infty$ sowie
$\int_{-\infty}^\infty \abs{t f(t)}\dt < \infty$. Definiere $g(t) = t f(t)$. Ist $f$ differentierbar, dann ist 
$F_f$ ebenfalls differentierbar und es gilt:
\begin{equation*}
    F_f'(x) = \ddx F_f(x) = -i F_g(x)
\end{equation*}

\subsection{Fouriertransformation der Ableitung}
Sei $f: \R \to \C$ stetig differentierbar und $\int_{-\infty}^\infty \abs{f(t)} \dt < \infty$ und
$\int_{-\infty}^\infty \abs{f'(t)} \dt < \infty$ sowie $\lim{t \to \infty} f(t) = \lim_{t \to - \infty} = 0$, dann
folgt
\begin{equation*}
    F_{f'} (x) = ixF_f(x)
\end{equation*}

\subsection{Definition inverse Fouriertransformation}
Sei $f: \R \to \R$ Lipschitz-Stetig und stückweise stetig differentierbar mit $\int_{-\infty}^\infty < \infty$
dann gilt:
\begin{equation*}
    f(t) = \frac{1}{2\pi} \int_{-\infty}^\infty \exp(itx) F_f(x) \dx
\end{equation*}

\subsection{Plancherel}
Ist $f$ Lipschitz stetig auf $\R$ und stückweise stetig differentierbar mit $\int_{-\infty}^\infty \abs{f(t)} \dt
<\infty$, dann konvergiert $\int_{-\infty}^\infty \abs{f(t)}^2 \dt < \infty$ und es gilt:
\begin{equation*}
    \int_{-\infty}^\infty \abs{f(t)}^2 \dt = \int_{-\infty}^\infty \abs{F_f(x)}^2 \dx
\end{equation*}

\subsection{Definition Faltung}
Seien $f,g: \R \to \C$ gegeben und das Integral $\int_{-\infty}^\infty f(x-t) g(t) \dt$ existiert für alle
$x \in \R$, dann heißt die Funktion $f * g: \R \to \C$ mit
\begin{equation*}
    (f*g)(x) := \int_{-\infty}^\infty f(x-t)g(t) \dt
\end{equation*}
die Faltung von $f$ und $g$.

\subsection{Fouriertransformation der Faltung}
Sei $f,g: \R \to \C$ stetig und $\int_{-\infty}^\infty \abs{f(t)} \dt < \infty$ und 
$\int_{-\infty}^\infty \abs{g(t)} \dt < \infty$. Ferner sei $g$ beschränkt, dann konvergiert
$\int_{-\infty}^\infty f(x-t)g(t) \dt\ \forall x \in \R$ und es gilt: 
\begin{equation*}
    F_{f*g}(x) = F_f(x) \cdot F_g(x)
\end{equation*}

\subsection{Fouriertransformation im Distributionenellen Sinne}
Für eine Distribution $f \in D'$ definieren wir die Fouriertransformation $F_f$ durch:
\begin{equation*}
    <F_f, \Phi> = <f, F_\Phi> \forall \Phi \in D
\end{equation*}
