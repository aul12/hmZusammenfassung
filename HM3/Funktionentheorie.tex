\section{Grundlagen}
\subsection{Definition Stetigkeit}
Sei $M\subseteq\C$ und $f: M \to \C$
\begin{enumerate}
    \item $f$ heißt stetig in $z_0 \in M$, wenn gilt
        \begin{equation*}
            \forall \varepsilon > 0\ \exists \delta > 0: \abs{f(z) - f(z_o)} < \varepsilon\ \forall z \in M
            \land \abs{z-z_0} < \delta
        \end{equation*}
    \item $f$ heißt stetig aus $M$, wenn $f$ in jedem $z_0 \in M$ stetig ist
    \item Eine stetige Funktion $\delta[a,b] \subseteq \R \to \C$ heißt Kurve in $\C$
\end{enumerate}

\subsection{Definition Argument}
Sei $z \in \C, z \neq 0$. Jede Zahl $\phi \in \R$ mit
\begin{equation*}
    \frac{z}{\abs{z}} = \exp(i \phi)
\end{equation*}
heißt ein Argument vo $z$ und wird mit $\arg(z)$ bezeichnet. Das einedeutig bestimmte Argument aus dem Intervall
$[0, 2\pi)$ heißt Hauptargument von $z$ und wird mit $\Arg(z)$ bezeichnet.

\subsubsection{Bemerkung:}
Das Intervall des Hauptarguments könnte man auch als $[-\pi, \pi)$ oder $(-\pi, \pi]$ wählen.

\subsection{Komplexe Wurzel}
\begin{enumerate}
    \item Es existieren genau zwei stetige Funktionen $\C\backslash(-\infty, 0] \to \C$ mit $g(z) = z^2$.
        Diese sind:
        \begin{eqnarray*}
            g_1(z) &=& \sqrt{\abs{x}} \exp(i \frac{\Arg(z)}{2}) \\
            g_2(z) &=& -\sqrt{\abs{x}} \exp(i \frac{\Arg(z)}{2}) 
        \end{eqnarray*}
    \item Es existiert keine stetige Funktion $g: \C \to \C$ mit $g(z^2) = z$
\end{enumerate}

\subsection{Definition komplexer Logarithmus}
Zu jedem $z \in \C \backslash \{0\}$ existiert ein eindeutig bestimmtes $w \in \C$
\begin{enumerate}
    \item $\Im(w) \in (-\pi, \pi]$
    \item $\exp(w) = z$
\end{enumerate}
dieses $w$ nennen wir den Hauptwert des Logarithmus von $z$ und bezeichnen diesen mit $w = \Log(z)$.

\subsubsection{Bemerkung:}
Durch die Funktion
\begin{equation*}
    \Log: \C \backslash \{0\} \to \{z \in \C: \Im(z) \in (-\pi, \pi]\}
\end{equation*}
mit $\Log(z) := \log(\abs{z}) + i \Arg(z)$ bezeichnen wir den Hauptwert des komplexen Logarithmus.

\section{Komplexe Differenzierbarkeit}
\subsection{Definition}
Sei $M \subseteq \C$ offen und $f: M \to \C$:
$f$ heißt differentierbar in $z_0 \in M$, wenn der Grenzwert
\begin{equation*}
    f'(z_0) = \lim_{z \to z_0} \frac{f(z)- f(z_0)}{z-z_0}
\end{equation*}
existiert. Dieser Grenzwert heißt dann Ableitung von $f$ bei $z_0$.

\subsection{Cauchy-Riemann'sche Differentialgleichungen}
Sei $M \subseteq \C$ und $M \neq \emptyset, f:M \to \C$ weiter sei $z_0 = x_0 + i y_0 \in M$ und $x_0, y_0 \in \R$.
Definiere:
\begin{eqnarray*}
    u(x,y) &=& \Re(f(x+iy)) \\
    v(x,y) &=& \Im(f(x+iy))
\end{eqnarray*}
sowie $F: \tilde{M} \subseteq \R^2 \to \R^2$ mit $F(x,y) = (u(x,y), v(x,y))^T$ dann gilt:
\begin{equation*}
    f \text{ ist differentierbar in } z_0 \Leftrightarrow \text{In } z_0 \text{ gelten die CR-DGLen}
\end{equation*}
Die CR-DGLen lauten:
\begin{eqnarray*}
    u_x(x_0, y_0) &=& v_y(x_0, y_0) \\
    u_y(x_0, y_0) &=& v_x(x_0, y_0)
\end{eqnarray*}

\subsubsection{Bemerkung:}
Wenn die CR-DGLen erfüllt sind ist die Jacobi-Matrix $f'(x_0)$ eine Dreh-Matrix.

\subsection{Definition Holomorphe Funktionen}
Sei $M \subseteq \C$ offen, $M \neq \emptyset, z \in M, f: M \to \C$:
\begin{enumerate}
    \item $f$ heißt in $z_0$ holomorph, wenn ein $\varepsilon > 0$ existiert, so dass $f$ auf $U_\varepsilon(z_0)$
        komplex differentierbar ist
    \item $f$ heißt auf $M$ holomorph, wenn $f$ in jedem $z_0 \in M$ holomorph ist
\end{enumerate}

\subsubsection{Bemerkung:}
Statt Holomorph sagt man auch analytisch.

\subsection{Definition orientierter Winkel}
Für $z_1, z_2 \in \C \backslash \{0\}$ heißt $\arg(\frac{z_2}{z_1}) := arg(z_2) - arg(z_1)$ der orientierte Winkel
von $z_1$ nach $z_2$.

\subsection{Definition Winkeltreue}
Seien $\gamma_1: I_1 \to \C, \gamma_2: I_2 \to \C$ mit reelem Intervallen $I_1, I_2$ zwei stetig differentierbare
Kurven. Gelte etwa $c = \gamma_1(t_1) = \gamma_2(t_2)$ für $t_1 \in I_1, t_2 \in I_2$ und weiter sei $M$ eine
offene Menge mit $c \in M$ und $f \in H(M)$ mit $f'(c) \neq 0$. Dann schneiden sich die Kurven $\gamma_1, \gamma_2$
im gleichen Winkel wie die abgebildeten Kurven $f \circ \gamma_1, f \circ \gamma_2$.

\subsection{Biholomorphe Funktionen}
Seien $M_1, M_2 \subseteq \C, M_1, M_2 \neq \emptyset$. Eine bijektive Funktion $f: M_1 \to M_2$ heißt biholomorph,
wenn $f \in H(M_1)$ und $f^{-1} \in H(M_2)$.

\subsection{Ableitung der Umkehrfunktion}
Seien $M_1, M_2 \subseteq \C$ und nicht leer $f: M_1 \to M_2$ biholomorph und $f'(z) \neq 0\ \forall z \in M_1$ dann
gilt:
\begin{equation*}
    \frac{\text{d}}{\text{d}z} f^{-1}(z) = \frac{1}{f'(f^{-1}(z))}
\end{equation*}

\section{Komplexe Kurvenintegrale}
\subsection{Eigenschaften komplexer Kurvenintegrale}
Für $f,g: [a,b] \to \C$ stetig, $\alpha,\beta \in \C$ gilt
\begin{enumerate}
    \item
        \begin{equation*}
            \int_a^b (\alpha f + \beta g)(t)\dt = \alpha \int_a^b f(t) \dt + \beta \int_a^b g(t) \dt
        \end{equation*}
    \item
        \begin{equation*}
            \int_a^c f(t) \dt + \int_c^b f(t) \dt = \int_a^b f(t) \dt
        \end{equation*}
    \item
        Gilt $F'(t) = f(t)\ \forall t$ dann folgt
        \begin{equation*}
            \int_a^b f(t) \dt = F(b) - F(a)
        \end{equation*}
    \item
        \begin{eqnarray*}
            \ReP{} \left( \int_a^b f(t) \dt \right) &=& \int_a^b \ReP{} ( f(t)) \dt \\
            \ImP{} \left( \int_a^b f(t) \dt \right) &=& \int_a^b \ImP{} ( f(t)) \dt
        \end{eqnarray*}
    \item
        \begin{equation*}
            \abs{ \int_a^b f(t) \dt} \leq \int_a^b \abs{f(t)} \dt
        \end{equation*}
\end{enumerate}

\subsection{Definition Kurveneigenschaften}
Sei $\gamma: [a,b] \subseteq \R \to \C$ eine Kurve (d.h. stetige Funktion) dann definiert man
\begin{enumerate}
    \item $\gamma$ ist glatt, wenn $\gamma$ stetig differentierbar ist
    \item $\gamma$ ist stückweise glatt, wenn $\gamma$ stückweise stetig differentierbar ist
    \item Ist $\gamma$ stückweise glatt, dann nennt man
        \begin{equation*}
            L(\gamma) = \int_a^b \abs{\gamma'(t)} \dt
        \end{equation*}
        die Länge von $\gamma$.
\end{enumerate}

\subsection{Komplexe Kurvenintegrale}
Sei $M \subseteq \C, f: M \to \C$ und $\gamma: [a,b] \to \C$ stückweise glatt, dann definieren wir:
\begin{equation*}
    \int_\gamma f(z) \dz := \int_a^b f(\gamma(t))\gamma'(t) \dt
\end{equation*}

\subsection{Konvention zu kreisförmigen Kurven}
Für Kurvenintegrale bei denen $\gamma$ einen positiv durchlaufenen Kreis darstellt schreiben wir
\begin{equation*}
    \int_{\abs{z-z_0} = r} f(z) \dz = \int_\gamma f(z) \dz 
\end{equation*}
es gilt also:
\begin{equation*}
    \gamma: [0, 2 \pi] \to \C\ \gamma(t) \mapsto z_0 + r e^{it}  
\end{equation*}

\section{Cauchy-Integralsatz}
\subsection{Geschlossene Kurvenintegrale}
Sei $G \subseteq \C$ ein Gebiet und $p \in G$ und $f: G \to \C$ stetig, sowie
$f: G\backslash\{p\} \to \C$ holomorph.
Dann gitl für jedes Dreieck $\Delta \subseteq G$ (dessen Rand mit $\partial \Delta$ bezeichnet wird):
\begin{equation*}
    \int_{\partial \Delta} f = \int_{\partial \Delta} f(z) \dz = 0 
\end{equation*}

\subsection{Definition Windungszahl}
Sei $\gamma$ eine geschlossene, stückweise glatte Kurve und $z_0$ ein Punkt der nicht auf $\gamma$ liegt, dann heißt
\begin{equation*}
    N_\gamma (z_0) = \frac{1}{2\pi i} \int_\gamma \frac{1}{z-z_0} \dz 
\end{equation*}
die Windungszahl von $\gamma$ um $z_0$.

\subsubsection{Bemerkung:}
Es gilt $N_\gamma(z_0) \in \Z$.

\subsection{Eigenschaften der Windungszahl}
Sei $\gamma \to \C$ stetig, stückweise glatt und geschlossen, $G$ ein Gebiet mit 
$G \subseteq \C \backslash \gamma([a,b])$ (das heißt ohne den Träger von $\gamma$), dann gilt:
\begin{enumerate}
    \item $N_\gamma(z)$ ist konstant auf $G$
    \item Im Fall $\{z \in \C: \abs{z} > \abs{\gamma(t)}\ \forall t \in [a,b]\}$ gilt $N_\gamma(z) = 0$
\end{enumerate}

\subsection{Windungszahl über zusammenhängende Gebiete}
Sei $\gamma: [a,b] \to \C$ stückweise glatt und geschlossen, $G$ ein Gebiet mit 
$G \subseteq \C \backslash \gamma([a,b])$. Dann gilt:
\begin{enumerate}
    \item $N_\gamma(z)$ ist stückweise konstant, das heißt existiert für zwei Punkte $z_1, z_2$ eine Kurve
        $\Psi \in G$, die $z_1, z_2$ verbindet, dann ist
        \begin{equation*}
            N_\gamma(z_1) = N_\gamma(z_2)
        \end{equation*}
    \item Für ein $z \in G$ mit $G = \{z \in \C: \abs{z} > \abs{\gamma(t)}\ \forall t \in [a,b]\}$ gilt
        $N_\gamma(z) = 0$.
\end{enumerate}

\subsection{Cauchy-Integralformel für sternförmige Gebiete}
Sei $G \subseteq \C$ bezüglich einem $c \in G$ sternförmig, $\gamma$ eine stückweise glatte, geschlossene Kurve,
$f \in H(G)$ dann gilt für alle $z_0$, die nicht auf $\gamma$ liegen mit $\gamma, z_0 \in G$:
\begin{equation*}
    N_\gamma (z_0) f(z_0) = \frac{1}{2 \pi i} \int_\gamma \frac{f(z)}{z-z_0} \dz
\end{equation*}

\subsection{Mittelwerteigenschaften der Cauchy-Integralformel}
Für jede Funktion $f \in H(U_R(z_0))$ und $r \in (0, R)$ gilt:
\begin{equation*}
    f(z_0) = \frac{1}{2 \pi} \int_0^{2\pi} f(z_0 + r e^{it}) \dt
\end{equation*}

\subsection{Definition n-te Ableitung}
Sei $M \subseteq \C$ offen und $f: M \to \C$. Wir setzten $f^{(0)} = f(z)\ \forall z \in M$ ist $f^{(n)}$ in einer
Umgebung von $z_0 \in M$ holomorph, definieren wir 
$f^{(n+1)}(z) = \frac{\text{d}}{\text{d}z} f^{(n)}(z)$ für $z$ in dieser Umgebung.

\subsection{Cauchy-Integralformel für n-te Ableitung}
Sei $G \subseteq \C$ ein Gebiet und $f \in H(G)$, dann gilt
\begin{enumerate}
    \item $f^{(n)}(z)$ existiert für alle $z \in G$ und für alle $n \in \N_0$
    \item Für alle $z \in G$ und $R > 0$ mit $\overline{U_R(z_0)} \subseteq G$ gilt
        \begin{equation*}
            f^{(n)}(z_1) = \frac{n!}{2 \pi i} \int_{\abs{z-z_0} < R} \frac{f(z)}{{(z-z_1)}^{n+1}} \dz
        \end{equation*}
        für $z_1 \in U_R(z_0)$.
\end{enumerate}

\section{Eigenschaften holomorpher Funktionen}
\subsection{Holomorphe Funktionen und Potenzreihen}
Sei $G \in \C$ ein Gebiet, $z_0 \in G, R > 0$ mit $U_R(z_0) \subseteq G$. Weiter sei $f \in H(G)$, dann gilt:
\begin{equation*}
    f(z) = \sum_{n=0}^\infty a_n {(z-z_0)}^n\ \forall z \in U_R(z_0)
\end{equation*}

\subsection{Abschätzung der Ableitung}
Sei $G \subseteq \C$ und $f \in H(G)$, dann gilt für alle $n \in \N_0$ und alle $z_0 \in G$ und $r>0$ mit
$\overline{U_R(z_0)} \subseteq G$:
\begin{equation*}
    \abs{f^{(n)}(z_0)} \leq \frac{n!}{r^n} \max_{\abs{z-z_0} = r} f(z)
\end{equation*}

\subsection{Definition ganze Funktion}
Eine Funktion $f \in H(\C)$ heißt ganze Funktion.

\subsection{Satz von Lionville}
Jede ganze, beschränkte Funktion ist konstant.

\subsection{Fundamentalsatz der Algebra}
Ist $p$ ein Polynom vom Grad $n$ mit $n \geq 1$, dann besitzt $p$ eine Nullstelle (in $\C$).

\subsection{Identitätssatz für holomorphe Funktionen}
Sei $G \subseteq \C$ und $f \in  H(G)$ dann sind folgende Aussagen äquivalent:
\begin{enumerate}
    \item $f \equiv 0$ auf $G$
    \item Die Menge der Nullstellen von $f$, d.h.
            $\{z \in \C: f(z) = 0\}$ hat einen Häufungspunkt $z_0 \in G$
    \item
        \begin{equation*}
            \exists z_0: f^{(k)}(z_0) = 0\ \forall k \in \N_0
        \end{equation*}
\end{enumerate}

\subsection{Maximumsprinzip}
Sei $G \subseteq \C$ ein Gebiet und $f \in H(G)$, dann gilt:
\begin{enumerate}
    \item Besitzt $\abs{f}$ ein lokales Maximum, dann ist $\abs{f}$ konstant.
    \item Ist $G$ beschränkt und $f$ stetig auf $\overline{G}$, so nimmt $\abs{f}$ sein Maximum auf dem
        Rand an.
\end{enumerate}

\subsection{Abschätzung von Potenzreihen}
Sei $f(z) = \sum_{k=0}^\infty a_k {(z-z_0)}^k$ eine Potenzreihe mit Konvergenzradius $R > 0$, dann gilt:
\begin{equation*}
    \sum_{k=0}^\infty \abs{a_k}^2 R^{2k} = \frac{1}{2 \pi} \int_0^{2\pi} \abs{f(z_0 + R e^{it})}^2 \dt
    \leq \max_{\abs{z-z_0} = R} \abs{f(z)}^2
\end{equation*}

\section{Isolierte Singularitäten}
\subsection{Definition isolierte Singularitäten}
Sei $M \in \C$ und $f \in H(M)$, dann heißt ein isolierter Punkt von $M^C$ eine isolierte Singularität von $f$.

\subsection{Charakterisierung von isolierten Singularitäten}
Eine isolierte Singularität $z_0$ von $f$ heißt:
\begin{enumerate}
    \item hebbar, falls $f$ in einer punktierten $\varepsilon$-Umgebung von $z_0$ beschränkt ist
    \item Polstelle (oder Pol) $n$-ter Ordnung, wenn ein $n \in \N$ existiert, so dass die durch ${(z-z_0)} f(z)$
        definierte Funktion in $z_0$ eine hebbare Singularität besitzt.
    \item wesentliche Singularität, wenn $z_0$ weder eine hebbare Singularität oder Polstelle ist
\end{enumerate}

\subsection{Riemannscher Hebbarkeitssatz}
Sei $G \in \C$ ein Gebiet, $f \in H(G)$ und $z_0$ eine hebbare Singularität von $f$, dann gilt:
\begin{equation*}
    z_0 \text{ hebbar} \Leftrightarrow \exists g \in H(U_\varepsilon(z_0)):\ \varepsilon > 0 \land g(z) = f(z)\ 
    \forall z \in \dot{U}_\varepsilon(z_0)
\end{equation*}

\subsection{Zusammenhang ganzrationale Funktionen und Polstellen}
Sei $G \subseteq \C$ ein Gebiet, $ f \in H(G), z_0$ eine isolierte Singularität von $f$, dann gilt:
\begin{equation*}
    z_0 \text{ ist Pol mit Ordnung } m \Leftrightarrow \exists g \in H(U_\varepsilon(z_0)):\ \varepsilon > 0 \land
    g(z_0) \neq 0 \land f(z) = \frac{g(z)}{{(z-z_0)}^m}
\end{equation*}

\subsection{Eigenschaften wesentlicher Singularitäten}
Sei $G \subseteq \C$ ein Gebiet, $f \in H(G), z_0$ eine isolierte Singularität, dann gilt:
\begin{equation*}
    z_0 \text{ ist eine wesentliche Singularität} \Leftrightarrow f \text{ kommt auf jder punktierten } \varepsilon
    \text{-Umgebung von } z_0 \text{ jedem Wert von } \C \text{ beliebig nahe} \Leftrightarrow
    \forall \varepsilon > 0: \overline{f(\dot{U}_\varepsilon (z_0))} = \C
\end{equation*}

\subsection{Variation von Kurven}
Sei $R=\{z \in \C: r_1 \leq \abs{z} \leq r_2\}$ mit $0 \leq r_1 < r_2 \leq \infty$ ein Ringgebiet und $f \in H(R)$.
Dann gilt:
\begin{equation*}
    \int_{\abs{z} = d_1} f(z) \dz = \int_{\abs{z} = d_2} f(z) \dz
\end{equation*}
für $d_1, d_2$ mit $r_1 < d_1, d_2 < r_2$.

\subsection{Holomorphie der Stammfunktion}
Sei $r>0$ und $h: \{ w \in \C: \abs{w} = r\} \to \C$ stetig, dann ist die Funktion $F: U_r(0) \to \C$ mit
\begin{equation*}
    F(z) = \int_{\abs{w} = r} \frac{h(w)}{z-w} \intd{w}
\end{equation*}
holomorph.

\subsection{Laurentzerlegung}
Sei $G = \{z \in \C: r<\abs{z-z_0} < R\}$ mit $0 \leq r < R \leq \infty$ und $f\in H(G)$. Dann existiert
\begin{enumerate}
    \item $g \in H(U_R(0))$
    \item $h \in H(U_{\frac{1}{r}} (0)$
\end{enumerate}
und es gilt
\begin{equation*}
    f(z) = g(z) + h(1/z)
\end{equation*}
Setzt man zusätzlich die Bedinung $h(0) = 0$ voraus, dann ist die Zerlegung eindeutig.

\subsection{Definition Laurentreihe}
Eine unendliche Reihe der Form
\begin{equation*}
    \sum_{k=-\infty}^\infty a_k {(z-z_0)}^k
\end{equation*}
heißt formale Laurent-Reihe. Dabei heißt
\begin{equation*}
    \sum_{k=-\infty}^{-1} a_k {(z-z_0)}^k
\end{equation*}
der Hauptteil der Laurentreihe und
\begin{equation*}
    \sum_{k=0}^\infty a_k {(z-z_0)}^k
\end{equation*}
der Nebenteil der Laurentreihe. Die Laurentreihe ist konvergent wenn der Hauptteil und der Nebenteil (einzeln)
konvergieren. In dem Fall ist:
\begin{equation*}
    \sum_{-\infty}^\infty a_k {(z-z_0)}^k = \sum_{0}^\infty a_k {(z-z_0)}^k + 
        \sum_{-\infty}^{-1} a_k {(z-z_0)}^k
\end{equation*}

\subsubsection{Bemerkung:}
Das Konvergenzgebiet einer Laurentreihe ist ein Ringgebiet.

\subsection{Berechnung der Laurent-Koeffizienten mit Cauchy und Taylor}
Sei $f(z) = \sum_{k=-\infty}^\infty a_k {(z-z_0)}^k$ für $z$ mit $r < \abs{z-z_0} < R$ dann gilt:
\begin{equation*}
    a_k = \frac{1}{2 \pi i} \int_{\abs{w-z_0} = d} \frac{f(w)}{{w-z_0}^{k+1}} \intd{w}
\end{equation*}

\subsection{Zusammenhang Holomorphie und Laurentreihen}
Ist $f$ holomorph auf einem Ringgebiet $G = \{z:\ r < \abs{z-z_0} < R\}$ dann ist $f$ als Laurentreihe auf $G$
darstellbar und die Koeffizienten sind eindeutig festgelegt.

\section{Residuensatz}
\subsection{Definition Resiuduum}
Sei $f \in H(\dot{U}_R(z_0)), R>0$ mit zugehöriger Laurentreihe 
$f(z) = \sum_{k=-\infty}^\infty a_k {(z-z_0)}^k$ dann
heißt $a_{-1}$ das Residuum von $f$ bei $z_0$ und wird mit
\begin{equation*}
    \Res(f, z_0) = a_{-1} 
\end{equation*}
\subsubsection{Bemerkung:}
\begin{enumerate}
    \item Es gilt: 
        \begin{equation*}
            \Res(f,z_0) = \frac{1}{2\pi i} 
            \int_{\abs{z-z_0} = r} f(w) \intd{w}
        \end{equation*}
        mit $0 < r < R$
    \item Falls $f$ bei $z_0$ holomorph ist oder eine
        hebbare Singularität hat ist
        \begin{equation*}
            \Res(f, z_0) = 0 
        \end{equation*}
\end{enumerate}

\subsection{Bestimmung des Residuums}
Sei $f \in H(\dot{U}_R (z_0)), R > 0, z_0$ ein Pol
$m$-ter Ordnung, dann gilt:
\begin{equation*}
    \Res(f, z_0) = \frac{1}{m-1} 
    {\left( \frac{\text{d}}{\text{d}z} \right)}^{m-1}
    {(z-z_0)}^n f(z) \big|_{z=z_0}
\end{equation*}

\subsection{L'Hospital für Residuen}
Seien $f,g \in H(U_R(z_0))$ mit $f(z_0) \neq 0,
g(z_0) = 0, g'(z_0) \neq 0$ dann besitzt $\frac{f}{g}$
bei $z_0$ einen Pol 1-ter Ordnung und
\begin{equation*}
    \Res(\frac{f}{g}, z_0) = \frac{f(z_0)}{g'(z_0)} 
\end{equation*}

\subsection{Residuensatz}
Sei $G \subseteq \C$ ein Elementargebiet und
$z_1, \ldots z_n \in G$ verschiedene Punkte. Weiter sei
$f \in H(G\backslash \{z_1, \ldots, z_n\}$ und $\gamma$
eine geschlossene, stückweise glatte Kurve in
$G \backslash \{z_1, \ldots, z_n\}$. Dann gilt:
\begin{equation*}
    \int_\gamma f(z) \intd{z} = 2 \pi i \sum_{k=1}^m
    N_\gamma (z_0) Res(f, z_0)
\end{equation*}

\subsection{Anwendung des Residuensatz auf uneigentliche Integrale}
Seien $z_1, \ldots, z_m (m \in \N)$ verschieden Punkte
der oberen (bzw. unteren) Halbebene und $G$ die obere
(bzw. untere) Halbebene von $\C$, dann gilt:
\begin{equation*}
    \lim_{\abs{z} \to \infty} z f(z) = 0 \Rightarrow
    \int_{-\infty}^\infty f(x) \dx = 
    2 \pi i \sum_{k=1}^m \Res(f, z_k) \left(
    \text{bzw. } -2 \pi i \sum_{k=1}^m \Res(f, z_k)
    \right)
\end{equation*}

\subsection{Anwendung des Residuensatz auf bestimmte uneigentliche Integrale}
Für Polynone $p, q$ gilt (in der oberen Halbebene):
\begin{enumerate}
    \item 
        \begin{equation*}
            \int_{-\infty}^\infty \frac{p(x)}{q(x)}\dx=
            2 \pi i \int_{k=1}^m \Res(\frac{p}{q},z_k)
        \end{equation*}
        falls Grad $p \geq$ Grad $q +2$
    \item 
        \begin{equation*}
            \int_{-\infty}^\infty \frac{p(x)}{q(x)} 
            e^{iwx} \dx=
            2 \pi i \int_{k=1}^m \Res(\frac{p}{q}
            e^{iwx},z_k)
        \end{equation*}
        falls Grad $p \geq$ Grad $q +1$
\end{enumerate}



