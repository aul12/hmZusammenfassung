\section{Grundlagen}
\subsection{Definition Stetigkeit}
Sei $M\subseteq\C$ und $f: M \to \C$
\begin{enumerate}
    \item $f$ heißt stetig in $z_0 \in M$, wenn gilt
        \begin{equation*}
            \forall \varepsilon > 0\ \exists \delta > 0: \abs{f(z) - f(z_o)} < \varepsilon\ \forall z \in M
            \land \abs{z-z_0} < \delta
        \end{equation*}
    \item $f$ heißt stetig aus $M$, wenn $f$ in jedem $z_0 \in M$ stetig ist
    \item Eine stetige Funktion $\delta[a,b] \subseteq \R \to \C$ heißt Kurve in $\C$
\end{enumerate}

\subsection{Definition Argument}
Sei $z \in \C, z \neq 0$. Jede Zahl $\phi \in \R$ mit
\begin{equation*}
    \frac{z}{\abs{z}} = \exp(i \phi)
\end{equation*}
heißt ein Argument vo $z$ und wird mit $\arg(z)$ bezeichnet. Das einedeutig bestimmte Argument aus dem Intervall
$[0, 2\pi)$ heißt Hauptargument von $z$ und wird mit $\Arg(z)$ bezeichnet.

\subsubsection{Bemerkung:}
Das Intervall des Hauptarguments könnte man auch als $[-\pi, \pi)$ oder $(-\pi, \pi]$ wählen.

\subsection{Komplexe Wurzel}
\begin{enumerate}
    \item Es existieren genau zwei stetige Funktionen $\C\backslash(-\infty, 0] \to \C$ mit $g(z) = z^2$.
        Diese sind:
        \begin{eqnarray*}
            g_1(z) &=& \sqrt{\abs{x}} \exp(i \frac{\Arg(z)}{2}) \\
            g_2(z) &=& -\sqrt{\abs{x}} \exp(i \frac{\Arg(z)}{2}) 
        \end{eqnarray*}
    \item Es existiert keine stetige Funktion $g: \C \to \C$ mit $g(z^2) = z$
\end{enumerate}

\subsection{Definition komplexer Logarithmus}
Zu jedem $z \in \C \backslash \{0\}$ existiert ein eindeutig bestimmtes $w \in \C$
\begin{enumerate}
    \item $\Im(w) \in (-\pi, \pi]$
    \item $\exp(w) = z$
\end{enumerate}
dieses $w$ nennen wir den Hauptwert des Logarithmus von $z$ und bezeichnen diesen mit $w = \Log(z)$.

\subsubsection{Bemerkung:}
Durch die Funktion
\begin{equation*}
    \Log: \C \backslash \{0\} \to \{z \in \C: \Im(z) \in (-\pi, \pi]\}
\end{equation*}
mit $\Log(z) := \log(\abs{z}) + i \Arg(z)$ bezeichnen wir den Hauptwert des komplexen Logarithmus.

\section{Komplexe Differenzierbarkeit}
\subsection{Definition}
Sei $M \subseteq \C$ offen und $f: M \to \C$:
$f$ heißt differentierbar in $z_0 \in M$, wenn der Grenzwert
\begin{equation*}
    f'(z_0) = \lim_{z \to z_0} \frac{f(z)- f(z_0)}{z-z_0}
\end{equation*}
existiert. Dieser Grenzwert heißt dann Ableitung von $f$ bei $z_0$.

\subsection{Cauchy-Riemann'sche Differentialgleichungen}
Sei $M \subseteq \C$ und $M \neq \emptyset, f:M \to \C$ weiter sei $z_0 = x_0 + i y_0 \in M$ und $x_0, y_0 \in \R$.
Definiere:
\begin{eqnarray*}
    u(x,y) &=& \Re(f(x+iy)) \\
    v(x,y) &=& \Im(f(x+iy))
\end{eqnarray*}
sowie $F: \tilde{M} \subseteq \R^2 \to \R^2$ mit $F(x,y) = (u(x,y), v(x,y))^T$ dann gilt:
\begin{equation*}
    f \text{ ist differentierbar in } z_0 \Leftrightarrow \text{In } z_0 \text{ gelten die CR-DGLen}
\end{equation*}
Die CR-DGLen lauten:
\begin{eqnarray*}
    u_x(x_0, y_0) &=& v_y(x_0, y_0) \\
    u_y(x_0, y_0) &=& v_x(x_0, y_0)
\end{eqnarray*}

\subsubsection{Bemerkung:}
Wenn die CR-DGLen erfüllt sind ist die Jacobi-Matrix $f'(x_0)$ eine Dreh-Matrix.

\subsection{Definition Holomorphe Funktionen}
Sei $M \subseteq \C$ offen, $M \neq \emptyset, z \in M, f: M \to \C$:
\begin{enumerate}
    \item $f$ heißt in $z_0$ holomorph, wenn ein $\varepsilon > 0$ existiert, so dass $f$ auf $U_\varepsilon(z_0)$
        komplex differentierbar ist
    \item $f$ heißt auf $M$ holomorph, wenn $f$ in jedem $z_0 \in M$ holomorph ist
\end{enumerate}

\subsubsection{Bemerkung:}
Statt Holomorph sagt man auch analytisch.

\subsection{Definition orientierter Winkel}
Für $z_1, z_2 \in \C \backslash \{0\}$ heißt $\arg(\frac{z_2}{z_1}) := arg(z_2) - arg(z_1)$ der orientierte Winkel
von $z_1$ nach $z_2$.

\subsection{Definition Winkeltreue}
Seien $\gamma_1: I_1 \to \C, \gamma_2: I_2 \to \C$ mit reelem Intervallen $I_1, I_2$ zwei stetig differentierbare
Kurven. Gelte etwa $c = \gamma_1(t_1) = \gamma_2(t_2)$ für $t_1 \in I_1, t_2 \in I_2$ und weiter sei $M$ eine
offene Menge mit $c \in M$ und $f \in H(M)$ mit $f'(c) \neq 0$. Dann schneiden sich die Kurven $\gamma_1, \gamma_2$
im gleichen Winkel wie die abgebildeten Kurven $f \circ \gamma_1, f \circ \gamma_2$.

\subsection{Biholomorphe Funktionen}
Seien $M_1, M_2 \subseteq \C, M_1, M_2 \neq \emptyset$. Eine bijektive Funktion $f: M_1 \to M_2$ heißt biholomorph,
wenn $f \in H(M_1)$ und $f^{-1} \in H(M_2)$.

\subsection{Ableitung der Umkehrfunktion}
Seien $M_1, M_2 \subseteq \C$ und nicht leer $f: M_1 \to M_2$ biholomorph und $f'(z) \neq 0\ \forall z \in M_1$ dann
gilt:
\begin{equation*}
    \frac{\text{d}}{\text{d}z} f^{-1}(z) = \frac{1}{f'(f^{-1}(z))}
\end{equation*}

\section{Komplexe Kurvenintegrale}
\subsection{Eigenschaften komplexer Kurvenintegrale}
Für $f,g: [a,b] \to \C$ stetig, $\alpha,\beta \in \C$ gilt
\begin{enumerate}
    \item
        \begin{equation*}
            \int_a^b (\alpha f + \beta g)(t)\dt = \alpha \int_a^b f(t) \dt + \beta \int_a^b g(t) \dt
        \end{equation*}
    \item
        \begin{equation*}
            \int_a^c f(t) \dt + \int_c^b f(t) \dt = \int_a^b f(t) \dt
        \end{equation*}
    \item
        Gilt $F'(t) = f(t)\ \forall t$ dann folgt
        \begin{equation*}
            \int_a^b f(t) \dt = F(b) - F(a)
        \end{equation*}
    \item
        \begin{eqnarray*}
            \ReP{} \left( \int_a^b f(t) \dt \right) &=& \int_a^b \ReP{} ( f(t)) \dt \\
            \ImP{} \left( \int_a^b f(t) \dt \right) &=& \int_a^b \ImP{} ( f(t)) \dt
        \end{eqnarray*}
    \item
        \begin{equation*}
            \abs{ \int_a^b f(t) \dt} \leq \int_a^b \abs{f(t)} \dt
        \end{equation*}
\end{enumerate}

\subsection{Definition Kurveneigenschaften}
Sei $\gamma: [a,b] \subseteq \R \to \C$ eine Kurve (d.h. stetige Funktion) dann definiert man
\begin{enumerate}
    \item $\gamma$ ist glatt, wenn $\gamma$ stetig differentierbar ist
    \item $\gamma$ ist stückweise glatt, wenn $\gamma$ stückweise stetig differentierbar ist
    \item Ist $\gamma$ stückweise glatt, dann nennt man
        \begin{equation*}
            L(\gamma) = \int_a^b \abs{\gamma'(t)} \dt
        \end{equation*}
        die Länge von $\gamma$.
\end{enumerate}

\subsection{Komplexe Kurvenintegrale}
Sei $M \subseteq \C, f: M \to \C$ und $\gamma: [a,b] \to \C$ stückweise glatt, dann definieren wir:
\begin{equation*}
    \int_\gamma f(z) \dz := \int_a^b f(\gamma(t))\gamma'(t) \dt
\end{equation*}

\subsection{Konvention zu kreisförmigen Kurven}
Für Kurvenintegrale bei denen $\gamma$ einen positiv durchlaufenen Kreis darstellt schreiben wir
\begin{equation*}
    \int_{\abs{z-z_0} = r} f(z) \dz = \int_\gamma f(z) \dz 
\end{equation*}
es gilt also:
\begin{equation*}
    \gamma: [0, 2 \pi] \to \C\ \gamma(t) \mapsto z_0 + r e^{it}  
\end{equation*}

\section{Cauchy-Integralsatz}
\subsection{Geschlossene Kurvenintegrale}
Sei $G \subseteq \C$ ein Gebiet und $p \in G$ und $f: G \to \C$ stetig, sowie
$f: G\backslash\{p\} \to \C$ holomorph.
Dann gitl für jedes Dreieck $\Delta \subseteq G$ (dessen Rand mit $\partial \Delta$ bezeichnet wird):
\begin{equation*}
    \int_{\partial \Delta} f = \int_{\partial \Delta} f(z) \dz = 0 
\end{equation*}

\subsection{Definition Windungszahl}
Sei $\gamma$ eine geschlossene, stückweise glatte Kurve und $z_0$ ein Punkt der nicht auf $\gamma$ liegt, dann heißt
\begin{equation*}
    N_\gamma (z_0) = \frac{1}{2\pi i} \int_\gamma \frac{1}{z-z_0} \dz 
\end{equation*}
die Windungszahl von $\gamma$ um $z_0$.

\subsubsection{Bemerkung:}
Es gilt $N_\gamma(z_0) \in \Z$.

\subsection{Eigenschaften der Windungszahl}
Sei $\gamma \to \C$ stetig, stückweise glatt und geschlossen, $G$ ein Gebiet mit 
$G \subseteq \C \backslash \gamma([a,b])$ (das heißt ohne den Träger von $\gamma$), dann gilt:
\begin{enumerate}
    \item $N_\gamma(z)$ ist konstant auf $G$
    \item Im Fall $\{z \in \C: \abs{z} > \abs{\gamma(t)}\ \forall t \in [a,b]\}$ gilt $N_\gamma(z) = 0$
\end{enumerate}

\subsection{Windungszahl über zusammenhängende Gebiete}
Sei $\gamma: [a,b] \to \C$ stückweise glatt und geschlossen, $G$ ein Gebiet mit 
$G \subseteq \C \backslash \gamma([a,b])$. Dann gilt:
\begin{enumerate}
    \item $N_\gamma(z)$ ist stückweise konstant, das heißt existiert für zwei Punkte $z_1, z_2$ eine Kurve
        $\Psi \in G$, die $z_1, z_2$ verbindet, dann ist
        \begin{equation*}
            N_\gamma(z_1) = N_\gamma(z_2)
        \end{equation*}
    \item Für ein $z \in G$ mit $G = \{z \in \C: \abs{z} > \abs{\gamma(t)}\ \forall t \in [a,b]\}$ gilt
        $N_\gamma(z) = 0$.
\end{enumerate}

\subsection{Cauchy-Integralformel für sternförmige Gebiete}
Sei $G \subseteq \C$ bezüglich einem $c \in G$ sternförmig, $\gamma$ eine stückweise glatte, geschlossene Kurve,
$f \in H(G)$ dann gilt für alle $z_0$, die nicht auf $\gamma$ liegen mit $\gamma, z_0 \in G$:
\begin{equation*}
    N_\gamma (z_0) f(z_0) = \frac{1}{2 \pi i} \int_\gamma \frac{f(z)}{z-z_0} \dz
\end{equation*}

\subsection{Mittelwerteigenschaften der Cauchy-Integralformel}
Für jede Funktion $f \in H(U_R(z_0))$ und $r \in (0, R)$ gilt:
\begin{equation*}
    f(z_0) = \frac{1}{2 \pi} \int_0^{2\pi} f(z_0 + r e^{it}) \dt
\end{equation*}

\subsection{Definition n-te Ableitung}
Sei $M \subseteq \C$ offen und $f: M \to \C$. Wir setzten $f^{(0)} = f(z)\ \forall z \in M$ ist $f^{(n)}$ in einer
Umgebung von $z_0 \in M$ holomorph, definieren wir 
$f^{(n+1)}(z) = \frac{\text{d}}{\text{d}z} f^{(n)}(z)$ für $z$ in dieser Umgebung.

\subsection{Cauchy-Integralformel für n-te Ableitung}
Sei $G \subseteq \C$ ein Gebiet und $f \in H(G)$, dann gilt
\begin{enumerate}
    \item $f^{(n)}(z)$ existiert für alle $z \in G$ und für alle $n \in \N_0$
    \item Für alle $z \in G$ und $R > 0$ mit $\overline{U_R(z_0)} \subseteq G$ gilt
        \begin{equation*}
            f^{(n)}(z_1) = \frac{n!}{2 \pi i} \int_{\abs{z-z_0} < R} \frac{f(z)}{{z-z_1}^{n+1}} \dz
        \end{equation*}
        für $z_1 \in U_R(z_0)$.
\end{enumerate}
