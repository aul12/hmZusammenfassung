Hier findest du eine kurze Übersicht über alle Themen, die du für die jeweilige
Klausur beherrschen solltest:
Wichtige Definitionen und Beweise, die man gut in der Klausur abfragen kann,
besonders trickreiche Aufgaben, die mehrmals in der Vorlesung oder in der Übung
besprochen wurden und generelle Kompetenzen, die höchstwahrscheinlich von dir
verlangt werden.
\chapter{HM1}
%@TODO
\chapter{HM2}
\section{Integration}

\subsection{Wichtige Beweise}
\begin{itemize}
  \item 1.\ und 2. Mittelwertsatz der Integralrechnung
  \item Hauptsatz der Differential- und Integralrechnung
\end{itemize}

\subsection{Typische Aufgaben}
Berechne den GW von z.B. folgender Reihe (hast du also das Prinzip der Riemann-Summen verstanden?)
\begin{equation*}
    \lim\limits_{n \rightarrow \infty}{\frac{1}{n}\sum_{k=0}^n \sin{\frac{k\pi}{n}}}
\end{equation*}

Untersuche Reihen auf Konvergenz (wende das Integralkriterium an)
\begin{equation*}
    \sum_{n=-m}^\infty \frac{1}{1+n^2} \quad (m \in \N)
\end{equation*}
Oder diese hier (Tipp: Eulersche Gammafunktion)
\begin{equation*}
    \sum_{n=0}^\infty n^3 e^{-n^2}
\end{equation*}

Untersuche uneigentliche Integrale auf Konvergenz

\subsection{Trickreiche Aufgaben}
Schwierige uneigentliche Integrale. Konvergiert beispielsweise dieses Integral? (Ja, tut es)
\begin{equation*}
    \int_1^\infty \frac{\sin{x}}{x}\dx
\end{equation*}

\subsection{Weitere hilfreiche Dinge}
Schau dir uneigentliche Integrale an, die man gut als Majorante oder Minorante
verwenden kann, z.B.:
\begin{equation*}
    \int_0^1 \frac{1}{x^\alpha}\dx
\end{equation*}

\section{Gleichmäßige Konvergenz}
\subsection{Wichtige Beweise}
\begin{itemize}
  \item Stetigkeit der Grenzfunktion
  \item Satz von Dini (ziemlich tricky, aber die Idee sollte man im Kopf haben)
\end{itemize}

\subsection{Typische Aufgaben}
Untersuche Reihen auf gleichmäßige Konvergenz, z.B.:
\begin{equation*}
    \sum_{k=0}^\infty x{(1-x)}^k,\quad \forall x \in [0,1] \quad \text{bzw} \quad \forall x \in [a,1] \quad \text{mit} \quad 0 < a \leq 1
\end{equation*}

\subsection{Trickreiche Aufgaben}
Auf welchem Intervall konvergiert die Riemannsche Zeta-Funktion gleichmäßig?
\begin{equation*}
    \zeta(s):=\sum_{n=1}^\infty \frac{1}{n^s}
\end{equation*}

\section{Differentialrechnung mit mehreren Veränderlichen}
\subsection{Wichtige Beweise}
\begin{itemize}
  \item Beweise über z.B. die Vereinigung von beliebig vielen offenen Mengen
  \item Mehrdimensionaler Mittelwertsatz
  \item Notwendige Bedingung für Extrema
  \item Satz über konstante Funktionen
  \item Satz von Taylor
  \item Hinreichende Bedingung für Extrema
  \item Herleitung für die Ableitung der Auflösung (kann sehr hilfreich sein,
        wenn man die Formel vergessen hat)
\end{itemize}

\subsection{Typische Aufgaben}
Kannst du:
\begin{itemize}
  \item dir Mengen vorstellen und zeichnen?
  \item Funktionsgrenzwerte berechnen?
  \item Funktionen auf Stetigkeit prüfen?
  \item partielle Ableitungen und Richtungsableitungen berechnen?
  \item Funktionen auf totale Diff'barkeit prüfen?
  \item Extremwerte von Funktionen finden und klassifizieren?
  \item mit Matrizen rechnen und Inverse bestimmen?
  \item prüfen, ob eine Funktion umkehrbar ist und die Umkehrung bestimmen?
  \item die Ableitung einer unbekannten Umkehrfunktion bestimmen?
  \item prüfen, ob eine Funktion nach einer / mehreren Variablen auflösbar ist und die Auflösung bestimmen?
  \item die Ableitung einer unbekannten Auflösung berechen?
  \item Extrema unter Nebenbedingung bestimmen?
\end{itemize}

\section{Integration in mehreren Veränderlichen}
\subsection{Wichtige Beweise}
\begin{itemize}
	\item Fubini
	\item Ableitung eines Parameterintegrals
	\item Leibniz-Formel herleiten können
	\item 1. Hauptsatz für Kurvenintegrale
\end{itemize}

\subsection{Typische Aufgaben}
Kannst du:
\begin{itemize}
	\item die Länge von Kurven bestimmen?
	\item Funktionen auf Wegunabhängigkeit prüfen?
	\item Potentiale und Stammfunktionen berechnen?
	\item Flächen und Volumina berechnen mit:
	\begin{itemize}
		\item Fubini und Cavaleri?
		\item der Substitutionsregel?
	\end{itemize}
	\item Integralsätze:
	\begin{itemize}
		\item verifizieren?
		\item geschickt anwenden?
	\end{itemize}
\end{itemize}

\subsection{Trickreiche Aufgaben}
Schau dir schwierige uneigentliche Integrale an, wie z.B.:
\begin{equation*}
	\int_0^1{\frac{t^b-t^a}{\log{t}}\dt}
\end{equation*}
oder
\begin{equation*}
	\int_{-\infty}^{\infty}{e^{-x^2}\dx}
\end{equation*}

\section{Lineare Algebra}

\subsection{Typische Aufgaben}
Kannst du:
\begin{itemize}
	\item prüfen, ob eine Menge ein Vektorraum ist?
	\item prüfen, ob eine Menge ein Unterraum ist?
	\item lineare Unabhängigkeit nachprüfen?
	\item die Basis eines VR bestimmen?
	\item die Dimension einers VR bestimmen?
	\item LGS lösen?
\end{itemize}


