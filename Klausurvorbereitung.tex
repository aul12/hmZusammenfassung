Hier findest du eine kurze Übersicht über alle Themen, die du für die jeweilige
Klausur beherrschen solltest:
Wichtige Definitionen und Beweise, die man gut in der Klausur abfragen kann,
besonders trickreiche Aufgaben, die mehrmals in der Vorlesung oder in der Übung
besprochen wurden und generelle Kompetenzen, die höchstwahrscheinlich von dir
verlangt werden.
\chapter{HM1}
%@TODO
\chapter{HM2}
\section{Integration}

\subsection{Wichtige Beweise}
\begin{itemize}
  \item 1.\ und 2. Mittelwertsatz der Integralrechnung
  \item Hauptsatz der Differential- und Integralrechnung
\end{itemize}

\subsection{Typische Aufgaben}
Berechne den GW von z.B. folgender Reihe (hast du also das Prinzip der Riemann-Summen verstanden?)
\begin{equation*}
    \lim\limits_{n \rightarrow \infty}{\frac{1}{n}\sum_{k=0}^n \sin{\frac{k\pi}{n}}}
\end{equation*}

Untersuche Reihen auf Konvergenz (wende das Integralkriterium an)
\begin{equation*}
    \sum_{n=-m}^\infty \frac{1}{1+n^2} \quad (m \in \N)
\end{equation*}
Oder diese hier (Tipp: Eulersche Gammafunktion)
\begin{equation*}
    \sum_{n=0}^\infty n^3 e^{-n^2}
\end{equation*}

Untersuche uneigentliche Integrale auf Konvergenz

\subsection{Trickreiche Aufgaben}
Schwierige uneigentliche Integrale. Konvergiert beispielsweise dieses Integral? (Ja, tut es)
\begin{equation*}
    \int_1^\infty \frac{\sin{x}}{x}\dx
\end{equation*}

\subsection{Weitere hilfreiche Dinge}
Schau dir uneigentliche Integrale an, die man gut als Majorante oder Minorante
verwenden kann, z.B.:
\begin{equation*}
    \int_0^1 \frac{1}{x^\alpha}\dx
\end{equation*}

\section{Gleichmäßige Konvergenz}
\subsection{Wichtige Beweise}
\begin{itemize}
  \item Stetigkeit der Grenzfunktion
  \item Satz von Dini (ziemlich tricky, aber die Idee sollte man im Kopf haben)
\end{itemize}

\subsection{Typische Aufgaben}
Untersuche Reihen auf gleichmäßige Konvergenz, z.B.:
\begin{equation*}
    \sum_{k=0}^\infty x{(1-x)}^k,\quad \forall x \in [0,1] \quad \text{bzw} \quad \forall x \in [a,1] \quad \text{mit} \quad 0 < a \leq 1
\end{equation*}

\subsection{Trickreiche Aufgaben}
Auf welchem Intervall konvergiert die Riemannsche Zeta-Funktion gleichmäßig?
\begin{equation*}
    \zeta(s):=\sum_{n=1}^\infty \frac{1}{n^s}
\end{equation*}
