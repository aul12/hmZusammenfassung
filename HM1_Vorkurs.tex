\section{Aussagenlogik}
\subsection{Definition Aussage}
Eine Aussage ist ein Satz, der entweder wahr oder falsch ist.

\subsubsection{Bemerkung}Wir beschäftigen uns mit der klassischen zweiwertigen
Logik. Es gibt auch Logiken mit 3 bzw.\ 4 Werten.

\subsection{Verknüpfungen}
Formal kann eine Oder-Verknüfung mit dem $\lor$-Zeichen durch eine
Wahrheitstabelle definiert werden:
\begin{center}
    \begin{tabular}{ccc}
        \toprule
        $A$ & $B$ & $A \lor B$ \\
        \midrule
        $1$ & $1$ & $1$ \\
        $1$ & $0$ & $1$ \\
        $0$ & $1$ & $1$ \\
        $0$ & $0$ & $0$ \\
        \bottomrule
    \end{tabular}
\end{center}

Analog kann eine Und-Verknüpfung mit dem $\land$-Zeichen durch eine
Wahrheitstabelle definiert werden:
\begin{center}
    \begin{tabular}{ccc}
        \toprule
        $A$ & $B$ & $A \land B$ \\
        \midrule
        $1$ & $1$ & $1$ \\
        $1$ & $0$ & $0$ \\
        $0$ & $1$ & $0$ \\
        $0$ & $0$ & $0$ \\
        \bottomrule
    \end{tabular}
\end{center}

Und eine Negation wird definiert durch:
\begin{center}
    \begin{tabular}{cc}
        \toprule
        $A$ & $\lnot A$ \\
        \midrule
        $1$ & $0$ \\
        $0$ & $1$ \\
        \bottomrule
    \end{tabular}
\end{center}

Eine sog.\ Implikation wird durch das $\Rightarrow$-Zeichen dargestellt und
ist definiert durch:
\begin{center}
    \begin{tabular}{ccc}
        \toprule
        $A$ & $B$ & $A \Rightarrow B$ \\
        \midrule
        $1$ & $1$ & $1$ \\
        $1$ & $0$ & $0$ \\
        $0$ & $1$ & $1$ \\
        $0$ & $0$ & $1$ \\
        \bottomrule
    \end{tabular}
\end{center}

\subsubsection{Bemerkung}
Bei mehr als einer Verknüpfung muss klar sein welche Verknüpfung
als erstes ausgewerted werden muss, hierfür werden Klammern verwendet.

\subsection{Mehr zu Implikationen}
Bei der Aussage $A \Rightarrow B$ bezeichnet man $A$ als hinreichende
Bedingung und $B$ als notwendige Bedingung.

Die Aussage $A \Rightarrow B$ ist äquivalent zu $\lnot B \Rightarrow \lnot A$.

\subsection{Bezeichnung von Aussagen}
Eine Aussageform heißt:
\begin{enumerate}[label= (\alph*)]
    \item Allgemeingültig (oder Tautologie), wenn sie als Wahrheitswert stets
        den Wert wahr annimmt.
    \item Erfüllbar, wenn die Wahrheitstabelle mindestens einmal den Wert
        wahr enthält.
    \item Unerfüllbar (oder Kontradiction), wenn die Wahrheitstabelle nur
        falsch-Einträge enthält.
\end{enumerate}

\subsection{Satz der Identität}
Mit $A \Leftrightarrow B$ kürzen wir die Aussage:
\begin{equation*}
    (A \Rightarrow B) \land (B \Rightarrow A)
\end{equation*}
ab.

\subsubsection{Bemerkung}
Für den allg. Fall sagt man zu $A \Leftrightarrow B$: A ist äquivalent
zu B. Das heißt aber nicht, dass $A=B$ ist.


\section{Mengen}
\subsection{Defintion: Mengen nach Cantor}
Unter einer Menge versteht man eine Zusammenfassung
bestimmter wohlunterscheidbarer Objekte unsere Anschauung oder unseres Denkens
zu einem Ganzen.

\subsection{Begrifflichkeiten und Schreibweise}
Objekte einer Menge bezeichnet man als Elemente einer Menge.

Schreibweise:
\begin{enumerate}[label= (\alph*)]
    \item $x \in M$ oder $x \notin M$
    \item Mengen können durch Aufzählen der Elemente beschrieben werden:
        $M= \{ a, b, c \}$
    \item Mengen können durch Eigenschaften der Elemente beschrieben werden:
        $M= \{x: x \text{ hat Eigenschaft\ldots} \}$
\end{enumerate}

\subsection{Leere Menge, Teilmengen}
\begin{enumerate}[label= (\alph*)]
    \item Die Menge, die kein Element enthält, heißt leere Menge.
        Wir bezeichnen diese mit $\emptyset$.
    \item Eine Menge $M_1$ heißt Teilmenge einer Menge $M_2$
        (Schreibweise $M_1 \subseteq M_2$) falls jedes Element von $M_1$ auch
        Element von $M_2$ ist. D.h.\ es gilt:
        \begin{equation*}
            x \in M_1 \Rightarrow x \in M_2
        \end{equation*}
    \item Zwei Mengen sind gleich wenn gilt:
        \begin{equation*}
            M_1 = M_2 \Leftrightarrow M_1 \subseteq M_2 \land M_2 \subseteq M_1
        \end{equation*}
    \item $M_1$ heißt echte Teilmenge von $M_2$ wenn gilt:
        \begin{equation*}
            M_1 \subseteq M_2 \land M_1 \neq M_2
        \end{equation*}
        Schreibweise: $M_1 \subset M_2$ oder $M_1 \subsetneq M_2$.
\end{enumerate}

\subsection{Transitivität u.a.}
Für Mengen $M, M_1, M_2, M_3$ gilt stets:
\begin{enumerate}[label= (\alph*)]
    \item Aus $M_1 \subseteq M_2$ und $M_2 \subseteq M_3$ folgt stets:
        $M_1 \subseteq M_3$
    \item
        $M_1 = M_2 \Leftrightarrow M_1 \subseteq M_2 \land M_2 \subseteq M_1$
    \item $M \subseteq M$ und $\emptyset \subseteq M$
\end{enumerate}

\subsection{Verknüpfung von Mengen}
Für Mengen $M_1$ und $M_2$ definiert man:
\begin{enumerate}[label= (\alph*)]
    \item Die Vereinigung von $M_1$ und $M_2$ durch:
        \begin{equation*}
            M_1 \cup M_2 := \{ x: x \in M_1 \lor x \in M_2 \}
        \end{equation*}
    \item Den Schnitt von $M_1$ und $M_2$ durch:
        \begin{equation*}
            M_1 \cap M_2 := \{ x: x \in M_1 \land x \in M_2 \}
        \end{equation*}
    \item Die Differenz von  $M_1$ und $M_2$ durch:
        \begin{equation*}
            M_1 \backslash M_2 := \{ x: x \in M_1 \land x \notin M_2 \}
        \end{equation*}
    \item Das Kartesische Produkt von  $M_1$ und $M_2$ durch:
        \begin{equation*}
            M_1 \times M_2 := \{(a, b): a \in M_1 \land b \in M_2 \}
        \end{equation*}
    \item Das Kartesische Produkt von $M_1$ und $M_1$ durch;
        \begin{equation*}
            {(M_1)}^2 := M_1 \times M_1
        \end{equation*}
\end{enumerate}

\subsection{Potenzmenge}
Für eine Menge $M$ ist durch
\begin{equation*}
    P(M) := \{ A: A \subseteq M \}
\end{equation*}
die Potenzmenge definiert (Menge aller Teilmengen von M).

\subsubsection{Bemerkung}
Hier gilt $\emptyset \in P(M)$.

\subsection{Rechenregeln für Mengen}
Für bel. Mengen $M_1, M_2, M_3$ gilt:
\begin{enumerate}[label= (\alph*)]
    \item Kommutativität:
        \begin{equation*}
            M_1 \cup M_2 = M_2 \cup M_1 \text{ und }
            M_1 \cap M_2 = M_2 \cap M_1
        \end{equation*}
    \item Assoziativität:
        \begin{equation*}
            (M_1 \cup M_2) \cup M_3 = M_1 \cup (M_2 \cup M_3) \text{ und }
            (M_1 \cap M_2) \cap M_3 = M_1 \cap (M_2 \cap M_3)
        \end{equation*}
    \item Distributivgesetz:
        \begin{equation*}
            M_1 \cap (M_2 \cup M_3) = (M_1 \cap M_2) \cup (M_1 \cap M_3)
            \text{ und }
            M_1 \cup (M_2 \cap M_3) = (M_1 \cup M_2) \cap (M_1 \cup M_3)
        \end{equation*}
\end{enumerate}

\subsection{Komplement}
Ist $X$ eine feste Menge und $M \subseteq X$ beliebig, so heißt
\begin{equation*}
    M^c := X \backslash M
\end{equation*}
das Komplement von $M$ (bzgl, $X$).

\subsection{Bemerkung}
Die Schreibweise erfordert das $X$ aus dem Kontext bekannt sein muss.

\subsection{Verknüpfungen über mehrere Elemente}
Für Mengen $M_1, M_2, \ldots, M_n$ mit $n \in \N$ definieren wir die Notation:
\begin{enumerate}[label= (\alph*)]
    \item
        \begin{equation*}
            \setbigcup_{k=1}^n M_k = M_1 \cup M_2 \cup \ldots \cup M_n
        \end{equation*}
    \item
        \begin{equation*}
            \setbigcap_{k=1}^n M_k = M_1 \cap M_2 \cap \ldots \cap M_n
        \end{equation*}
    \item
        \begin{equation*}
            \setbigtimes_{k=1}^n M_k = M_1 \times M_2 \times \cdots \times M_n
        \end{equation*}
\end{enumerate}

\subsection{Wichtige Zusammenhänge}
\begin{enumerate}[label= (\alph*)]
    \item ${(M^c)}^c = M$
    \item $M_1 \subseteq M_2 \Rightarrow {M_2}^c \subseteq {M_1}^c$
    \item ${(M_1 \cup M_2)}^c = {M_1}^c \cap {M_2}^c$
\end{enumerate}

\section{Vollständige Induktion}

\subsection{Summen und Produktzeichen}
Für $m, n \in \Z, m \leq n$ und $a_m, a_{m+1}, \ldots a_n \in \R$ definieren
wir:

\begin{equation*}
    \sum_{k=n}^n a_k := a_m + a_{m+1} + \ldots + a_n
\end{equation*}
und
\begin{equation*}
    \prod_{k=m}^n a_k := a_m \cdot a_{m+1} \cdot \ldots \cdot a_n
\end{equation*}
Falls $m>n$ ist definieren wir $\sum_{k=m}^n a_k := 0$ und
$\prod_{k=m}^n a_k := 1$

\subsection{Prinzip der Vollständigen Induktion}
Gegen seien Aussagen $A(n)$ für $n \geq n_0$ mit $n_0, n \in \Z$
($n_0$ beliebig aber fest). Und es gelte:
\begin{enumerate}[label= (\alph*)]
    \item $A(n_0)$ ist wahr
    \item Für alle $n \geq n_0$ gilt: $A(n) \Rightarrow A(n+1)$
\end{enumerate}

\subsubsection{Bemerkung}
\begin{enumerate}[label= (\alph*)]
    \item $n_0$ wird als Induktionsanfang, $n$ als Induktionsschritt bezeichnet
    \item Nachteil: wir wissen nicht wieso etwas gilt, nur dass es gilt
\end{enumerate}

\subsection{Rechenregeln für Summen}
Für $m,n \in \Z$ und $a_k, b_k, c \in \R$ gilt:
\begin{enumerate}[label= (\alph*)]
    \item Indexverschiebung:
        \begin{equation*}
            \sum_{k=m}^n a_k = \sum_{k=m+l}^{n+l} a_{k-l}
        \end{equation*}
        für beliebiges $l \in \Z$
    \item Trennen von Summen:
        \begin{equation*}
            \sum_{k=m}^n (a_k + b_k) = \sum_{k=m}^n a_k + \sum_{k=m}^n b_k
        \end{equation*}
    \item Konstante Faktoren können aus der Summe ``gezogen'' werden:
        \begin{equation*}
            \sum_{k=m}^n c \cdot a_k = c \cdot \sum_{k=m}^n a_k
        \end{equation*}
    \item ``Teleskopsummen'':
        \begin{equation*}
            \sum_{k=m}^n (a_k - a_{k+1}) = a_m - a_{n+1}
        \end{equation*}
    \item Summe über Konstanten:
        \begin{equation*}
            \sum_{k=m}^n c = c \cdot (n-m+1)
        \end{equation*}
\end{enumerate}

\subsection{Doppelsummen}
Für $n \in \N$ und $a_{ij} \in \R$, $1 \leq i \leq j \leq n$ gilt:
\begin{equation*}
    \sum_{i=1}^n \sum_{j=i}^n a_{ij} = \sum_{j=1}^n \sum_{i=1}^j a_{ij}
\end{equation*}

\subsection{Fakultät und Binomialkoeffizient}
Für $n \in \N_0$ und ein $\alpha \in \R$ heißt
\begin{enumerate}[label= (\alph*)]
    \item die Fakultät von $n$
        \begin{equation*}
            n! := \begin{cases}
                n \cdot (n-1)!&;n \neq 0\\
                1&;n=0
            \end{cases}
        \end{equation*}
    \item den Binomielkoeffizienten
        \begin{equation*}
            {\alpha \choose n} := \frac{\prod\limits_{k=1}^n (\alpha-k+1)}{n!}
        \end{equation*}
\end{enumerate}

\subsection{Rechenregeln für den Binomialkoeffizienten}
Für $n, m \in \N_0$ mit $m \geq n$ und $\alpha \in \R$ gilt:
\begin{enumerate}[label= (\alph*)]
    \item
        \begin{equation*}
            {\alpha \choose n} + {\alpha \choose n+1} =
            {\alpha + 1 \choose n +1}
        \end{equation*}
     \item
        \begin{equation*}
            {m \choose n} = \frac{m!}{n!(m-n)!}
        \end{equation*}
\end{enumerate}

\subsection{Binomischer Lehrsatz}
Für $a, b \in \R$ und $n \in \N_0$ gilt:
\begin{equation*}
    {(a+b)}^n = \sum_{k=0}^n {n \choose k} a^k b^{n-k}
\end{equation*}

\subsection{Definition Betrag}
Für $x \in \R$ heißt
\begin{equation*}
    \abs{x} :=
        \begin{cases}
            x &,x\geq0\\
            -x &,x<0
        \end{cases}
\end{equation*}
der Betrag von $x$

\subsubsection{Bemerkung}
Es gilt:
\begin{enumerate}[label= (\alph*)]
    \item $\abs{x} \geq 0\ \forall x \in \R$
    \item $\abs{x \cdot y} = \abs{x} \cdot \abs{y}\ \forall x,y \in \R$
    \item $\abs{x-a} < \varepsilon \Leftrightarrow a-\varepsilon < x < a+\varepsilon$
    \item $\abs{x} = \max{\{x, -x\}}\ \forall x \in \R$
\end{enumerate}

\subsection{Dreiecksungleichung}
Für alle $x, y \in \R$ gilt:
\begin{enumerate}[label= (\alph*)]
    \item $\abs{x+y} \leq \abs{x} + \abs{y}$ (obere Dreiecksungleichung)
    \item $\abs{x+y} \geq \abs{\abs{x}-\abs{y}}$ (untere Dreiecksungleichung)
\end{enumerate}

\subsubsection{Bemerkung}
Es gilt $x \leq \abs{x}\ \forall x \in \R$.

\section{Funktion und Differentiation}
Eine Funktion (bzw. Abbildung, Operator) $f$ von $X$ nach $Y$ ist eine
Vorschrift, die jedem $x \in X$ ein eindeutig bestimmtes $y \in Y$ zuordnet.
Das $x \in X$ zugeordnete Element aus $Y$ wird mit $f(x)$ bezeichnet.

\subsubsection{Schreibweise}
\begin{equation*}
    f: X \rightarrow Y,\quad x \mapsto f(x)
\end{equation*}


\subsubsection{Bemerkung}
$X$ heißt Definitionsbereich, $Y := \{y \in Y: \exists x \in X \text{ mit }
y=f(x)\}$ die Zielmenge.

\subsection{Injektivität, Surjektivität, Bijektivität}
\begin{enumerate}[label= (\alph*)]
    \item Eine Funktion heißt injektiv, falls gilt:
        \begin{equation*}
            x \neq y \Rightarrow f(x) \neq f(y)\ forall x,y\in X
        \end{equation*}
    \item Eine Funktion heißt surjektiv, falls gilt:
         \begin{equation*}
            \forall y \in Y\ \exists x \in X: y=f(x)
         \end{equation*}
     \item Eine Funktion heißt bijektiv, wenn sie injektiv und
        surjektiv ist.
\end{enumerate}
