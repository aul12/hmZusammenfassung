\section{Aussagenlogik}
\subsection{Definition Aussage}
Eine Aussage ist ein Satz, der entweder wahr oder falsch ist.

\subsubsection{Bemerkung}Wir beschäftigen uns mit der klassischen zweiwertigen
Logik. Es gibt auch Logiken mit 3 bzw.\ 4 Werten.

\subsection{Verknüpfungen}
Formal kann eine Oder-Verknüfung mit dem $\lor$-Zeichen durch eine
Wahrheitstabelle definiert werden:
\begin{center}
    \begin{tabular}{ccc}
        \toprule
        $A$ & $B$ & $A \lor B$ \\
        \midrule
        $1$ & $1$ & $1$ \\
        $1$ & $0$ & $1$ \\
        $0$ & $1$ & $1$ \\
        $0$ & $0$ & $0$ \\
        \bottomrule
    \end{tabular}
\end{center}

Analog kann eine Und-Verknüpfung mit dem $\land$-Zeichen durch eine
Wahrheitstabelle definiert werden:
\begin{center}
    \begin{tabular}{ccc}
        \toprule
        $A$ & $B$ & $A \land B$ \\
        \midrule
        $1$ & $1$ & $1$ \\
        $1$ & $0$ & $0$ \\
        $0$ & $1$ & $0$ \\
        $0$ & $0$ & $0$ \\
        \bottomrule
    \end{tabular}
\end{center}

Und eine Negation wird definiert durch:
\begin{center}
    \begin{tabular}{cc}
        \toprule
        $A$ & $\lnot A$ \\
        \midrule
        $1$ & $0$ \\
        $0$ & $1$ \\
        \bottomrule
    \end{tabular}
\end{center}

Eine sog.\ Implikation wird durch das $\Rightarrow$-Zeichen dargestellt und
ist definiert durch:
\begin{center}
    \begin{tabular}{ccc}
        \toprule
        $A$ & $B$ & $A \Rightarrow B$ \\
        \midrule
        $1$ & $1$ & $1$ \\
        $1$ & $0$ & $0$ \\
        $0$ & $1$ & $1$ \\
        $0$ & $0$ & $1$ \\
        \bottomrule
    \end{tabular}
\end{center}

\subsubsection{Bemerkung}
Bei mehr als einer Verknüpfung muss klar sein welche Verknüpfung
als erstes ausgewerted werden muss, hierfür werden Klammern verwendet.

\subsection{Mehr zu Implikationen}
Bei der Aussage $A \Rightarrow B$ bezeichnet man $A$ als hinreichende
Bedingung und $B$ als notwendige Bedingung.

Die Aussage $A \Rightarrow B$ ist äquivalent zu $\lnot B \Rightarrow \lnot A$.

\subsection{Bezeichnung von Aussagen}
Eine Aussageform heißt:
\begin{enumerate}[label= (\alph*)]
    \item Allgemeingültig (oder Tautologie), wenn sie als Wahrheitswert stets
        den Wert wahr annimmt.
    \item Erfüllbar, wenn die Wahrheitstabelle mindestens einmal den Wert
        wahr enthält.
    \item Unerfüllbar (oder Kontradiction), wenn die Wahrheitstabelle nur
        falsch-Einträge enthält.
\end{enumerate}

\subsection{Satz der Identität}
Mit $A \Leftrightarrow B$ kürzen wir die Aussage:
\begin{equation*}
    (A \Rightarrow B) \land (B \Rightarrow A)
\end{equation*}
ab.

\subsubsection{Bemerkung}
Für den allg. Fall sagt man zu $A \Leftrightarrow B$: A ist äquivalent
zu B. Das heißt aber nicht, dass $A=B$ ist.
