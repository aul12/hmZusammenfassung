\section{Ableitung}

\subsection{Definition Differnzen-Quotient}
Sei $f: D \subseteq \R \rightarrow \R $. Dann heißt $f$ in
$x_0 \in D$ differentierbar, falls
\begin{equation*}
    f'(x_0) := \lim_{x \rightarrow x_0}
        \frac{f(x) - f(x_0)}{x - x_0}
\end{equation*}
für alle $x_0 \in D$ existiert.

\subsection{Rechtsseitige und linksseitige Ableitung}
Im Fall der existenz heißen
\begin{eqnarray*}
    f'(x_0^+) &:=& \lim_{x \rightarrow x_0^+}
        \frac{f(x) - f(x_0)}{x-x_0} \text{ bzw.}\\
    f'(x_0^-) &:=& \lim_{x \rightarrow x_0^-}
        \frac{f(x) - f(x_0)}{x-x_0}
\end{eqnarray*}
die rechts- bzw.\ linksseitige Ableitung in $x_0$

\subsubsection{Bemerkung}
\begin{equation*}
    f'(x_0) \text{ ex.} \Leftrightarrow
    f'(x_0^+) \text{ und } f(x_0^-) \text{ ex.\ und }
    f'(x_0^+) = f'(x_0^-)
\end{equation*}

\subsection{Ableitungsrechenregeln}
Seien $f, g: D \rightarrow \R$ differentierbar in $x_0 \in D$, dann gilt:
\begin{enumerate}[label= (\alph*)]
    \item $(f+g)'(x_0) = f'(x_0) + g'(x_0)$
    \item $(f \cdot g)'(x_0) = f'(x_0) \cdot g(x_0) +
                                f(x_0) \cdot g'(x_0)$
    \item Falls $g(x_0) \neq 0$: $\left( \dfrac{f}{g} \right) '(x_0) =
            \dfrac{f'(x_0) g(x_0) - f(x_0) g'(x_0)}{{g(x_0)}^2}$
\end{enumerate}

\subsection{Alternative Definition der Ableitung}
Sei $f: D \subseteq \R \rightarrow \R$ und $x_0 \in D$. Dann gilt:
f differenzierbar in $x_0 \Leftrightarrow$
\begin{equation*}
    \exists A \in \R \text{ und } r: D \rightarrow \R \text{ mit }
    \lim_{x \rightarrow x_0} r(x) = 0 \text{ so dass gilt: }
    f(x) = f(x_0) + A \cdot (x - x_0)  + r(x) \cdot (x-x_0)
\end{equation*}

\subsection{Zusammenhang Differntierbarkeit --- Stetigkeit}
Ist $f: D \rightarrow \R$ differentierbar in $x_0 \in D \Rightarrow$
$f$ stetig in $x_0$

\subsection{Differentiation von Potenzreihen}
Sei $f(x) = \sum_{k=0}^\infty a_k {(x-x_0)}^k$ eine Potenzreihe mit $R>0$, dann
ist $f$ für $x$ mit $\abs{x-x_0}<R$ differentierbar, und es gilt:
\begin{equation*}
    f'(x) = \sum_{k=1}^\infty a_k \cdot k \cdot {(x - x_0)}^{k-1}
\end{equation*}

\subsubsection{Bemerkung}
Der Konvergenzradius von $f'(x)$ ebenfalls $R$ ist.

\subsection{Ableitung der Umkehrfunktion}
Seien $I, J \subseteq \R$ Intervalle und $f: I \rightarrow J$ sei differentiarbar
und bijektiv, dann ist auch $f^{-1}: J \rightarrow I$ differentierbar und es
gilt:
\begin{equation*}
    {(f^{-1})}'(y_0) = \frac{d}{dx} f^{-1}(y_0) =
    \frac{1}{f'(f^{-1}(y_0))} \forall y_0 \in J \text{ für ein } y_0=f(x_0)
    \text{ und } f'(y_0) \neq 0
\end{equation*}

\subsection{Ketternregel}
Seien $f: A \rightarrow B$, $g: B \rightarrow \R$ mit $A,B \subseteq \R$
differentierbar auf $A$ bzw. $B$, dann ist auch $g \circ f$ auf $A$
differentierbar und es gilt:
\begin{equation*}
    (g \circ f)'(x_0) = g'(f(x_0)) \cdot f'(x_0)\ \forall x_0 \in A
\end{equation*}

\section{Mittelwertsätze}

\subsection{Satz von Rolle}
Sei $f: [a, b] \rightarrow \R$ stetig und auf $(a,b)$ differentierbar. Falls
$f(a) = f(b)$ gilt, existiert ein $x_0 \in (a,b)$ mit $f'(x_0)=0$

\subsection{Definition lokaler Extrempunkt}
Sei $f: D \rightarrow \R$ und $x_0 \in D$. Dann besitzt$f$ in $x_0$ ein
lokales Maximum (bzw. Minimum)$:\Leftrightarrow$
\begin{equation*}
    \exists \delta > 0: f(x) \leq f(x_0)
    \text{ (bzw. } f(x) \geq f(x_0) \text{) } \forall x \in D \cap U_\delta(x_0)
\end{equation*}

\subsection{Notwendige Bedingung für lokale Extrema}
Sei $f: D \rightarrow \R$ differentierbar in $x_0 \in D$ und $x_0$ sei kein
Randpunkt, dann gilt:

Liegt bei $x_0$ ein lokales Maximum/Minimum $\Rightarrow f'(x_0)=0$.
