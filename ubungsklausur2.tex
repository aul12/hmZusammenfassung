\documentclass[10pt]{article}
\usepackage[utf8]{inputenc}
\usepackage{amsfonts}
\usepackage{amsmath}
\usepackage{commath}
\usepackage[ngerman]{babel}
\usepackage{enumitem}
\usepackage{booktabs}


\newcommand{\K}{$\mathbb{K}$}
\newcommand{\N}{\mathbb{N}}
\newcommand{\Z}{\mathbb{Z}}
\newcommand{\Q}{\mathbb{Q}}
\newcommand{\R}{\mathbb{R}}
\newcommand{\C}{\mathbb{C}}
\newcommand{\an}{{(a_n)}_{n=1}^\infty}
\newcommand{\sn}{{(s_n)}_{n=1}^\infty}
\newcommand{\ReP}{\operatorname{Re}}
\newcommand{\ImP}{\operatorname{Im}}
\newcommand*{\nthSqrt}[2]{\sqrt[\leftroot{-2}\uproot{2}#1]{#2}}


\setlength\parindent{0pt}

\title{Zusammenfassung HM I --- Übungsklausur 2}
\author{Paul Nykiel}

\begin{document}
    \maketitle
    \pagebreak
    \tableofcontents
    \pagebreak

    \part{Grenzwerte}
    \section{Gruppen und Körper}
    \subsection{Gruppen}
    Eine Gruppe ist definiert als ein Tuppel aus einer (nicht-leeren) Menge
    und einer Verknüpfung.
    Eine Gruppe erfüllt die folgenden Axiome (seien $a,b,c \in \mathbb{G}$):
    \begin{align*}
        a \circ (b \circ c) &= (a \circ b) \circ c &&\text{(Assoziativität)}\\
        a \circ \varepsilon &= a &&\text{(Rechtsneutrales Element)}\\
        a \circ a' &= \varepsilon  &&\text{(Rechtsinverses Element)}
    \end{align*}
    Eine abelsche Gruppe erfüllt des weiteren:
    \begin{align*}
        a \circ b &= b \circ a  &&\text{(Kommutativität)}
    \end{align*}

    \subsection{Körper}
    Ein Körper ist definiert als eine Menge mit mindestens zwei Elementen
    (0 und 1) und zwei Verknüfungen.
    \begin{eqnarray*}
        +: \mathbb{K} \times \mathbb{K} &\rightarrow& \mathbb{K}\\
        \cdot: \mathbb{K} \times \mathbb{K} &\rightarrow& \mathbb{K}
    \end{eqnarray*}
    \K\ ist bezüglich der Addition und der Multiplikation (genauer: $\mathbb{K} \backslash \{0\}$) ein abelscher
    Körper, das heißt es gilt (seien $a,b,c \in \mathbb{K}$):
    \begin{align*}
        a + (b + c) &= (a + b) + c &&\text{(Assoziativität bez.\ der Addition)}\\
        a + 0 &= a &&\text{(Existenz einer 0)}\\
        a + (-a) &= 0 &&\text{(Existenz eines Inversen bez.\ der Addition)}\\
        a + b &= b + a &&\text{(Kommutativität bez.\ der Addition)}\\
        a \cdot (b \cdot c) &= (a \cdot b) \cdot c &&\text{(Assoziativität bez.\ der Multiplikation)}\\
        a \cdot 1 &= a &&\text{(Existenz einer 1)}\\
        a \cdot a^-1 &= 1\quad \forall a \neq 0 &&\text{(Existenz eines Inversen bez.\ der Multiplikation)}\\
        a \cdot b &= b \cdot a &&\text{(Kommutativität bezüglich der Multiplikation)}
    \end{align*}
    außerdem gilt:
    \begin{align*}
        a \cdot (b + c) &= (a \cdot b) + (a \cdot c) &&\text{(Distributivgesetz)}
    \end{align*}
    \textbf{Bem.:}
    $\Q$, $\R$ und $\C$ sind Körper.
    $\Z$ und $\N$ nicht (kein additiv inverses bei $\N$,
    kein multiplikativ inverses bei beiden).


    \subsection{Angeordnete Körper}
    Ein Körper heißt angeordent wenn folgende Axiome erfüllt sind
    (seien $a,b,c \in \mathbb{K}$):
    \begin{eqnarray*}
        a<b\ \vee\ &b<a&\ \vee\ a=b\\
        a<b\ \wedge\ b<c\ &\Rightarrow&\ a<c\\
        a<b &\Rightarrow& a+c<b+c\\
        a<b \wedge c>0 &\Rightarrow& a*c<b*c
    \end{eqnarray*}
    \textbf{Bem.:}
    $\Q$ und $\R$ sind angeordnete Körper. Für $\C$
    kann keine Ordnungsrelation definiert werden so das alle Axiome erfüllt
    sind.

    \subsubsection{Gebräuchliche Definition zu angeordenten Körpern}
    Für gewöhnlich gilt $0<1$.

     Die Ordnungsrelation wird dann definiert durch:
    \begin{eqnarray*}
        2&:=& 1+1 \\
        3&:=& 2+1 \\
        4&:=&3+1 \\
        &\vdots&
    \end{eqnarray*}

     Die Natürlichen Zahlen werden Induktiv definiert:
    \begin{enumerate}
        \item $1 \in \mathbb{N}$
        \item $n \in \mathbb{N} \Rightarrow (n+1)\in\mathbb{N}$
    \end{enumerate}

    \textbf{Bem:}
    Aus 2.\ lässt sich direkt ableiten das $\N$ nach oben unbeschränkt ist (Archimedisches Prinzip).

    \subsubsection{Vollständig Angeordnete Körper}
    Ein Körper heißt Vollständig, falls jede nach oben beschränkte, nicht-leere
    Teilmenge ein Supremum besitzt.

    $\Rightarrow$ $\R$ ist der einzige Vollständig angeordnete Körper.

    \textbf{Bem:} $\Q$ ist nicht vollständig angeordnet, da
    $A := \{x | x^2 \leq 2\} \subset \mathbb{Q}$ kein obere Schrank besitzt
    (obere Schranke ist $\sqrt{2} \notin \mathbb{Q}$).


    \subsection{Minimum und Maximum}
    Sei \K\ ein angeordnter Körper und $A \subset$ \K\ dann heißt
    $m$ Minimum falls gilt:
    \begin{enumerate}
        \item $m \in \mathbb{K}$
        \item $a \geq m\ \forall a \in A$
    \end{enumerate}
    Analog ist das Maximum definiert:
    Sei \K\ ein angeordnter Körper und $A \subset$ \K\ dann heißt
    $m$ Maximum falls gilt:
    \begin{enumerate}
        \item $m \in \mathbb{K}$
        \item $a \leq m\ \forall a \in A$
    \end{enumerate}
    \textbf{Schreibweisen:}
    $m = \min{(A)}$ bzw.\ $m = \max{(A)}$\\
    \textbf{Bem.:}
    Minimum und Maximum exisitieren nicht immer.\\
    \textbf{Beispiel:} $A := \{x | x>0\}\subset\ \mathbb{R}$
    hat nicht 0 als Minimum da $0 \notin A$ und kein beliebiges $m$ da $\tilde{m} := \frac{m}{2} < m\ \forall m \in A$

    \subsection{Obere und untere Schranke}
    Sei \K\ ein angeordenter Körper und $A \subset$ \K\ dann ist $s$ untere
    Schranke falls gilt:
    \begin{itemize}
        \item $s \leq a\ \forall a \in A$
    \end{itemize}

     Analog ist die obere Schranke definiert:
    Sei \K\ ein angeordenter Körper und $A \subset$ \K\ dann ist $s$ obere
    Schranke falls gilt:
    \begin{itemize}
        \item $s \geq a\ \forall a \in A$
    \end{itemize}

    \textbf{Bem.:} Hat eine Menge eine obere (bzw.\ untere) Schranke
    heißt er nach oben (bzw.\ unten) beschränkt. Ist eine Menge nach unten und
    oben beschränkt bezeichnet man sie als beschränkt.

    \subsection{Supremum und Infimum}
    $s$ heißt Infimum (größte untere Schranke) falls gilt:
    \begin{itemize}
        \item $s$ ist untere Schranke
        \item Falls $\tilde{s}$ ebenfalls untere Schranke ist gilt
        $s\geq\tilde{s}$
    \end{itemize}

     Analog ist das Supremum definiert: $s$ heißt Supremum (kleinste obere Schranke) falls gilt:
    \begin{itemize}
        \item $s$ ist obere Schranke
        \item Falls $\tilde{s}$ ebenfalls obere Schranke ist gilt
        $s\leq\tilde{s}$
    \end{itemize}

    \textbf{Bem.:}
    Wenn Minimum (bzw. Maximum) existieren sind diese gleich dem
    Infimum (bzw. Supremum).

    \textbf{Schreibweise:}
    $s = \inf{(A)}$ bzw. $s = \sup{(A)}$


    \section{Folgen}
    Eine Folge $a_n$ ist definiert als eine Funktion:
    \begin{equation*}
        a_n := \varphi: \mathbb{N} \rightarrow \mathbb{M} \subset \mathbb{R}
    \end{equation*}
    oder auch $\an$.

    \subsection{Konvergenz}
    Eine Folge $a_n$ heißt konvergent wenn gilt:
    \begin{equation*}
        \forall \varepsilon>0\ \exists~n_0(\varepsilon):\ \abs{a_n - a} < \varepsilon\ \forall n > n_0(\varepsilon)
    \end{equation*}

    \textbf{Bem.:}
    Der Grenzwert ist eindeutig, d.h.\ es existiert nur ein Grenzwert.

    \subsubsection{Schreibweise}
    Falls $a_n$ gegen $a$ konvergiert schreibt man:
    \begin{equation*}
        \lim_{n \rightarrow \infty} a_n = a
    \end{equation*}

    \subsection{Bestimmte Divergenz}
    Eine Folge $a_n$ heißt bestimmt Divergent wenn gilt
    \begin{equation*}
        \forall x \in \mathbb{R}\ \exists n(x):\ a_n>x \text{ bzw. } a_n<x
    \end{equation*}
    \textbf{Schreibweise:}
    \begin{equation*}
        \lim_{n \rightarrow \infty} a_n = \infty \text{ bzw. } -\infty
    \end{equation*}

    \subsection{Beschränktheit}
    Eine Folge heißt beschränkt wenn gilt:
    \begin{equation*}
        \abs{a_n} < c\ \forall n
    \end{equation*}

    \subsubsection{Beschränktheit nach oben/unten}
    Eine Folge heißt nach oben (bzw.~unten) beschränkt wenn gilt:
    \begin{equation*}
        a_n < c\ \forall n \in \mathbb{N} \text{ bzw. }a_n > c\ \forall n \in \mathbb{N}
    \end{equation*}

    \subsection{Zusammenhang Konvergenz --- Beschränktheit}
    Jede konvergente Folge ist beschränkt.

    \subsection{Grenzwertrechenregeln}
    Seien $\an$, ${(b_n)}_{n=1}^\infty$, ${(c_n)}_{n=1}^\infty$ Folgen
    in $\C$ mit:
    \begin{equation*}
        \lim_{n \rightarrow \infty} a_n = a \text{ und }
        \lim_{n \rightarrow \infty} b_n = b
    \end{equation*}

    Dann gilt:

    \begin{itemize}
        \item $\lim\limits_{n \rightarrow \infty} \abs{a_n} = \abs{a}$
        \item $\lim\limits_{n \rightarrow \infty}(a_n + b_n) = a + b$
        \item $\lim\limits_{n \rightarrow \infty}(a_n \cdot b_n) = a \cdot b$
        \item Falls $b \neq 0$:
        $\lim\limits_{n \rightarrow \infty}\dfrac{a_n}{b_n} = \dfrac{a}{b}$
    \end{itemize}

    \subsection{Sandwich Theorem u.a.}
    Seien $\an$, ${(b_n)}_{n=1}^\infty$, ${(c_n)}_{n=1}^\infty$
    Folgen in $\R$ mit:
    \begin{equation*}
        \lim_{n \rightarrow \infty}a_n = a \text{, }
        \lim_{n \rightarrow \infty}b_n = b \text{ und } \gamma \in \mathbb{R}
    \end{equation*}

    Dann gilt:

    \begin{itemize}
        \item $a_n \leq \gamma\ \forall n \in \mathbb{N} \Rightarrow a \leq \gamma$
        \item $a_n \geq \gamma\ \forall n \in \mathbb{N} \Rightarrow a \geq \gamma$
        \item $a_n \leq b_n\ \forall n \in \mathbb{N} \Rightarrow a \leq b$
        \item $a_n \leq c_n \leq b_n\ \forall n \in \mathbb{N} \wedge a=b
        \Rightarrow c=\lim\limits_{n \rightarrow \infty} c_n = a = b$
    \end{itemize}

    \subsection{Monotonie}
    Eine Folge $\an$ in $\R$ heißt:
    \begin{itemize}
        \item Monoton wachsend falls: $a_{n+1} \geq {a_n}\ \forall n \in \mathbb{N}$ (Schreibweise: $a_n \nearrow$)
        \item Monoton fallend falls: $a_{n+1} \leq {a_n}\ \forall n \in \mathbb{N}$ (Schreibweise: $a_n \searrow$)
        \item Streng monoton wachsend falls: $a_{n+1} > {a_n}\ \forall n \in \mathbb{N}$ (Schreibweise: $a_n \uparrow$)
        \item Streng monoton fallend falls: $a_{n+1} < {a_n}\ \forall n \in \mathbb{N}$ (Schreibweise: $a_n \downarrow$)
    \end{itemize}

    \subsection{Zusammenhang Monotonie und Beschränktheit}
    Jede Monotone und beschränkte Folge konvergiert.

    \section{Häufungswerte}
    Häufungswerte sind Grenzwerte einer Teilfolge.

    \subsection{Teilfolgen}
    Eine Folge ${(b_n)}_{n=1}^\infty$ heißt Teilfolge von $\an$, wenn
    eine streng monotone Funktion $\varphi: \mathbb{N} \rightarrow \mathbb{N}$ exisitiert
    mit $b_n = a_{\varphi(n)}$.

    \subsection{Teilfolgen einer Konvergenten Folge}
    Sei $\an$ eine konvergente Folge in $\C$ mit:
    $\lim\limits_{n \rightarrow \infty} a_n = a$ und ${(b_n)}_{n=1}^\infty$
    sei eine Teilfolge. Dann gilt $\lim\limits_{n \rightarrow \infty} b_n = a$.

    \subsection{Häufungswerte} Sei $\an$ eine Folge in $\C$. Dann heißt
    $a \in\C$\ ein Häufungswert einer Folge, falls eine Teilfolge gegen $a$ konvergiert.

    \subsection{Limes superior/inferior}
    Sei $\an$ eine reele Folge, dann heißt:
    \begin{equation*}
        \lim_{n \rightarrow \infty} \sup a_n :=
        \varlimsup_{n \rightarrow \infty} a_n :=
        \sup \{ x \in \R, a_n > x\ \infty\text{-oft}\}
    \end{equation*}
    der Limes superior von $\an$ und
    \begin{equation*}
        \lim_{n \rightarrow \infty} \inf a_n :=
        \varliminf_{n \rightarrow \infty} a_n :=
        \inf \{ x \in \R, a_n < x\ \infty\text{-oft}\}
    \end{equation*}
    der Limes inferior von $\an$.

    \subsection{Konvergenz und limsup/liminf}
    Eine beschränkte Folge $\an$ in $\R$ konvergiert $\Leftrightarrow$
    \begin{equation*}
         \varlimsup_{n \rightarrow \infty} a_n
        = \varliminf_{n \rightarrow \infty} a_n
    \end{equation*}

    \subsection{Satz von Bolzano-Weierstraß}
    Jede beschränkte Folge in $\C$ besitzt eine konvergente Teilfolge.

    \subsection{Cauchy-Kriterium}
    Sei $\an$ eine Folge in $\C$, dann gilt
    \begin{equation*}
        \an \text{ konv.} \Leftrightarrow
        \forall \varepsilon > 0\ \exists n_0(\varepsilon):\
        \abs{a_n - a_m} < \varepsilon\ \forall n,m > n_0(\varepsilon)
    \end{equation*}
    \textbf{Bem.:} Im Gegensatz zur Definition der Folgenkonvergenz muss der
    Grenzwert nicht bekannt sein.

    \section{Unendliche Reihen}
    \subsection{Definition}
    Sei $\an$ eine Folge in $\C$, dan  heißt die durch
    \begin{equation*}
        s_n = \sum_{k=1}^n a_k
    \end{equation*}
    definiert Folge $\sn$ eine Folge von Partialsummen der
    unendlichen Reihe:
    \begin{equation*}
        \sum_{k=1}^\infty a_k
    \end{equation*}
    Falls die Folge $\sn$ konvergiert setzten wir:
    \begin{equation*}
        \lim_{n \rightarrow \infty} s_n =: \sum_{k=1}^\infty a_k
    \end{equation*}

    \subsection{Cauchy-Kriterium für unendliche Reihen}
    Sei $\sum_{k=1}^\infty a_k$ eine $\infty$-Reihe, dann gilt:
    \begin{equation*}
        \sum_{k=1}^\infty a_k \text{ konv.} \Leftrightarrow
        \forall \varepsilon > 0\ \exists n_0(\varepsilon):\
        \abs{\sum_{k=m}^n a_k} < \varepsilon\
        \forall n, m>n_0(\varepsilon)
    \end{equation*}
    und:
    \begin{equation*}
        \sum_{k=1}^\infty a_k \text{ konv.} \Rightarrow
        \lim_{n \rightarrow \infty} a_n = 0
    \end{equation*}

    \subsection{Grenzwertrechenregeln für unendliche Reihen}
    Seien
    \begin{equation*}
        \sum_{n=1}^\infty a_k \text{ und } \sum_{n=1}^\infty b_k
        \text{ gegeben und } \alpha,\beta \in \C
    \end{equation*}
    dann gilt:
    \begin{enumerate}[label= (\alph*)]
        \item
            \begin{eqnarray*}
                \sum_{n=1}^\infty a_k \text{ und } \sum_{n=1}^\infty b_k \text{ konv.:}\\
                \Rightarrow \sum_{k=1}^\infty(\alpha a_k &+& \beta b_k) \text{ konv.}\\
                \text{und: }
                \sum_{k=1}^\infty(\alpha a_k + \beta b_k) &=& \alpha \sum_{n=1}^\infty a_k
                + \beta \sum_{n=1}^\infty b_k
            \end{eqnarray*}\\
        \item
            \begin{equation*}
                \sum_{k=1}^\infty a_k \text{ konv.} \Leftrightarrow
                \sum_{k=1}^\infty \ReP(a_k) \text{ und }
                \sum_{k=1}^\infty \ImP(a_k) \text{ konv.}
            \end{equation*}
        \item
            \begin{equation*}
                \sum_{k=1}^\infty a_k \text{ konv.} \Leftrightarrow
                \text{ die Restreihe } R_n := \sum_{k=n}^\infty a_k \text{ konv.\ gegen }0\\
                \Rightarrow \lim_{n \rightarrow \infty} R_n = 0
            \end{equation*}
    \end{enumerate}

    \subsection{Positive Folgen}
    Es sei $\an$ eine Folge mit $\an \in \left[0, \infty\right)$ dann gilt:
    \begin{equation*}
        \sum_{k=1}^\infty a_k \text{ konv. } \Leftrightarrow
        \text{ Folge der Partialsummen }\sum_{k=1}^n a_k \text{ ist beschr.}
    \end{equation*}

    \subsection{Leibniz-Kriterium}
    Sei $\an$ eine monoton fallende, stetige Folge. Dann gilt falls
    $\lim\limits_{n \rightarrow \infty} a_n = 0$ ist, konv.\ die sogennante
    alternierende Reihe
    \begin{equation*}
        \sum_{k=1}^\infty {(-1)}^k a_k
    \end{equation*}


    \subsection{Absolute Konvergenz}
    Eine Reihe $\sum_{k=1}^\infty a_k$ heißt absolut konvergent, wenn
    \begin{equation*}
        \sum_{k=1}^\infty \abs{a_k}
    \end{equation*}
    konvergiert.

    \textbf{Bem.:} Jede absolut konvergente Reihe ist auch konvergent.

    \subsection{Majorantenkriterium}
    Seien $\sum_{k=1}^\infty a_k$ und $\sum_{k=1}^\infty b_k$ mit $b_k \geq 0$
    gegeben.

    Wenn $\sum_{k=1}^\infty b_k$ konv.\ und ein $c>0$ ex.\ mit
    $$\abs{a_k} \leq c \cdot \abs{b_k}$$ für fast alle k, dann konv.
    $\sum_{k=1}^\infty a_k$ absolut.

    \subsection{Minorantenkriterium}
    Falls ein $c > 0$ ex.\ mit $a_k \geq c \cdot b_k > 0$ für fast alle k,
    dann:
    \begin{equation*}
        \sum_{k=1}^\infty b_k \text{ div. }
        \Rightarrow \sum_{k=1}^\infty a_k \text{ div.}
    \end{equation*}

    \subsection{Wurzel- und Quotientenkriterium}
    Sei $\sum_{k=1}^\infty a_k$ gegeben. Dann gilt:
    \begin{enumerate}[label= (\alph*)]
        \item Wenn
            \begin{equation*}
                \varlimsup_{n \rightarrow \infty}
                \nthSqrt{n}{\abs{a_n}} < 1
            \end{equation*}
            gilt, dann konv. $\sum_{k=1}^\infty a_k$ absolut.

            Wenn
            \begin{equation*}
                \varlimsup_{n \rightarrow \infty}
                \nthSqrt{n}{\abs{a_n}} > 1
            \end{equation*}
            gilt, dann div. $\sum_{k=1}^\infty a_k$.
        \item Wenn $a_n \neq 0\ \forall n$ und
            \begin{equation*}
                \varlimsup_{n \rightarrow \infty}
                \abs{\frac{a_{n+1}}{a_n}} < 1
            \end{equation*}
            gilt, dann konv. $\sum_{k=1}^\infty a_k$ absolut.

            Wenn $a_n \neq 0\ \forall n$ und
            \begin{equation*}
                \varlimsup_{n \rightarrow \infty}
                \abs{\frac{a_{n+1}}{a_n}} > 1
            \end{equation*}
            gilt, dann divergiert. $\sum_{k=1}^\infty a_k$.
    \end{enumerate}
    \textbf{Bem.:} Wenn das Wurzelkriterium keine Aussage macht, kann das
    Quotientenkriterium trotzdem eine Aussage machen.

    \subsection{Umordnung einer Reihe}
    Eine Reihe $\sum_{k=1}^\infty b_k$ heißt Umordnung der Reihe
    $\sum_{k=1}^\infty a_k$,
    wenn eine bij. Abb $\varphi: \N \rightarrow \N$ ex.\ mit $b_k = a_{\varphi(k)}$.

    \textbf{Bem.:}
    Die Reihe konvergiert nur gegen den selben Wert, wenn $\sum_{k=1}^\infty a_k$
    absolut konvergent ist.

    \subsection{Cauchy-Produkt}
    Die Reihen  $\sum_{k=1}^\infty b_k$ und $\sum_{k=1}^\infty a_k$ seien absolut
    konv.. Dann gilt:
    \begin{equation*}
        \left(\sum_{k=0}^\infty a_k\right) \cdot \left(\sum_{k=0}^\infty b_k\right) =
        \sum_{k=0}^\infty \left(\sum_{j=0}^k a_j \cdot b_{k-j}\right) =
        \sum_{k=0}^\infty c_k
    \end{equation*}
    und $\sum_{k=0}^\infty c_k$ konv.\ ebenfalls absolut.

    \subsection{Cauchy-Verdichtungssatz}
    \begin{equation*}
        \sum_{n=1}^\infty a_n \text{ konv. } \Leftrightarrow
        \sum_{k=1}^\infty 2^k a_{2^k} \text{ konv.}
    \end{equation*}


    \section{Potenzreihen}

    \subsection{Definition}
    Sei $\an$ eine Folge in $\C$ und $z_0 \in \C$. Dann heißt
    \begin{equation*}
        \sum_{k=0}^\infty a_k \cdot {(z - z_0)}^k
    \end{equation*}
    eine Potenzreihe mit Entwicklungspunkt $z_0$ und Koeffizienten
    $a_n$.

    \textbf{Bem.:} Viele wichtige Funktionen können als Potenzreihen dargestellt
    werden.

    \subsection{Hadamard}
    Sei $\sum_{k=0}^\infty a_k {(z-z_o)}^k$ eine PR.\ Definiere
    \begin{equation*}
        R := \frac{1}{\varlimsup\limits_{n \rightarrow \infty} \nthSqrt{n}{\abs{a_n}}}
    \end{equation*}

    Dabei sei $R:=\infty$, falls
    $\varlimsup\limits_{n \rightarrow \infty} \nthSqrt{n}{\abs{a_n}} = 0$ und
    $R=0$ falls
    $\varlimsup\limits_{n \rightarrow \infty} \nthSqrt{n}{\abs{a_n}} = \infty$.

    Dann konv.\ die PR absolut, falls $\abs{z - z_0} < R$ und divergiert falls
    $\abs{z - z_0} > R$.

    \textbf{Bem. I:} Für $\abs{z - z_0} = R$ wird keine Aussage gemacht.

    \textbf{Bem. II:} $R$ heißt der Konvergenzradius der Potenzreihe.

    \subsection{Hinweis}
    Es gilt:
    \begin{equation*}
        \lim_{n \rightarrow \infty} \nthSqrt{n}{n} = 1
    \end{equation*}

    \subsection{Integration und Differentiation von Potenzreihen}
    Sei $\sum_{k=0}^\infty a_k {(z - z_0)}^k$ mit Konvergenzradius $R$. Dann besitzen
    auch die Potenzreihen
    \begin{equation*}
        \sum_{k=0}^\infty k\: a_k {(z - z_0)}^{k-1} \text{ und }
        \sum_{k=0}^\infty \frac{a_k}{k+1} {(z-z_0)}^{k+1}
    \end{equation*}
    den Konvergenzradius R.

    \subsection{Cauchy-Produkt für Potenzreihen}
    Seien $\sum_{k=0}^\infty a_k {(z-z_0)}^k$ und
    $\sum_{k=0}^\infty b_k {(z-z_0)}^k$ Potenzreihen, die den Konvergenzradius
    $R_1$ bzw.\ $R_2$ besitzen. Dann besitzt
    \begin{equation*}
            \sum_{k=0}^\infty c_k {(z-z_0)}^k \text{ mit }
            c_k = \sum_{l=0}^k a_l \cdot b_{k-l}
    \end{equation*}
    den Konvergenzradius $R = \min \{R_1, R_2\}$.

    \subsection{Wichtige Potenzreihen}
    \begin{enumerate}[label= (\alph*)]
        \item Die Expontentialfunktion ist definiert durch:
            \begin{equation*}
                \exp: \C \rightarrow \C\quad z \mapsto \exp(z) :=
                \sum_{k=0}^\infty \frac{z^k}{k!}
            \end{equation*}
        \item Die Trigonometrischen Funktionen sind definiert durch:
            \begin{eqnarray*}
                \sin: \C \rightarrow \C\quad z \mapsto \sin(z) &:=&
                \sum_{k=0}^\infty \frac{{(-1)}^k}{(2k+1)!}z^{2k+1}\\
                \cos: \C \rightarrow \C\quad z \mapsto \cos(z) &:=&
                \sum_{k=0}^\infty \frac{{(-1)}^k}{(2k)!}z^{2k}
            \end{eqnarray*}
        \item Tangens und Cotangens sind dann definiert als:
            \begin{eqnarray*}
                \tan: \{z \in \C:\ \cos(z) \neq 0 \} \rightarrow \C\quad
                z \mapsto \tan(z)&:=&\frac{\sin(z)}{\cos(z)}\\
                \cot: \{z \in \C:\ \sin(z) \neq 0 \} \rightarrow \C\quad
                z \mapsto \cot(z)&:=&\frac{\cos(z)}{\sin(z)}
            \end{eqnarray*}
    \end{enumerate}

    \subsection{Alternative Definiton der Exponentialfunktion}
    \begin{equation*}
        \forall z \in \C \text{ gilt }
        \lim_{n \rightarrow \infty} {\left(1 + \frac{z}{n}\right)}^n = \exp{(z)}
    \end{equation*}

    \section{Funktionsgrenzwerte}

    \subsection{Bemerkung}
    In diesem Intervall bezeichnet $I$ stets ein offenes Intervall und
    $\overline{I}$ dessen sog.\ Abschluss z.B.:
    \begin{enumerate}[label= (\alph*)]
        \item $I = (a, b)$ und $\overline{I} = [a, b]$
        \item $I = (-\infty, b)$ und $\overline{I} = (-\infty, b]$
        \item $I = (a, \infty)$ und $\overline{I} = [a, \infty)$
        \item $I = (\infty, \infty)$ und $\overline{I} = (\infty, \infty)$
    \end{enumerate}

    \subsection{Epsilon-Umgebung}
    Für $x_0 \in \R$ und $\varepsilon > 0$ heißt
    \begin{equation*}
        U_e(x_0) := \{ x \in \R: \abs{x - x_0}<\varepsilon \} =
        (x_0 - \varepsilon, x_0+\varepsilon)
    \end{equation*}
    die $\varepsilon$-Umgebung von $x_0$. Und
    \begin{equation*}
        \dot{U}_e(x_0) := U_e(x_0)\backslash \{0\} =
        (x_0-\varepsilon, x_0) \cup (x_0, x_0+\varepsilon)
    \end{equation*}
    die punktierte $\varepsilon$-Umgebung von $x_0$.

    \subsection{Funktionsgrenzwerte (über Delta-Epsilon-Kriterium)}
    Sei $f: I \rightarrow \R$ und $x_0 \in I$
    \begin{enumerate}[label= (\alph*)]
        \item $f$ konv.\ gegen ein $a\in \R$ für $x \rightarrow x_0$
            (kurz: $\lim\limits_{x \rightarrow x_0} f(x)=a$) wenn gilt
            \begin{equation*}
                \forall \varepsilon > 0\ \exists \delta(\varepsilon):\
                \abs{f(x) - a} < \varepsilon\ \forall x \text{ mit }
                \abs{x- x_0} < \delta(\varepsilon) \text{ und } x \neq x_0
            \end{equation*}
            Schreibweise:
            \begin{equation*}
                \lim_{x \rightarrow x_0} f(x) = a
                \text{ oder } f(x)=a \text{ für } x \rightarrow x_0
            \end{equation*}
        \item Sei $x_o \in I$, dann konv.\ f einseitig von links
            gegen $a\in\R$ wenn gilt:
            \begin{equation*}
                \forall \varepsilon > 0\  \exists \delta(\varepsilon):\
                \abs{f(x) -a} < \varepsilon\ \forall x \in
                (x_0 - \delta{\varepsilon}, x_0)
             \end{equation*}
            Schreibweise:
            \begin{equation*}
                \lim_{x \rightarrow x_{0^-}} f(x) = a
            \end{equation*}
        \item Sei $x_o \in I$, dann konv.\ f einseitig von rechts
            gegen $a\in\R$ wenn gilt:
            \begin{equation*}
                \forall \varepsilon > 0\  \exists \delta(\varepsilon):\
                \abs{f(x) -a} < \varepsilon\ \forall x \in
                (x_0 , x_0 + \delta{\varepsilon})
             \end{equation*}
            Schreibweise:
            \begin{equation*}
                \lim_{x \rightarrow x_{0^+}} f(x) = a
            \end{equation*}
        \item Sei $I = (\alpha, \infty)$ (bzw. $I = (-\infty, \beta)$) dann
            konv. $f$ gegen $a$ für $x \rightarrow \infty$
            (bzw. $x \rightarrow -\infty$) wenn gilt:
            \begin{equation*}
                \forall \varepsilon > 0\ \exists x_1(\varepsilon):\
                \abs{f(x) - a}<\varepsilon\ \forall x \in I:\ x > x_1(\varepsilon)
                \text{ (bzw. } x < x_1(\varepsilon) \text{)}
            \end{equation*}
    \end{enumerate}

    \subsection{Folgenkriterium}
    Sei $f: I \rightarrow \R$ und $x_0 \in \overline{I}, u \in \R$ dann gilt
    $\lim\limits_{x \rightarrow \infty} f(x) = a \Leftrightarrow$

    \begin{equation*}
        \begin{cases}
            \text{Für eine beliebe Folge } {(x_n)}_{n=1}^\infty \text{ mit}\\
            \text{(i)} x_n \neq x_0 \forall n\\
            \text{(ii)} \lim\limits_{n \rightarrow \infty} x_n = x_0\\
            \text{gilt stets:}\\
            \lim\limits_{n \rightarrow \infty} f(x_n) = a
        \end{cases}
    \end{equation*}

    \subsection{Rechenregeln für Funktionsgrenzwerte}
    Seien $f, g: I \rightarrow \R$ und $x_0 \in I$ und gelte
    \begin{equation*}
        \lim_{x \rightarrow x_0} f(x) = a \text{, }
        \lim_{x \rightarrow x_0} g(x) = b
    \end{equation*}
    Dann gilt:
    \begin{enumerate}[label= (\alph*)]
        \item
            \begin{equation*}
                \lim_{x \rightarrow x_0} (\alpha \cdot f(x)) = \alpha \cdot a
            \end{equation*}
        \item
            \begin{equation*}
                \lim_{x \rightarrow x_0} (g(x) + f(x)) = a + b
            \end{equation*}
        \item
            \begin{equation*}
                \lim_{x \rightarrow x_0} (g(x) \cdot f(x)) = a \cdot b
            \end{equation*}
        \item
            \begin{equation*}
                \lim_{x \rightarrow x_0} \left(\frac{f(x)}{g(x)}\right) = \frac{a}{b}
                \qquad\text{falls } b \neq 0
            \end{equation*}
    \end{enumerate}

    \subsection{Cauchy-Kriterium für Funktionsgrenzwerte}
    Sei $f: I \rightarrow \R$ und $x_0 \in I$ dann ex.
    $\lim\limits_{x \rightarrow x_0} f(x) \Leftrightarrow$
    \begin{equation*}
        \forall \varepsilon > 0\ \exists \delta(\varepsilon):\
        \abs{f(x) - f(y)}<\varepsilon\ \forall x,y \in I \text{ mit }
        0 < \abs{x-x_o} < \delta(\varepsilon) \text{ und }
        0 < \abs{y-x_0} < \delta(\varepsilon)
    \end{equation*}

    \subsection{Bestimmte Divergenz}
    Sei $f: I \rightarrow \R ,\ x_0 \in I$ dann definieren wir die bestimmte
    Divergenz (uneigentliche Konvergenz) von ($f \rightarrow \infty$) durch
    \begin{equation*}
        \lim_{x \rightarrow x_0} f(x) = \infty \Leftrightarrow
        \forall c>0\ \exists \delta(c): f(x) > c\ \forall x \text{ mit }
        0 < \abs{x - x_0} < \delta(c)
    \end{equation*}
    Analog definieren man links- und rechtsseitig Divergenz gegen $\infty$ bzw.
    $-\infty$.

    \subsection{Monotone Funktionen}
    Sei $f: I \rightarrow \R$ dann heißt (auf $I$)
    \begin{enumerate}[label= (\alph*)]
        \item monoton wachsend ($f \nearrow$), falls gilt
            \begin{equation*}
                x < y \Rightarrow f(x) \leq f(y)
            \end{equation*}
        \item streng monoton wachsend ($f \uparrow$), falls gilt
            \begin{equation*}
                x < y \Rightarrow f(x) < f(y)
            \end{equation*}
        \item monoton fallend ($f \searrow$), falls gilt
            \begin{equation*}
                x < y \Rightarrow f(x) \geq f(y)
            \end{equation*}
        \item streng monoton fallend ($f \downarrow$)
            \begin{equation*}
                x < y \Rightarrow f(x) > f(y)
            \end{equation*}
        \item monoton fallend falls $f$ monoton fallend oder monoton steigend ist
        \item streng monoton falls $f$ streng monoton fallend oder streng monoton
            steigend ist
        \item Beschränkt falls gilt:
            \begin{equation*}
                \exists c:\ \abs{f(x)} < c\ \forall x \in I
            \end{equation*}
    \end{enumerate}

    \subsection{Grenzwerte an Intervallgrenzen}
    Sei $a \leq b$ und $f: (a, b) \rightarrow \R$ monoton und beschränkt, dann
    ex.
    \begin{equation*}
        \lim_{x \rightarrow b^-} f(x) \text{ und } \lim_{x \rightarrow a^+} f(x)
    \end{equation*}

    \section{Stetigkeit}

    \subsection{Anschaulich}
    Graph einer Funktion kann ohne Absetzen gezeichnet werden $\Leftrightarrow$

    Es gibt keine Sprünge $\Leftrightarrow$

    $f: I \rightarrow \R$ an keiner Stelle $x_0 \in I$ ist ein Sprung $\Leftrightarrow$

    $\forall x_0 \in I:\ \lim\limits_{x \rightarrow x_0} f(x) = f(x_0)$


    \subsection{Stetigkeit: Delta-Epsilon-Kriterium}
    Sei $f: I \rightarrow \R$ und $x_0 \in I$, dann ist $f$ in $x_0$ stetig
    falls gilt:
    \begin{equation*}
        \forall \varepsilon > 0\ \exists \delta(\varepsilon):\
        \abs{f(x)-f(x_0)}<\varepsilon\ \forall x \in I \text{ mit }
        \abs{x-x_0}<\delta(\varepsilon)
    \end{equation*}
    Und f ist stetig (auf $I$), wenn $f$ in jedem $x_0 \in I$ stetig ist.

    \subsection{Bemerkungen}
    \begin{enumerate}[label= (\alph*)]
        \item $f$ ist stetig in $x_0 \Leftrightarrow$
            \begin{equation*}
                \lim_{x \rightarrow x_0} f(x) = f(x_0)
            \end{equation*}
            gilt.
        \item $f$ ist stetig in $x_0$ dann gilt:
            \begin{equation*}
                \lim_{n \rightarrow \infty} x_n = x_0 \Rightarrow
                \lim_{n \rightarrow \infty} f(x_n) = f(x_0)
            \end{equation*}
    \end{enumerate}

    \subsection{Rechenregeln für Stetigkeit}
    Sind $f,g: I \rightarrow \R$ stetig, dann sind auch die Funktionen
    \begin{enumerate}[label= (\alph*)]
        \item $c \cdot f$ (für $c \in \R$)
        \item $f + g$
        \item $f \cdot g$
        \item und falls $g(x) \neq 0 \forall x \in I \frac{f}{g}$
    \end{enumerate}
    stetig.

    Ist $f: I \rightarrow J, g: I \rightarrow \R$ und beide stetig dann
    ist auch $g \circ f$ stetig.

    \subsection{Stetigkeit von Potenzreihen}
    Sei $f(x) = \sum_{k=0}^\infty a_k {(x - x_0)}^k$ eine Potenzereihe mit
    Konvergenzradius $R > 0$, dann gilt für $x_1 \in U_R(x_0)$, dass
    $\lim\limits_{x \rightarrow x_1} f(x) = f(x_1)$ (d.h. Potenzreihen
    sind innerhalb des Konvergenzradius stetig).

    \subsection{Umgebung positiver Werte}
    Sei $f: I \rightarrow \R$ stetig in $x_0$, dann gilt:
    \begin{equation*}
        f(x_0) > 0 \Rightarrow \exists \delta > 0:\ f(x)>0\
        \forall x \in I \text{ mit } \abs{x - x_1} < \delta
    \end{equation*}

    \subsection{Zwischenwertsatz}
    Sei $D = [a, b]$ (also abgeschlossen) und $f: D \rightarrow \R$ stetig dann
    ex.\ zu jedem $y$ zwischen $f(a)$ und $f(b)$ ein $x \in [a,b]$ mit $f(x) =y$.

    \textbf{Genauer:}
    \begin{equation*}
        \forall y \in [m, M]\ \exists x \in [a,b] \text{ mit } f(x)=y
    \end{equation*}
    Wobei $m = \min \{ f(a), f(b) \}$ und $M = \max \{ f(a), f(b) \}$.

    \textbf{Bem.:}
    Bei einer Funktion ist das Bild eines Intervals wieder ein Interval. D.h.
    \begin{equation*}
        f([a,b]) = [c, d]
    \end{equation*}

    \pagebreak
    \part{Appendix}
    \section{Konvergenzkriterien}
    Zusammenfassung verschiedener Konvergenzkriterien nach Wikipedia (Seite: Konvergenzkriterium):
    \begin{center}
        \begin{tabular}{lcccccccp{2cm}}
             \toprule
             Kriterium & {nur f.\ mon. F.} & Konv. & Div. & abs. Konv. & Absch. & Fehlerabsch.\\
             \midrule
             Nullfolgenkriterium &  &  & x &  &  & \\
             Monotoniekriterium &  & x &  & x &  & \\
             Leibniz-Kriterium & x & x &  &  & x & x\\
             Cauchy-Kriterium &  & x & x &  &  & \\
             Abel-Kriterium & x & x &  &  &  & \\
             Dirichlet-Kriterium & x & x &  &  &  & \\
             Majorantenkriterium &  & x &  & x &  & \\
             Minorantenkriterium &  &  & x &  &  & \\
             Wurzelkriterium &  & x & x & x &  & x\\
             Integralkriterium & x & x & x & x & x & \\
             Cauchy-Kriterium & x & x & x & x &  & \\
             Grenzwertkriterium &  & x & x &  &  & \\
             Quotientenkriterium &  & x & x & x &  & x\\
             Gauß-Kriterium &  & x & x & x &  & \\
             Raabe-Kriterium &  & x & x & x &  & \\
             Kummer-Kriterium &  & x & x & x &  & \\
             Bertrand-Kriterium &  & x & x & x &  & \\
             Ermakoff-Kriterium & x & x & x & x &  & \\
             \bottomrule
        \end{tabular}
    \end{center}

    \section{Beweis-Ansätze}
    Ansatz für die einzelnen Beweise.
    \begin{center}
        \begin{tabular}{lp{5cm}}
            \toprule
                Lemma / Satz & Beweisansatz\\
            \midrule
                Eindeutigkeit des GW einer Folge & Zeige, dass GW a = GW b, nahrhafte 0\\
                Konvergente Folgen sind beschränkt & Nahrhafte 0, Dreiecks-ugl.\\
                Grenzwertrechenregeln & Nahrhafte 0, Dreiecks-ugl. \\
                $a_n < \gamma\ \forall n \Rightarrow a < \gamma$  & Ausgehend von a über nahrh. 0 zu Def Konvergenz \\
                $a_n < b_n\ \forall n \Rightarrow a<b$ & Definiere Hilfsfolge, argumentiere nach s.o \\
                SWT & Zeige, dass $-\varepsilon < c_n < \varepsilon$  (Quasi Epsilon-Schlauch) \\
                Monotoniekriterium & Da $\abs{a_n} < c \forall n$, argumentiere über das Supremum der Menge, die aus  besteht \\
                GW einer konv. Folge = GW jeder Teilfolge & Def. Konvergenz + Def Teilfolge \\
                Charakterisierung limSup und limInf & Argumentiere über Eigneschaften sup und inf \\
                Folge konv limSup = limInf & Hin: Eindeutigkeit des GW;\@Rück: Charakterisierung limSup und limInf \\
                Bolzano-Weierstraß & Zunächst für reelle Folge (trivial), dann für komplex: Realteil ist klar, Imaginärteil: Teilfolge konstruieren \\
                Cauchykriterium & Hin: nahrhafte 0; Rück: zeige Beschränktheit, dann folge daraus, dass ein HW ex und benutze diesen als GW-Kandidat \\
                Reihe konv.  Folge ist Nullfolge & Cauchy für Reihen \\
                GWRR für Reihen & GWRR für Folgen \\
                Reihe konv  0 & Restreihe als Differenz darstellen \\
                Leibniz & Cauchy für Reihen \\
                Absolut konv.\  konv. & Cauchy und Dreiecks-ugl.\\
                Majorantenkrit.-Kriterium & Cauchy\\
                Minorantenkrit. & Kontradiktion von Majorantenkrit.\\
                Wurzelkriterium & Majorantenkrit: geom. Summe über $Q:=q+\varepsilon<1$, in $q$ das Wurzelkrit einsetzen, Char. LimSup\\
                Quotientenkrit. & Majorantenkrit: setze in $q$ das Quotientenkrit ein u.\ arg. über LimSup \\
                Hadamard & Wurzelkrit + Fallunterscheidung für Sonderfälle\\
                Differenzieren / Integrieren von PR & Wurzelkriterium\\
                Lemma zu sin, cos und exp & Cauchy-Produkt + Definitionen\\
                $e^z \neq 0$ und $e^{-z} = \frac{1}{z}$ & Inverses Element der Multiplikation\\
                Pythagoras & 3. binomische Formel\\
                $e^x > 0 \forall x \in \R$ & Betrachte $x \geq 0$, angeordneter Körper\\
                $1+x \leq e^x \forall x \in \R$ & Bernoulli\\
                $x<y \Rightarrow e^x < e^y$ & nahrhafte 0\\
                Folgenkriterium & Hin: Def. Folgenkonv. und dann Def FunktionsGW einsetzen; \@Rück: Wähle versch. $\delta$ und zeige Widerspruch\\
                Cauchy für Funktionen & Hin: Def. FunktionsGW + nahrhafte 0; \@Rück: Cauchy für Folgen\\
                Grenzwerte an Intervallgrenzen & Argumentiere über Supremum / Infimum\\
                Verknüpfungen von stetigen Funktionen sind stetig & Folgenkriterium\\
                Potenzreihen sind innerhalb des KR stetig & Abschätzung: $\exists r>0 : abs{x-x_0 \text{bzw} x_1} \leq r$, dann einfach $abs{f(x)-f(x_1)}$ nach oben abschätzen\\
                Umgebung pos. Funktionswerte & Wähle $\varepsilon = frac{f(x_0)}{2}$, Def. Stetigkeit\\
                Zwischenwertsatz & Definiere $x_0 := sup{x \in [a,b] : f(x) \leq y}$ und zwei Hilfsfolgen, die gegen $x_0$ konvergieren\\
            \bottomrule
        \end{tabular}
    \end{center}
\end{document}
