\documentclass[12pt]{article}
\usepackage[utf8]{inputenc}
\usepackage{amsfonts}
\usepackage{amsmath}
\usepackage{commath}
\usepackage[ngerman]{babel}

\newcommand{\Korp}{$\mathbb{K}$}
\setlength\parindent{0pt}

\title{Zusammenfassung HM I}
\author{Paul Nykiel}

\begin{document}
    \maketitle
    \pagebreak
    \tableofcontents
    \pagebreak

    \part{Grenzwerte}
    \section{Gruppen und Körper}
    \subsection{Gruppen}
    Eine Gruppe ist definiert als ein Tuppel aus einer (nicht-leeren) Menge
    und einer Gruppe.
    Eine Gruppe erfüllt die folgenden Axiome (seien $a,b,c \in \mathbb{G}$):
    \begin{itemize}
        \item $a \circ (b \circ c) = (a \circ b) \circ c$ (Assoziativität)
        \item $a \circ \varepsilon = a$ (Rechtsneutrales Element)
        \item $a \circ a' = \varepsilon$ (Rechtsinverses Element)
    \end{itemize}
    Eine abelsche Gruppe erfüllt des weiteren:
    \begin{itemize}
        \item $a \circ b = b \circ a$ (Kommutativität)
    \end{itemize}


    \subsection{Körper}
    Ein Körper ist definiert als eine Menge mit mindestens zwei Elementen
    (0 und 1) und zwei Verknüfungen.
    \begin{itemize}
        \item $+: \mathbb{K} \times \mathbb{K} \rightarrow \mathbb{K}$
        \item $\cdot: \mathbb{K} \times \mathbb{K} \rightarrow \mathbb{K}$
    \end{itemize}
    \Korp\ ist bezüglich der Addition und der Multiplikation (genau: $\mathbb{K} \backslash \{0\}$) ein abelscher
    Körper, das heißt es gilt (seien $a,b,c \in \mathbb{K}$):
    \begin{itemize}
        \item $a + (b + c) = (a + b) + c$ (Assoziativität bez.\ der Addition)
        \item $a + 0 = a$ (Existenz einer Null)
        \item $a + (-a) = 0$ (Existenz eines Inversen bez.\ der Addition)
        \item $a + b = b + a$ (Kommutativität bez.\ der Addition)
        \item $a \cdot (b \cdot c) = (a \cdot b) \cdot c$ (Assoziativität bez.\ der Multiplikation)
        \item $a \cdot 1 = a$ (Existenz einer 1)
        \item $a \cdot a^-1 = 1$ (Existenz eines Inversen bez.\ der Multiplikation)
        \item $a \cdot b = b \cdot a$ (Kommutativität bezüglich der Multiplikation)
    \end{itemize}
    außerdem gilt:
    \begin{itemize}
        \item $a \cdot (b + c) = (a \cdot b) + (a \cdot c)$ (Distributivgesetz)
    \end{itemize}
    \textbf{Bem.:}
    $\mathbb{Q}$, $\mathbb{R}$ und $\mathbb{C}$ sind Körper.
    $\mathbb{Z}$ und $\mathbb{N}$ nicht (kein additiv inverses bei $\mathbb{N}$,
    kein multiplikativ inverses bei beiden).


    \subsection{Angeordnete Körper}
    Ein Körper heißt angeordent wenn folgende Axiome erfüllt sind
    (seien $a,b,c \in \mathbb{K}$):
    \begin{itemize}
         \item $a<b\ \vee\ b<a\ \vee\ a=b$
         \item $a<b\ \wedge\ b<c\ \Rightarrow\ a<c$
         \item $a<b \Rightarrow a+c<b+c$
         \item $a<b \wedge c>0 \Rightarrow a*c<b*c$
    \end{itemize}
    \textbf{Bem.:}
    $\mathbb{Q}$ und $\mathbb{R}$ sind angeordnete Körper. Für $\mathbb{C}$
    kann keine Ordnungsrelation definiert werden so das alle Axiome erfüllt
    sind.

    \subsubsection{Gebräuchliche Definition zu angeordenten Körpern}
    Für gewöhnlich gilt $0<1$.

     Die Ordnungsrelation wird dann definiert durch:
    \begin{eqnarray*}
        2&:=& 1+1 \\
        3&:=& 2+1 \\
        4&:=&3+1 \\
        &\cdots&
    \end{eqnarray*}

     Die Natürlichen Zahlen werden Induktiv definiert:
    \begin{enumerate}
        \item $1 :\in \mathbb{N}$
        \item $n \in \mathbb{N} \Rightarrow (n+1)\in\mathbb{N}$
    \end{enumerate}

    \textbf{Bem:}
    Aus 2.\ lässt sich direkt ableiten das $\mathbb{N}$ nach oben unbeschränkt ist.

    \subsubsection{Vollständig Angeordnete Körper}
    Ein Körper heißt Vollständig, falls jede nach oben beschränkte, nicht-leere
    teilmenge ein Supremum besitzt.

    \textbf{Bem I:} $\mathbb{R}$ ist der einzige Vollständig angeordnete Körper.

    \textbf{Bem II:} $\mathbb{Q}$ ist nicht vollständig angeordnet, da
    $A := \{x | x^2 \leq 2\} \subset \mathbb{Q}$ kein obere Schrank besitzt
    (obere Schranke ist $\sqrt{2} \notin \mathbb{Q}$).


    \subsection{Minimum und Maximum}
    Sei \Korp\ ein angeordnter Körper und $A \subset$ \Korp\ dann heißt
    $m$ Minimum falls gilt:
    \begin{enumerate}
        \item $m \in \mathbb{K}$
        \item $a \geq m\ \forall a \in A$
    \end{enumerate}
    Analog ist das Maximum definiert:
    Sei \Korp\ ein angeordnter Körper und $A \subset$ \Korp\ dann heißt
    $m$ Maximum falls gilt:
    \begin{enumerate}
        \item $m \in \mathbb{K}$
        \item $a \leq m\ \forall a \in A$
    \end{enumerate}
    \textbf{Schreibweisen:}
    $m = \min{(A)}$ bzw.\ $m = \max{(A)}$\\
    \textbf{Bem.:}
    Minimum und Maximum exisitieren nicht immer.\\
    \textbf{Beispiel:} $A := \{x | x>0\}\subset\ \mathbb{R}$
    hat nicht 0 als Minimum da $0 \notin A$ und kein beliebiges $m$ da $\tilde{m} := \frac{m}{2} < m\ \forall m \in A$

    \subsection{Obere und untere Schranke}
    Sei \Korp\ ein angeordenter Körper und $A \subset$ \Korp\ dann ist $s$ untere
    Schranke falls gilt:
    \begin{itemize}
        \item $s \geq a\ \forall a \in A$
    \end{itemize}

     Analog ist die obere Schranke definiert:
    Sei \Korp\ ein angeordenter Körper und $A \subset$ \Korp\ dann ist $s$ obere
    Schranke falls gilt:
    \begin{itemize}
        \item $s \leq a\ \forall a \in A$
    \end{itemize}

    \textbf{Bem.:} Hat eine Menge eine obere (bzw.\ untere) Schranke
    heißt er nach oben (bzw.\ unten) beschränkt. Ist eine Menge nach unten und
    oben beschränkt bezeichnet man sie als beschränkt.

    \subsection{Supremum und Infimum}
    $s$ heißt Infimum (größte untere Schranke) falls gilt:
    \begin{itemize}
        \item $s$ ist untere Schranke
        \item Falls $\tilde{s}$ ebenfalls untere Schranke ist gilt
        $s\geq\tilde{s}$
    \end{itemize}

     Analog ist das Supremum definiert: $s$ heißt Supremum (kleinste obere Schranke) falls gilt:
    \begin{itemize}
        \item $s$ ist obere Schranke
        \item Falls $\tilde{s}$ ebenfalls obere Schranke ist gilt
        $s\leq\tilde{s}$
    \end{itemize}

    \textbf{Bem.:}
    Wenn Minimum (bzw. Maximum) existieren sind diese gleich dem
    Infimum (bzw. Supremum).

    \textbf{Schreibweise:}
    $s = \inf{(A)}$ bzw. $s = \sup{(A)}$


    \section{Folgen}
    Eine Folge $a_n$ ist definiert als eine Funktion:
    \begin{equation*}
        a_n := \varphi: \mathbb{N} \rightarrow \mathbb{M} \subset \mathbb{R}
    \end{equation*}
    oder auch ${(a_n)}_{n=1}^\infty$.

    \subsection{Konvergenz}
    Eine Folge $a_n$ heißt konvergent wenn gilt:
    \begin{equation*}
        \forall \varepsilon>0\ \exists~n_0(\varepsilon):\ \abs{a_n - a} < \varepsilon\ \forall \varepsilon > n_0(\varepsilon)
    \end{equation*}

    \textbf{Bem.:}
    Der Grenzwert ist eindeutig, d.h.\ es existiert nur ein Grenzwert.

    \subsubsection{Schreibweise}
    Falls $a_n$ gegen $a$ konvergiert schreibt man:
    \begin{equation*}
        \lim_{n \rightarrow \infty} a_n = a
    \end{equation*}

    \subsection{Bestimmte Divergenz}
    Eine Folge $a_n$ heißt bestimmt Divergent wenn gilt
    \begin{equation*}
        \forall x \in \mathbb{R}\ \exists n(x):\ a_n>x \text{ bzw. } a_n<x
    \end{equation*}
    \textbf{Schreibweise:}
    \begin{equation*}
        \lim_{n \rightarrow \infty} a_n = \infty \text{ bzw. } -\infty
    \end{equation*}

    \subsection{Beschränktheit}
    Eine Folge heißt beschränkt wenn gilt:
    \begin{equation*}
        \abs{a_n} < c\ \forall n
    \end{equation*}

    \subsubsection{Beschränktheit nach oben/unten}
    Eine Folge heißt nach oben (bzw.~unten) beschränkt wenn gilt:
    \begin{equation*}
        a_n < n\ \forall n \in \mathbb{N} \text{ bzw. }a_n > c\ \forall n \in \mathbb{N}
    \end{equation*}

    \subsection{Zusammenhang Konvergenz --- Beschränktheit}
    Jede Konvergente Folge ist beschränkt.

    \subsection{Grenzwertrechenregeln}
    Seien ${(a_n)}_{n=1}^\infty$, ${(b_n)}_{n=1}^\infty$, ${(c_n)}_{n=1}^\infty$ Folgen
    in $\mathbb{C}$ mit:
    \begin{equation*}
        \lim_{n \rightarrow \infty} a_n = a \text{ und }
        \lim_{n \rightarrow \infty} b_n = b
    \end{equation*}

    Dann gilt:

    \begin{itemize}
        \item $\lim\limits_{n \rightarrow \infty} \abs{a_n} = \abs{a}$
        \item $\lim\limits_{n \rightarrow \infty}(a_n + b_n) = a + b$
        \item $\lim\limits_{n \rightarrow \infty}(a_n \cdot b_n) = a \cdot b$
        \item Falls $b \neq 0$:
        $\lim\limits_{n \rightarrow \infty}\dfrac{a_n}{b_n} = \dfrac{a}{b}$
    \end{itemize}

    \subsection{Sandwich Theorem u.a.}
    Seien ${(a_n)}_{n=1}^\infty$, ${(b_n)}_{n=1}^\infty$, ${(c_n)}_{n=1}^\infty$
    Folgen in $\mathbb{R}$ mit:
    \begin{equation*}
        \lim_{n \rightarrow \infty}a_n = a \text{, }
        \lim_{n \rightarrow \infty}b_n = b \text{ und } \gamma \in \mathbb{R}
    \end{equation*}

    Dann gilt:

    \begin{itemize}
        \item $a_n \leq \gamma\ \forall n \in \mathbb{N} \Rightarrow a \leq \gamma$
        \item $a_n \geq \gamma\ \forall n \in \mathbb{N} \Rightarrow a \geq \gamma$
        \item $a_n \leq b_n\ \forall n \in \mathbb{N} \Rightarrow a \leq b$
        \item $a_n \leq c_n \leq b_n\ \forall n \in \mathbb{N} \wedge a=b
        \Rightarrow c=\lim\limits_{n \rightarrow \infty} c_n = a = b$
    \end{itemize}

    \subsection{Monotonie}
    Eine Folge ${(a_n)}_{n=1}^\infty$ in $\mathbb{R}$ heißt:
    \begin{itemize}
        \item Monoton wachsend falls: $a_{n+1} \geq {a_n}\ \forall n \in \mathbb{N}$ (Schreibweise: $a_n \nearrow$)
        \item Monoton fallend falls: $a_{n+1} \leq {a_n}\ \forall n \in \mathbb{N}$ (Schreibweise: $a_n \searrow$)
        \item Streng Monoton wachsend falls: $a_{n+1} > {a_n}\ \forall n \in \mathbb{N}$ (Schreibweise: $a_n \uparrow$)
        \item Streng Monoton fallend falls: $a_{n+1} < {a_n}\ \forall n \in \mathbb{N}$ (Schreibweise: $a_n \downarrow$)
    \end{itemize}

    \subsection{Zusammenhang Monotonie und Beschränktheit}
    Jede Monotone und beschränkte Folge konvergiert.

    \section{Häufungswerte}
\end{document}
